% Created 2021-03-11 Thu 23:05
% Intended LaTeX compiler: pdflatex
\documentclass[11pt]{article}
\usepackage[utf8]{inputenc}
\usepackage[T1]{fontenc}
\usepackage{graphicx}
\usepackage{grffile}
\usepackage{longtable}
\usepackage{wrapfig}
\usepackage{rotating}
\usepackage[normalem]{ulem}
\usepackage{amsmath}
\usepackage{textcomp}
\usepackage{amssymb}
\usepackage{capt-of}
\usepackage{hyperref}
\author{Alexis}
\date{\today}
\title{Topo\textsubscript{natremie}}
\hypersetup{
 pdfauthor={Alexis},
 pdftitle={Topo\textsubscript{natremie}},
 pdfkeywords={},
 pdfsubject={},
 pdfcreator={Emacs 27.1 (Org mode 9.5)}, 
 pdflang={English}}
\begin{document}

\maketitle
\tableofcontents

On veut éviter que le sang ne soit trop dilué, ou on contraire pas assez dilué.
Pour ça, on regarde la concentration des molécules qui sont \textbf{osmotiquement actives}, c'est-à-dire qui peuvent franchir la paroi des vaisseaux sanguins.
En gros il y a surtout du sodium et un peu de glucose. C'est pour ça qu'on calcule l'osmolalité plasmatique :

$$ 2 [NA^+] + glycémie $$

Donc la natrémie n'est qu'une partie de l'équation ! Mais on regarde ça en pratique clinique.

\section{Cas 1: hyponatrémie (sang trop dilué ?)}
\label{sec:org34334d9}
\end{document}
