\input header

\begin{document}
\title{Fiches de cardiologie}
\author{Alexis Praga}
\maketitle
\tableofcontents

\section{218 : Athérome}%
\label{sec:1_atherome}

Épidémio : 1ere cause de mortalité dans le monde. 

En France : incidence \male = 5$\times$\female. 

Mortalité $\searrow$ mais prévalence $\nearrow$

\subsection{Mécanisme}
Contient centre lipiqude, cellules {spumeuses,muscularise, inflammatoire} +
chape fibreuse

Évolution de la plaque :
\begin{itemize}
  \item rupture (plus probable si plaque jeune !)
  \item progression par poussées
  \item hémorragie intraplaque
  \item régression ?
\end{itemize}
Remodelage

Anévrismes

\paragraph{Localisations}
Surtout : carotides (AVC), coronaires (cardiopathies ischémiques), membre inférieure
(AOMI)

\paragraph{Évolution} Aggravation par étapes silencieuses. \danger gravité pas
toujours proportionnelle à l'ancienneté/étendue

FDR : tabagisme, HTA, dyslipidémie, diabète

\paragraph{Thérapeutiques}
Prévention du développement de l'athérome : diminuer lésion endothéliale,
diminuer accumulation LDL, stabiliser plaques, diminuer volume des plaques,
diminuer l'inflammation, diminuer les contraintes mécaniques

\subsection{Polyathéromateux}

$\ge 2$ territoire artériels différents

Évaluer FdR, bilan des lésions

Thérapeutiques :
\begin{itemize}
  \item arrêt tabac, diététique, activité physique
  \item aspirine en systématique (colpidogrel si intolérance)
  \item statines en prévention secondaire
  \item IEC\footnote{Inhibiteurs de l'enzyme de conversion}, ARA
      II\footnote{antagonistes des récepteurs de l'angiotensine}
\end{itemize}

PEC spécifique : chirurgie anévrisme ($\diameter \ge 5.5cm$), endartériectomie
(sténose carotide > 60\%), revasc. myocardique (sd coronaire aigü $\wedge$
sténose coronaires > 70\%)

\section{219 : Facteur de risques cardio-vasculaires}%
\label{sec:219_facteur_de_risques_cardio_vasculaires}

Facteur de risque (FR) : causalité avec la maladie $\neq$ marqueur de risque
(simple témoin)

\subsection{FR}
Non modifiables : 10 ans + tôt chez \male, hérédité = plutôto environnement
familial

Modifiables : 
\begin{itemize}
  \item risque : {tabagisme, hypercholestérolémie, HTA, diabète, obésité abdo,
    psychosociaux}
  \item protecteur : {fruit et légumes, activité physique, alcool modéré}
\end{itemize}

\paragraph{Tabac}
1ère cause de mortalité évitable.

Conséquence : \dec HDL, \inc risque thrombose, altère vasomotricité artérielles,
\inc [CO]

2eme FR de l'IDM : $\propto$ consommation, $\forall$ tabac, jeunes, passif

Rôle : AOMI, anévrisme aorte abdo, AVC

\paragraph{Hypercholestérolémie}
3eme FR IDM : \inc LDL et \dec HDL = mauvais signe $\implies$ exploration d'une
anomalie lipidique à jeun

Majorité = alimentaire mais génétique possible (hétérozygote/homozygote)

\paragraph{HTA}
Stade 1 : [140-159]/[90-99] mmHg
Stade 2 : [160-179]/[100-109] mmHg
Stade 3 : > 180/110 mmHg

Silencieuse. Impact coeur (insuf. coronire, cardiaque), cerveau (AVC), rein (IR)

Augmente avec l'âge.

3 mesure espaces d'1 semaine

\paragraph{Diabète}
90\% de diabète 2 (résistance insuline). Déf :
\begin{itemize}
  \item diabète si glycémie à jeun > 1.26g/L
  \item hyperglycémie non diab : glycémie jeun $\in [1.10, 1.26]$ g/L
  \item intolérance hydrates de carbones : < 1.26 (jeun), $\ge 2$ (provoquée)
    puis $\in [1.40, 2]$
\end{itemize}
Hérédité. Complications microvasc, macrovasc

\paragraph{Surpoids}
IMC $\in [25, 29.9]$ = surpoids, IMC $\ge 30$ = obésité. 

Obésité centrale = (\diameter abdo $\ge 94 $cm (\male) ou $\ge 80$cm (\female))
\land 2 FR


\subsection{Évaluation}
\label{subsec:fr}
Score
\begin{itemize}
  \item +1 si {tabac $\le 3$ ans, LDL > 1.6g/L, HTA, diabète, HDL < 0.40g/L, âge > 50
(\male) ou 60 (\female), ATCD coronaires}
  \item  -1 si HDL $\ge 0.60$
\end{itemize}

ATD personnels CV

\subsection{Prévention}
\paragraph{Secondaire}

BASIC : $\beta$bloquants, Antiagrégants, Statine, Inhibiteurs de l'enzyme de
conversion, Contrôle des FR

\begin{itemize}
  \item statine pour LDL < 1g/L
\item sevrage tabac : substituts nicotinique, {bupropion, varénicline},
  anxiété/dépression, TCG
  \item pression artérielle : hygiénodiététique (échec à 3 mois : médic)
  \item contrôle glycémie (diabète)
  \item activité physique régulière
  \item enquête familiale
\end{itemize}

\paragraph{Primaire}
Cholestérol : 0FR : LDL < 2.20, 1FR : LDl < 1.90, 2FR : LDl < 1.60, $\ge 3$ FR
: LDL < 1.30 (haut risque : LDL < 1g/L)







\section{220 : Dyslipidémies}%
\label{sec:220_dyslipidemies}

Risques : maladies CV athéromateuses

LDL = total - HDL - TG\footnote{tryglycérides}

Bilan normal : \begin{itemize}
  \item LDL  < 1.6g/L
  \item HDL  > 0.4g/L
  \item TG  < 1.5g/L
\end{itemize}

\paragraph{Hyperlipidémies secondaire } hypothyroïdie, cholestase, sd néphrotique, IR chronique, alcoolisme,
diabète, hyperlipidémie iatrogènee, oestrogènes, corticoïdes, rétinoïdes,
antirétroviraux, ciclosporine, diurétiques

\paragraph{Hyperlipidémies primitives}
\textit{Fréquentes}  : hypercholestérolémie familiale monogénique (HCFM) (mutation LDL récepteur
hétérozygote), hypercholestérolémie polygénique, hyperlipidémie familiale
combinée

\textit{Rares}  : HCFM (mutation apolipoprotéine B),
dysbêtalipoprotéinémie, hypertriglycéridémie familiale, hypechylomicronémie
primitives

\paragraph{Risque} faible (0 FR), intermédiarie ($\ge 1$ FR), haut (ATCD)

FR semblables au~\hyperref[subsec:fr]{score précédent} : tabac $\le 3$ ans, HTA, diabète, HDL < 0.40g/L, âge > 50
(\male) ou 60 (\female), ATCD familaux IDM ou mort subite

\subsection{Traitement}

\paragraph{Diététique}
\begin{itemize}
  \item lipides < 40\%
  \item graisses saturées < 12\%
  \item plutôt mono- et polyinsaturées
  \item cholestérol alimentaire < 300mg/j
  \item 5 fruits ou légumes/j
  \item sodium < 6g/j
  \item diminuer excès pondéral
\end{itemize}
Si hypertriglycéridémie (HTG) : \dec poids, alcool, sucres simples

\paragraph{Médicaments}
Primaire : seulement +3 mois après diététique. Secondaire : d'emblée.

\danger pas de statine si grossesse

Hypercholestérolémie, hyperlipidémie mixtes : statines (1er)

HTG : fibrate si TG > 4g/L, diététique sinon



\section{334 : Syndromes coronariers aigüs}%
\label{sec:334_syndromes_coronariers_aigus}

Sd coronaire aigü (SCA) : lésions athérothrombotiques aigües

Angor stable à l'effort : lésions fibro-athéromateuses

\subsection{Angine de poitrine (angor) stable}
Ici : pas de thrombus

Inadéquation besoin/apport $O_2$ : 95\% sténoses athéromateuses coronariennes
serrées (parfois : spasme coronaire, \inc besoins, "à coronires saines")

Mécanisme d'aptation d'apport en $O_2$ du myocorde = vasodilatation (surtout par
sécrétion monoxyde d'azode par l'endothélium)

Cascade ischémique : \dec perfusion myocarde [scinti] \thus altération
contractilité [écho stress] \thus signes ECG \thus douleur (pas toujours)

Athérome : risque = fracture de plaque \thus (thrombose) mort subit/IDM, angor
instable

\paragraph{Diagnostic}
Douleur angineuse\footnote{Classes de 1 à 4}
\begin{itemize}
  \item typique : rétrosternal en barre horizontale, irradiant (épaules,
    avant-bras, poignet, machoîres), constrictive, angoissante, \textbf{à
    l'effort}, sensible à trinitrine
  \item atypique ou silencieuse possible
\end{itemize}
Exaen clinique souvent négatif mais chercher souffle aortique, souffle vasc, HTA

\paragraph{Examens}
\begin{itemize}
  \item ECG : intercritique = normal, percritique : (sus/sous)-décalage
    ST, ondes T (négatives symétriques, amples positives symétrique)
  \item ECG d'effort : \textit{1ere intention} . Positive si douleur thoracique \lor
    sous-decalage ST
  \item Tomoscintigraphie myocardique de perfussion d'effort ou injection
    vasodilatateur (dipyridamole) : segment normal/ischémie/nécrotique.
    \textit{Lorsque VP ECG insuffisante}. Coûteux. Éviter si BBG\footnote{bloc
    de branche gauche}
  \item Échocardiographie d'effort ou dobutamine. \textit{Mêmes indication que
    scinti} 
  \item IRM stresse : rare
  \item Coronarographie (parfois + venticulalographie) : sténose si > 70\%
    lumière. Invasif, complications rare. \textit{Angor suspecté et examen
    d'ischémie positif}\footnote{Examens complémentaires : test Méthergin pour
      forcer un spasme, FFR (fraction flow reserve) pour vérifier sténose}
  \item Scanner coronaire : non recommandé
\end{itemize}

\danger CI des épreuves de stresse : angor instable, troubles rythme ventriculaire
graves, fibrillation auriculaire rapide, HTA repos > 220/120mmHg

\paragraph{Mauvais pronostic} : 
\begin{itemize}
  \item angor classe 3/4
  \item ischémie pour charge/fréquence cardiaque faible, baisse PA à l'effort
  \item plusieurs segments ischémique, fraction d'éjection <
    40\%\footnote{normale si > 55\%}
  \item lésions pluritronculaire, tronc coronaire g, IVA proximale
\end{itemize}

\paragraph{Traitements}
Crise : arrêt effort, dérivés nitrés.

Correction FR (tabac, hypolipides, activisé physique, HTA, diabètes, statine,
IEC)

Anti-ischémique (traitement de fond) en première intention : $\beta$bloquant
(anticalcique/ivbradine si intolérance) $\pm$ {dérivés nitrés, molsidomine,
nicorandil}

Antiagrégants plaquettaires :
\begin{itemize}
  \item aspirine (sauf CI) 75mg/j
  \item clopidogrel sinon 75mg
\end{itemize}
Revascularisation si échec médicament ou pour améliorer le pronostic vital
\begin{itemize}
  \item intervention coronaire percutanée (IPC) : stent
  \item pontage coronaire
\end{itemize}

\paragraph{Angor de Prinzmetal} Vaspastique = douleur sensible trinitrine et
\begin{itemize}
  \item au repos, 2eme partie de nuit, récupération = angor de Prinzmetal
  \item sur un effort = surimposé à une sténose
\end{itemize}
Diagnostic : coronarographie \thus test provocation spasme (pendant coronaro)

Ttt : inhibiteurs calcique (2 molécules).

Bon pronostic si traité

\subsection{SCA sans sus-décalage ST}

= {angor instable, IDm sans sus-décalage ST persistant }. Ici thrombus non
occlusif

\subsection{IDM}
Ici thrombus occlusif après réaction thrombotique

\paragraph{Diagnostic}
Même douleur que l'angor stable mais 
\begin{itemize}
  \item spontané > 20min, régressant spontanément ou à trinitrine
  \item angor d'effort récent (2-3)
  \item aggravation d'un angor stable
  \item IDM + 1mois
\end{itemize}
Examen clinique normal mais chercher râles crépitants, galop

ECG en urgence \skull puis +6h
\begin{itemize}
  \item percritique : sous-décalage ST (rarement sus), (grandes T négatives \lor
    repositivation T). Si normal, diagnostic peu probable
  \item post-critique (être très prudent !) : sous-déclage ST, T négative
    profonde
\end{itemize}
Doser troponine ssi suspicion !

Échocardiographie pour DD

Coronarographie suivant le risque :
\begin{itemize}
  \item très haut risque : en urgence !
  \item haut risque : < 24h (score GRACE > 140) ou < 72 (GRACE $\in [109, 140]$)
  item bas risque (GRACE < 109)  à discuter 
\end{itemize}

\paragraph{Traitement}
USIC : suivi ECG; dosage troponine, crétaninie, glycémie, NFS

Aspirine, inhibiteurs de la pompe à prontos, anti ischémique et :
\begin{itemize}
  \item bas risque : inhib P2Y12 (clopidogrel) et anticoag (fondaparinux)
  \item (très) haut risque : inhib P2Y12 (ticagrelor/prasugrel) et anticoag
    (énoxaparine/héparine) (+ anti-GPIIb/IIIa suivant)
\end{itemize}

\subsection{IDM}

Ici, obstruction par thrombus

5 catégories : 1 à 5. Type 1 = 
\begin{itemize}
  \item sus-ST : désobstruer ASAP
  \item sans sus-ST : prévenir
\end{itemize}
\danger urgence ! \skull

Physiopatho : accident vasculaire coronaire athérothrombotique occlusif ou
occlusion coronaire aigüe (segmente : nécrose totale à 12h, akinésie)

\paragraph{Diagnostic}
Douleur précordiale : angineuse au repos > 30min, trinitrorésistante (peut
manquer !)

Examen clinique normal

ECG : sus-décalage ST sur $\ge 2$ dérivations contiguës. Donne la topographie
(antérieur/latéral, inférieur/postérieur).
Parfois en miroir

\textbf{Douleur thoracique > 30min \land ECG = IDM ST} 

\paragraph{Évolution}
Sd de reperfusion : \dec douleur, négativation ondes T, T = 38
à +6h

Onde Q de nécrose (diagnostic a posteriori)

Marqueur = troponine (ASAP, +6h, +12h), éventuellement myoglobine (rapide++) ou
CPK-MB si récidive

\paragraph{DD} 
Douleur thoracique : péricardite aigüe, EP, dissection aortique, sous-diaphragme (cholécystite aigüe,
ulcère perforé, pancréatite aigüe).

Simule IDM : Penser à mycocardite aigüe (IRm), cardiomypoathie de stresse
(coronarographie)

\paragraph{Complications précoces}
Rythme/conduction : 
\begin{itemize}
  \item rythme ventriculaire : extrasystole < tachycaride < fibrillation
    ventriculaire (FV = plupart des morts subites ! Besoin d'un choc électrique)
  \item supra-ventriculaire : décompensation hémodynamique, accidents emboliques
  \item BAV\footnote{bloc auricolventriulaire} (transitoire/définitif) ou
    hypevagotonie (bradycardie, hypotension. Ttt : atropine, remplissage
    macromoléculaire)
\end{itemize}

Hémodynamiques
\begin{itemize}
  \item insuf. ventriculaire G : grave, faire échocardio vite (4 stades)
  \item choc cardiogénique : diagnostic si hypotension artérielle mal tolérée,
    ne répond pas au rempilssage macromoléculaire. Souvent OCA + 24/48h.
    Mortalité > 70\%
  \item infarctus ventricule D : hypotension, champs pulmonaires clairs,
    turgescence jugulaire. Regarder dérivations droites (!) : sus-ST.
    Échocardiographie
\end{itemize}
Mécaniques :
\begin{itemize}
  \item rupture paroi libre ventricule G : rapidement fatal
  \item rupture septale : +24-48gh. Échocardiographie doppler. Forte mortalité
  \item insuf mitrale : fuite par prolapsus valvlaire. Ttt chir
\end{itemize}
Thrombotique : thrombus intra-VG, embolies systémique : échocardio. (Thrombose
veineuse, EP)

Péricardituqe : sd inflammatoire, souvent asymptomatique.

Récidive ischémique \thus récidive IDM. Épreuve d'effort à  +5 jours.

\paragraph{Complications tardives}
Péricardite à +3 semaines (sd de Dressler)

Dysfonction ventricule G : scint/échocardio de stress/IRM cardiaque. Évolue en
dilatation VG/anévrisme

Troubles rythmes ventriculaires sévères : défibrillateurs automatique
implantable

\subsection{Traitement}
Reperfusion !!
\begin{itemize}
  \item ICP si < 120min. Aspirine, inhib récepteurs P2Y12, anticoagulant
  \item sinon fibrinolyse VI par TNK-tPA
\end{itemize}
Efficacité : reperfusion dans 90min (50\%). Sd reperfusion

Complications : AVC, réocclusion (surtout si ttt antiagrégant interrompu)

\paragraph{Associé}
\begin{itemize}
  \item antalgique
  \item antiagrégant : aspirine et inhib récepteur P2Y12 [\danger clopidogrel
    seulement si fibronolyse]
  \item anticoagulant : bivalirudine si ICP
  \item $\beta$bloquant (avec prudence)
  \item IEC dans 24h
  \item Éplérénone précocement
\end{itemize}

\paragraph{Des complications}
Troubles rythmes ventriculaire : amiodarone

Troubles rythmes supra-ventriculaire : AVK si mal toléré (hémodynamique)

BAV transitoire : atropine. BAV après IDm antérieur : sonde d'entraînement
électrosystoliques.

Insuf ventriculaire G : diurétique, IEC, épléronone

Choc cardiogénique : lutter contre {hypovolémie, troubles rythme}, sidération
(dobutamine). Assistance circulatoire/cardiaque/cardiocirculation,
revascularisation

Mécanique : rupture paroi libre = mortelle, Septable = suture chir. mitrale =
remplacement valvulaire.

\subsection{Suivi}
\begin{itemize}
  \item antiagrégants plaquettaires : aspirine + clopidogrel (sauf si angor
    stable : aspirine)
  \item statines : si SCA/ango stable
  \item $\beta$bloquant : si infarctus
  \item EAC si coronarions post-infact
  \item épléronone : IDM étendu FEVG < 40à%
\end{itemize}
Éventuellement DAI

\end{document}
