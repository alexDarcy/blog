\documentclass{article}
\usepackage{hyperref}
\input header
\def\arrow{$\rightarrow$}

\usepackage[linesnumbered,ruled,vlined]{algorithm2e}
\usepackage{enumitem}

\newacronym{AT}{AT}{Antithrombine}
\newacronym{TIH}{TIH}{Thrombopénie induite par l'héparine}
\newacronym{FO}{FO}{Fond d'oeil}
\newacronym{ARA II}{ARA II}{Antagonistes des récepteurs de l'angiotensine}
\newacronym{BAV}{BAV}{Bloc auriculoventriculaire}
\newacronym{BBG}{BBG}{Bloc de branche gauche}
\newacronym{BSA}{BSA}{Bloc sinuso-atrial}
\newacronym{DAI}{DAI}{Défibrillateur automatique implantable}
\newacronym{EI}{EI}{Endocardite infectieuse}
\newacronym{EP}{EP}{Embolie pulmonaire}
\newacronym{FA}{FA}{Fibrillation atriale}
\newacronym{FE}{FE}{Fraction d'ejection}
\newacronym{FIV}{FIV}{Fibrinolyse intra-veineuse}
\newacronym{HVG}{HVG}{Hypertrophie ventriculaire gauche}
\newacronym{IC}{IC}{Insuffisance cardiaque} 
\newacronym{IEC}{IEC}{Inhibiteurs de l'enzyme de conversion}
\newacronym{IPC}{IPC}{Intervention coronaire percutanée}
\newacronym{IVA}{IVA}{Artère intraventriculaire antérieure}
\newacronym{JPDC}{JPDC}{Jusqu'à preuve du contraire}
\newacronym{MTEV}{MTEV}{Maladie thromboemolique veineuse}
\newacronym{OCA}{OCA}{Occlusion coronaire aigüe}
\newacronym{OG}{OG}{Oreillette gauche}
\newacronym{PAD}{PAD}{Pression artérielle diastolique}
\newacronym{PAPO}{PAPO}{Pression artérielle pulmonaire d'occlusion, $\approx$ pression oreillette G} 
\newacronym{PAPm}{PAPm}{Pression de l'artère pulmonaire moyenne}
\newacronym{PAS}{PAS}{Pression artérielle systolique}
\newacronym{PA}{PA}{Pression artérielle}
\newacronym{RCT}{RCT}{Rapport cardiothor}
\newacronym{SAS}{SAS}{Syndrome d'apnée du sommeil}
\newacronym{SPT}{SPT}{Syndrome post-thrombotique}
\newacronym{TAVI}{TAVI}{Transcatheter Aortic Valve Implantation}
\newacronym{TG}{TG}{Tryglycérides}
\newacronym{TVP}{TVP}{Thrombose veineuse profonde}
\newacronym{TVS}{TVS}{Thrombose veineuse superficielle}
\newacronym{VG}{VG}{Ventricule gauche}
\glsaddall % Add all entries
\glsunsetall % Acronyms always in short form

\begin{document}
\title{Fiches de cardiologie}
\author{Alexis Praga}
\maketitle
\tableofcontents

\input bacteries-header

\section{218 : Athérome}%
\label{sec:1_atherome}

Épidémio : 1ere cause de mortalité dans le monde. 

En France : incidence \male = 5$\times$\female. 

Mortalité $\searrow$ mais prévalence $\nearrow$

\subsection{Mécanisme}
Contient centre lipidique, cellules \{spumeuses, inflammatoire\} +
chape fibreuse + support musculaire (migration vers l'endothelium)

Évolution de la plaque :
\begin{itemize}
  \item rupture (plus probable si plaque jeune !)
  \item progression par poussées
  \item hémorragie intraplaque
  \item régression ?
\end{itemize}
Remodelage

Anévrismes

\paragraph{Localisations}
Surtout : carotides (AVC), coronaires (cardiopathies ischémiques), membre inférieure
(AOMI), aorte (anévrysme)

\paragraph{Évolution} Aggravation par étapes silencieuses. \danger gravité pas
toujours proportionnelle à l'ancienneté/étendue

FDR : tabagisme, HTA, dyslipidémie, diabète

\paragraph{Thérapeutiques}
\begin{itemize}
\item Prévention : FR, statine, aspirine, hypertenseur
\item Rupture de plaque : antiplaquettaires, héparines
\item PEC des sténoses, complications CV
\item Angioplastie, chir
\end{itemize}
\subsection{Polyathéromateux}

$\ge 2$ territoire artériels différents

Évaluer FdR, bilan des lésions

Thérapeutiques :
\begin{itemize}
  \item arrêt tabac, diététique, activité physique
  \item aspirine en systématique (colpidogrel si intolérance)
  \item statines en prévention secondaire
  \item \gls{IEC}, \gls{ARA II}
\end{itemize}

PEC spécifique : chirurgie anévrisme ($\diameter \ge 5.5cm$), endartériectomie
(sténose carotide > 60\%), revasc. myocardique (sd coronaire aigü $\wedge$
sténose coronaires > 70\%)

\section{219 : Facteur de risques cardio-vasculaires}%
\label{sec:219_facteur_de_risques_cardio_vasculaires}

Facteur de risque (FR) : causalité avec la maladie $\neq$ marqueur de risque
(simple témoin)

\subsection{FR}
Non modifiables : 10 ans + tôt chez \male, hérédité = plutôt environnement
familial

Modifiables : 
\begin{itemize}
  \item risque : {tabagisme, hypercholestérolémie, HTA, diabète, obésité abdo,
    psychosociaux}
  \item protecteur : {fruit et légumes, activité physique, alcool modéré}
\end{itemize}

\paragraph{Tabac}
1ère cause de mortalité évitable.

Conséquence : \dec HDL, \inc risque thrombose, altère vasomotricité artérielles,
\inc [CO]

2eme FR de l'IDM : $\propto$ consommation, $\forall$ tabac, sujet jeune, tabagisme passif

Rôle : AOMI, anévrisme aorte abdo, AVC

\paragraph{Hypercholestérolémie}
3eme FR IDM : \inc LDL et \dec HDL = mauvais signe $\implies$ exploration d'une
anomalie lipidique à jeun

Majorité = alimentaire mais génétique possible (hétérozygote/homozygote)

\paragraph{HTA}
Voir table~\ref{tab:hta_stades}.
\begin{table}
  \centering
  \begin{tabular}{cc}
      Stade 1 & [140-159]/[90-99] mmHg\\
Stade 2 & [160-179]/[100-109] mmHg\\
Stade 3 & > 180/110 mmHg
  \end{tabular}
  \caption{Stades d'HTA}
  \label{tab:hta_stades}
\end{table}

Silencieuse. Impact c\oe{}ur (insuf. coronaire, cardiaque), cerveau (AVC), rein (IR)

Augmente avec l'âge.

3 mesure espacées d'1 semaine

\paragraph{Diabète}
90\% de diabète 2 (résistance insuline). Déf :
\begin{itemize}
  \item diabète si glycémie à jeun > 1.26g/L
  \item hyperglycémie non diab : glycémie jeun $\in [1.10, 1.26]$ g/L
  \item intolérance hydrates de carbones : < 1.26 (jeun), $\ge 2$ (provoquée)
    puis $\in [1.40, 2]$
\end{itemize}
Hérédité. Complications microvasc, macrovasc

\paragraph{Surpoids}
IMC $\in [25, 29.9]$ = surpoids, IMC $\ge 30$ = obésité. 

Obésité centrale = (\diameter{} abdo $\ge 94 $cm (\male) ou $\ge 80$cm (\female))
et 2 FR


\subsection{Évaluation}
\label{subsec:fr}
Score
\begin{itemize}
  \item +1 si {arrêt tabac $\le 3$ ans, LDL > 1.6g/L, HTA, diabète, HDL < 0.40g/L, âge > 50
(\male) ou 60 (\female), ATCD coronaires}
  \item  -1 si HDL $\ge 0.60$
\end{itemize}

ATD personnels CV

\subsection{Prévention}
\paragraph{Secondaire}
\label{subsec:basic_prev}        
BASIC : $\beta$bloquants, Antiagrégants, Statine, Inhibiteurs de l'enzyme de
conversion, Contrôle des FR

\begin{itemize}
  \item statine pour LDL < 1g/L
\item sevrage tabac : substituts nicotinique, {bupropion, varénicline},
  anxiété/dépression, TCG
  \item pression artérielle : hygiénodiététique (échec à 3 mois : médic)
  \item contrôle glycémie (diabète)
  \item activité physique régulière
  \item enquête familiale
\end{itemize}

\paragraph{Primaire}
Voir table~\ref{tab:cholestérol}.

\begin{table}
  \centering
  \begin{tabular}{cc}
    0 FR & LDL < 2.20 \\
    1 FR & LDL < 1.90\\
    2 FR & LDL < 1.60\\
    $\ge 3$ FR& LDL < 1.30 \\
    haut risque & LDL < 1g/L
  \end{tabular}
  \caption{FR du cholestérol}
  \label{tab:cholesterol}
\end{table}
\section{220 : Dyslipidémies}%
\label{sec:220_dyslipidemies}

Risques : maladies CV athéromateuses

LDL = total - HDL - \gls{TG}

Bilan normal : \begin{itemize}
  \item LDL  < 1.6g/L
  \item HDL  > 0.4g/L
  \item TG  < 1.5g/L
\end{itemize}

\paragraph{Hyperlipidémies secondaire } hypothyroïdie, cholestase, sd néphrotique, IR chronique, alcoolisme,
diabète, hyperlipidémie iatrogènee, oestrogènes, corticoïdes, rétinoïdes,
antirétroviraux, ciclosporine, diurétiques

\paragraph{Hyperlipidémies primitives}\mbox{}\\
\textit{Fréquentes}  : hypercholestérolémie familiale monogénique (HCFM) (mutation LDL récepteur
hétérozygote), hypercholestérolémie polygénique, hyperlipidémie familiale
combinée

\textit{Rares}  : HCFM (mutation apolipoprotéine B),
dysbêtalipoprotéinémie, hypertriglycéridémie familiale, hypechylomicronémie
primitives

\paragraph{Risque} faible (0 FR), intermédiaire ($\ge 1$ FR), haut (ATCD)

FR semblables au~\hyperref[subsec:fr]{score précédent} : tabac $\le 3$ ans, HTA, diabète, HDL < 0.40g/L, âge > 50
(\male) ou 60 (\female), ATCD familiaux IDM ou mort subite

\subsection{Traitement}

\paragraph{Diététique}
\begin{itemize}
  \item lipides < 40\%
  \item graisses saturées < 12\%
  \item plutôt mono- et polyinsaturées
  \item cholestérol alimentaire < 300mg/j
  \item 5 fruits ou légumes/j
  \item sodium < 6g/j
  \item diminuer excès pondéral
\end{itemize}
Si hypertriglycéridémie (HTG) : \dec poids, alcool, sucres simples

\paragraph{Médicaments}
Primaire : seulement +3 mois après diététique. Secondaire : d'emblée.

\danger pas de statine si grossesse

Hypercholestérolémie, hyperlipidémie mixtes : statines (1er)

HTG : fibrate si TG > 4g/L, diététique sinon



\section{334 : Syndromes coronariens aigüs}%
\label{sec:334_syndromes_coronariens_aigus}

Sd coronaire aigü (SCA) : lésions athérothrombotiques aigües

Angor stable à l'effort : lésions fibro-athéromateuses

\subsection{Angine de poitrine (angor) stable}
Ici : pas de thrombus

Inadéquation besoin/apport $O_2$ : 95\% sténoses athéromateuses coronariennes
serrées (parfois : spasme coronaire, \inc besoins, "à coronaires saines")

Donc le myocarde s'adapte en vasodilatant (pour apport $O_2$)\footnote{Surtout par
sécrétion monoxyde d'azode par l'endothélium}

Donc cascade ischémique : \dec perfusion myocarde [scinti] \thus altération
contractilité [écho stress] \thus signes ECG \thus douleur (pas toujours)

Athérome : risque = fracture de plaque \thus (thrombose) mort subite/IDM, angor
instable

\paragraph{Diagnostic}
Douleur angineuse\footnote{Classes de 1 à 4}
\begin{itemize}
  \item typique : rétrosternal en barre horizontale, irradiant (épaules,
    avant-bras, poignet, machoîres), constrictive, angoissante, \textbf{à
    l'effort}, \underline{sensible à trinitrine}
  \item atypique ou silencieuse possible
\end{itemize}
Exaen clinique souvent négatif mais chercher souffle aortique, souffle vasc, HTA

\paragraph{Examens}
\begin{itemize}
  \item ECG : intercritique = normal, percritique : (sus/sous)-décalage
    ST, ondes T (négatives symétriques, amples positives symétrique)
  \item ECG d'effort : \textit{1ere intention} . Positive si douleur thoracique ou
    sous-decalage ST
  \item Tomoscintigraphie myocardique de perfusion d'effort ou injection
    vasodilatateur (dipyridamole) : segment normal/ischémie/nécrotique.
    \textit{Lorsque VP ECG insuffisante}. Coûteux. Éviter si \gls{BBG}
  \item Échocardiographie d'effort ou dobutamine. \textit{Mêmes indication que
    scinti} 
  \item IRM de stress : rare
  \item Coronarographie (parfois + ventriculographie) : sténose si > 70\%
    lumière. Invasif, complications rare. \textit{Si angor suspecté et examen
    d'ischémie positif}\footnote{Examens complémentaires : test Méthergin pour
      forcer un spasme, FFR (fraction flow reserve) pour vérifier sténose}
  \item Scanner coronaire : non recommandé
\end{itemize}

\danger CI des épreuves de stress : angor instable, troubles rythme ventriculaire
graves, fibrillation auriculaire rapide, HTA repos > 220/120mmHg

\paragraph{Mauvais pronostic} : 
\begin{itemize}
  \item angor classe 3/4
  \item ischémie pour charge/fréquence cardiaque faible, baisse PA à l'effort
  \item plusieurs segments ischémique, fraction d'éjection <
    40\%\footnote{Normale si > 55\%}
  \item lésions pluritronculaires, tronc coronaire G, \gls{IVA} proximale
\end{itemize}

\paragraph{PEC}
\begin{enumerate}
\item Crise : arrêt effort, dérivés nitrés.
\item Correction FR (tabac, hypolipides, activité physique, HTA, diabètes, statine,
IEC)
\item aspirine\footnote{Anti-agrégant plaquettaire} 75mg/j (ou
  clopidogrel\footnote{Idem mais inhibiteur des récepteurs P2Y12} 75mg) +
  $\beta$-bloquant (anticalcique/ivabradine si intolérance) $\pm$ {dérivés nitrés, molsidomine,
nicorandil}
\item Revascularisation si échec médicament ou pour améliorer le pronostic vital
  : \gls{IPC} (stent) ou pontage coronaire
\end{enumerate}

\paragraph{Angor de Prinzmetal} Vasospastique = douleur sensible à la trinitrine
et, soit:
\begin{itemize}
  \item au repos, 2eme partie de nuit, récupération d'effort = angor de Prinzmetal
  \item sur un effort = angor surimposé à une sténose
\end{itemize}
Diagnostic : coronarographie \thus test provocation spasme (pendant coronaro)

Ttt : inhibiteurs calcique (2 molécules).
Bon pronostic si traité

\subsection{SCA sans sus-décalage ST}

= \{angor instable, IDM sans sus-décalage ST persistant \}. Ici thrombus non
occlusif

\begin{figure}[htpb]
  \centering
  \resizebox{0.6\linewidth}{!}{
    \tikz \graph [
    % Labels at the middle 
    edge quotes mid,
    % Needed for multi-lines
    nodes={align=center},
    sibling distance=3cm,
    layer distance=2cm,
    edges={nodes={fill=white}}, 
    layered layout]
    {
      "SCA sans sus-décalage ST" ->{
        Angor instable [>"tropo=0"];
        "IDM ST-"[>"tropo +"];
      };
      "SCA avec sus-décalage ST" -> "IDM ST+"[>"tropo +"];
    };
  }
  \caption{Classification des SCA (hor}
\end{figure}

\paragraph{Diagnostic}
Même douleur que l'angor stable mais 
\begin{itemize}
  \item \textbf{spontané > 20min}, régressant spontanément ou non à trinitrine
  \item angor d'effort récent (2-3)
  \item aggravation d'un angor stable
  \item IDM + 1mois
\end{itemize}
Examen clinique normal mais chercher râles crépitants, galop

ECG en urgence \skull puis +6h
\begin{itemize}
  \item percritique : sous-décalage ST (rarement sus), (grandes T négatives ou
    repositivation T). Si normal, diagnostic peu probable
  \item post-critique (être très prudent !) : sous-décalage ST, T négative
    profonde
\end{itemize}

\paragraph{PEC}
\begin{enumerate}
\item USIC en urgence ! Avec ECG, dosage troponine, créatinine, glycémie, NFS
  \begin{itemize}
  \item aspirine
  \item +
    $\begin{cases}
      \text{clopidogrel + fondaparinux si bas risque}\footnotemark\\
      \addtocounter{footnote}{-1}
\text{ticagrelor/prasugrel + HNF/HBPM (+ anti-GPIIb/IIIa) si haut risque}\footnotemark

\end{cases}$
\footnotetext{Anti-agrégant plaquettaire et anticoagulant respectivement}
  \item + $\beta$-bloquant + statine $\pm$ dérivé nitré $\pm$ inhibiteur
    calcique\footnote{Anti-ischémiques}
  \end{itemize}
\item si (risque élevé et Grace > 140) ou (risque faible mais élevé à +6/12h) :
  poursuite médic + coronarographie + angioplastie
\item sinon, tests non invasifs
\end{enumerate}

Notes :
\begin{itemize}
\item Doser troponine ssi suspicion !
\item Échocardiographie pour DD
\item Coronarographie suivant le risque :
\begin{itemize}
  \item très haut risque : en urgence !
  \item haut risque : < 24h (score GRACE > 140) ou < 72 (GRACE $\in [109, 140]$)
  item bas risque (GRACE < 109)  à discuter 
\end{itemize}
\end{itemize}

\subsection{IDM}

Ici, obstruction par thrombus

5 catégories : 1 à 5. Type 1 (spontané) =
\begin{itemize}
  \item sus-ST : désobstruer ASAP
  \item sans sus-ST : prévenir
\end{itemize}
\danger urgence ! \skull

Physiopatho : accident vasculaire coronaire athérothrombotique occlusif ou
occlusion coronaire aigüe (segmente : nécrose totale à 12h, akinésie)

\paragraph{Diagnostic}
Douleur précordiale : angineuse \textbf{au repos > 30min},
\underline{trinitrorésistante} (la douleur peut manquer !)

Examen clinique normal

ECG : sus-décalage ST sur $\ge 2$ dérivations contiguës. Donne la topographie
(antérieur/latéral, inférieur/postérieur).
Parfois en miroir

\fbox{(Douleur thoracique > 30min) et ECG = IDM ST} 

\paragraph{Évolution}
Sd de reperfusion : \dec douleur, négativation ondes T, T = $38^{\circ}$
à +6h

Onde Q de nécrose (diagnostic a posteriori)

Marqueur = troponine (ASAP, +6h, +12h), éventuellement myoglobine (rapide++) ou
CPK-MB si récidive

\paragraph{DD} 
Douleur thoracique : péricardite aigüe, EP, dissection aortique, sous-diaphragme (cholécystite aigüe,
ulcère perforé, pancréatite aigüe).

Simule IDM : Penser à mycocardite aigüe (IRM), cardiomyopathie de stress
(coronarographie)

\paragraph{Complications précoces}
Rythme/conduction : 
\begin{itemize}
  \item rythme ventriculaire : extrasystole < tachycardie < fibrillation
    ventriculaire (FV = plupart des morts subites ! Besoin d'un choc électrique)
  \item supra-ventriculaire : décompensation hémodynamique, accidents emboliques
  \item \gls{BAV} (transitoire/définitif) ou
    hypervagotonie\footnote{Bradycardie, hypotension} (Ttt : atropine, remplissage
    macromoléculaire)
\end{itemize}

Hémodynamiques
\begin{itemize}
  \item insuf. ventriculaire G : grave, faire échocardio vite (4 stades)
  \item choc cardiogénique : diagnostic si hypotension artérielle mal tolérée,
    ne répond pas au rempilssage macromoléculaire. Souvent \gls{OCA} + 24/48h.
    Mortalité > 70\%
  \item infarctus ventricule D : hypotension, champs pulmonaires clairs,
    turgescence jugulaire. Regarder dérivations droites (!) : sus-ST.
    Échocardiographie
\end{itemize}
Mécaniques :
\begin{itemize}
  \item rupture paroi libre ventricule G : rapidement fatal
  \item rupture septale : +24-48h. Échocardiographie doppler. Forte mortalité
  \item insuf mitrale : fuite par prolapsus valvlaire. Ttt chir
\end{itemize}
Thrombotique : thrombus intra-VG, embolies systémique : échocardio. (Thrombose
veineuse, EP)

Péricardite : sd inflammatoire, souvent asymptomatique.

Récidive ischémique \thus récidive IDM. Épreuve d'effort à  +5 jours.

\paragraph{Complications tardives}
Péricardite à +3 semaines (sd de Dressler)

Dysfonction ventricule G : scinti/échocardio de stress/IRM cardiaque. Évolue en
dilatation VG/anévrisme

Troubles rythmes ventriculaires sévères : \gls{DAI}

\subsection{PEC}
Reperfusion !!
\begin{enumerate}
\item Si angioplastie faisable < 120min ou CI à la \gls{FIV} : salle de cathétérisme +
  angioplastie
\item sinon FIV par TNK-tPA. Si échec, angioplastie de sautage
\item \textbf{en même temps } 
  \begin{itemize}
  \item antalgique $\pm O_2$
  \item + aspirine + (clopidogrel (si fibronolyse) ou prasugrel ou ticagrelor)
  \item + (HNF ou énoxaparine ou bivalirudine [\danger{} pas si FIV]
  \item + $\beta$-bloquant (avec prudence)
  \item + IEC dans 24h
  \item + éplérénone précocement (si FEVG < 40\% ou insuf cardiaque)
\end{itemize}
\end{enumerate}
  
Efficacité : reperfusion dans 90min (50\%). Sd reperfusion

Complications : AVC, réocclusion (surtout si ttt antiagrégant interrompu)

\paragraph{Tttt des complications}
Troubles rythmes ventriculaire : amiodarone

Troubles rythmes supra-ventriculaire : AVK si mal toléré (hémodynamique)

BAV transitoire : atropine.

BAV après IDM antérieur : sonde d'entraînement électrosystoliques.

Insuf ventriculaire G : diurétique, IEC, épléronone

Choc cardiogénique : lutter contre {hypovolémie, troubles rythme}, sidération
(dobutamine). Assistance circulatoire/cardiaque/cardiocirculation,
revascularisation

Mécanique : rupture paroi libre = mortelle, septale = suture chir, mitrale =
remplacement valvulaire.

\subsection{Suivi} BASIC (cf section~\ref{subsec:basic_prev}).
\begin{itemize}
  \item antiagrégants plaquettaires : aspirine + clopidogrel (sauf si angor
    stable : aspirine)
  \item statines : si SCA/angor stable
  \item $\beta$-bloquant : si infarctus
  \item IEC si coronariens post-infact
  \item épléronone : IDM étendu FEVG < 40à%
\end{itemize}
Éventuellement DAI

\section{228 : Douleur thoracique aigüe}%
\label{sec:228_douleur_thoracique_aigue}

\subsection{CAT}
Détresse vitale ?
\begin{itemize}
  \item respi : FR < 10 ou > 30/min, tirage, sueurs, cyanose, $SpO_2$
  \item hémodynamique : arrêt circulatoire, choc, c\oe{}ur pulmonaire, pouls
    paradoxal
  \item trouble conscience
\end{itemize}

4 urgences vasculaire : PIED (péricardite, infarctus, embolie pulmonaire,
dissection)

Examens : ECG 12 + 5 dérivations, radio poumon, troponinémie

Transfert USIC

\subsection{Urgences}

\paragraph{Sd coronarien aigü}
\begin{itemize}
  \item FR, ATCD
  \item douleur spontanée de repos > 20min : constriction, pesanteur, brûlure,
    rétrosternale, irradie  cou/épaule/avant-bras/tête. \danger présentation
    \textbf{atypique} possible
  \item examen clinique, radio normale
  \item ECG : sus/sous décalage ST
  \item doser myoglobine (< 6h) ou troponine
\end{itemize}

\paragraph{Dissection aortique}
\begin{itemize}
  \item Favorisé par : HTA ancienne, sd de Marfan, maladie de Turner
  \item Douleur aigüe, prolongée, intense, déchirement, irradie dans dos, descend
    vers lombes
  \item Clinique : $\Delta$PAS > 20mmHg (bras), abolition 1 pouls, souffle
    insuffisance aortique, déficit neuro
  \item ECG : normal ou SCA
  \item Radio : élargissement médiastin
  \item \textit{Échocardio et (ETO ou scanner)} 
  \item Chir en urgence, contrôle pression artérielle
\end{itemize}

\paragraph{Embolie pulmonaire} Y penser si douleur thoracique, dyspnée, radio
normale \skull
\begin{itemize}
  \item Terrain
  \item 2 tableaux
    \begin{itemize}
      \item infarctus pulmonaire : douleur basithoracique, hémoptysie noire
      \item c\oe{}ur pulmonaire aigü\footnote{Insuffisance ventricule D suite à
          une augmentation brutale de sa post-charge} : dyspnée, défaillance ventriculaire
    \end{itemize}
  \item EC : parfois thrombose veineuse
  \item radio normale
  \item ECG : c\oe{}ur pulmonaire droit
  \item \textit{D-Ddimère \thus doppler veineux MI, angioscan ou scinti}. HBPM sans
    attendre !
\end{itemize}

\paragraph{Péricardite aigüe}
Tamponnade\footnote{Accumulation de liquide dans le péricarde} péricardite = urgence \skull
\begin{itemize}
  \item douleur thoracique, dyspnée, polypnée \thus orthopnée, toux
  \item turgescence jugulaire, reflux hépatojugulaire
  \item Choc : tachycardie, PAS < 90mmHg
  \item Pouls paradoxal
  \item ECG . microvoltage
  \item radio : cardiomégalie
  \item \textit{échocardio}  (compression VG par VD)
\end{itemize}

Péricardite non compliquée (plus bénin) :
\begin{itemize}
  \item terrain
  \item douleur thoracique augmente inspiration, decubitus. Calmée par
    antéflexion
  \item ECGA : sus-ST diffus, sous-PQ, microvoltage
  \item \textit{échocardio, troponine} 
\end{itemize}

\paragraph{Myopéricardite}
Douleur type péricardite mais \textbf{peut simuler SCA} .

Échocardio + (coronarographie normale)

\subsection{Chroniques cardiaque}
Angor stable

Douleur d'angor : d'effort du rétrécissement aortique serré, fonction des
tachycardies chroniques

Douleur d'effort de myoacardiopathie obstructives.

(HTA pulmonaire)

\subsection{Extra-cardiaques}
Urgences moyennes : 4 P = \{pneumothorax, pleurésie,
pneumonies, pancréatite\}, ulcère gastrique/duodénale, cholécystite, douleurs
radiculaires

\section{223 : Artériopathie oblitérante (aorte, MI)}%
\label{sec:223_arteriopathie_obliterante_aorte_mi_}
\subsection{AOMI\footnote{Artériopathie oblitérante des membres inférieurs}}

Épidémio : \male > \female. Pic = 60-75 (\male), 70-80 (\female). Prévalence :
1-2\%

\paragraph{Clinique}
Classif de Rutherford : 
\begin{enumerate}[label=\Roman*]
  \item asymptomatique 
  \item claudication légère/modérée/sévère
  \item douleur ischémique de repos 
  \item perte de substance faible/majeur(ulcère/gangrène)
\end{enumerate}

Claudication intermittente : douleur "crampe" au mollet après $x$m de marche.
Disparaît en 5min. Sévère si $x < 200$m. \danger{} Sévère $\neq$ symptomatique

Puis au repos : 
\begin{itemize}
  \item douleurs de décubitus : brûlure orteils, avant-pied. Amélioré par
    déclivité
  \item trouble trophiques : peau mince, fragile, perte pilosité. Puis plaies,
    ulcères, gangrène
  \item ischémie permanente : douleur > 10 j, antalgique résisntant. Critique si
    PF\footnote{Pression de perfusion} < 50mmHg (cheville) ou 30mmHg (gros
    orteil) !
\end{itemize}
Physique : 
\begin{itemize}
  \item inspection : pâle, cyanosé. Interdigitaux++)
  \item palpation : froid, douleur à palpation musc si sévère, pouls, temps recoloration cutané, anévrisme
abdo, poplité
  \item auscult : souffle
\end{itemize}
AOMI si IPS\footnote{Index de pression systolique = pression systolique
cheville/bras} < 0.90, sévére si < 0.60

\paragraph{Paraclinique}
\begin{itemize}
  \item Test de marche (6min ou tapis roulant) : -30mmHG ou -20\% évoque AOMI
  \item Transcutané de la $PO_2$ : hypoxie si < 35mmHg, critique si < 10mmHg
  \item écho-doppler artériel des MI
  \item Si revascularisation : angioscanner des MI, angiographie par RM,
    artériographie des MI
\end{itemize}

\paragraph{DD} 
\begin{itemize}
  \item Douleurs hanches : neuro, rhumato, veineuse, musc
  \item Douleurs de décubitus : neuropathie sensorielle, sd régionaux douloureux
    complexes, compression radiculaire
  \item Ulcères : veineux, microcirculatoire, neuropathie, trauma...
\end{itemize}

\paragraph{Étiologie} : atteinte athéromateuse = 95\%. Sinon : arthériopathies
inflammatoires, dysplasie fibromusculaire, coarctation de l'aorte, atteinte
post-radique ou post-trauma, toxique, gelures, compressions extrinsèques,
atteinte de l'artère poplitée
    
\paragraph{Traitement}
Local : FR, antiagrégant plaquettaire (risque CV), statine (LDL), IEC (PA) $\pm$
$\beta$-bloquants si coronaire

Local : arrêt tabac, marche. Éventuellement statine (périmètre de marche),
prostaglandine (ischémie critique non revasc.)

Revascularisation si ischémie permanente : endovasc. (stent) ou chir (pontage).
Association possible. Parfois endartériectomie ou amputation

\paragraph{Pronostic} : grave, espérance de vie -10 ans

\subsection{Anévrismes}
Dilatation du \diameter{} > 50\%. Artères cérébrales, aorte, artères poplitées, iliaques

\paragraph{Aorte abdominale}
FR : tabac, ATCD familiaux, âge. Risque de rupture > \female. Haut risque CV

90\% des cas : si maladie athéromateuses. Associés à athérosclérose (90\%).
Formes familiales, évolution aortite.

Clinique : 
\begin{itemize}
  \item asymptomatique : dépister si FR
  \item symptomatique : douleur abdo/lobmaire $\pm$ choc hémorragique. Risque de
    rupture imminente \thus scanner en urgence \skull
  \item autre : complication embolique, compression, sd inflammatoire
\end{itemize}
Paraclinique : écho abdo (dépistage), scanner abdo-pelvien ou IRM = réf

\paragraph{PEC}
Asymptomatique : surveillance si \diameter < 50cm sinon chir (pontage) ou
endoprothèse (si haut risque chir)

Symptomatique : \danger anévrisme rompu = urgence chir \danger. Ne pas attendre
résultat

Suivi : écho-doppler si prothèse viasc, scanner/écho si endoprothèse.

\paragraph{Anévrisme poplité}
Asymptomatique : masse battante. Opéré si > 20mm
Au contraire, complication = embolie (ou ischémie)

\subsection{Ischémie aigüe des MI}
\danger urgence vasculaire !

Chronologie : +2h cellules nerveuses, +6h rhabdomyolyse, +24h nécrose. Sd des
loges.

Reperfusion : sd de reperfusion ou troubles métaboliques, insuf. rénale (ou
choc)

\paragraph{Diagnostic} Clinique, ne pas retarder la chirurgie \danger

Douleur brutale, intense, broiement, impotence fonctionnelle. 

Membre livide et froid, douleur à palpation musc, pouls abolis en aval, anesthésie, paralysie

\paragraph{Étiologie} 2 tableaux (qui peuvent se mélanger) :
\begin{itemize}
  \item thrombose artérielle in situ (surtout AOMI)
  \item embolie sur artères saines (surtout cardiaque : fibrillation atriale)
\end{itemize}
Donc ascultation cardiaque, ECG, palpation abdo, bilan coagulation

Évaluer état général, fonction cardiaque, comorbidité

\paragraph{Traitement}
Médical : HNF, antalgique niveau 3, oxygène, soins locaux.

revascularisation : chirurgie (embolectomie par sonde Fogarty) voire fibrinolyse
$\pm$ angioplastie, aponévrotomie. Amputation possible

Surveiller acidose métabolique, hyperK, insuf rénale : diurèse, iono, urée,
créat.

\section{231 : Rétrécissement aortique}%
\label{sec:231_retrecissement_aortique}
Obstruction à l'éjection du VG\footnote{Ventricule gauche}, ici au niveau de la
valve aortique

\paragraph{Étiologies} 
\begin{itemize}
  \item bicuspidie < 65 ans
  \item dégénératif après (rarement post-rhumatismal)
\end{itemize}

Physiopatho: \inc pression \thus hypertrophie pariétale (compense un temps
l'élévation de pression) \thus dysfonction systolique \thus dysfonction
diastolique (altération compliance)

\paragraph{Complication}
Insuf cardiaque, fibrillation auriculaire, troubles conduction, mort subite++

\subsection{Clinique}

Pronostic vital mis en jeu si symptômes ! \skull

Dyspnée d'effort, angor d'effort, syncope (d'effort ?), (hémorragie digestive)

Examen : 
\begin{itemize}
  \item auscultation : \{souffle mésosystolique éjectionnel, rude, râpeux\},
    abolition B2 si calcifié
  \item frémissement palpatoire (foyer aortique), (choc de point dévié en bas à
    gauche) 
\end{itemize}

\subsection{Explorations}
Radio thorax : dilatation VG ?, surcharge pulmonaire

ECG : souvent hypertrophie VG et auriculaire G, troubles conduction/rythme

Cathétérisme : pas habituellement mais coronarographie pour pré-op si \male > 40 ans, FR, angor d'effort ou insuf
cardiaque

Scanner cardiaque : pré-op si \gls{TAVI}

\paragraph{Échocardiographie-doppler transthoracique} : examen clé. Critères
\begin{itemize}
  \item V max > 4m/s
  \item gradient moyen > 40 mmHg
  \item surface aortique < 1 $\text{cm}^2$
\end{itemize}
Évalue conséquences sur VG, débit cardiaque, pressions droites

Examiner taille aorte, valve mitrale, tricuspide

\subsection{Traitement}
\begin{itemize}
  \item valve chirurgical : mécanique si jeune mais AVK à vie, sinon prothèse bio (> 65
    ans)
  \item valvulopathie percutanée abandonnée
  \item implantation percutanée d'une valve aortique (TAVI)
\end{itemize}
Si symptomatique, opérer. Sinon (et FEVG normale), test d'effort.

NB : si FE < 35\%, échocardio de stress sous dobutamine pour risque opératoire


\section{231 : Insuffisance mitrale}%
\label{sec:231_insuffisance_mitrale}
Reflux de sang depuis le VG vers l'OG pendant la systole.

Classif de Carpentier
\begin{enumerate}[label=\Roman*]
  \item Valves restent dans le plan de l'anneau (perforations)
  \item Au moins une valve au-dessus du plan de l'anneau (prolapsus)
  \item Au moins une valve sous le plan de l'anneau
\end{enumerate}

\paragraph{Étiologies}
\begin{itemize}
  \item Rhumatismale (rare) : type III
  \item Dystrophique (fréquente++) : type II. Soit "dégénerescences myxoïdes"
    (trop de tissu, trop de mobilité), soit dégénerescences fibroélastiques
    (rupture de cordage)
  \item Sur endocardite : type I (perforations) ou II (rupture de cordage)
  \item Ischémique : soit aigüe (rupture de pilier, urgence \skull !), soit
    chronique (type III)
  \item Fonctionnelle : souvent une évolution de cardiopathie avec dilatation VG
    et atteinte systolique
\end{itemize}

Causes des insuf. mitrales aigües : rupture de cordage ou de pilier, dysfonction
de pilier ischémique, perforation par endocardite.

Tableau hémodynamique \thus urgence vitale \danger 

Conséquences hémodynamiques : altération contractilité VG (aval), HTAP pouvant
être importante si aigü (amont)


\subsection{Diagnostic}
\danger{} peut être asymptomatique

Dyspnée : d'effort (lente et progressif), de repos, orthopnée, paroxystique
nocturne, OAP

Examen : 
\begin{itemize}
  \item palpation : frémissement systolique apex, (déviation et abaissement choc
    de pointe)
  \item auscultation : souffle systolique de régurgitation, en "jet de vapeur",
    souffle holosystolique de B1 à post-B2, irradie vers l'aisselle ou la base
  \item (autres : galop B3, roulement mésodiastolique, éclat B2, souffle
    d'insuf. tricuspide)
  \item poumon : râles de stase
\end{itemize}

Para clinique
\begin{itemize}
  \item ECG longtemps normal. hypertrophie OG, VG, VD, fibrillation atriale, 
  \item Radio thorax : normale si $\le$ modérée. cardiomégalie, dilatation OG,
    HTAP
  \item \textit{ETT} référence (et ETO). Sévérité côtée en 4 grades\footnote{Selon
      la Surface de l'orifice régurgitant, Volume régurgité} (Vérifier la
    tricuspide)
  \item Cathétérisme : coronarographie seulement en pré-op si \male > 40 ans ou
    \female{} monopausée avec FR
  \item Épreuve d'effort, échocardio d'effort
\end{itemize}

Évolution : si constitué, bien toléré pendant longtemps. Si brutal : évolue vers
oedème pulmonaire. Complication : endocardite infectieuse, fibrillation/flutter
atrial, insuf cardiaque, complications thromboembolique

\subsection{Prolapsus valvulaire mitral}
Primitif ou associé. \female. Formes familiales.

Signes fonctionnels absents ou ceux IM.

Clinique : clic méso-/télosystoliques, souffle d'IM.

Examen : échocardio.

Évolution bénigne ou complication

\subsection{Traitement}
\begin{itemize}
  \item aigüe mal tolérée : chir urgence 
  \item chronique III ou IV symptomatique : chir
  \item chronique III ou IV asymptomatique : chir si retentissement VG ou
    surveillance échodoppler 6 mois (chir si symptômes, retentissement, troubles
    rythmes supraventriculaire)
\end{itemize}

Chirurgie : idéalement plastie reconstructicie, sinon remplacement valvulaire
(mécanique si jeune mais anticoag à vie, bioprothèse si > 65 ans)

Médical :
\begin{itemize}
  \item IM aigüe : ttt OAP ou choc, chir en urgence
  \item poussée insuf. cardiaque : diurétiques de l'anse, vasodilatateurs,
    digitalique (fibrillation atriale), anticoagulant oraux (fibrillation
    atriale)
\end{itemize}

\section{231 : Insuffisance aortique}%
\label{sec:231_insuffisance_aortique}
Régurgitation de sang vers VG en diastole.

\paragraph{Physiopatho} 
\begin{itemize}
  \item Chronique
surcharge volume et pression. Aorte : \inc
PA systolique, \dec PA diastoliques. 
Hypertrophie compense (parfois pendant des années !!) puis fibrose
\item Aigüe : surtout \gls{EI}, surcharge brutale,
  \inc pression puis oedème pulmonaire
\end{itemize}

\subsection{Étiologies}
Chronique :
\begin{itemize}
  \item dystrophique(freq++) : annulo-ectasiante (valves normale mais anneau
    dilaté), sd des valves flasques
  \item EI qui perfore les valvules
  \item malformative (bicuspidie aortique)
  \item rhumatismale (rare)
  \item inflammatoire, infectieuses, médicamenteuse
\end{itemize}
Aigü : EI, dissection aortique, rupture d'anévrisme d'un sinus de Valsalva,
traumatique

Prothèse : désinsertion partielle, dysfonction

\subsection{Clinique}
Fonctionnel : dyspnée d'effort, (angor d'effort ,), insuf cardiaque (rare,
tardive)

Physique : 
\begin{itemize}
  \item ascult : souffle diastolique++\footnote{Holosystolique si IA importante}, "doux, lointain, humé, aspiratif",
    souffle systolique éjectionnel d'accompagnement, roulement de Flint
    apexien/galop
  \item palpation : choc de pointe étalé, en bas à gauche
  \item hyperpulsatilité artérielle périphérique (pouls++), \dec PA
    diastolique++
\end{itemize}

ECG : normal ou \gls{HVG} diastolique, (ou HVG systolique)

Radio : \inc index cardiothoracique si volumineuse chronique

\textit{Échocardio-doppler (ETT)} = confirmer, quantifie dilatation VG 

Coronarographie : pré-op, \male > 40 ans ou \female monopausée, FR

IRM/scanner : dimension aorte, surveillance

\paragraph{DD} 
\begin{itemize}
  \item souffle diastolique : insuf pulmonaire
  \item double souffle (rupture sinus Valsalva), souffle continu, frottement
    péricardique
\end{itemize}
\paragraph{Évolution}
\begin{itemize}
  \item Chronique : si volumineuses, sévère dès les symptômes \thus surveillance
\danger, opération même si asympto. \danger dystrophique, bicuspidies
  \item Aigu : OAP, mort subite \thus chir précoce
\end{itemize}

Complications : EI ++, insuf cardiaque (tardive), rupture aortique, (mort subite)
   
\paragraph{Surveillance} 
Chronique : 1-2/an si fuite importante, sinon tous 2-3ans

Aigü : chir rapidement

\subsection{Traitement}
Médical : 
\begin{itemize}
  \item si volumineuse et IVG : IEC, diurétique et chir rapidement
  \item dilatation de l'aorte : beta-bloquant, losartan
\end{itemize}
Hygiène dentaire, examen tous 6 mois pour prévenir EI

Chirurgie : 
\begin{itemize}
  \item remplacement valvulaire si IA isolée
  \item valve + aorte si dystrophique ou (bicuspidie et dilatation aortique)
\end{itemize}

Quand faire la chir ?
\begin{itemize}
  \item chronique volumineuse
    \begin{itemize}
      \item  symptomatique : urgent \danger
      \item asymptomatique : si FEVG < 50\%, dilatation aorte $\ge 55$mm, diamètre
        VG télédiastolique > 70mm, télésystolique > 50mm
    \end{itemize}
  \item dystrophique et dilatation aorte asc : dès $\ge 55$mm
  \item aigüe volumineuse : urgence
\end{itemize}

\section{150 : Surveillance des porteurs de valves, prothèses vasculaires}%
\label{sec:150_surveillance_des_porteurs_de_valves_protheses_vasculaires}

\begin{itemize}
  \item Prothèses mécaniques : double ailette, à vie, anticoagulant à vie
    (risque thrombose)
  \item Biologiques : pas d'anticoagulant, chez > 65 ans (aortique) ou > 70
    (mitrale) [faible durée de vie\footnote{40\% durent 15 ans}].
\end{itemize}
Risque majeur d'EI $\forall$ prothèse !

\subsection{Complications}
\begin{itemize}
  \item Thromboemboliques (freq++) : surtout mécanique, surtout prothèses mitrale,
    anciennes, fibrillation atriale
    \begin{itemize}
      \item Embolie systémiques : souvent cérébrales
      \item Thromboses de prothèse mécanique : accidents brutaux (OAP, syncope,
        choc, mort subite). Diagnostic difficile : apparition d'un
        souffle/roulement. Diagnostic : \textit{ETT, ETO} \\
        Chir d'urgence possible
        \danger DD avec EI parfois difficile
    \end{itemize}
  \item Désinsertions de prothèses (5\%) : spontané, EI. À évoquer si apparition d'un
    souffle, anémie hémolytique, insuf. cardiaque. Confirmé par ETT, ETO(++)

  \item Infectieuses
    \begin{itemize}
      \item médiastinie post-op (1\%)
      \item Endocardite infectieuses : \textbf{redoutable} \skull\\
        Précoce (50\%) ou tardive. Diagnostic : ETT, ETO++\\
        Prévention/traitement de tout foyer infectieux (ORL, dentaire)\\
        Hémocultures systémiques devant fièvre inexpliquée
    \end{itemize}
  \item Traitement anticoagulant : risque hémorragique 1.2\% patients-années
  \item Dégénérescence bioprothèses
\end{itemize}

\subsection{Surveillance}
Post-op : AVK (à vie si mécanique, 3 mois si bio). ETT à +3mois (référence !)

Puis : 1/mois puis tous les 3 mois. Cardiologue à +3 mois puis 1-2/an.

Clinique : 
\begin{itemize}
  \item surveiller symptômes, dyspnée, insuf cardiaque
  \item ascult : attention à \dec intensité bruits (ou variables), \inc
    intensité d'un souffle, bruit diastolique surajouté
\end{itemize}
Radio, ECG mais surtout ETT, ETO

\textit{Biologie} ++ : équilibre AVK parfait, à vie \thus INR tous les mois $\in
[2.5, 4]$.

FR : valve non aortique, ATCD, fibrillation atriale, \diameter OG > 50mm,
contraste spontané dense OG, sténose mitrale, FE < 35\%, hypercoagulabilité

Ne pas interrompre AVK sauf pronostic vital !. Si chir extracardiaque : HNF
pendant l'arrêt AVK

\section{149 : Endocardite infectieuse}%
\label{sec:149_endocardite_infectieuse}
Infections des valves cardiaque ou de l'endocarde pariétal. Dominées par les
staphylocoques

\subsection{Physiopatho}
Bactéries adhèrent sur une lésion préexistante \thus
\begin{itemize}
  \item insuffisance valvulaire, souffle, risque de défaillance cardiaque
  \item végétations \thus embolies septiques, lésions de vascularite, anévrisme
    "mycotique"
\end{itemize}

Cardiopathies à haut risque : prothèses valvulaires, cardiopathies congénitales
cyanogènes, ATCD EI

50\% des EI sur c\oe{}ur présumé sain !

Hémocultures positives (90\%)
\begin{itemize}
  \item streptocoques oraux, streptocoques du groupe D
  \item staphylocoques : blanc, coagulase négative
\end{itemize}
Hémocultures négatives :
\begin{itemize}
   \item ATB
   \item croissante lente : HACEK\footnote{Haemophilus, Actinobacillus,
       Cardiobacterium, Capnocytophaga, Eikenella, Kingella}, Brucella, champignons
   \item intra-cellulaire : \bact{burnetii}, Chlamydia, Bartonella,
     \bact{whipplei}
\end{itemize}

\subsection{Clinique}
\danger Manif trompeuses. Y penser si souffle cardiaque et fébrile, AVC,
purpura, lombalgies fébriles

\begin{itemize}
  \item Sd infectieux : fièvre, AEG, splénomégalie
  \item Apparition/modif souffle, insuf cardiaque
  \item cutané (nodosité d'Osler !), respi, ophtalmo, rhumato (freq), neuro,
    rénale
\end{itemize}

Diagnostic : hémoculture, échocardio

Autres : NFS, \{CRP, électrophorèse\}, complexes immuns circulants, \{urée,
créat\}, BNP

Classif de Duke : 2 majeurs ou (1 majeur et 3 mineurs) ou (5 mineurs)
\begin{itemize}
  \item majeurs
    \begin{itemize}
      \item Hémocultures : (micro-org typique d'une EI sur $\ge 2$ HC) ou (HC
        positives sur > 12h) ou (une HC positive à \bact{burnetii})
      \item (Échocardio avec végétation, abcès, désinsertion prothétique) ou (nouveau souffle de régurgitation valvulaire)
    \end{itemize}
  \item mineurs
    \begin{itemize}
      \item cardiopathie à risque/toxicomanie
      \item $\ge 38^{\circ}$
      \item complication vasc\footnote{Cérébrale, EP septique, anévrisme mycotique...}
      \item immunologique\footnote{Glomérulonéphrite, faux panaris d'Osler,
          taches de Roth...}
      \item hémoc/séro positive
    \end{itemize}
\end{itemize}

\paragraph{Évolution}
Complications : insuf cardiaque (1ere cause DC), neuro (2eme cause DC), embolies (septiques,
cérébrales, splénique, rénales, coronaires), infarctus splénique, arythmies et
troubles de conduction

Penser à scanner cérébral et abdo-pelvien !

Pronostic : 
\begin{itemize}
  \item sur aortique : chir
  \item staph ou prothèse : mortalité++
  \item pneumocoque, bacilles Gram négatif : destruction valvulaire graves
  \item levure : grosse végétations
\end{itemize}

\subsection{Traitement}
Bithérapie IV

Fonction rénale pour aminosides et vancomycine !

\begin{itemize}
  \item Strepto oraux/groupe D : amoxicilline et gentamicine (2 semaine bi, 4
    semaines mono) [vancomycine + gentamicine si allergie]
  \item entérocoques : idem
  \item staph : si sensible : cloxacilline (+gentamicine + rifampicine si sur
    prothèse). Sinon vancomycine (+gentamicine + rifampicine)
  \item hémoc négative :  amox + acide clavilanique + gentamicine en attendant
\end{itemize}

Chirurgie : valve native si possible. Intervention si insuf cardiaque ou sd
infectieux non contrôlé

\paragraph{Prévention}
Hémoc avant antibio \danger

ATBprophylaxie : amoxicilline (clindamycine si allergie) avant geste (région
apical/gingivale, perforation muqueuse orale ou (extraction dentaire et haut
risque))

\section{236 : Souffle cardiaque chez l'enfant}%
\label{sec:236_souffle_cardiaque_chez_l_enfant}
Très fréquent.

Malformation congénitale (1\%), souffle fonctionnel, cardiomyopathie/myocardite
aigüe (rarement), acquises (exceptionnelles)

\paragraph{Auscultation chez l'enfant} Rythme rapide, irrégulier.

B2 dédoublé : anormal si large et fixe.

Éclat B2 : HTA pulmonaire, malposition des gros vaisseaux

B3 physiologique (apex)

Clic possible

\subsection{Clinique}
Fonctionnel : souvent absent, dyspnée d'effort. \danger douleur thoraciques =
rarement cardiaques !

Souffle :
\begin{itemize}
  \item varie en temps et position : innocent
  \item bruyant, irradiant largement : organique
  \item diastolique : organique
  \item frémissant : organique
  \item holosystolique, de régurgitation : organique
  \item Localisation probables : cou et sus-sternal $\approx$ aortique ;
    dos $\approx$ pulmonaire ; irradiant $\approx$ 
    communication intra-V
\end{itemize}
Associés : 
\begin{itemize}
  \item regarder $SaO_2$
  \item troubles alimentaires, dyspnée, sueur, retard staturopondéral : large
    shunt
  \item HTA, pas de pouls fémoraux : coarctation aortique
\end{itemize}

\paragraph{Complémentaire}
Radio thorax : cardiomégalie (\danger "fausses")
\begin{itemize}
  \item saillie arc moyen G : shunt gauche-droite
  \item arc moyen G conctave : hypoplasie voie pulmonaire
\end{itemize}

ECG : fréquence diminue avec l'âge. Ondes T < 0 de $V_1$ à $V_4$

\textit{Échocardio} = examen clé

Autres : effort, holter ECG, IRM cardiaque, scanner multibarettes, cathétérisme
cardiaque (rare)

\subsection{Cardiopathies}
\paragraph{[Naissance, +2 mois]}
\begin{itemize}
  \item Souffle isolé : examen clinique, ECG, radio pulmonaire, échocardio
  \item Insuf cardiaque : coarctation préductale \thus chir urgente
  \item Cyanose : transposition des gros vaissaux \thus chir avant N\footnote{Naissance}+15 jours
\end{itemize}

\paragraph{[N+2 mois, marche]}
\begin{itemize}
  \item Insuf cardiaque : shunts gauche-droite surtout (\thus opérer avant 1 an
    si large !!), communication intra-V
    large, persistance canal artériel, canal atrioventriculaire
  \item cyanose : tétralogie de Fallot\footnote{Communication intra-V,
      hypertrophie ventriculaire D, sténose pulmonaire, dextroposition de
    l'aorte}
\end{itemize}
\paragraph{2 à 16 ans}
\begin{itemize}
  \item Malformatives : rares, bien tolérées
  \item Souffles "innocents" (1/3) : asymptomatique, systolique, éjectionnels,
    faible intensité, (intensité varie avec position), doux. Ne rien faire
\end{itemize}

\section{337 : Malaise, perte de connaissance}%
\label{sec:337_malaise_perte_de_connaissance}
\begin{itemize}
  \item Syncope : trouble de conscience, hypotonie, début brutal/rapide, souvent
    bref. Comportement, orientation normaux après retour conscience
  \item Lipoythmie : sensation de perte de connaissance
  \item Stokes-Adams : syncope à l'emporte-pièce
  \item Autres : coma, confusion mentale, crise comitiale, AVC, cataplexie,
    narcolepsie
\end{itemize}

\paragraph{Physiopatho}
Hypoperfusion de la substance réticulée du tronc cérébral (< 60 mmHg ou arrêt >
6 secondes) \thus perte conscience, tonus, myclonies si > 30s

\subsection{Étiologies}
Cause cardiaques mécaniques
\begin{itemize}
  \item rétrécissement aortique : à l'effort
  \item cardiomyopathies hypertrophiques obstructives : génétique, à l'effort ou
    post-effort immédiat. Auscut : souffle systolique sternum gauche, ECG :
    hypertrophie VG
  \item EP massive
  \item tamponnade brutale
\end{itemize}

Cause cardiaques électriques :
\begin{itemize}
  \item tachycardie
  \item BAV
  \item dysfonction sinusale
  \item défaillance stimulateur cardiaque
\end{itemize}

Hypotension :
\begin{itemize}
  \item avec tachycardie sinusale : iatrogènes, orthostatique
  \item avec bradycardie sinusale : hypotension réflexe, vasovagale
\end{itemize}

\paragraph{DD} : 
\begin{itemize}
  \item métaboliques (hypoglycémie, hypoxie-hypercapnie,
encéphalopathie hépatique)
\item toxiques (toxico, médical, alcool++, CO++)
\item psy (trouble de conversion, attaque de panique, simulation)
\item neuro (vasc) : infarctus cérébraux, AIT, insuf. vertébrobasilaire,
  drop-attacks
\end{itemize}

\subsection{PEC}
\begin{algorithm}
  \caption{PEC des malaises}
   Perte de connaissance brève, pas de crise comitale ? Si non : \textit{épilepsie,
    AVC/AIT, coma, intoxication, céphalée, SAS} \faHandStopO\;
   Syncope. Signe de gravité ? Si oui : urgence = SCA, EP... \faHandStopO\;
   Interrogatoire, cliinque, ECG ? Si cause évidente (méca, électrique,
    hypotension) \faHandStopO\;
   Cardiopathie sous-jacente ? Si oui : holter, électrophysio\;
   Sinon probablement neurocardiogénique
\end{algorithm}

Interrogatoire :
\begin{itemize}
  \item âge, ATCD : mort subite (famille), cardiopathie si âgé, médicaments
  \item prodrome, postures, activité
  \item mouvements anormaux, durée, réveil, courbature
\end{itemize}
Examen neuro (déficit), CV (pression artérielle)

ECG : diagnostic si bradycardie < 40/min, tachycardie (supra)ventriculaire, BAV
complet ou 2eme degré, défaillance stimulateur cardiaque

\paragraph{Paraclinique}
Éliminer cardiopathie sous-jacente : \textit{échochardio} , test d'effort, BNP,
troponine

Autres : Holter-ECG (dysfonction sinusale, trouble conduction AV). Sinon
étude électrophysiologique endocavitaire\footnote{Déclenche tachycardie
ventriculaire}, test d'inclinaison\footnote{Déclenche syncope vasovagale},
hyperréflexie sinocartidienne, ECG implantable

\paragraph{Gravité}
\begin{itemize}
  \item Trouble du rythme ventriculaire/de conduction supposé
  \item syncope inexpliquée chez cardiaque
  \item suspicion maladie génétique chez jeune
  \item syncope et trauma grave
  \item syncope d'effort
  \item syncope de décubitus
\end{itemize}

\paragraph{Formes typiques}
\begin{itemize}
  \item syncope neurocardiogénique : vasovagale (debout, vue du sang,
    \textbf{jeune} ), réflexe
    (miction), hyperréflexie sinocarotidienne (rasage, \textbf{âgé} )
  \item hypotension artériel : âgé, iatrogène, debout prolongé
  \item troubles du rythme/conduction : tachycardie
    ventriculaire++. Diagnostic = étude électrophysiologique endocavitaire
\end{itemize}

\section{230 : Fibrillation atriale}%
\label{sec:230_fibrillation_atriale}

Tachycardie irrégulière due à une activité anarchique des oreillettes
(400-600/min) > 30 secondes.

Noeud AV filtre à 130-180/min \thus tachy irrégulière \thus risque
d'insuffisance cardiaque et thromboembolique (stase). Évolue : fibrose
oreillettes, dilatation atriale

Fréquente chez âgé

\paragraph{Classification}
\begin{itemize}
  \item Premier épisode
  \item Paroxystique : retour en sinusal < 7 j
  \item Persistante :  retour en sinusal > 7 j ou après cardioversion
  \item Permanente : échec cardioversion/non tentée
\end{itemize}

\subsection{Diagnostic}
Signes usuels : palpitations, dyspnée d'effort, angor fonctionnel, asthénie
inexpliquée...

Auscul : bruits irréguliers, rythme $\pm$ rapide

\textit{ECG} : indispensable \danger
\begin{itemize}
  \item petites/grosse mailles
  \item QRS lents réguliers
  \item dysfonction sinusale à l'arrêt de \gls{FA} (brady-tachy)
\end{itemize}

Autres : \{iono, créat, TSHus, NFS\}, radio thorax, echocardio

\paragraph{Étiologies}
\begin{enumerate}
  \item HTA (âgé)
  \item Valvulopathie (mitrale)
  \item Autres : respi (SAS !), cardiomyopathies, SCA, hyperthyroïdie (y penser
    !), péricardites, chir cardiaque récente, cardiopathies congénitales,
    phéochromocytome
\end{enumerate}

\paragraph{Tableaux cliniques}\mbox{}\\
\textit{FA isolé, c\oe{}ur normal} : quinqua, palpitation nocturnes $\pm$ (angor
fonctionnel ou dypsnée d'effort). Échocardio normale. Exclure SAS et HTA !\\
\hspace*{10pt}\thus seulement anti-arythmique (flécaïdine)

\textit{FA avec insuf cardiaque} : souvent séquelle infarctus sévère ou
cardiomyopathie dilatée à coronaires saines. OAP/ décompensation cardiaque
globale.\\
\hspace*{10pt}\thus antiocoagulants oraux \arrow{} cardioversion (parfois urgence) \arrow{}
anticoag. au long cours, amiodarone\footnote{Maintien rythme sinusal}

\textit{FA valvulaire post-rhumatismale}  : persistante/permanente sur maladie
mitrale. \\
\hspace*{10pt}\thus à discuter, AVK au long cours

\textit{Embolie artérielle systémique} : souvent cérébrale. FA méconnue chez
\female{} âgée avec FR embolique (HTA, diabète). Écarter SCA (tropo, ECG)\\
\hspace*{10pt}\thus aigü : (thrombolyse), aspirine \arrow{} AVK, héparine

\textit{Maladie de l'oreillette}  : alternance FA paroxystique rapide-brady\\
\hspace*{10pt}\thus stimulateur cardiaque définitif

\subsection{Traitement}
\paragraph{Risque thromboembolique}
Cardioversion : à risque par défaut ! Donc héparine ou anticoag. oral. 
Si risque très élevée, vérifier l'absence de thrombus atrial G

Chronique :
\begin{itemize}
  \item FA valvulaire : risque très élevé
  \item FA isolé sur c\oe{}ur sain : risque faible
  \item sinon score CHADS2 \danger{} pas si FA valvulaire !!! : \texttt{Congestion +1,
    Hypertension artérielle +1, Âge > 75 ans +1, Diabète +1, Stroke +2}\\
    anticoag si CHADS2 > 1
\end{itemize}

\paragraph{FA persistante ou premier accès < 7 j}
\begin{itemize}
  \item Prévention thromboembolique par HNF IV (AVK/nouveau anticoag
    directement si bien toléré et pas à haut risque)
  \item Cardioversion : antiocoag orale -3 semaine et + 4 semaines. Choc
    électrique sous anesthésie générale ou médicament (amiodarone). Rarement en
    urgence.
\end{itemize}

\paragraph{Entretien}
\begin{itemize}
  \item Anticoagulant selon terrain : AVK (INR !), inhib trombine (dabigatran), inhib
facteur X (rivaroxaban, apixaban, edoxaban)\\
Si FA valvulaire : seuls AVK
\item respect FA et seul contrôle FC (oui) ou contrôle FA (paroxystique,
  persistante)
\end{itemize}

\paragraph{Éducation du patient}
HTA, risque d'embole cérébrale, effets secondaire amiodarone (thyroïde,
photosensibilisation, dépôts cornéens)

\section{234 : Troubles de la conduction intracardiaque}%
\label{sec:234_troubles_de_la_conduction_intracardiaque}
Fréquences d'échappement :
\begin{itemize}
  \item noeud AV : 40-50/min
  \item faisceau de His : 35-45/min
  \item branches et ventricules : < 30/min
\end{itemize}
Dysfonction sinale et BAV peuvent être symptomatiques. Les BAV isolé non.

\subsection{ECG}%
\paragraph{Dysfonction sinusale}
Arrêt par le noeud sinusal ou non-transmission à l'atrium.

\begin{itemize}
  \item Tracé plat sans P bloqué (!)
  \item BSA II si pause après P = plusieurs cycles normaux
  \item Si arrêt sinusal ou BSA complet : asystolie ou bradycardie
  \item Bradycardie sinusale inappropriée (éveil)
\end{itemize}

\paragraph{Blocs atrioventriculaires}
Dans le faisceau de His ou infra : rythme très lent donc grave \skull
\begin{itemize}
  \item BAV I : PR constant mais > 0.2s\footnote{1 mm = 0.04s}
  \item BAV II Wenckebach : allongement PR progressif puis bloqué (souvent QRS
    < 0.12s)
  \item BAV II Möbitz : PR normal, multiple P bloqué\footnote{Wenckebach 2:1 ou
    Möbitz 2:1 ne peuvent être différenciés sur l'ECG}
  \item BAV III : aucun P ne passe, ventricule à leur rythme, plus lent.
    \danger{} DC possible (torsade de pointes)
  \item BAV III + FA : bradycardie (!), rythme
    régulier (!)
\end{itemize}

\paragraph{Blocs de branches}
\danger{} BdB gauche gêne le diagnostic d'infarctus !!
\begin{itemize}
  \item Droit : QRS > 0.12s et RsR' en V1 
  \item Gauche : QRS > 0.12s et R exclusif en V6 
  \item Hémi-bloc\footnote{Hémibloc : forcément de la branche gauche qui se
    divise en faisceau antérieur et postérieur} antérieur : déviation axiale
  gauche, QRS < 0.12s, $Q_1S_3$ et $S_3 > S_2$
  \item Hémi-bloc postérieur : déviation axiale droite, QRS < 0.12s, $S_1Q_3$

\end{itemize}

\subsection{Clinique}
\paragraph{Dysfonction sinusale}
Asymptomatique, lipoythmie, syncopes...

Fréquent si âgé

Étiologies : \{médic (bradycardisant), vagal\}, cardiaque, maladies systémiques, neuromusculaire,
post-chir, HTIC, hypothermie, (septicémies), ictères rétentionnels sévères,
\{hypoxie,hypercapnie, acidose sévère\}, hypothyroïdie.

Diagnostic : \textit{ECG} ! 
\begin{itemize}
  \item bradycardie en éveil, pas d'accélération à l'effort
  \item pauses P sans ondes > 3 s
  \item BSA II
  \item bradycardie avec rythme d'échappement atrial/jonctionnel
  \item sd bradycardie-tachycardique
\end{itemize}

Cliniques usuelles :
\begin{itemize}
  \item dégénérative liée à l'âge : \female, multiple médicaments. Souvent + FA
    $\pm$ troubles conductif sur noeud AV. Traiter !
  \item hypervagotonie : sportif. ECG : brady < 50/min en éveil. Test
    atropine/d'effort normalise. Ne pas traiter.
\end{itemize}

\paragraph{BAV}
Cf dysfonction sinusale. Peut avoir fibrillation ventriculaire suite à torsade
de pointe. Fréquent si âgé

Étiologies :
\begin{itemize}
\item hyperkaliémie+++ 
\item fibrose, rétrécissement aortique dégénératif, causes ischémiques
du SCA (mauvais pronostic si (infra)-hissien !), infectieux, \{médic, vagal\},
systémiques, neuromusculaire, post-chir, postcathétérisme, postradiothérapie,
néoplasique, congénital
\end{itemize}

Diagnostic : préciser degré, paroxistique/permanent, siège++ 
\begin{itemize}
  \item nodaux : souvent BAV I, BAV II Wenckebach, BAV III à QRS fins \thus
   Holter
  \item (infra)hissiens : sur des BdB ou BAV II Möbitz. \skull{} si complet DC
    possible !\\
    \thus étude endocavitaire
\end{itemize}

Cliniques usuelles :
\begin{itemize}
  \item BAV complet sur infarctus antérieur : régressif sous 15 j(sinon stimulateur ?),
    sensible à l'atropine
  \item BAV dégénératif (âgé)
  \item BAV congénital (risque = insuf cardiaque, DC)
\end{itemize}

\paragraph{BdB}
Toujours asymptomatique si isolé. Grave si lipoythimie/syncope \danger \thus
étude endocavitaire

Étiologies :
\begin{itemize}
  \item Droit : peut être bénin. Surtout dans patho pulmonaires
  \item gauche : jamais bénin ! (dégénératif ou cardiopathie). SCA de cause
    ischémique possible \skull
\end{itemize}

Diagnostic : incomplèt si QRS < 120ms, complet sinon. Droit/gauche/ bi- ou
trifasculaire. Chercher cardiopathie sous-jacente

\subsection{PEC}
\begin{itemize}
  \item 
Dysfonction sinusale : Confirmer l'ECG par Holter (à répéter éventuellement).
Si vagal possible, test d'inclinaison. Si âgé, on peut chercher une
hyperréflectivité sinocarotidienne.
\item BAV : médicament, SCA (territoire inférieur), myocardite ? \\
  Bloc permanent ? ECG suffit. Sinon enregistrement Holter \\
  Si suspicion infra-hissien, étude endocavitaire possible.\\
  Échocardio et troponine dans tous les cas
\item BdB : HTA ou cardiopathie ?\\
  Droit chez jeune asymptomatique $\approx$ variante normale\\
  si syncope sur cardiopathie : cherche tachycardie ventriculaire
\end{itemize}

\subsection{Traitement}
Bradycardie grave = urgence ! (rea) \skull

Brady avec BAV III plus grave que brady par dysfonction sinusale

Médicaments tachycardisants (atropine, catécholamine), stimulation cardiaque
temporaire (percutanée, transthoracique)

Stimulateur pour 
\begin{itemize}
  \item dysfonction sinusale symptomatique seulement
  \item BAV III si non curable
  \item BAV II si infra-hissiens ou symptomatiques
  \item BdB avec symptômes et BAV paroxistique (sinon non !)
\end{itemize}
Toujours traiter cause

\section{229 : ECG}%
\label{sec:229_ecg}
Normales : FC $\in [60, 100]$ battements/min, P < 120ms

\subsection{Hypertrophies}

Atriales
\begin{itemize}
  \item droite : P > 2.5mm en D2 ou > 2 mm en $V_1$ ou $V_2$
  \item gauche : P > 0.12s
\end{itemize}
Ventriculaires
\begin{itemize}
  \item gauche : Sokolov : S$V_1$ + r$V_5$ > 35mm. \danger{} QS ou sus-ST peut
    mimer un infarctus !
  \item droite : +110$^{\circ}$
\end{itemize}

\subsection{Troubles de conduction}
BdB :
\begin{itemize}
  \item droit : QRS > 0.12s, RsR' en $V_1$ 
  \item gauche : QRS > 0.12s et rS ou QS en $V_1$
\end{itemize}
Hémibloc : 
\begin{itemize}
  \item antérieur: -30$^{\circ}$, $S_3$ > $S_2$
  \item postérieur : +90 $^{\circ}$, $S_1 Q_3$
\end{itemize}
Bifasciculaire : BdB droit + (un des hémibloc)\\
Trifasciculaire \skull : (alterne BdB droit et gauche) ou (BdB droit et
alternance hémiblocs)

BAV
\begin{itemize}
  \item I : PR constant > 200ms
  \item II : PR croissant jusque P bloqué ou un seul P sur plusieurs
  \item III : aucun P, QRS réguliers lents
\end{itemize}

Dysfonction sinusale : asystole, bloc sino-atrial II

\subsection{Troubles du rythme supraventriculaire}
Man\oe{}uvres vagales : Valsalva, (jeune : compression carotidienne unilat) sinon
vagomimétique

Fibrillation atriale : 
\begin{itemize}
  \item 100-200/min, QRS irréguliers, atriale = mailles amples ou fines.
  \item \danger{} : BAV III, brady-tachy ou BdB possibles !
\end{itemize}
Flutters atriaux : (souvent + FA)
\begin{itemize}
  \item 300/min avec "dents de scie" en $D_1$, $D_3$ aVF
  \item ventriculaire : rapide (pas toujours), régulières (pas toujours).
    Ralentie par man\oe{}uvre vagale !
\end{itemize}
Tachycardie atriale : (moins fréq)
\begin{itemize}
  \item 120-200/min
  \item tachy régulières à QRS fin, souvent coupés de retours en rythme sinusal
\end{itemize}
Tachycardies jonctionnelles (fréq++)
\begin{itemize}
  \item 130-260/min
  \item pas d'activité atriale, retour en sinusal à man\oe{}uvre vagale
\end{itemize}
Extrasystole (freq, physio). Si un battement sur 2, bigéminisme

\subsection{Troubles du rythme ventriculaire}
\fbox{Toute tachycardie à QRS larges est une tachycardie ventriculaire \gls{JPDC} \skull}

Tachycardies ventriculaires
\begin{itemize}
  \item > 100/min
  \item QRS > 0.12s pendant $\ge 3$ battements
\end{itemize}
Fibrillation ventriculaire : \textbf{urgence absolue}  \skull

Torsade de pointe : si allongement QT, bradycardie

Extrasystoles ventriculaires : banales, sur c\oe{}ur sain, regarder étiologie

\subsection{Autres}
\begin{itemize}
  \item Hypokaliémie : T plates/négative, sous-ST, QRS normale, allongement QT
  \item Hyperkaliémie : T ample pointe, allongement PR, élargissement QRS
  \item Péricardites : 4 phases = 1. (microvoltage, sus-ST, sous-PQ), 2. (T plate), 3. (T
    négative), 4. (retour à la normale)
  \item Sd Wolf-Parkinson-White : PR < 0.12s, "empâtement" QRS
\end{itemize}

Maladie coronaires : sus-ST
\begin{itemize}
  \item chercher miroir, +2mm en précordial, +1 mm en frontal
  \item sur au moins 2 dérivations
  \item BdB gauche complet suffit !
\end{itemize}
Ondes Q de nécrose : +6h, > 1/3 du QRS

\section{235 : Palpitations}%
\label{sec:235_palpitations}
Sensation que le c\oe{}ur bat trop fort/vite/irrégulièrement

Interrogatoire : 
\begin{itemize}
  \item fréquence, effort, durée
  \item \danger{} douleur thoracique, perte de connaissance, dyspnée
\end{itemize}
Gravité ?
\begin{itemize}
  \item ATCD personnels : post-infarctus, HTA, troubles du rythme, stimulateur,
    médic
  \item ATCD familiaux : mort subite < 35 ans
  \item clinique : pouls > 150 /min, hypotension artérielle, angor, insuf
    cardiaque, neuro
  \item ECG : tachy à QRS large = urgence absolue \skull\\
    autres : anomalie repolarisation (SCA ?), BAV II ou III (rare), tachy à QRS
    fins + clinique
\end{itemize}
Diagnostic : chercher cardiopathie sous-jacente, ECG concomitant
\begin{itemize}
  \item interrogatoire : alcool++, fièvre++, déshydratation, SAS,
    hyperthyroïdie, grossesse
  \item ECG, echocardio, ECG d'effort
\end{itemize}

\paragraph{Étiologies fréquentes}\mbox{}\\
Extrasystoles : cherche (extra) cardiaque :
\begin{itemize}
  \item alcool, électrocution, pneumopathie, hyperthyroïdie, anomalie
    électrolytique, anxiété, grossesse, SAS
  \item \danger{} obèse/diabétique : bien vérifier si fibrillation atriale !
\end{itemize}
Tachycardie sinusale :
\begin{itemize}
  \item cardio (avec dyspnée) : insuf cardiaque, EP, épanchement péricarde...
  \item extra : fièvre, anémie, hypoxie, hyperthyroïde, grossesse, alcool,
    hypotension artérielle, SAS...
\end{itemize}
Troubles supra-ventriculaires
\begin{itemize}
  \item flutters/tachy atriale
  \item tachycardie jonctionnelle : jeune, coeur normal, polyurie en fin d'accès,
    arrêt par vagal
\end{itemize}
Troubles ventriculaires : rares, plutôt syncope. Sur cardiopathi et signes
gravité.

Névrose cardiaque = élimination

\section{232 : Insuffisance cardiaque}%
\label{sec:insuffisance_cardiaque}
Déf clinique : symptômes d'\gls{IC} (dyspnée, oedèmes
chevilles, fatigue...) et signes d'IC (crépitant, turgescence jugulaire...)
et anomalie de structure/fonction du coeur

Prévalence : 1-2\%, augmente avec l'âge

Adaptation :
\begin{itemize}
  \item cardiaque : remodelage (dilatation ventriculaire, hypertrophie), \inc
    inotropie\footnote{Force de contractation musculaire}, tachycardie
  \item extra-cardiaque : vasoconstriction, rétention hydrosodée, activation
    neurohormonale
\end{itemize}
Si IC non réversible et non curable, la fonction systolique est :
\begin{itemize}
  \item soit diminuée (défaut contraction donc dilatation)
  \item soit préservée (parois épaissies)
\end{itemize}

\subsection{Diagnostic}
Fonctionnels :
\begin{itemize}
  \item respi : dyspnée d'effort (cf NYHA\footnote{I = 0 dyspnée, II = marche
      rapide, III = marche terrain plat, IV = au repos}, orthopnée, dyspnée paroxystique nocturne
  \item \danger{} asthme, toux, hémoptysie aussi 
  \item autres : fatigue (repos/effort), faiblesse musculaire, palpitations
  \item si sévère : respi, neuro, digestif
  \item IC droite : hépatalgie
\end{itemize}
Physique : pauvre donc des signes sont facteur de gravité
\begin{itemize}
  \item cardiaque : palpation : choc de pointe en bas à gauche\\
    auscult : tachy, galop B3, éclat B2 en pulmonaire, souffle d'insuf
    mitrale/tricuspide, souffle de valvulopathie
  \item pulmonaire : râles (sous-)crépitants, épanchement pleural
  \item artériel : pouls rapide. si PAS < 100mHg, facteur de gravité
  \item \textbf{signes périphériques dIC droite} : turgescence jugulaire, reflux
    hépatojugulaire, hépatomégalie, oedèmes périph, ascite
\end{itemize}

\textit{ECG}  peu contributif.

\textit{RX thorax}  :
\begin{itemize}
  \item cardiomégalie (\gls{RCT} > 0.5)
  \item stase pulmonaire :
    \begin{itemize}
      \item redistribution vasc de la base vers sommets
      \item oèdème interstitiel (ligne B de Kerley, gros vaisseaux hilaires flou,
        réticulo-nodulaire aux base)
      \item oedème alvéolaire ("ailes de papillons")
    \end{itemize}
  \item épanchement pleural
\end{itemize}

Bio : Na+, K+, créat, bilan hépatique, TSH us, NFS, fer

\textit{Dosage (NT-pro)BNP}  : intéressant si normaux ou très élevés

\textit{ETT} : indispensable ! Peut orienter : ischémique, valvulopathie,
hypertrophie

Autres :
\begin{itemize}
  \item coronarographie : important ! Revasc ou peut orienter vers cardiomyopathie
    dilatée
  \item IRM cardiaque : peut compléter échocardio (mesure \gls{FE})
  \item scintigraphie : mesure FE
  \item Holter : troubles du rythme ventriculaire ou supra-ventriculaire
  \item Épreuve d'effort
  \item Cathétérisme : mesure pression pulmonaires, débit cardiaque
\end{itemize}

\subsection{Étiologies}
Toute patho cardiaque peut donner une IC....

\begin{itemize}
  \item Cardiopathies ischémique : 1ere cause ! Souvent plusieurs IDM
  \item HTA \thus hypertrophie, IDM/atteinte petites coronaires
  \item Cardiomyopathies : dilatées (25\% familiale), hypertrophique (surtout
    familiable), restrictive (rare)
  \item Valvulopathies : gauche
  \item Troubles rythme (supra)-ventriculaire
  \item Péricarde
  \item IC droit : conséquence IC gauche ou isolé (patho pulmonaire, HTA
    pulmonaire...)
  \item À débit augmenté
\end{itemize}

\subsection{Formes cliniques}
\paragraph{ Insuf aigüe}
OAP sutout : détresse respi aigüe (inondation alvéolaire) 
\begin{itemize}
  \item polypnée, orthopnée
  \item sueurs, anxiété, cyanose, grésillement laryngé, toux + expect mousseuse
    rose saumon
  \item râles crépitant
\end{itemize}
\thus PEC immédiate \skull
   
Choc cardiogénique possible : < 85mmHg, extrémités froides, marbrures, oligurie

Toujours chercher facteurs favorisants : rupture traitement, surinfection
bronchique, troubles du rythme, anémie, EP, dysthyroïdes, iatrogène, poussée
hyppertensive

\paragraph{Autres} : chronique, à fonction systolique préservée (50\%)

\paragraph{Complications}
\begin{itemize}
  \item DC : 50\% à 5 ans
  \item IC aigǜe avec hospit
  \item troubles rythme (supra)ventriculaire
  \item thromboemboliques
  \item hypotension artérielle
  \item troubles hydroélectrolytiques, insuf rénale
  \item anémie, carence martiale
\end{itemize}

\subsection{Traitement IC chronique}
\paragraph{Étiologie}
FR, revasc

\paragraph{Hygiène}
< 5g de sel, conserver poids, pas de tabac, diminuer alcool, activité physique,
pas d'efforts importants au travail, vaccins : grippe (âgé), pneumocoque,
contraception pour éviter grossesse

\paragraph{Médicaments}
\begin{enumerate}
  \item Diurétiques (congestion) + IEC (diminue angiotensine II) + betabloquants
    (\skull pas immédiatement si crise aigüe \danger{}) $\pm$ antagonistes des
    récepteurs aux minéralo-corticoïdes (diurétiques)
  \item Si échec : + ivabradine (diminue FC)
  \item Si échec : défibrillateur automatique (+ resynchronisation si QRS >
    120ms)
  \item si échec : digoxine (n'améliore pas la survie) ou nitrés (vasodilatateur), voire greffe/assistance
\end{enumerate}

NB : IC à FE conservée mal codifié\footnote{Bloquer régine-angiotensine (IEC/ARA II), ralentir FC
($\beta$-bloquants), diurétiques (thiazidiques/inhib de l'anse de Henlé)}.

\subsection{Traitement IC aigüe}
OAP : 
\begin{itemize}
  \item domicile : assis, furosémide IV, dérivés nitrés, SAMU (?)
  \item hôpital : assis, apport IV G5\%, $O_2$, furosémide, dérivés nitrés si
    PAS < 100 mmHg, (morphine), HPBM systémique !
\end{itemize}
Poussée sans OAP franc : (hôpital), diurétique VI, rééquilibration traitement,
cause ??

Choc cardiogénique : sonde urinaire++, inotropes (ex: dobutamine)

\section{222 : HTA pulmonaire artérielle}%
\label{sec:hta_pulmonaire_arterielle}

Classif des HT pulmonaires : 
\begin{enumerate}
  \item HTAP
  \item maladies cardiaques (seules post-capillaires)
  \item appareil respi ou hypoxémiantes
  \item thromboemboliques chroniques
  \item mécanismes multiples/complèexes
\end{enumerate}
\subsection{Adulte}
Rare, Pas de cause connue. Plutôt \female{}, 40-50ans. Traitement à vie, pas curatif

Épaissisement des paroi artères/artérioles, perte vasodilatation \thus \inc résistance

Définition : 
\begin{itemize}
  \item \gls{PAPm} $\ge 25$ mmHg 
  \item \gls{PAPO} $\le 15$ mmHg
\item débit cardiaque normal/diminué
\end{itemize}

\paragraph{Clinique}
\begin{itemize}
  \item Dyspnée++, très progressive (attention sous-estimation). 
  \item puis asthénie, fatigabilité à l'effort, douleurs thoraciques angineuses,
    palpitations, lipothymies à l'effort
  \item insuf. cardiaque droite : hépatalgie, oedème membres inférieurs
\end{itemize}
Interrogatoire : ATCD, anorexigènes, toxiques, maladie (sclérodermie : sd de
Raynaud, dysphagie, dyspepsie)

Y penser si clinique normale !

CV : signes d'HT (non spécifiques), insuf cardiaque droite
\paragraph{Complémentaires}
\begin{itemize}
  \item Radio thorax normale (DD: fibrose, bronchoemphysème chronique,
    sarcoïdose)
  \item EFR : normales
  \item gazométrie artérielle : normale (ou hypercapnie)
  \item ECG: normal ou \{grande onde P, dév. axiale droite, BdB droit, trouble
    repolarisation\}
\end{itemize}
\textit{Échocardio}  = examen clé, 

\textit{Cathétérisme cardiaque droit}  = confirmation

\subsection{Enfant}

Augmentation du débit ou des résistances

\paragraph{Clinique}
Nouveau-né : cyanose réfractaire à l'$O_2$, détresse respi/circ \thus échocardio en
urgence \danger

Enfant : signes tardifs : dyspnée d'effort, syncope, fatigue
\thus dépistage si cardiopathie congénitale, patho. respi. chronique, maladie de
systèmes, ATCD familiaux

\paragraph{Examens}
Radio thorax, ECG aspécifiques.

Échocardio : diagnosic (+ cherche cardioapathie), confirmé par cathétérisme 

Scanner (parenchyme), écho hépatiques (shunt porto-cave)

\section{199 : Dysnée aigüe et chronique}%
\label{sec:199_dysnee_aigue_et_chronique}
Difficulté respi, fréquent++

Sémiologique : Terrain, rapidité, circonstance, intensité (classif NYAH), nombre d'oreillers

Clinique : \{inspi/expi/les 2\}, \{tachy (> 20), brady (<10)\}, rythme
(Kussmaul, Cheynes-Stokes), intensité (poly/hypo/oligopnée)

\subsection{Aigüe}
\textit{Examens}  : NFS, ECG, radiothorax, gazométrie artérielle, D-dimère, BNP

\textit{Orientation} 
\begin{itemize}
  \item Signes de choc ? Traiter
  \item Sinon, dyspnée inspiratoire ? Si oui, laryngé : 
    \begin{itemize}
      \item brady inspiratoire, cornage, tirage
      \item {corps étranger, épiglotite, laryngite} chez l'enfant
      \item trachéal
    \end{itemize}
  \item Sinon  dyspnée expiratoire ? Si oui : 
    \begin{itemize}
      \item crise asthme : sibilants. À distinguer de l'asthme aigü
        grave\footnote{Thorax bloqué en inflation, 0 sibilant, 0 élocution}
        (urgence \danger)
      \item exacerbation BPCO : ATCD, sibilants, hippocratisme digital (\danger
        EP possible)
      \item pseudo-asthme cardiaque
    \end{itemize}
  \item Sinon, crépitants ? Si oui, 
    \begin{itemize}
      \item OAP : orthopnée, wheezing, crépitants bilat, terrain, RX
        (cardiomégalie, oedeme alvéolaire), ECG (Q nécrose, FA), BNP \inc
      \item pneumopathie infectieuses : sd infectieux, fièvre, toux, expect
        purulente, crépitants, RX : opacité parenchymateuses systématisée
    \end{itemize}
  \item Sinon, sd pleural ? Si oui, pneumothorax, épanchement pleural (asymétrie
    ascult, RX)
  \item Sinon : 
    \begin{itemize}
      \item EP : fréq++, brutale, auscultations normales, gazométrie : effet
        shunt
      \item sepsis sévère, acidose, anémie, neuromusc
    \end{itemize}
  \item Sinon psychogène
\end{itemize}

Autres :
\begin{itemize}
  \item cardiaque : tamponnade (orthopnée, tachy, assourdissement bruits, ascult
    pulmonaire normale, turgescene jugulaire, pouls paradoxal), troubles
    (supra)-ventriculaire, choc cardiogénique
  \item pulmonaire : SDRA, décompensation aigüe, atélectasies, trauma
\end{itemize}

\subsection{Chronique}
Examens : NFS, EFR, échocardio, radiothorax, épreuve d'effort, scanner thoracique,
cathétérisme

\begin{itemize}
\item Cardiaque : ECG, radiothorax, BNP\footnote{Si normales, peu probable}
    \begin{itemize}
      \item insuf cardiaque
      \item constriction péricardique
    \end{itemize}
  \item Pulmonaire :
    \begin{itemize}
      \item obstructif : BPCO, asthme à dyspnée continue
      \item restrictif : PID, pneumoconioses, post-tuberculose, paralysie phrénique,
        cyphoscoliose, obésité morbide
    \end{itemize}
  \item HTAP
  \item HT pulmonaire post-embolique
\end{itemize}

\section{224 : Thrombose veineuse profonde et embolie pulmonaire}%
\label{sec:224_thrombose_veineuse_profonde_et_embolie_pulmonaire}
Complications de la \gls{TVP} : \gls{EP} (précoce), \gls{SPT} (tardif)

TVP : obstruction thrombotique d'un tronc veineux profond. EP : idem mais
artères pulmonaires ou leurs branches (secondaires TVP à 70\%)

3eme cause de DC en france

Facteurs de risques temporaires (chirurgie orthopédique, trauma, alitement > 3
jours), permaments (ATCD \gls{MTEV})

\paragraph{Physiopatho} stase veineuse, lésions pariétales, anomalies de l'hémostase \thus
obstruction/lyse. Si migre dans les artères pulmonaires : symptômes si
obstruction à 30-50\% \thus anomalie hémodynamique \thus insuf respi

\paragraph{Évolution}
TVP distales asymptomatique (postop) : 20\% deviennent proximales\\
TVP distales symptomatique : récidive (9\% si ttt anticoagulant)\\
TVP proximal symptomatique : risque important !\\
EP : TVP + 3/7 jours, mortelle dans l'heure à 10\%

\subsection{Thrombose veineuse profonde}

\paragraph{Clinique} = orientation\\
Douleur du MI, oedème unilatéral, signes inflammatoires, dilatation veines
superficielles ou asymptomatique

\textit{DD}  : \{traumatisme, claquage musc\}, kyste synovial, \{SPT, insuf veineuse
primaire\}, \{sciatique, compression extrinsèque\}, \{érysipèle, lymphangite,
cellulite\}, lymphoedeme, insuf cardiaque droite/rénale/hépatique

Score de probablitié clinique : Wells (faible/interm/forte)

\begin{figure}[htpb]
  \centering
  \resizebox{0.6\linewidth}{!}{
    \tikz \graph [
    % Labels at the middle 
    edge quotes mid,
    % Needed for multi-lines
    nodes={align=center},
    sibling distance=6cm,
    layer distance=2cm,
    edges={nodes={fill=white, align=center}}, 
    tree layout]
    {
      criteres/"âge < 80 ans,\\pas de K,\\pas de chirurgie < 30 jours" -> {
        ddimere/"D-dimères" [>"oui"];
        contention/"Contention\\
          (+HBPM si proba clinique\\$\ge$ intermediaire)" [>"non"];
      };
      ddimere -> {                                                                   
        contention [>"$> 500\mu$g/L"] -> "écho-doppler veineux\footnotemark" -> {                 
          "Anticoag\\contention" [>"positif"];                                      
          "Surveillance\\$\pm$ anticoag, contention" [>"négatif\\forte suspicion"]; 
          dd/"Chercher DD" [>"négatif"];                                                          
        };                                                                             
        dd [>"$< 500\mu$g/L"];                                           
      };
    };
  }
  \caption{Diagnostic TVP}
\end{figure}

\footnotetext{Signes : thrombus, incompressibilité de la veine à la
pression, \dec signal, remplissage partiel du throbus.}

\paragraph{Étiologies}
\begin{itemize}
  \item FR transitoire (chir, fracture < 3 mois, immobilisation > 3 j) ? Sinon, "non
provoquée"
\item Recherche de thrombophilie si
  \begin{itemize}
  \item 1er épisode :  (non provoqué < 60 ans) ou (femme âge procrééer)
  \item ou récidive : (MTEV proximale) ou (TVP distale non provoquée)
  \end{itemize}
  \thus 1ere intention : \gls{AT}, prot. C et S, mutation G20210A,
  homocysténiméie, Ac antiphospholipides
\item Recherche cancer : > 40 ans ou bilan thrombophilie négatif
\end{itemize}

Formes particulières :
\begin{itemize}
  \item thromboses veineuse superficielles : (sur trajet saphène, douloureux,
    rouge, inflammatoire, cordon induré \thus écho-doppler
  \item TVP pelvienne
  \item thrombose veine cave inférieure
  \item phlébite bleue : très rare mais grave
\end{itemize}

Si grossesse : bilan thrombophilie si ATCD familaux/personnels MTEV

Si cancer : HBPM long cours

\paragraph{Évolution} : favorable si bien conduit mais récidive toujours
possible. Complications : 
\begin{itemize}
  \item SPT : lourdeur de jambes, oedeme de cheville, dilat veineuses
    superficielles, troubles trophiques sans ulcère, ulcères sus-malléolaires
  \item EP
\end{itemize}

\subsection{Embolie pulmonaires}

\paragraph{Clinique} pas spécifique : dyspnée, douleur thoracique, syncope,
crachats hémoptoïques, asymptomatique

Radiothorax : aspécifique. 

Gaz du sang : hypoxie-hypocapnie

ECG : normale /tachy sinusale / souffrance VD

\thus score probablitié (Genève, Wells)

\paragraph{PEC} voir fig.~\ref{fig:pec_ep_risque}, \ref{pec_ep}.
\begin{figure}[htpb]                    
  \centering                              
  \resizebox{0.6\linewidth}{!}{            
    \tikz \graph [
      edge quotes mid,
      % Needed for multi-lines              
      nodes={align=center},                 
      sibling distance=3cm,                 
      layer distance=2cm,                   
      edges={nodes={fill=white}},
      component order=by first specified node, % important here
      tree layout]                       
    {                                     
      "scanner disponible immédiat.?" -> {
        ETT [>"non"] -> {
          "Cherche autre"[>"non"];
          "scanner dispo ?" [>"oui", sibling distance=5cm] -> {
            scanner [>"oui"] -> {
              ttt/"thrombolyse\\embolectomie" [>"positif"];
              autre [>"négatif"];
            };
            ttt [>"non"];
          };
        };
        scanner;
      };
    };
  }
  \caption{PEC : EP supectée à haut risque}
  \label{fig:pec_ep_risque}  
\end{figure}                            

\begin{algorithm}
  Proba clinique faible/intermédiaire : D-Dimère. Si positif, goto 2\;
  Si scanner ou écho MI positif, traitement\;
  \caption{EP supectée non haut risque}
  \label{fig:pec_ep}
\end{algorithm}

NB : 
\begin{itemize}
  \item si suspicion d'EP, TVP \textbf{proximale} à l'echo des MI suffit au
diagnostic.
  \item Scinti : si normale, pas d'EP
  \item ETT : en urgence, signe de surcharge peut suffire à poser le diagnostic
\end{itemize}

\textit{DD} :
\begin{itemize}
  \item douleur thoracique : IDM, péricardite, dissection aortique,
    pneumothorax
  \item dyspnée aigüe : oedème aigu pulmonaire, crise d'asthme, décompensation
    BPCO, pneumopathie
\end{itemize}

\paragraph{Pronostic}
Risque haut/intermédiaire/faible selon : choc/hypotension (H) \footnote{PAS < 90mmHg
ou -40mmHg > 15 min sans cause rythmique/hypovolémique/septique}, dysfonction
VD (H + I), ischémie (H + I)\footnote{H=risque haut, I = intermédiaire}

\paragraph{Évolution} favorable. Complications : choc cardiogénique réfractaire
(DC), rédicive, HTAP chronique post-embolique (rare mais grave)

\subsection{Traitement curatif}
Confirmer diagnostic avant traitement antiocagulant (sauf si 0 risque
hémorragique et probabilité $\ge$ intermédiaire)

\begin{itemize}
  \item HNF : 500UI/kg/jour puis selon test d'hémostase. Pour insuffisants
    rénaux sévères ou instables
  \item HBPM, fondaparinux : pas de surveillance bio
\end{itemize}
Pas de surveillance systématique des plaquettes si : HBPM hors contexte post-op,
fondaparinux 

Thrombolytique : seulement si EP grave !

Relai per os AVK précoce. Objectif INR = 2.5

Durée : $\ge$ 3 mois si TVP proximale ou EP

Compression élastique $\ge 2$ ans, ASAP. 

Hospitalisation ? 
\begin{itemize}
  \item EP
  \item TVP proximale chez insuf rénaux sévère, risque hémorragique, syndrome
    obstructif ou localisation iliocave
\end{itemize}

Filtre cave : non systématique

\paragraph{Cas particuliers}
\begin{itemize}
  \item 
TVP distale : symptomatique : anticoagulement 6 semaines seulement si premier épisode. Dans tous les cas,
compression $\ge 2$ ans
\item \gls{TVS} : \textbf{pas} d'AINS, antiocoagulants en curatif, chirurgie
\item cancer : HBPM, arrêt si plaquettes < 50g/L
\end{itemize}
Prévention : 
\begin{itemize}
  \item risque modéré : HBPM/HNF/fondaparinux, compression
  \item risque élevé : idem sans HNF
\end{itemize}

Nouveau anticoagulant oraux (rivaroxaban) :
\begin{itemize}
  \item prévention chir hanche-genou
  \item curatif TVP, EP
  \item pas si insuf rénale sévère ou insuf hépatique
  \item pas de surveillance bio
\end{itemize}

\section{221 : Hypertension artérielle}%
\label{sec:221_hypertension_arterielle}
Grades (\gls{PAS}/\gls{PAD})):
\begin{itemize}
  \item 1 : 140/90 - 159/99
  \item 2 : 160/100 - 759/109
  \item 3 : > 180/110
  \item systolique isolé : > 140 et < 90
\end{itemize}

HTA modérée = plus fréquente\\
\inc avec l'âge. Plus fréquent chez femmes, noirs, obèse, consommation sel,
défavorisé. Génétique 30\%.

Risque estimé par PAS, PAD (Après 60 ans, utiliser la PAS et Pression pulsée = PAS - PAD)

\paragraph{Physiopatho}
Régulation :
\begin{itemize}
  \item court terme : sympathique 
  \item moyen : rénine-angiotensine-aldostérone et peptide natriurétique (ANP,
    BNP)
  \item long : natriurièse de pression, arginine-vasopressine
\end{itemize}

90\% sont essentielles.

\subsection{Évolution}
\paragraph{Complications}
\begin{itemize}
  \item 
\textit{Neuro}  : AVC ischémique, hémorragie cérébrale/méningée, encéphalopathie
hypertensive, lacune cérébrale, démence vasculaire, rétinopathie hypertensive
\item \textit{CV}  : insuf cardiaque systolique, insuf VG (anomalie de
  \textit{remplissage}), cardiopathie ischémique, FA, arythmie
  ventriculaire, complications artérielles. Mortalité CV $\times 5$ (\male) ou
  $\times 3$ (\female)
\item \textit{rénales} : évolution vers l'insuf rénale par néphroangioscélose,
  sténose athéromateuse de l'artère rénale, diurétiques, IEC
\item rénale
\end{itemize}

\paragraph{Urgences hypertensives} : HTA sévère et atteinte aigüe des organes
\begin{itemize}
  \item Urgences : SCA, insuf VG, dissection aortique, encéphalopathie
    hypertensivee, hémorragie méningée/AVC, phéochromocytome, \{amphétamines, 
    LSD, cocaïne\}, péri-op, (pré)éclampise, sd hémolytique et urémique
  \item examens : bio, ECG, radiothorax, échocardio, \gls{FO},
    scanner cérébral, imagerie aortique
\end{itemize}

\paragraph{HTA maligne} rare. Hypovolémie (natriurèse). \\
Tableau : PAD > 130mmHg, oèdeme papillaire au FO, insuf VG, insuf rénale
aigüe++. \\
Évolution en quelques mois++

\subsection{Bilan inital}
Mesure de la pression : 
\begin{itemize}
  \item ascult (attention : effet bouse blanche, HTA masquée, rigidité des
    artères si âgé)
  \item MAPA (24h, toutes 15min), obj < 130/80 en 24h
  \item automesure 
\end{itemize}
Évaluation :
\begin{itemize}
  \item interrogatoire : \{ancienneté, FR, organes (cerveaux, yeux, coeur rein,
    artères)\}, \{secondaire (médica : contraceptifs oraux++, AINS++...)\}
  \item clinique : atteinte des organes, secondaire, obésité
  \item complémentaires : gylcémie, cholestérol (total, HDL, trygylcéride, LDL),
kaliémie, créat, BU, ECG repos)
\end{itemize}
Atteintes spécifiques :
\begin{itemize}
  \item coeur : ECG (HVG, hypertrophie atriale G), échocardio (HVG = masse VG >
    125g/$m^2$ (110 chez \female)
  \item écho carotides (AOMI : IPS < 0.9)
  \item rein : \inc créat ou \dec clairance créat
  \item FO : stade 3 (hémorragie, exsudats), 4 (oedème papillaire)
\end{itemize}

\paragraph{Calcul du risque}
Élevé si grade 3 ou ((grade 1 ou grade 2) et $\ge 3$ FR)\\
FR : âge, tabac, ATCD familaux d'accident CV précoce, diabète, dyslipidémie

\subsection{HTA secondaire}
Dépistage si point d'appel, grade 3, < 30 ans, HTA résistante

\begin{itemize}
  \item néphropathie parenchymateuse : palpation de masses abdo bilat \thus écho
    abdo et créat, protéinurie, sédiment urinaire
  \item HTA rénovasculaire : clinique (souffle abdo lat, OAP récidivant sans
    explication), bio (hypoK et hyperaldostéronisme ou insuf rénale). Diag
    écho, confirmé par angio-IRM. \\
    Ttt : hygiénodiététique, aspirine, statine, antihypertenseurs
  \item phéochromocytome : triade de Ménard (céphalée, sueurs, palpitations).
    Dosage urinaire métanéphrines, catécholamines\\
    Localisation tumeur (IRM), $\alpha$ et $\beta$ bloqueurs, exérèse chir
  \item Sd de Conn : dépisté par kaliémie. \inc aldostérone/rénine. Localisation
    tumeur scanner/IRM\\
    Chir éventuelle
  \item Coarctation aortique (enfant, adultee jeune) : clinique, confirmée IRM.
    Ttt chir ou endoluminal
  \item SAS : obèse, si HTA résistante. Diag par polysomnographie
  \item Médicaments : cocaïne, amphétamines, AINS, corticoïdes, ciclosporine,
    contraceptifs oraux, réglisse
  \item HTA gravidique : 16-22 SA. Primipare, noires ou obèses.\\
    DD : HTA > 20 SA, grossesse chez hypertendue, prééclampsie\\
    Complications : éclampsie++, rénale, cardiqaue, RCIU, HELLP\\
    \textbf{Pas} régime sans sel, ni IEC, ARA II, diurétiques
\end{itemize}

\subsection{Traitement}
Objectif : < 140/90\\
Si grade 3 ou (grade 1/2 et $\ge 3$FR)

Hygiénodiététique, traitement FR, éducation.

\paragraph{Médicaments}
\begin{itemize}
  \item inhibiteurs calciques
  \item IEC
  \item antagonistes récepteurs angiotensine II
  \item beta-bloquants
  \item diurétiques thiazidique
\end{itemize}
\danger : pas de beta-bloquants ni diurétiques si FR métaboliques.\\

En association la plupart du temps

Éventuellement :
\begin{itemize}
  \item antiagrégant : aspirine 75mg/j si ATCD CV ou créat ou risque CV élevé
  \item hypolipidémiant : diabétique 2 ou maladie CV ou haut risque CV
\end{itemize}

Autres :
\begin{itemize}
  \item HTA résistante si $\ge 3$ classes médicaments dont 1 diurétiques
  \item âgé : \danger{} hypotension orthostatique, \textbf{pas} régime sans sel,
    $\ge 3$ médic, objectif : PAS < 150\\
    ttt : diurétique, calcium bloqueurs
  \item Urgence : éviter baisse brutale tension et hypotension \skull
\end{itemize}
\section{225 : Insuffisance veineuse chronique}%
\label{sec:225_insuffisance_veineuse_chronique}

Rappel : réseau superificiel = veine saphène interne, saphène externe\\
Insuffisance veineuse quand dysfonction du retour, soit par valvules, soit par
péristaltisme (contraction musculaire, écrasement voûte plantaire)

\begin{itemize}
  \item reflux dans le réseau superficiel = \textit{varices}  (>
    3mm\footnote{$\neq$ télangiectasies < 1mm}) : essentielles ou
    secondaires
  \item \textit{post-thrombotique} : reflux (destruction valvulaire) ou obstruction par thrombose. 
  \item \textit{Insuf valvulaire profonde primitive} : rare
  \item \textit{Déficience pompe musculaire mollet} 
\end{itemize}

\paragraph{FR} Varices : âge, ATCD, obésité, grossesse, \female\\
MTEV : immobilisation, cancer, anomalies hémostase

\paragraph{Symptômes} jambes lourdes, aggravé par station debout prolongée,
fatigue vespérale, chaleur. Calmée par le froid, marche, surélévation

\paragraph{Clinique}
\begin{itemize}
  \item dermite ocre
  \item télangiectasies, veines réticulaires (plantaire, malléole)
  \item oedème cheville
  \item lipodermatosclérose
  \item atrophie blanche
  \item varices
  \item ulcère veineux (stade ultime) : périmalléolaire, oval, peu algique, peu
    creusant, non nécrotique, exsudatif
\end{itemize}
NB : classif CEAP (clinique, étio, anat, physiopatho)

Examens : \textit{écho-doppler} veineux des MI

\subsection{Traitement}
\begin{itemize}
  \item Compression élastique (bas, bandes) : à vie si chronique. Attention CI si AOMI
  \item Hygiène de vie
  \item Invasif : sclérothérapie, traitement endoveineux, chirurgie (conservatrice/exérèse), (recanalisation)
\end{itemize}
NB : CI à la chir = sd obstructif veineux profond

\section{233 Péricardte aigüe}%
\label{sec:233_pericardite_aigue}

\subsection{Diagnostic}
2 parmi
\begin{itemize}
  \item douleur thoracique : résistant trinitrine, majorée décubitus, calmée par
    l'antéflexion
  \item frottement péricardique
  \item ECG typique : 4 stades = \{ sus-ST, T plates, T négatives,
    normalisation\}. Aussi : sous-PQ, tachycardie sinusale, microvoltage
  \item épanchement péricardique
\end{itemize}
$\pm$ fièvre modérée, dyspnée, épanchement pleural

Examens complémentaires (hors ECG) :
\begin{itemize}
  \item bio : inflammation, nécrose, iono, urée, créat, hémoc si fièvre
  \item radiothorax normale (rectitude gauche ou cardiomégalie)
  \item échocardio : épanchement péricardique, masse péricardique
\end{itemize}
Parfois : IRM (2eme intention), ponction ou drainage

Hospit ? si étiologie ou risque (symptôme > plusieurs jours)

\paragraph{Étiologies} 90\% virale ou inconnue
\begin{itemize}
  \item aigüe virale : fréq++, sérologie inutile si typique sans gravité.
    Évolution favorable\footnote{Tamponnade/constriction péricardique rare}.\\
    Fréquent pendant VIH
  \item purulente : rare mais grave. Immunodéprimé, infections sévère.\\
    Évolue vers tamponnade/constriction péricardique\\
    ATB
  \item tuberculeuse : AEG, fièvre modéré persistante. Sujet tuberculeux, âgé,
    greffé, VIH, alcoolique.\\
    BK, PCR\\
    Évolue vers tamponnade/constriction péricardique\\
    ATB
  \item néoplasique : souvent métastase. Échocardio\\
    Ponction/biopsie essentielle. Récidive fréquente
  \item systémique auto-immunes : lupus, polyarthrite rhumatoïde, sclérodermie,
    pérartérite noueuse, dermatomyosite
  \item IDM : précoce (5j) : favorable, tardive (2-16sem) : sd Dressler
  \item insuf rénale chronique : urémique $\ne$ dialysée au long cours
  \item sd post-péricardotomie : post chir ou greff cardiaque.
\end{itemize}

\subsection{Complications}
\begin{itemize}
  \item Tamponnade : compression des cavités D par épanchement.\\
    Clinique : douleur thoracique, signes droits, choc, bruits coeur assourdis,
    pouls paradoxal. \\
    ECG, radiothorax, échocardio ("swining heart")
  \item Myocardite : insuf cardiaque fébrile. Échocardio, IRM+++
  \item Péricardite récidivante : colchicine ?
  \item Péricardite chronique > 3 mois : souvent néoplasique
  \item Péricardite chronique constrictive : adiastolie avec égalisation
    pression télédiastoliques\\
    Insuf cardiaque droite et gauche. Diago écho. Chir ?
\end{itemize}

\subsection{Traitement}
Bénigne : hospit, repos, douleur, AINS ou aspirine, protection gastrique,
colchicine (CI si insuf rénale sévère)

Tamponnade : urgence \skull, remplissage, ponction/drainage

\section{327 : Arrêt cardiocirculatoire}%
\label{sec:327_arret_cardiocirculatoire}

Mort subite : aigǜe, après symptômes < 1h. Arrêt cardiorespi (ACR) : plus d'activité
mécanique cardiaque\\
"No flow" : intervalle sans rénimation = \textbf{important}. 
"Low flow" : intervalle sans rétablissement HD

Chaîne de survie : alerte, réa (RCP), défibrillation, RCP spécialisée

Si > 10 min de fibrillation ou d'arrêt : < 5\% de récupérations

\subsection{Étiologies}
Fibrillation ventriculaire puis brady extrême puis asystolies.
Autres :
\begin{itemize}
  \item SCA inauguraux (40-77\=)
  \item autres CV : troubles rythme sur ischémie ancienne, hypertrophique,
    trouble du rythme/conduction indépendant, tamponnade, dissection aortique...
  \item non vasc : toxique, traumatique, insuf respi aigüe, noyades
\end{itemize}

\paragraph{Diagnostic} ne bouge plus, ne réagit plus, ne respire plus, plus de
pouls

\subsection{CAT}
ABCD : maintien voies Aériennes, assistance respi (B), Circulation (massage
cardiaque), Défibrillation et Drogue

\paragraph{Médicaments}
\begin{itemize}
  \item Adrénaline (vasoconstricteur) : 1mg/4min, avant 2eme choc
  \item Antiarythmique (après 2eme choc): amiodarone 300mg dans 30mL de sérum
    salé isotonique (lidocaïne sinon), (sulfate
    de magnésium)
  \item Bradycardie : atropine (isoprénaline)
  \item (bicarbonate de sodium équimolaire pour alcaliniser)
  \item (thrombolyse)
\end{itemize}

Survie -10\% par minute sans réa

Asystolie = marque arrêt ancien (FV peut évoluer en asystolie)

\paragraph{PEC}
3 phases :
\begin{itemize}
  \item < 12h : acidoses, radicaux libres, enzymes musc
  \item 12h - 3eme jour : atteinte organes
  \item  > 3 j : sd septique
\end{itemize}

Cardiaque : dobutamine (trouble contractilité), monitoring échocardio $\pm$
coronaro/angioplastie

Cerveau :
\begin{itemize}
  \item $O_2$, ventilation importants ! (éviter hypertension intracrânen
    fatable)
  \item sédation 24-49h
  \item éviter hyperglycémie !
  \item hypothermie à 34 degré
\end{itemize}

\section{264 : Diurétiques}%
\label{sec:264_diuretiques}

\subsection{Modes d'actions}
\paragraph{Diurétiques de l'anse} Furosémide

Inhibe réabsorption Na+, K+, Cl- branche ascendante anse de Henlé

Action rapide et courte. Vasodilatation veineuse (utile pour OAP)

\paragraph{Thiazidiques} Hydrochlorothiazide

Inhibe réabsorption NaCl au segment proximal du tube contourné distal. Augment
excrétion $K^+$, $Cl^-$ mais diminue $Ca^{2+}$.

Inefficace si insuf rénale sévère

\paragraph{Épargnant le potassium} spironolactone

Diminue excrétion $K^+$, $Cl^-$ au tube contourné distal. Effet inférieur au 2
autres

\subsection{Indications}
HTA : 
\begin{itemize}
  \item toujours un diurétique
  \item doses faibles
  \item spironolactone seulement si hyperaldostéronisme primaire
  \item si insuf rénale sévère seulement diurétique de l'anse
\end{itemize}
Insuf cardiaque :
\begin{itemize}
  \item furosémide++
  \item dose minimale efficace
  \item aigü : diurétique de l'anse 
\end{itemize}

\subsection{Prescription}
Toujours régime hydrosodé

\paragraph{Effets secondaires}
Hydroélectriques 
\begin{itemize}
  \item déshydratation : âgé, peut entrainer IR aigue
  \item hyponatrémie 
  \item hypokaliémie : fréquente, modérée, surveillance, apport alimentaires
    !
  \item hyperkaliémie : dangereuse \danger{} bradycardie sévères, troubles
    rythme ventriculaire. Favorisé par IEC, ARA II, insuf rénale
  \item hypovolémie : parfois hypotension orthostatique
\end{itemize}
Autres : 
\begin{itemize}
  \item hyperglycémie, hyperuricémie (anse, thiazidiques)
  \item augmentation cholestérol (thiazidique)
  \item gynécomastie (spironolactone)
  \item ototoxicité (anse)
  \item interaction : lithium, hypokaliémiant, AINS
\end{itemize}

\section{326 : Antithrombotiques}%
\label{sec:326_antithrombotiques}
% ???
%%%%%%%%%%%%%%%%%%%%%%%%%%%%%%%%%%%%%%%%%%%%%%%%%%%%%%%%%%%%%%%%%%%%%%%%%%%%%%%%%
% Cliniques usuelles :                                                          %
% \begin{itemize}                                                               %
%   \item Bifasciculaire du sujet âgé avec perte de conaissance : BdB droite et %
%     hémibloc antérieur gauche. Chercher cardiopathie sous-jacente. souvent    %
%     endocavitaire.                                                            %
%   \item BdB gauche de l'infarctus antérieur                                   %
%   \item Bloc alternant : (BdB droit III et gauche III) ou (alternance         %
%     entre les 2 hémibloc avec BdB droit III)                                  %
% \end{itemize}                                                                 %
%%%%%%%%%%%%%%%%%%%%%%%%%%%%%%%%%%%%%%%%%%%%%%%%%%%%%%%%%%%%%%%%%%%%%%%%%%%%%%%%%
\subsection{Antiagrégants plaquettaires}
\paragraph{Aspirine}
Inhibe Cox1. Irréversible. Effets antalgique, anti-inflammatoires (doses plus
fortes), anticancéreux

Posologie : 300mg + 75mg/j.

Indications : 
\begin{itemize}
  \item prev. secondaire : coronaropathie, artériopathie des MI, AVC (à
vie)
\item  prev. primaire : coronaropathie, AVC
\end{itemize}

EI : saignements, intolérances gastriques (d'où IPP)

Situations à risque : attendre 6 semaines/3-6 mois tout acte invasif à risque
hémorragique. Si risque très important, arrêt 5 j seulement.

\paragraph{Thiénopyridines, ticagrelor}
Bloque récepteur P2Y12
\begin{itemize}
  \item Clopidogrel : 300-600mg + 75mg/j. SCA et post-angioplastie coronaire
    (avec aspirine)
  \item Prasugrel : 60mg + 10mg/j. SCA post-angioplastie 
  \item Ticagrelor : 180mg + 90x2mg/j. SCA 
\end{itemize}

CI absolue : prasugrel si ATCD accident cérébral.

\paragraph{Autres\footnote{Dipyridamole : n'est plus utilisé}}
\begin{itemize}
  \item Anti-GPIIb-IIIa : voie veineuse, courtes périodes, conditions très particulières
\end{itemize}

\subsection{Héparines}%
\begin{itemize}
  \item HNF : IV, effet immédiat. Antidote : sulfate de protamine. 80 UI/kg puis
    18 UI/kg/h \\
    \textbf{Surveiller TCA !} \danger
  \item HPBM : demi-vie plus longue. Attention au rein !!. 100 I/kg x2/j
  \item fondaparinux : demi-vie plus longue. Attention au rein !!
\end{itemize}

Indications : pour anticoag urgente = thromboses veineuses profondes, EP,
troubles du rythmes, SCA.

EI : complications hémorragiques, \gls{TIH}

Apparentés :
\begin{itemize}
  \item danaparoïde : anticoag si ATCD de TIH
  \item bivalirudine : pour angioplastie coronaire
\end{itemize}

\subsection{Anti-vitamines K}
Agit sur facteurs dépendant de vitamine K. Oral, traitement longue durée.

Coumadine (ref), fluindione (Previscan) = demi-vie longue. Acénocoumarole = demi-vie courte.

Utiliser l'héparine en relais des AVK avec 4-5j de chevauchement et 2 INR
efficaces à 24h

\textbf{Surveiller INR}  1/mois \danger. Cible : INR $\in [2,3]$

Antidotes : PPSB, vitamine K

Indications : FA, TVP, EP, valve cardiaque mécanique,
complications de l'IDM/insuf cardiaque

Situations à risque très hémorragique :
\begin{itemize}
  \item arrêt 3-4j avant
  \item ou arrêt 4-5 j avant et relais héparine (pour TVP, EP > 3 mois, FA
    à risque embolique élevée, valves mécaniques)
\end{itemize}

\subsection{Nouveau anticoagulants oraux}
Dabigatran, rivaroxaban, apixaban. Inhibe facteur II ou X.

Action rapide (2h) mais pas d'antidote

CI : dabigatran si fonction rénale altérée. Suivre rein pour tous !!!

\subsection{Thrombolytiques}
Active la fibrinolyse physio. Suffixes : -kinase, -téplase

Indications spécifiques (IV) : 
\begin{itemize}
  \item IDM < 6-12h
  \item AVC < 4h30
  \item EP grave
\end{itemize}

\subsection{Accident anticoagulants}
\paragraph{Héparines}
Hémorragique : 
\begin{itemize}
  \item 1-4\%.
  \item clinique : TCA > 3 témoin, asymptomatique, anémie microcytaire
    ferriprive, hématome
  \item respecter prescription (CI si insuf rénale sévère)
  \item si accident majeur : continuer ? Antidote ? Remplissage IV dans tous les
    cas
\end{itemize}
Thrombopénies induites par les héparines : 
\begin{itemize}
  \item type I = bénin, non immun. Type II = grave, immun, +7/10jours.
  \item mécanisme thrombotique (!)
  \item clinique : plaquettes < 100G/L, thromboses veineuses/artérielles,
    résistance héparine, thrombose/thrombopénie juste après arrêt héparine
  \item CAT : confirmer, éliminer causes infectieuses, médicam, test ELISA
  \item puis arrêt héparine, danaparoïde sodique à la place (éventuellement
    AVK), NFS tous les jours, déclaration \textbf{obligatoire} 
  \item prévention : relais AVK, remplacer par HBPM ou fondaparinux
\end{itemize}

\paragraph{AVK}
Hémorragiques : 
\begin{itemize}
  \item AVK = 1ere cause d'hospit iatrogène !
  \item hémorragie grave : extériorisé non contrôlable, instabilité
    hémodynamique, geste hémostatique, transfusion de culots globulaires,
    pronostic vital/fonctionnelle
  \item CAT (si grave) : arrêt AVK, INR urgence et vitamine K + CCP (antidote),
    surveillance biologique
\end{itemize}

\appendix
 \resizebox{\linewidth}{!}{           
\begin{tikzpicture}
    \graph [
    % Labels at the middle 
    %edge quotes mid,
    % Needed for multi-lines
    nodes={align=center},
    % right-angle arrow
    skip loop/.style = {to path={-- ++(0,#1) |- (\tikztotarget)}},
    vh path/.style = {to path={ |- (\tikztotarget)}},
    hv path/.style = {to path={ -| (\tikztotarget)}},
    medic/.style = {rectangle, draw=red},
    %sibling distance=6cm,
    %layer distance=1.5cm,
    %edges={nodes={fill=white}}, 
    branch down=9mm,
   grow right sep]
   % Ugly hack : using "above" should not be used with "right
    {
      at/angiotensinogène -- p1 [coordinate] -> at1/"Angiotensine I" 
      -- p2 [coordinate] -> at2/"Angiotensine II" 
      -- p3 [coordinate] -> "+" -> [vh path, "lol"] {
        "Système sympathique";
        "Réabsorption Na+ et Cl-\\Excrétion K+";
        "Aldostérone\\(cortex surrénalien)";
        "Vasoconstriction artérioles";
        adh/"ADH (hypophyse)";
      }
      --[hv path] p4 [coordinate] -> rt/"Rétention hydrosodée\\Augmentation volémie";
      Rein[above=-1mm of p1];
      Foie[above=6cm of at];
      Foie -> at;
      rt -- [vh path] neg/"-"[below=-6cm of adh] -> Rein ->["+"] p1;
      p1 --["Rénine"] Rein;
      IEC[above=8cm of p2, medic] ->["-"] p2;
    };
\end{tikzpicture}
}
\printglossary
\printglossary[type=\acronymtype]

\end{document}

%%% Local Variables:
%%% mode: latex
%%% TeX-master: t
%%% TeX-engine: luatex
%%% End:
