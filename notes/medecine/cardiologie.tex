\input header

\begin{document}
\title{Fiches de cardiologie}
\author{Alexis Praga}
\maketitle
\tableofcontents

\input bacteries-header

\section{218 : Athérome}%
\label{sec:1_atherome}

Épidémio : 1ere cause de mortalité dans le monde. 

En France : incidence \male = 5$\times$\female. 

Mortalité $\searrow$ mais prévalence $\nearrow$

\subsection{Mécanisme}
Contient centre lipiqude, cellules {spumeuses,muscularise, inflammatoire} +
chape fibreuse

Évolution de la plaque :
\begin{itemize}
  \item rupture (plus probable si plaque jeune !)
  \item progression par poussées
  \item hémorragie intraplaque
  \item régression ?
\end{itemize}
Remodelage

Anévrismes

\paragraph{Localisations}
Surtout : carotides (AVC), coronaires (cardiopathies ischémiques), membre inférieure
(AOMI)

\paragraph{Évolution} Aggravation par étapes silencieuses. \danger gravité pas
toujours proportionnelle à l'ancienneté/étendue

FDR : tabagisme, HTA, dyslipidémie, diabète

\paragraph{Thérapeutiques}
Prévention du développement de l'athérome : diminuer lésion endothéliale,
diminuer accumulation LDL, stabiliser plaques, diminuer volume des plaques,
diminuer l'inflammation, diminuer les contraintes mécaniques

\subsection{Polyathéromateux}

$\ge 2$ territoire artériels différents

Évaluer FdR, bilan des lésions

Thérapeutiques :
\begin{itemize}
  \item arrêt tabac, diététique, activité physique
  \item aspirine en systématique (colpidogrel si intolérance)
  \item statines en prévention secondaire
  \item IEC\footnote{Inhibiteurs de l'enzyme de conversion}, ARA
      II\footnote{Antagonistes des récepteurs de l'angiotensine}
\end{itemize}

PEC spécifique : chirurgie anévrisme ($\diameter \ge 5.5cm$), endartériectomie
(sténose carotide > 60\%), revasc. myocardique (sd coronaire aigü $\wedge$
sténose coronaires > 70\%)

\section{219 : Facteur de risques cardio-vasculaires}%
\label{sec:219_facteur_de_risques_cardio_vasculaires}

Facteur de risque (FR) : causalité avec la maladie $\neq$ marqueur de risque
(simple témoin)

\subsection{FR}
Non modifiables : 10 ans + tôt chez \male, hérédité = plutôto environnement
familial

Modifiables : 
\begin{itemize}
  \item risque : {tabagisme, hypercholestérolémie, HTA, diabète, obésité abdo,
    psychosociaux}
  \item protecteur : {fruit et légumes, activité physique, alcool modéré}
\end{itemize}

\paragraph{Tabac}
1ère cause de mortalité évitable.

Conséquence : \dec HDL, \inc risque thrombose, altère vasomotricité artérielles,
\inc [CO]

2eme FR de l'IDM : $\propto$ consommation, $\forall$ tabac, jeunes, passif

Rôle : AOMI, anévrisme aorte abdo, AVC

\paragraph{Hypercholestérolémie}
3eme FR IDM : \inc LDL et \dec HDL = mauvais signe $\implies$ exploration d'une
anomalie lipidique à jeun

Majorité = alimentaire mais génétique possible (hétérozygote/homozygote)

\paragraph{HTA}
Stade 1 : [140-159]/[90-99] mmHg
Stade 2 : [160-179]/[100-109] mmHg
Stade 3 : > 180/110 mmHg

Silencieuse. Impact coeur (insuf. coronire, cardiaque), cerveau (AVC), rein (IR)

Augmente avec l'âge.

3 mesure espaces d'1 semaine

\paragraph{Diabète}
90\% de diabète 2 (résistance insuline). Déf :
\begin{itemize}
  \item diabète si glycémie à jeun > 1.26g/L
  \item hyperglycémie non diab : glycémie jeun $\in [1.10, 1.26]$ g/L
  \item intolérance hydrates de carbones : < 1.26 (jeun), $\ge 2$ (provoquée)
    puis $\in [1.40, 2]$
\end{itemize}
Hérédité. Complications microvasc, macrovasc

\paragraph{Surpoids}
IMC $\in [25, 29.9]$ = surpoids, IMC $\ge 30$ = obésité. 

Obésité centrale = (\diameter abdo $\ge 94 $cm (\male) ou $\ge 80$cm (\female))
\land{} 2 FR


\subsection{Évaluation}
\label{subsec:fr}
Score
\begin{itemize}
  \item +1 si {tabac $\le 3$ ans, LDL > 1.6g/L, HTA, diabète, HDL < 0.40g/L, âge > 50
(\male) ou 60 (\female), ATCD coronaires}
  \item  -1 si HDL $\ge 0.60$
\end{itemize}

ATD personnels CV

\subsection{Prévention}
\paragraph{Secondaire}

BASIC : $\beta$bloquants, Antiagrégants, Statine, Inhibiteurs de l'enzyme de
conversion, Contrôle des FR

\begin{itemize}
  \item statine pour LDL < 1g/L
\item sevrage tabac : substituts nicotinique, {bupropion, varénicline},
  anxiété/dépression, TCG
  \item pression artérielle : hygiénodiététique (échec à 3 mois : médic)
  \item contrôle glycémie (diabète)
  \item activité physique régulière
  \item enquête familiale
\end{itemize}

\paragraph{Primaire}
Cholestérol : 0FR : LDL < 2.20, 1FR : LDl < 1.90, 2FR : LDl < 1.60, $\ge 3$ FR
: LDL < 1.30 (haut risque : LDL < 1g/L)







\section{220 : Dyslipidémies}%
\label{sec:220_dyslipidemies}

Risques : maladies CV athéromateuses

LDL = total - HDL - TG\footnote{Tryglycérides}

Bilan normal : \begin{itemize}
  \item LDL  < 1.6g/L
  \item HDL  > 0.4g/L
  \item TG  < 1.5g/L
\end{itemize}

\paragraph{Hyperlipidémies secondaire } hypothyroïdie, cholestase, sd néphrotique, IR chronique, alcoolisme,
diabète, hyperlipidémie iatrogènee, oestrogènes, corticoïdes, rétinoïdes,
antirétroviraux, ciclosporine, diurétiques

\paragraph{Hyperlipidémies primitives}
\textit{Fréquentes}  : hypercholestérolémie familiale monogénique (HCFM) (mutation LDL récepteur
hétérozygote), hypercholestérolémie polygénique, hyperlipidémie familiale
combinée

\textit{Rares}  : HCFM (mutation apolipoprotéine B),
dysbêtalipoprotéinémie, hypertriglycéridémie familiale, hypechylomicronémie
primitives

\paragraph{Risque} faible (0 FR), intermédiarie ($\ge 1$ FR), haut (ATCD)

FR semblables au~\hyperref[subsec:fr]{score précédent} : tabac $\le 3$ ans, HTA, diabète, HDL < 0.40g/L, âge > 50
(\male) ou 60 (\female), ATCD familaux IDM ou mort subite

\subsection{Traitement}

\paragraph{Diététique}
\begin{itemize}
  \item lipides < 40\%
  \item graisses saturées < 12\%
  \item plutôt mono- et polyinsaturées
  \item cholestérol alimentaire < 300mg/j
  \item 5 fruits ou légumes/j
  \item sodium < 6g/j
  \item diminuer excès pondéral
\end{itemize}
Si hypertriglycéridémie (HTG) : \dec poids, alcool, sucres simples

\paragraph{Médicaments}
Primaire : seulement +3 mois après diététique. Secondaire : d'emblée.

\danger pas de statine si grossesse

Hypercholestérolémie, hyperlipidémie mixtes : statines (1er)

HTG : fibrate si TG > 4g/L, diététique sinon



\section{334 : Syndromes coronariens aigüs}%
\label{sec:334_syndromes_coronariens_aigus}

Sd coronaire aigü (SCA) : lésions athérothrombotiques aigües

Angor stable à l'effort : lésions fibro-athéromateuses

\subsection{Angine de poitrine (angor) stable}
Ici : pas de thrombus

Inadéquation besoin/apport $O_2$ : 95\% sténoses athéromateuses coronariennes
serrées (parfois : spasme coronaire, \inc besoins, "à coronires saines")

Mécanisme d'adaptation d'apport en $O_2$ du myocorde = vasodilatation (surtout par
sécrétion monoxyde d'azode par l'endothélium)

Cascade ischémique : \dec perfusion myocarde [scinti] \thus altération
contractilité [écho stress] \thus signes ECG \thus douleur (pas toujours)

Athérome : risque = fracture de plaque \thus (thrombose) mort subit/IDM, angor
instable

\paragraph{Diagnostic}
Douleur angineuse\footnote{Classes de 1 à 4}
\begin{itemize}
  \item typique : rétrosternal en barre horizontale, irradiant (épaules,
    avant-bras, poignet, machoîres), constrictive, angoissante, \textbf{à
    l'effort}, sensible à trinitrine
  \item atypique ou silencieuse possible
\end{itemize}
Exaen clinique souvent négatif mais chercher souffle aortique, souffle vasc, HTA

\paragraph{Examens}
\begin{itemize}
  \item ECG : intercritique = normal, percritique : (sus/sous)-décalage
    ST, ondes T (négatives symétriques, amples positives symétrique)
  \item ECG d'effort : \textit{1ere intention} . Positive si douleur thoracique \lor{}
    sous-decalage ST
  \item Tomoscintigraphie myocardique de perfusion d'effort ou injection
    vasodilatateur (dipyridamole) : segment normal/ischémie/nécrotique.
    \textit{Lorsque VP ECG insuffisante}. Coûteux. Éviter si BBG\footnote{Bloc
    de branche gauche}
  \item Échocardiographie d'effort ou dobutamine. \textit{Mêmes indication que
    scinti} 
  \item IRM stresse : rare
  \item Coronarographie (parfois + venticulalographie) : sténose si > 70\%
    lumière. Invasif, complications rare. \textit{Angor suspecté et examen
    d'ischémie positif}\footnote{Examens complémentaires : test Méthergin pour
      forcer un spasme, FFR (fraction flow reserve) pour vérifier sténose}
  \item Scanner coronaire : non recommandé
\end{itemize}

\danger CI des épreuves de stresse : angor instable, troubles rythme ventriculaire
graves, fibrillation auriculaire rapide, HTA repos > 220/120mmHg

\paragraph{Mauvais pronostic} : 
\begin{itemize}
  \item angor classe 3/4
  \item ischémie pour charge/fréquence cardiaque faible, baisse PA à l'effort
  \item plusieurs segments ischémique, fraction d'éjection <
    40\%\footnote{Normale si > 55\%}
  \item lésions pluritronculaire, tronc coronaire g, IVA proximale
\end{itemize}

\paragraph{Traitements}
Crise : arrêt effort, dérivés nitrés.

Correction FR (tabac, hypolipides, activisé physique, HTA, diabètes, statine,
IEC)

Anti-ischémique (traitement de fond) en première intention : $\beta$bloquant
(anticalcique/ivbradine si intolérance) $\pm$ {dérivés nitrés, molsidomine,
nicorandil}

Antiagrégants plaquettaires :
\begin{itemize}
  \item aspirine (sauf CI) 75mg/j
  \item clopidogrel sinon 75mg
\end{itemize}
Revascularisation si échec médicament ou pour améliorer le pronostic vital
\begin{itemize}
  \item intervention coronaire percutanée (IPC) : stent
  \item pontage coronaire
\end{itemize}

\paragraph{Angor de Prinzmetal} Vaspastique = douleur sensible trinitrine et
\begin{itemize}
  \item au repos, 2eme partie de nuit, récupération = angor de Prinzmetal
  \item sur un effort = surimposé à une sténose
\end{itemize}
Diagnostic : coronarographie \thus test provocation spasme (pendant coronaro)

Ttt : inhibiteurs calcique (2 molécules).
Bon pronostic si traité

\subsection{SCA sans sus-décalage ST}

= \{angor instable, IDM sans sus-décalage ST persistant \}. Ici thrombus non
occlusif

\subsection{IDM}
Ici thrombus occlusif après réaction thrombotique

\paragraph{Diagnostic}
Même douleur que l'angor stable mais 
\begin{itemize}
  \item spontané > 20min, régressant spontanément ou à trinitrine
  \item angor d'effort récent (2-3)
  \item aggravation d'un angor stable
  \item IDM + 1mois
\end{itemize}
Examen clinique normal mais chercher râles crépitants, galop

ECG en urgence \skull puis +6h
\begin{itemize}
  \item percritique : sous-décalage ST (rarement sus), (grandes T négatives \lor{}
    repositivation T). Si normal, diagnostic peu probable
  \item post-critique (être très prudent !) : sous-déclage ST, T négative
    profonde
\end{itemize}
Doser troponine ssi suspicion !

Échocardiographie pour DD

Coronarographie suivant le risque :
\begin{itemize}
  \item très haut risque : en urgence !
  \item haut risque : < 24h (score GRACE > 140) ou < 72 (GRACE $\in [109, 140]$)
  item bas risque (GRACE < 109)  à discuter 
\end{itemize}

\paragraph{Traitement}
USIC : suivi ECG; dosage troponine, crétaninie, glycémie, NFS

Aspirine, inhibiteurs de la pompe à protons, anti ischémique et :
\begin{itemize}
  \item bas risque : inhib P2Y12 (clopidogrel) et anticoag (fondaparinux)
  \item (très) haut risque : inhib P2Y12 (ticagrelor/prasugrel) et anticoag
    (énoxaparine/héparine) (+ anti-GPIIb/IIIa suivant)
\end{itemize}

\subsection{IDM}

Ici, obstruction par thrombus

5 catégories : 1 à 5. Type 1 = 
\begin{itemize}
  \item sus-ST : désobstruer ASAP
  \item sans sus-ST : prévenir
\end{itemize}
\danger urgence ! \skull

Physiopatho : accident vasculaire coronaire athérothrombotique occlusif ou
occlusion coronaire aigüe (segmente : nécrose totale à 12h, akinésie)

\paragraph{Diagnostic}
Douleur précordiale : angineuse au repos > 30min, trinitrorésistante (peut
manquer !)

Examen clinique normal

ECG : sus-décalage ST sur $\ge 2$ dérivations contiguës. Donne la topographie
(antérieur/latéral, inférieur/postérieur).
Parfois en miroir

\fbox{(Douleur thoracique > 30min) \land{} ECG = IDM ST} 

\paragraph{Évolution}
Sd de reperfusion : \dec douleur, négativation ondes T, T = 38
à +6h

Onde Q de nécrose (diagnostic a posteriori)

Marqueur = troponine (ASAP, +6h, +12h), éventuellement myoglobine (rapide++) ou
CPK-MB si récidive

\paragraph{DD} 
Douleur thoracique : péricardite aigüe, EP, dissection aortique, sous-diaphragme (cholécystite aigüe,
ulcère perforé, pancréatite aigüe).

Simule IDM : Penser à mycocardite aigüe (IRM), cardiomypoathie de stresse
(coronarographie)

\paragraph{Complications précoces}
Rythme/conduction : 
\begin{itemize}
  \item rythme ventriculaire : extrasystole < tachycardie < fibrillation
    ventriculaire (FV = plupart des morts subites ! Besoin d'un choc électrique)
  \item supra-ventriculaire : décompensation hémodynamique, accidents emboliques
  \item BAV\footnote{Bloc auricolventriulaire} (transitoire/définitif) ou
    hypervagotonie\footnote{Bradycardie, hypotension} (Ttt : atropine, remplissage
    macromoléculaire)
\end{itemize}

Hémodynamiques
\begin{itemize}
  \item insuf. ventriculaire G : grave, faire échocardio vite (4 stades)
  \item choc cardiogénique : diagnostic si hypotension artérielle mal tolérée,
    ne répond pas au rempilssage macromoléculaire. Souvent OCA + 24/48h.
    Mortalité > 70\%
  \item infarctus ventricule D : hypotension, champs pulmonaires clairs,
    turgescence jugulaire. Regarder dérivations droites (!) : sus-ST.
    Échocardiographie
\end{itemize}
Mécaniques :
\begin{itemize}
  \item rupture paroi libre ventricule G : rapidement fatal
  \item rupture septale : +24-48gh. Échocardiographie doppler. Forte mortalité
  \item insuf mitrale : fuite par prolapsus valvlaire. Ttt chir
\end{itemize}
Thrombotique : thrombus intra-VG, embolies systémique : échocardio. (Thrombose
veineuse, EP)

Péricardituqe : sd inflammatoire, souvent asymptomatique.

Récidive ischémique \thus récidive IDM. Épreuve d'effort à  +5 jours.

\paragraph{Complications tardives}
Péricardite à +3 semaines (sd de Dressler)

Dysfonction ventricule G : scint/échocardio de stress/IRM cardiaque. Évolue en
dilatation VG/anévrisme

Troubles rythmes ventriculaires sévères : défibrillateurs automatique
implantable (DAI)

\subsection{Traitement}
Reperfusion !!
\begin{itemize}
  \item ICP si < 120min. Aspirine, inhib récepteurs P2Y12, anticoagulant
  \item sinon fibrinolyse VI par TNK-tPA
\end{itemize}
Efficacité : reperfusion dans 90min (50\%). Sd reperfusion

Complications : AVC, réocclusion (surtout si ttt antiagrégant interrompu)

\paragraph{Associé}
\begin{itemize}
  \item antalgique
  \item antiagrégant : aspirine et inhib récepteur P2Y12 [\danger clopidogrel
    seulement si fibronolyse]
  \item anticoagulant : bivalirudine si ICP
  \item $\beta$bloquant (avec prudence)
  \item IEC dans 24h
  \item Éplérénone précocement
\end{itemize}

\paragraph{Des complications}
Troubles rythmes ventriculaire : amiodarone

Troubles rythmes supra-ventriculaire : AVK si mal toléré (hémodynamique)

BAV transitoire : atropine. BAV après IDm antérieur : sonde d'entraînement
électrosystoliques.

Insuf ventriculaire G : diurétique, IEC, épléronone

Choc cardiogénique : lutter contre {hypovolémie, troubles rythme}, sidération
(dobutamine). Assistance circulatoire/cardiaque/cardiocirculation,
revascularisation

Mécanique : rupture paroi libre = mortelle, Septable = suture chir. mitrale =
remplacement valvulaire.

\subsection{Suivi}
\begin{itemize}
  \item antiagrégants plaquettaires : aspirine + clopidogrel (sauf si angor
    stable : aspirine)
  \item statines : si SCA/ango stable
  \item $\beta$bloquant : si infarctus
  \item EAC si coronarions post-infact
  \item épléronone : IDM étendu FEVG < 40à%
\end{itemize}
Éventuellement DAI

\section{228 : Douleur thoracique aigüe}%
\label{sec:228_douleur_thoracique_aigue}

\subsection{CAT}
Détresse vitale ?
\begin{itemize}
  \item respi : FR < 10 ou > 30/min, tirage, sueurs, cyanose, $SpO_2$
  \item hémodynamique : arrêt circulatoire, choc, coeur pulmonaire, pouls
    paradoxal
  \item trouble conscience
\end{itemize}

4 urgences vasculaire : PIED (péricardite, infarctus, embolie pulmonaire,
dissection)

Examens : ECG 12 + 5 dérivations, radio poumon, troponinémie

Transfert USIC

\subsection{Urgences}

\paragraph{Sd coronarien aigü}
\begin{itemize}
  \item FR, ATCD
  \item douleur spontanée de repos > 20min : constriction, pesanteur, brûlure,
    rétrosternale, irradie  cou/épaule/avant-bras/tête. \danger présentation
    \textbf{atypique} possible
  \item examen clinique, radio normale
  \item ECG : sus/sous décalage ST
  \item doser myoglobine (< 6h) \lor{} troponine
\end{itemize}

\paragraph{Dissection aortique}
\begin{itemize}
  \item HTA, sd de Marfan, maladie de Turner
  \item Douleur aigüe, prolongée, intense, déchirement, irradie dans dos, descend
    vers lombes
  \item Clinique : $\Delta$PAS > 20mmHg (bras), abolition 1 pouls, souffle
    insuffisance aortique, déficit neuro
  \item ECG : normal \lor{} SCA
  \item Radio : élargissement médiastin
  \item \textit{Échocardio et ETO \lor{} scanner} 
  \item Chir en urgence, contrôle pression artérielle
\end{itemize}

\paragraph{Embolie pulmonaire} Y penser si douleur thoracique, dyspnée, radio
normale \skull
\begin{itemize}
  \item Terrain
  \item 2 tableaux
    \begin{itemize}
      \item infarctus pulmonaire : douleur basithoracique, hémoptysie noire
      \item coeur pulmonaire aigü : dyspnée, défaillance ventriculaire
    \end{itemize}
  \item EC : parfois thrombose veineuse
  \item radio normale
  \item ECG : coeur pulmonaire droit
  \item \textit{D-Ddimère \thus doppler veineux MI, angoscan ou scinti}. HBPM sans
    attendre !
\end{itemize}

\paragraph{Péricardite aigüe}
Tamponnade péricardite = urgence \skull
\begin{itemize}
  \item douleur thoracique, dyspnée, polypnée \thus orthopnée, toux
  \item turgescence jugulaire, reflux hépatojugulaire
  \item Choc : tachycardie, PAS < 90mmHg
  \item Pouls paradoxal
  \item ECG . microvoltage
  \item radio : cardiomégalie
  \item \textit{échocardio}  (compression VG par VD)
\end{itemize}

Péricardite non compliquée (plus bénin) :
\begin{itemize}
  \item terrain
  \item douleur thoracique augmente inspiration, decubitus. Calmée par
    antéflexion
  \item ECGA : sus-ST diffus, sous-PQ, microvoltage
  \item \textit{échocardio, troponine} 
\end{itemize}

\paragraph{Myopéricardite}
Douleur type péricardite mais \textbf{peut simuler SCA} .

Échocarido + (coronarographie normale)

\subsection{Chroniques cardiaque}
Angor stable

Douleur d'angor : d'effort du rétrécissement aortique serré, fonction des
tachycardies chroniques

Douleur d'effort de myoacardiopathie obstructives.

(HTA pulmonaire)

\subsection{Extra-cardiques}
Urgences moyennes : 4 P = \{pneumothorax, pleurésie,
pneumonies, pancréatite\}, ulcère gastrique/duodénale, cholécystite, douleurs
radiculaires











\section{223 : Artériopathie oblitérante (aorte, MI)}%
\label{sec:223_arteriopathie_obliterante_aorte_mi_}
\subsection{AOMI\footnote{Artériopathie oblitérante des membres inférieurs}}

Épidémio : \male > \female. Pic = 60-75 (\male), 70-80 (female). Prévalence :
1-2\%

\paragraph{Clinique}
Classif de Rutherford : 
\begin{enumerate}[label=\Roman*]
  \item asymptomatique 
  \item claudication légère/modérée/sévère
  \item douleur ischémique de repos 
  \item perte de substance faible/majeur(ulcère/gangrène)
\end{enumerate}

Claudication intermittente : douleur "crampe" au mollet après $x$m de marche.
Disparaît en 5min. Sévère si $x < 200$m. \danger{} Sévère $\neq$ symptomatique

Puis au repos : 
\begin{itemize}
  \item douleurs de décubitus : brûlure orteils, avant-pied. Amélioré par
    déclivité
  \item trouble trophiques : peau mince, fragile, perte pilosité. Puis plaies,
    ulcères, gangrène
  \item ischémie permanente : douleur > 10 j, antalgique résisntant. Critique si
    PF\footnote{Pression de perfusion} < 50mmHg (cheville) ou 30mmHg (gros
    orteil) !
\end{itemize}
Physique : 
\begin{itemize}
  \item inspection : pâle, cyanosé. Interdigitaux++)
  \item palpation : froid, douleur à palpation musc si sévère), pouls, temps recoloration cutané, anévrisme
abdo, poplité
  \item auscult : souffle
\end{itemize}
AOMI si IPS\footnote{Index de pression systolique = pression systolique
cheville/bras} < 0.90, sévére si < 0.60

\paragraph{Paraclinique}
\begin{itemize}
  \item Test de marche (6min ou tapris roulant) : -30mmHG \lor{} -20\% évoque AOMI
  \item Transcutané de la $PO_2$ : hypoxie si < 35mmHg, critique si < 10mmHg
  \item écho-doppler artériel des MI
  \item Si revascularisation : angioscanner des MI, angiographie par RM,
    artériographie des MI
\end{itemize}

\paragraph{DD} 
\begin{itemize}
  \item Douleurs hanches : neuro, rhumato, veineuse, musc
  \item Douleurs de décubitus : neuropathie sensorielle, sd régionaux douloureux
    complexes, compression radiculaire
  \item Ulcères : veineux, microcirculatoire, neuropathie, trauma...
\end{itemize}

\paragraph{Étiologie} : atteinte athéromateuse = 95\%. Sinon : arthériopathies
inflammatoires, dysplasie fibromusculaire, coarctaation de l'aorte, atteinte
post-radique ou post-trauma, toxique, gelures, compressions extrinsèques,
atteinte de l'artère poplitée
    
\paragraph{Traitement}
Local : FR, antiagrégant plaquettaire (risque CV), statine (LDL), IEC (PA) $\pm$
$\beta$-bloquants si coronaire

Local : arrêt tabac, marche. Éventuellement statine (périmètre de marche),
prostaglandine (ischémie critique non revasc.)

Revascularisation si ischémie permanente : endovasc. (stent) ou chir (pontage).
Association possible. Parfois endartériectomie ou amputation

\paragraph{Pronostic} : grave, esperance de vie -10 ans

\subsection{Anévrismes}
Dilatation du \diameter{} > 50\%. Artères cérébrales, aorte, artères poplitées, iliaques

\paragraph{Aorte abdominale}
FR : tabac, ATCD familiaux, âge. Risque de rupture > \female. Haut risque CV

90\% des cas : si maladie athéromateuses. Associés à athérosclérose (90\%).
Formes familiales, évolution aortite.

Clinique : 
\begin{itemize}
  \item asymptomatique : dépister si FR
  \item symptomatique : douleur abdo/lobmaire $\pm$ choc hémorragique. Risque de
    rupture imminente \thus scanner en urgence \skull
  \item autre : complication embolique, compression, sd inflammatoire
\end{itemize}
Paraclinique : écho abdo (dépistage), scanner abdo-pelvien ou IRM = réf

\paragraph{PEC}
Asymptomatique : surveillance si \diameter < 50cm sinon chir (pontage) ou
endoprothèse (si haut risque chir)

Symptomatique : \danger anévrisme rompu = urgence chir \danger. Ne pas attendre
résultat

Suivi : écho-doppler si prothèse viasc, scanner/écho si endoprothèse.

\paragraph{Anévrisme poplité}
Asymptomatique : masse battante. Opéré si > 20mm
Au contraire, complication = embolie (ou ischémie)

\subsection{Ischémie aigüe des MI}
\danger urgence vasculaire !

Chronologie : +2h cellules nerveuses, +6h rhabdomyolyse, +24h nécrose. Sd des
loges.

Reperfusion : sd de reperfusion ou troubles métaboliques, insuf. rénale (ou
choc)

\paragraph{Diagnostic} Clinique, ne pas retarder la chirurgie \danger

Douleur brutale, intense, broiement, impotence fonctiennelle. 

Livie, douleur à palpation musc, pouls abolis en aval, anesthésie, paralysise

\paragraph{Étiologie} 2 tableaux (qui peuvent se mélanger) :
\begin{itemize}
  \item thrombose artérielle in situ (surtout AOMI)
  \item embolie sur artères saines (surtout cardiaque : fibrillation atriale)
\end{itemize}
Donc ascultation cardiaque, ECG, palpation abdo, bilan coagulation

Évualuer état général, fonction cardiaque, comorbidité

\paragraph{Traitement}
Médical : HNF, antalgique niveau 3, oxygène, soins locaux.

revascularisation : chirurgie (embolectomie sond Fogarty) voire fibrinolyse
$\pm$ angioplastie, aponévrotomie. Amputation possible

Surveiller acidose métabolique hyperK, insuf rénanel : diurèse, iono, urée,
créat.









\section{231 : Rétrécissement aortique}%
\label{sec:231_retrecissement_aortique}
Obstruction à l'éjection du VG\footnote{Ventricule gauche}, ici au niveau de la
valve aortique

\paragraph{Étiologies} 
\begin{itemize}
  \item bicuspidie < 65 ans
  \item dégénératif après (rarement post-rhumatismal)
\end{itemize}

Physiopatho: \inc pression \thus hypertrophie pariétale. Compense un temps
l'élévation de pression puis dysfonction systolique. Dysfonction diastolique

\paragraph{Complication}
Insuf cardiaque, fibrillation auriculaire, troubles conduction, mort subite++

\subsection{Clinique}

Pronostic vital mis en jeu si symptomes ! \skull

Dyspnée d'effort, angor d'effort, syncope (d'effort ?), (hémorragie digestive)

Examen : 
\begin{itemize}
  \item auscultation : \{souffle mésosystolique éjectionnel, rude, râpeux\},
    abolition B2 si calcifié
  \item frémissement palpatoire (foyer aortique), (choc de point dévié en bas à
    gauche 
\end{itemize}

\subsection{Explorations}
Radio thorax : éventullement dilatation VG, surcharge pulmonaire

ECG : souvent hypertrophie VG, auriculaire G, troubles conduction/rythme

Cathétérisme : pas habituellement mais coronarographie pour pré-op si \male > 40 ans, FR, angor d'effort ou insuf
cardiaque

Scanner cardiaque : pré-op si TAVI (cf supra)

\paragraph{Échocardiographie-doppler transthoracique} : examen clé. Critères
\begin{itemize}
  \item V max > 4m/s
  \item gradient moyen > 40 mmHg
  \item surface aoritque < 1 $\text{cm}^2$
\end{itemize}
Évalue conséquences sur VG, débit cardiaque, pressions droites

Examiner taille aort, valve mitrale, tricuspide

\subsection{Traitement}
\begin{itemize}
  \item valve chirurgical : mécanique si jeune mais AVK à vie, sinon prothèse bio (> 65
    ans)
  \item valvulopastie percutanée abandonnée
  \item implantation percutanée d'une valve aortique (TVAI)
\end{itemize}
Si symptomatique. Sinon test d'effort.

NB : si FE < 35\=, échocardio de stress sous dobutamine pour risque opératoire


\section{231 : Insuffisance mitrale}%
\label{sec:231_insuffisance_mitrale}
Reflux de sang depuis le VG\footnote{Ventricule gauche} vers l'OG\footnote{Oreillette gauche} pendant la systole.

Classif de Carpentier
\begin{enumerate}
  \item valves restent dans le plan de l'anneau (perforations)
  \item au moins une valve hors du plan de l'anneau (prolapsus)
  \item au moins une valve sous le plan de l'anneau
\end{enumerate}

\paragraph{Étiologies}
\begin{itemize}
  \item Rhumatismale (rare) : type III
  \item Dystrophique (fréquente++) : type II. Soit "dégénerescences myxoïdes"
    (trop de tissu, trop de mobilité), soit dégénerescences fibroélastiques
    (rupture de cordage)
  \item Sur endocardite : type I (perforations) ou II (rupture de cordage)
  \item Ischémique : soit aigüe (rupture de pilier, urgence \skull !), soit
    chronique (type III)
  \item Fonctionnelle : souvent une évolution de cardiopathie avec dilatation VG
    et atteinte systolique
\end{itemize}

Causes des insuf. mitrales aigües : rupture de portage ou de pliier, dysfonction
de pilier ischémique, perforation par endocardite.

Tableau hémodynamique \thus urgence vitale \danger 

Conséquences Hémodynamiques : altération contractilité VG (aval), HTAP pouvaint
être importante si aigü (amont)

\subsection{Diagnostic}
\danger{} peut être asymptomatique

Dyspnée : d'effort (lente et prgoressif), de repos, orthopnée, paroxystique
nocturne, OAP

Examen : 
\begin{itemize}
  \item palpation : frémissement systolique apex, (déviation et abissement choc
    de pointe)
  \item auscultation : souffle systolique de régurgitation, en "jet de vapeur",
    souffle holosystaque de B1 à post-B2, irradie vers l'aisselle ou la base
  \item (autres : galop B3, roulement mésodiastolique, éclat B2, souffle
    d'insuf. tricuspide)
  \item poumon : râles de stase
\end{itemize}

Para clinique
\begin{itemize}
  \item ECG longtemps normal. hypertrophie OG, VG, VD, fibrillation atriale, 
  \item Radio thorax : normale si $\le$ modérée. cardiomégalie, dilatation OG,
    HTAP
  \item \textit{ETT et ETO} : référence. Sévérité côté par SOR\footnote{surface
    de l'orifice régurgitant}, VR\footnote{Volume régurgité} (Vérifier la
    tricuspide)
  \item Cathétérisme : coronarographie seulement en pré-op si \male > 40 ans ou
    \female monopausée avec FR
  \item Épreuve d'effort, échocardio d'effort
\end{itemize}

Évolution : si constitué, bien toléré pendant longtemps. Si brutal : évolue vers
oedème pulmonaire. Complication : endocardite infectieuse, fibrillation/flutter
atrial, insuf cardiaque, complications thromboembolique

\subsection{Prolapsus valvulaire mitral}
Primitif ou associé. \female. Formes familiales.

Signes fonctionnels absents ou ceux IM.

Clinique : clic méso-/télosystoliques, souffle d'IM.

Examen : échocardio.

Évolution bénigne ou complication

\subsection{Traitement}
\begin{itemize}
  \item aigüe mal tolérée : chir urgence 
  \item chronique III ou IV symptomatique : chir
  \item chronique III ou IV asymptomatique : chir si retentissement VG ou
    surveillance échodoppler 6 mois (chir si symptômes, retentissement, troubles
    rythmes supraventriculaire)
\end{itemize}

Chirurgie : idéalement plastie reconstructicie, sinon remplacement valvulaire
(mécanique si jeune mais anticoag à vie, bioprothèse si > 65 ans)

Médical :
\begin{itemize}
  \item IM aigüe : ttt OAP ou choc, chir en urgence
  \item poussée insuf. cardiaque : diurétiques de l'anse, vasodilatateurs,
    digilatique (fibrillation atriale), anticoagulant oraux (fibrillation
    atriale)
\end{itemize}


\section{231 : Insuffisance aortique}%
\label{sec:231_insuffisance_aortique}
Régurgitation de sang vers VG en diastole.

\paragraph{Physiopatho} 
\begin{itemize}
  \item Chronique
surcharge volume et pression. Aorte : \inc
PA\footnote{Pression artérielle}
systolique, \dec PA diastoliques
Hypertrophie compense (parfois pendant des années !!) puis fibrèse
\item Aigüe : surtout EI\footnote{Endocardite infectieuse}, surcharge brutale,
  \inc pression puis oedème pulmonaire
\end{itemize}

\subsection{Étiologies}
Chronique :
\begin{itemize}
  \item dystrophique(freq++) : annulo-ectasiante (valves normale mais anneau
    dilaté), sd des valves flasques
  \item EI qui perfore les valvules
  \item malformative (bicuspidie aortique)
  \item rhumatismale (rare)
  \item inflammatoire, infectieuses, médicamenteuse
\end{itemize}
Aigü : EI, dissection aortique, rupture d'anévrisme d'un sinus de Valsalva,
traumatique

Prothèse : désinsertion partielle, dysfonction

\subsection{Clinique}
Fonctionnel : dyspnée d'effort, (angor d'effort ,), insuf cardiqaue (rare,
tardive)

Physique : 
\begin{itemize}
  \item ascult : souffle diastolique++, "doux, lointain, humé, aspiratif",
    souffle systolique éjectionnel d'accompagnement, roulement de Flint
    apexin/galop
  \item palpation : choc de pointe étalé, en bas à gauche
  \item hyperpulsatilité artérielle périphérique (pouls++), \dec PA
    diastolique++
\end{itemize}

ECG : normal ou HVG\footnote{Hypertrophie ventriculaire gauche} diastolique, (ou
HVG systolique)

Radio : \inc index cardiothoracique si volumineuse chronique

\textit{Échocardio-doppler (ETT)} = confirmer, quantifie dilatation VG 

Coronarographie : pré-op, \male > 40 ans ou \female monopausée, FR

IRM/scanner : dimension aorte, surveillance

\paragraph{DD} 
\begin{itemize}
  \item souffle diastolique : insuf pulmonaire
  \item double souffle (rupture sinus Valsalva), souffle continu, frottement
    péricardique
\end{itemize}
\paragraph{Évolution}
\begin{itemize}
  \item Chronique : si volumineuses, sévère dès les symptômes \thus surveillance
\danger, opération même si asympto. \danger dystrophique, bicuspidies
  \item Aigu : OAP, mort subite \thus chir précoce
\end{itemize}
Complications : EI ++, insuf cardiaque (tardive), rupture aortique, (mort subite)
   
\paragraph{Surveillance} 
Chronique : 1-2/an si fuite importante, sinon tous 2-3ans

Aigü : chir rapidement

\subsection{Traitement}
Médical : 
\begin{itemize}
  \item si volumineuse et IVG : IEC, diurétique et chir rapidement
  \item dilatation de l'aorte : beta-bloquant, losartan
\end{itemize}
Hygiène dentaire, examen tous 6 mois pour prévenir EI

Chirurgie : 
\begin{itemize}
  \item remplacement valvulaire si IA isolée
  \item valve + aorte si dystrophique ou (bicuspidie \land{} dilatation aortique)
\end{itemize}

Quand faire la chir ?
\begin{itemize}
  \item chronique volumineuse
    \begin{itemize}
      \item  symptomatique : urgent \danger
      \item asymptomatique : FEVG < 50\%, dilatation aorte $\ge 55$mm, diamètre
        VG télédiastolique > 70mm, télésystolique > 50mm
    \end{itemize}
  \item dystrophique et dilatation aorte asc : dès $\ge 55$mm
  \item aigüe volumineuse : urgence
\end{itemize}

\section{150 : Surveillance des porteurs de valves, prothèses vasculaires}%
\label{sec:150_surveillance_des_porteurs_de_valves_protheses_vasculaires}

\begin{itemize}
  \item Prothèses mécaniques : double ailette, à vie, anticoagulant à vie
    (risque thrombose)
  \item Biologiques : pas d'anticoagulant, 40\% à 15 ans.
\end{itemize}
Risque majeur d'EI $\forall$ prothèse !

\subsection{Complications}
\begin{itemize}
  \item Thromboemboliques (freq++) : surtout mécanique, surtout prothèses mitrale,
    anciennes, fibrillation atriale
    \begin{itemize}
      \item Embolie systémiques : souvent cérébrales
      \item Thromboses de prothèse mécanique : accidents brutaux (OAP, syncope,
        choc, mort subite). Diagnosic difficile : appartition d'un
        souffle/roulement. \textit{ETT, ETO} \\
        Chir d'urgence possible
        \danger DD avec EI parfois difficile
    \end{itemize}
  \item Désinsertions de prothèses (5\%) : spontané, EI. À évoquer si apparition d'un
    souffle, anémie hémolytique, insuf. cardiaque. Confirmé par ETT, ETO(++)

  \item Infectieuses
    \begin{itemize}
      \item médiastinie post-op (1\%)
      \item Endocardite infectieuses : \textbf{redoutable} \skull\\
        Précoce (50\%) ou tardive. Diagnostic : ETT, ETO++\\
        Prévention/traitement de tout foyer infectieux (ORL, dentaire)\\
        Hémocultures systémiques devant fièvre inexpliquée
    \end{itemize}
  \item Traitement anticoagulant : 1.2\% patiens-années risque hémorragique
  \item Dégéneresce bioprothèses
\end{itemize}

\subsection{Surveillance}
Post-op : AVK (à vie si mécanique, 3 mois si bio). ETT à +3mois (référence !)

Puis : 1/mois puis tous les 3 mois. Cardiologue à +3 mois puis 1-2/an.

Clinique : 
\begin{itemize}
  \item surveiller symptômes, dyspnée, insuf cardiaque
  \item ascult : attention à \dec intensité bruits (ou variables), \inc
    intensité d'un souffle, bruit diastolique surajouté
\end{itemize}
Radio, ECG mais surtout ETT, ETO

\textit{Biologie} ++ : équilibre AVK parfait, à vie \thus INR tous les mois $\in
[2.5, 4]$.

FR : valve non aortique, ATCD, fibrillation atriale, \diameter OG > 50mm,
contraste spontané dense OG, sténose mitrale, FE < 35\%, hypercoagulabilité

Ne pas interrompre AVK sauf pronostic vital !. Si chir extracardiaque : HNF
pendant l'arrêt AVK

\section{149 : Endocardite infectieuse}%
\label{sec:149_endocardite_infectieuse}
Infections des valves cardiaque ou de l'endocarde pariétal. Dominées par les
staphylocoques

\subsection{Physiopatho}
Bactéries adhèrent sur une lésion préexistante \thus
\begin{itemize}
  \item insuffisance valvulaire, souffle, risque de défaillance cardiaque
  \item végétations \thus embolies septiques, lésions de vasculairet, anévrisme
    "mycotique"
\end{itemize}

Cardiopathies à haut risque : prothèses valvilaire, cardiopathies congénitale
cyanogiène, ATCD EI

50\% des EI sur coeur présumé sain !

Hémocultures positives (90\%)
\begin{itemize}
  \item streptocoques oraux, streptocoques du groupe D
  \item staphylocoques : blanc, coagulase négative
\end{itemize}
Hémocultures négatives :
\begin{itemize}
   \item ATB
   \item croissante lente : HACEK, Brucella, champignons
   \item intra-cellulaire : \bact{burnetii}, Chlamydia, Bartonella,
     \bact{whipplei}
\end{itemize}

\subsection{Clinique}
\danger Manif trompeuses. Y penser si souffle cardiaque et fébrile, AVC,
purpura, lombalgies féribles

\begin{itemize}
  \item Sd infectieux : fièvre, AEG, splénomégalie
  \item Apparition/modif souffle, insuf cardiaque
  \item cutané (nodosité d'Osler !), respi, ophtalmo, rhumato(freq), neuro,
    rénale
\end{itemize}

Diagnosic : hémoculture, échocardio

Autres : NFS, \{CRP, électrophorèse\}, complexes immuns circulants, \{urée,
créat\}, BNP

Classif de Duke : 2 majeurs ou (1 majeur et 3 mineurs) ou (5 mineurs)
\begin{itemize}
  \item majeurs
    \begin{itemize}
      \item Hémocultures : micro-org typique, HC $\ge 2$ \lor{} positive > 12h
        \lor{} positive à \bact{burnetii}
      \item Échocardio avec végétation, abcès, désinsertion prothétique \lor{} nouveau souffle de régurgitation valvulaire
    \end{itemize}
  \item mineurs
    \begin{itemize}
      \item cardiopathie à risque/toxicomanie
      \item $\ge 38^{\circ}$
      \item comilcation vasc
      \item immunologique
      \item hémoc/séro positive
    \end{itemize}
\end{itemize}

\paragraph{Évolution}
Complications : insuf cardiaque (num. 1 DC), neuro (num 2), embolies (septiques,
cérébrales, splénique, rénales, coronaires), infarctus spnélique, arythmies et
troubles de conduction

Penser à scanner cérébral et abdo-pelvien !

Pronostic : 
\begin{itemize}
  \item sur aortique : chir
  \item staph ou prothèse : mortalité++
  \item pneumocoque, bacciles Gram négatif : destruction valvulaire graves
  \item levure : grosse végétations
\end{itemize}

\subsection{Traitement}
Bithérapie IV

Fonction rénale pour aminosides et vancomycine !

\begin{itemize}
  \item Strepto oraux/groupe D : amoxicilline et gentamicine (2 semaine bi, 4
    semaines mono) [vancomycine + gentamicine si allergie]
  \item entérocoques : idem
  \item staph : si sensible : cloxacilline (+gentamicine + rifampicine si sur
    prothèse). Sinon vancomycine (+gentamicine + rifampicine)
  \item hémoc négative : en attendant amox + acide clavilanique + gentamicine
\end{itemize}

Chirurgie : valvie native si possible. Intervention si insuf cardiaque ou sd
infectieux non contrôlé

\paragraph{Prévention}
Hémoc avant antio \danger

ATBprophylaxie : amoxicilline (clindamycine si allergie) avant geste (région
apical/gingivale, perforation muqueuse orale ou (extraction dentaire et haut
risque))

\section{236 : Souffle cardiaque chez l'enfant}%
\label{sec:236_souffle_cardiaque_chez_l_enfant}
Très fréquent.

Malformation congénitale (1\%), souffle fonction, cardiomyopathie/myocardite
aigüe (rarement), acquises (exceptionnelles)

\paragraph{Auscultation chez l'enfant} Rythme rapide, irrégulier.

B2 dédoublé : anormal si large et fixe.

Éclat B2 : HTA pulmonaire, malposition des gros vaisseaux

B3 physiologique (apex)

Clic possible

\subsection{Clinique}
Fonctionnel : souvent absent, dyspnée d'effort. \danger douleur thoraciques =
rarement cardiaques !

Souffle :
\begin{itemize}
  \item varie en temps et position : innocent
  \item bruyant, irradiant largement : organique
  \item diastolique : organique
  \item frémissant : organique
  \item holosystolique, de régurgitation : organique
  \item cou et sus-sternal : aortique ; dos : pulmonaire ;kirradiant :
    communication intra-V
\end{itemize}
Associés : 
\begin{itemize}
  \item regarder $SaO_2$
  \item troubles alimentaires, dyspnée, sueur, retard staturopondéral : large
    shunt
  \item HTA, pas de pouls fémoraux : coarctation aortique
\end{itemize}

\paragraph{Complémentaire}
Radio thorax : cardiomégalie (\danger "fausses")
\begin{itemize}
  \item saillie arc moyen G : shunt gauche-droite
  \item arc moyen G conctave : hypoplasie voie pulmonaire
\end{itemize}

ECG : fréquence diminue avec l'âge. T < 0 en $V_1$ $V_4$

\textit{Échocardio} = examen clé

Autres : effort, holter ECG, IRM cardiaque, scanner multibarettes, cathétérisme
cardiaque (rare)

\subsection{Cardiopathies}
\paragraph{Naissance, +2 mois}
\begin{itemize}
  \item Souffle isolé : examen clinique, ECG, radio pulmonaire, échocardio
  \item Insuf cardiaque : coarctation préductale \thus chir urgente
  \item Cyanose : transposition des gros vaissaux \thus chir avant N+15 jours
\end{itemize}

\paragraph{N+2 mois, marche}
\begin{itemize}
  \item Insuf cardiaque : shunts gauche-droite surtout (\thus opérer avant 1 an
    si large !!), communication intra-V
    large, persistance canal artériel, canal atrioventriculaire
  \item cyanose : tétralogie de Fallot\footnote{Communication intra-V,
      hypertrophie ventriculaire d, sténose pulmonaire, dextroposition de
    l'aorte}
\end{itemize}
\paragraph{2 à 16 ans}
\begin{itemize}
  \item Malformatives : raraes, bien tolérées
  \item Souffles "innocents" (1/3) : asymptomatique, systolique, éjectionnels,
    faible intensité, (intensité varie avec position), doux. Ne rien faire
\end{itemize}




\end{document}
