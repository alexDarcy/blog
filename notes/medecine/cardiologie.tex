\input header

\begin{document}
\title{Fiches de cardiologie}
\author{Alexis Praga}
\maketitle
\tableofcontents

\section{218 : Athérome}%
\label{sec:1_atherome}

Épidémio : 1ere cause de mortalité dans le monde. 

En France : incidence \male = 5$\times$\female. 

Mortalité $\searrow$ mais prévalence $\nearrow$

\subsection{Mécanisme}
Contient centre lipiqude, cellules {spumeuses,muscularise, inflammatoire} +
chape fibreuse

Évolution de la plaque :
\begin{itemize}
  \item rupture (plus probable si plaque jeune !)
  \item progression par poussées
  \item hémorragie intraplaque
  \item régression ?
\end{itemize}
Remodelage

Anévrismes

\paragraph{Localisations}
Surtout : carotides (AVC), coronaires (cardiopathies ischémiques), membre inférieure
(AOMI)

\paragraph{Évolution} Aggravation par étapes silencieuses. \danger gravité pas
toujours proportionnelle à l'ancienneté/étendue

FDR : tabagisme, HTA, dyslipidémie, diabète

\paragraph{Thérapeutiques}
Prévention du développement de l'athérome : diminuer lésion endothéliale,
diminuer accumulation LDL, stabiliser plaques, diminuer volume des plaques,
diminuer l'inflammation, diminuer les contraintes mécaniques

\subsection{Polyathéromateux}

$\ge 2$ territoire artériels différents

Évaluer FdR, bilan des lésions

Thérapeutiques :
\begin{itemize}
  \item arrêt tabac, diététique, activité physique
  \item aspirine en systématique (colpidogrel si intolérance)
  \item statines en prévention secondaire
  \item IEC\footnote{Inhibiteurs de l'enzyme de conversion}, ARA
      II\footnote{antagonistes des récepteurs de l'angiotensine}
\end{itemize}

PEC spécifique : chirurgie anévrisme ($\diameter \ge 5.5cm$), endartériectomie
(sténose carotide > 60\%), revasc. myocardique (sd coronaire aigü $\wedge$
sténose coronaires > 70\%)

\section{219 : Facteur de risques cardio-vasculaires}%
\label{sec:219_facteur_de_risques_cardio_vasculaires}

Facteur de risque (FR) : causalité avec la maladie $\neq$ marqueur de risque
(simple témoin)

\subsection{FR}
Non modifiables : 10 ans + tôt chez \male, hérédité = plutôto environnement
familial

Modifiables : 
\begin{itemize}
  \item risque : {tabagisme, hypercholestérolémie, HTA, diabète, obésité abdo,
    psychosociaux}
  \item protecteur : {fruit et légumes, activité physique, alcool modéré}
\end{itemize}

\paragraph{Tabac}
1ère cause de mortalité évitable.

Conséquence : \dec HDL, \inc risque thrombose, altère vasomotricité artérielles,
\inc [CO]

2eme FR de l'IDM : $\propto$ consommation, $\forall$ tabac, jeunes, passif

Rôle : AOMI, anévrisme aorte abdo, AVC

\paragraph{Hypercholestérolémie}
3eme FR IDM : \inc LDL et \dec HDL = mauvais signe $\implies$ exploration d'une
anomalie lipidique à jeun

Majorité = alimentaire mais génétique possible (hétérozygote/homozygote)

\paragraph{HTA}
Stade 1 : [140-159]/[90-99] mmHg
Stade 2 : [160-179]/[100-109] mmHg
Stade 3 : > 180/110 mmHg

Silencieuse. Impact coeur (insuf. coronire, cardiaque), cerveau (AVC), rein (IR)

Augmente avec l'âge.

3 mesure espaces d'1 semaine

\paragraph{Diabète}
90\% de diabète 2 (résistance insuline). Déf :
\begin{itemize}
  \item diabète si glycémie à jeun > 1.26g/L
  \item hyperglycémie non diab : glycémie jeun $\in [1.10, 1.26]$ g/L
  \item intolérance hydrates de carbones : < 1.26 (jeun), $\ge 2$ (provoquée)
    puis $\in [1.40, 2]$
\end{itemize}
Hérédité. Complications microvasc, macrovasc

\paragraph{Surpoids}
IMC $\in [25, 29.9]$ = surpoids, IMC $\ge 30$ = obésité. 

Obésité centrale = (\diameter abdo $\ge 94 $cm (\male) ou $\ge 80$cm (\female))
\land 2 FR


\subsection{Évaluation}
\label{subsec:fr}
Score
\begin{itemize}
  \item +1 si {tabac $\le 3$ ans, LDL > 1.6g/L, HTA, diabète, HDL < 0.40g/L, âge > 50
(\male) ou 60 (\female), ATCD coronaires}
  \item  -1 si HDL $\ge 0.60$
\end{itemize}

ATD personnels CV

\subsection{Prévention}
\paragraph{Secondaire}

BASIC : $\beta$bloquants, Antiagrégants, Statine, Inhibiteurs de l'enzyme de
conversion, Contrôle des FR

\begin{itemize}
  \item statine pour LDL < 1g/L
\item sevrage tabac : substituts nicotinique, {bupropion, varénicline},
  anxiété/dépression, TCG
  \item pression artérielle : hygiénodiététique (échec à 3 mois : médic)
  \item contrôle glycémie (diabète)
  \item activité physique régulière
  \item enquête familiale
\end{itemize}

\paragraph{Primaire}
Cholestérol : 0FR : LDL < 2.20, 1FR : LDl < 1.90, 2FR : LDl < 1.60, $\ge 3$ FR
: LDL < 1.30 (haut risque : LDL < 1g/L)







\section{220 : Dyslipidémies}%
\label{sec:220_dyslipidemies}

Risques : maladies CV athéromateuses

LDL = total - HDL - TG\footnote{tryglycérides}

Bilan normal : \begin{itemize}
  \item LDL  < 1.6g/L
  \item HDL  > 0.4g/L
  \item TG  < 1.5g/L
\end{itemize}

\paragraph{Hyperlipidémies secondaire } hypothyroïdie, cholestase, sd néphrotique, IR chronique, alcoolisme,
diabète, hyperlipidémie iatrogènee, oestrogènes, corticoïdes, rétinoïdes,
antirétroviraux, ciclosporine, diurétiques

\paragraph{Hyperlipidémies primitives}
\textit{Fréquentes}  : hypercholestérolémie familiale monogénique (HCFM) (mutation LDL récepteur
hétérozygote), hypercholestérolémie polygénique, hyperlipidémie familiale
combinée

\textit{Rares}  : HCFM (mutation apolipoprotéine B),
dysbêtalipoprotéinémie, hypertriglycéridémie familiale, hypechylomicronémie
primitives

\paragraph{Risque} faible (0 FR), intermédiarie ($\ge 1$ FR), haut (ATCD)

FR semblables au~\hyperref[subsec:fr]{score précédent} : tabac $\le 3$ ans, HTA, diabète, HDL < 0.40g/L, âge > 50
(\male) ou 60 (\female), ATCD familaux IDM ou mort subite

\subsection{Traitement}

\paragraph{Diététique}
\begin{itemize}
  \item lipides < 40\%
  \item graisses saturées < 12\%
  \item plutôt mono- et polyinsaturées
  \item cholestérol alimentaire < 300mg/j
  \item 5 fruits ou légumes/j
  \item sodium < 6g/j
  \item diminuer excès pondéral
\end{itemize}
Si hypertriglycéridémie (HTG) : \dec poids, alcool, sucres simples

\paragraph{Médicaments}
Primaire : seulement +3 mois après diététique. Secondaire : d'emblée.

\danger pas de statine si grossesse

Hypercholestérolémie, hyperlipidémie mixtes : statines (1er)

HTG : fibrate si TG > 4g/L, diététique sinon



\end{document}

