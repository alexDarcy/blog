\input header

\begin{document}
\title{Fiches de cardiologie}
\author{Alexis Praga}
\maketitle

\section{218 : Athérome}%
\label{sec:1_atherome}

Épidémio : 1ere cause de mortalité dans le monde. 

En France : incidence \male = 5$\times$\female. 

Mortalité $\searrow$ mais prévalence $\nearrow$

\subsection{Mécanisme}
Contient centre lipiqude, cellules {spumeuses,muscularise, inflammatoire} +
chape fibreuse

Évolution de la plaque :
\begin{itemize}
  \item rupture (plus probable si plaque jeune !)
  \item progression par poussées
  \item hémorragie intraplaque
  \item régression ?
\end{itemize}
Remodelage

Anévrismes

\paragraph{Localisations}
Surtout : carotides (AVC), coronaires (cardiopathies ischémiques), membre inférieure
(AOMI)

\paragraph{Évolution} Aggravation par étapes silencieuses. \danger gravité pas
toujours proportionnelle à l'ancienneté/étendue

FDR : tabagisme, HTA, dyslipidémie, diabète

\paragraph{Thérapeutiques}
Prévention du développement de l'athérome : diminuer lésion endothéliale,
diminuer accumulation LDL, stabiliser plaques, diminuer volume des plaques,
diminuer l'inflammation, diminuer les contraintes mécaniques

\subsection{Polyathéromateux}

$\ge 2$ territoire artériels différents

Évaluer FdR, bilan des lésions

Thérapeutiques :
\begin{itemize}
  \item arrêt tabac, diététique, activité physique
  \item aspirine en systématique (colpidogrel si intolérance)
  \item statines en prévention secondaire
  \item IEC\footnote{Inhibiteurs de l'enzyme de conversion}, ARA
      II\footnote{antagonistes des récepteurs de l'angiotensine}
\end{itemize}

PEC spécifique : chirurgie anévrisme ($\diameter \ge 5.5cm$), endartériectomie
(sténose carotide > 60\%), revasc. myocardique (sd coronaire aigü $\wedge$
sténose coronaires > 70\%)

\end{document}
