\documentclass{article}
\input header
\def\arrow{$\rightarrow$}
\hypersetup{
  colorlinks = true,
  linkcolor=teal
}
\usepackage[linesnumbered,ruled,vlined]{algorithm2e}
\usepackage[acronym, nonumberlist]{glossaries}

\makeglossaries

\newacronym{IEC}{IEC}{Inhibiteurs de l'enzyme de conversion}
\newacronym{ARA II}{ARA II}{Antagonistes des récepteurs de l'angiotensine}
\newacronym{TG}{TG}{Tryglycérides}
\newacronym{BBG}{BBG}{Bloc de branche gauche}
\newacronym{IVA}{IVA}{Artère intraventriculaire antérieure}
\newacronym{BAV}{BAV}{Bloc auriculoventriculaire}
\newacronym{IPC}{IPC}{Intervention coronaire percutanée}

\glsaddall % Add all entries
\glsunsetall % Acronyms always in short form

\begin{document}
\title{Fiches de cardiologie}
\author{Alexis Praga}
\maketitle
\tableofcontents

\input bacteries-header

\section{218 : Athérome}%
\label{sec:1_atherome}

Épidémio : 1ere cause de mortalité dans le monde. 

En France : incidence \male = 5$\times$\female. 

Mortalité $\searrow$ mais prévalence $\nearrow$

\subsection{Mécanisme}
Contient centre lipidique, cellules \{spumeuses, inflammatoire\} +
chape fibreuse + support musculaire (migration vers l'endothelium)

Évolution de la plaque :
\begin{itemize}
  \item rupture (plus probable si plaque jeune !)
  \item progression par poussées
  \item hémorragie intraplaque
  \item régression ?
\end{itemize}
Remodelage

Anévrismes

\paragraph{Localisations}
Surtout : carotides (AVC), coronaires (cardiopathies ischémiques), membre inférieure
(AOMI)

\paragraph{Évolution} Aggravation par étapes silencieuses. \danger gravité pas
toujours proportionnelle à l'ancienneté/étendue

FDR : tabagisme, HTA, dyslipidémie, diabète

\paragraph{Thérapeutiques}
Prévention du développement de l'athérome : diminuer lésion endothéliale,
$\searrow$ accumulation LDL, stabiliser plaques, $\searrow$ volume des plaques,
$\searrow$ l'inflammation, $\searrow$ les contraintes mécaniques

\subsection{Polyathéromateux}

$\ge 2$ territoire artériels différents

Évaluer FdR, bilan des lésions

Thérapeutiques :
\begin{itemize}
  \item arrêt tabac, diététique, activité physique
  \item aspirine en systématique (colpidogrel si intolérance)
  \item statines en prévention secondaire
  \item \gls{IEC}, \gls{ARA II}
\end{itemize}

PEC spécifique : chirurgie anévrisme ($\diameter \ge 5.5cm$), endartériectomie
(sténose carotide > 60\%), revasc. myocardique (sd coronaire aigü $\wedge$
sténose coronaires > 70\%)

\section{219 : Facteur de risques cardio-vasculaires}%
\label{sec:219_facteur_de_risques_cardio_vasculaires}

Facteur de risque (FR) : causalité avec la maladie $\neq$ marqueur de risque
(simple témoin)

\subsection{FR}
Non modifiables : 10 ans + tôt chez \male, hérédité = plutôt environnement
familial

Modifiables : 
\begin{itemize}
  \item risque : {tabagisme, hypercholestérolémie, HTA, diabète, obésité abdo,
    psychosociaux}
  \item protecteur : {fruit et légumes, activité physique, alcool modéré}
\end{itemize}

\paragraph{Tabac}
1ère cause de mortalité évitable.

Conséquence : \dec HDL, \inc risque thrombose, altère vasomotricité artérielles,
\inc [CO]

2eme FR de l'IDM : $\propto$ consommation, $\forall$ tabac, sujet jeune, tabagisme passif

Rôle : AOMI, anévrisme aorte abdo, AVC

\paragraph{Hypercholestérolémie}
3eme FR IDM : \inc LDL et \dec HDL = mauvais signe $\implies$ exploration d'une
anomalie lipidique à jeun

Majorité = alimentaire mais génétique possible (hétérozygote/homozygote)

\paragraph{HTA}
Voir table~\ref{tab:hta_stades}.
\begin{table}
  \centering
  \begin{tabular}{cc}
      Stade 1 & [140-159]/[90-99] mmHg\\
Stade 2 & [160-179]/[100-109] mmHg\\
Stade 3 & > 180/110 mmHg
  \end{tabular}
  \caption{Stades d'HTA}
  \label{tab:hta_stades}
\end{table}

Silencieuse. Impact c\oe{}ur (insuf. coronaire, cardiaque), cerveau (AVC), rein (IR)

Augmente avec l'âge.

3 mesure espacées d'1 semaine

\paragraph{Diabète}
90\% de diabète 2 (résistance insuline). Déf :
\begin{itemize}
  \item diabète si glycémie à jeun > 1.26g/L
  \item hyperglycémie non diab : glycémie jeun $\in [1.10, 1.26]$ g/L
  \item intolérance hydrates de carbones : < 1.26 (jeun), $\ge 2$ (provoquée)
    puis $\in [1.40, 2]$
\end{itemize}
Hérédité. Complications microvasc, macrovasc

\paragraph{Surpoids}
IMC $\in [25, 29.9]$ = surpoids, IMC $\ge 30$ = obésité. 

Obésité centrale = (\diameter{} abdo $\ge 94 $cm (\male) ou $\ge 80$cm (\female))
\land{} 2 FR


\subsection{Évaluation}
\label{subsec:fr}
Score
\begin{itemize}
  \item +1 si {arrêt tabac $\le 3$ ans, LDL > 1.6g/L, HTA, diabète, HDL < 0.40g/L, âge > 50
(\male) ou 60 (\female), ATCD coronaires}
  \item  -1 si HDL $\ge 0.60$
\end{itemize}

ATD personnels CV

\subsection{Prévention}
\paragraph{Secondaire}

BASIC : $\beta$bloquants, Antiagrégants, Statine, Inhibiteurs de l'enzyme de
conversion, Contrôle des FR

\begin{itemize}
  \item statine pour LDL < 1g/L
\item sevrage tabac : substituts nicotinique, {bupropion, varénicline},
  anxiété/dépression, TCG
  \item pression artérielle : hygiénodiététique (échec à 3 mois : médic)
  \item contrôle glycémie (diabète)
  \item activité physique régulière
  \item enquête familiale
\end{itemize}

\paragraph{Primaire}
Voir table~\ref{tab:cholestérol}.

\begin{table}
  \centering
  \begin{tabular}{cc}
    0 FR & LDL < 2.20 \\
    1 FR & LDL < 1.90\\
    2 FR & LDL < 1.60\\
    $\ge 3$ FR& LDL < 1.30 \\
    haut risque & LDL < 1g/L
  \end{tabular}
  \caption{FR du cholestérol}
  \label{tab:cholesterol}
\end{table}
\section{220 : Dyslipidémies}%
\label{sec:220_dyslipidemies}

Risques : maladies CV athéromateuses

LDL = total - HDL - \gls{TG}

Bilan normal : \begin{itemize}
  \item LDL  < 1.6g/L
  \item HDL  > 0.4g/L
  \item TG  < 1.5g/L
\end{itemize}

\paragraph{Hyperlipidémies secondaire } hypothyroïdie, cholestase, sd néphrotique, IR chronique, alcoolisme,
diabète, hyperlipidémie iatrogènee, oestrogènes, corticoïdes, rétinoïdes,
antirétroviraux, ciclosporine, diurétiques

\paragraph{Hyperlipidémies primitives}\mbox{}\\
\textit{Fréquentes}  : hypercholestérolémie familiale monogénique (HCFM) (mutation LDL récepteur
hétérozygote), hypercholestérolémie polygénique, hyperlipidémie familiale
combinée

\textit{Rares}  : HCFM (mutation apolipoprotéine B),
dysbêtalipoprotéinémie, hypertriglycéridémie familiale, hypechylomicronémie
primitives

\paragraph{Risque} faible (0 FR), intermédiaire ($\ge 1$ FR), haut (ATCD)

FR semblables au~\hyperref[subsec:fr]{score précédent} : tabac $\le 3$ ans, HTA, diabète, HDL < 0.40g/L, âge > 50
(\male) ou 60 (\female), ATCD familiaux IDM ou mort subite

\subsection{Traitement}

\paragraph{Diététique}
\begin{itemize}
  \item lipides < 40\%
  \item graisses saturées < 12\%
  \item plutôt mono- et polyinsaturées
  \item cholestérol alimentaire < 300mg/j
  \item 5 fruits ou légumes/j
  \item sodium < 6g/j
  \item diminuer excès pondéral
\end{itemize}
Si hypertriglycéridémie (HTG) : \dec poids, alcool, sucres simples

\paragraph{Médicaments}
Primaire : seulement +3 mois après diététique. Secondaire : d'emblée.

\danger pas de statine si grossesse

Hypercholestérolémie, hyperlipidémie mixtes : statines (1er)

HTG : fibrate si TG > 4g/L, diététique sinon



\section{334 : Syndromes coronariens aigüs}%
\label{sec:334_syndromes_coronariens_aigus}

Sd coronaire aigü (SCA) : lésions athérothrombotiques aigües

Angor stable à l'effort : lésions fibro-athéromateuses

\subsection{Angine de poitrine (angor) stable}
Ici : pas de thrombus

Inadéquation besoin/apport $O_2$ : 95\% sténoses athéromateuses coronariennes
serrées (parfois : spasme coronaire, \inc besoins, "à coronaires saines")

Donc le myocarde s'adapte en vasodilatant (pour apport $O_2$)\footnote{Surtout par
sécrétion monoxyde d'azode par l'endothélium}

Donc cascade ischémique : \dec perfusion myocarde [scinti] \thus altération
contractilité [écho stress] \thus signes ECG \thus douleur (pas toujours)

Athérome : risque = fracture de plaque \thus (thrombose) mort subite/IDM, angor
instable

\paragraph{Diagnostic}
Douleur angineuse\footnote{Classes de 1 à 4}
\begin{itemize}
  \item typique : rétrosternal en barre horizontale, irradiant (épaules,
    avant-bras, poignet, machoîres), constrictive, angoissante, \textbf{à
    l'effort}, \underline{sensible à trinitrine}
  \item atypique ou silencieuse possible
\end{itemize}
Exaen clinique souvent négatif mais chercher souffle aortique, souffle vasc, HTA

\paragraph{Examens}
\begin{itemize}
  \item ECG : intercritique = normal, percritique : (sus/sous)-décalage
    ST, ondes T (négatives symétriques, amples positives symétrique)
  \item ECG d'effort : \textit{1ere intention} . Positive si douleur thoracique \lor{}
    sous-decalage ST
  \item Tomoscintigraphie myocardique de perfusion d'effort ou injection
    vasodilatateur (dipyridamole) : segment normal/ischémie/nécrotique.
    \textit{Lorsque VP ECG insuffisante}. Coûteux. Éviter si \gls{BBG}
  \item Échocardiographie d'effort ou dobutamine. \textit{Mêmes indication que
    scinti} 
  \item IRM de stress : rare
  \item Coronarographie (parfois + ventriculographie) : sténose si > 70\%
    lumière. Invasif, complications rare. \textit{Si angor suspecté et examen
    d'ischémie positif}\footnote{Examens complémentaires : test Méthergin pour
      forcer un spasme, FFR (fraction flow reserve) pour vérifier sténose}
  \item Scanner coronaire : non recommandé
\end{itemize}

\danger CI des épreuves de stress : angor instable, troubles rythme ventriculaire
graves, fibrillation auriculaire rapide, HTA repos > 220/120mmHg

\paragraph{Mauvais pronostic} : 
\begin{itemize}
  \item angor classe 3/4
  \item ischémie pour charge/fréquence cardiaque faible, baisse PA à l'effort
  \item plusieurs segments ischémique, fraction d'éjection <
    40\%\footnote{Normale si > 55\%}
  \item lésions pluritronculaires, tronc coronaire G, \gls{IVA} proximale
\end{itemize}

\paragraph{PEC}
\begin{enumerate}
\item Crise : arrêt effort, dérivés nitrés.
\item Correction FR (tabac, hypolipides, activité physique, HTA, diabètes, statine,
IEC)
\item aspirine\footnote{Anti-agrégant plaquettaire} 75mg/j (ou
  clopidogrel\footnote{Idem mais inhibiteur des récepteurs P2Y12} 75mg) +
  $\beta$-bloquant (anticalcique/ivabradine si intolérance) $\pm$ {dérivés nitrés, molsidomine,
nicorandil}
\item Revascularisation si échec médicament ou pour améliorer le pronostic vital
  : \gls{IPC} (stent) ou pontage coronaire
\end{enumerate}

\paragraph{Angor de Prinzmetal} Vasospastique = douleur sensible à la trinitrine
et, soit:
\begin{itemize}
  \item au repos, 2eme partie de nuit, récupération = angor de Prinzmetal
  \item sur un effort = angor surimposé à une sténose
\end{itemize}
Diagnostic : coronarographie \thus test provocation spasme (pendant coronaro)

Ttt : inhibiteurs calcique (2 molécules).
Bon pronostic si traité

\subsection{SCA sans sus-décalage ST}

= \{angor instable, IDM sans sus-décalage ST persistant \}. Ici thrombus non
occlusif

\begin{figure}[htpb]
  \centering
  \resizebox{0.6\linewidth}{!}{
    \tikz \graph [
    % Labels at the middle 
    edge quotes mid,
    % Needed for multi-lines
    nodes={align=center},
    sibling distance=3cm,
    layer distance=2cm,
    edges={nodes={fill=white}}, 
    layered layout]
    {
      "SCA sans sus-décalage ST" ->{
        Angor instable [>"tropo=0"];
        "IDM ST-"[>"tropo +"];
      };
      "SCA avec sus-décalage ST" -> "IDM ST+"[>"tropo +"];
    };
  }
  \caption{Classification des SCA (hor}
\end{figure}

\paragraph{Diagnostic}
Même douleur que l'angor stable mais 
\begin{itemize}
  \item \textbf{spontané > 20min}, régressant spontanément ou non à trinitrine
  \item angor d'effort récent (2-3)
  \item aggravation d'un angor stable
  \item IDM + 1mois
\end{itemize}
Examen clinique normal mais chercher râles crépitants, galop

ECG en urgence \skull puis +6h
\begin{itemize}
  \item percritique : sous-décalage ST (rarement sus), (grandes T négatives \lor{}
    repositivation T). Si normal, diagnostic peu probable
  \item post-critique (être très prudent !) : sous-décalage ST, T négative
    profonde
\end{itemize}

\paragraph{PEC}
\begin{enumerate}
\item USIC en urgence ! Avec ECG, dosage troponine, créatinine, glycémie, NFS
  \begin{itemize}
  \item aspirine
  \item +
    $\begin{cases}
      \text{clopidogrel + fondaparinux si bas risque}\footnotemark\\
      \addtocounter{footnote}{-1}
\text{ticagrelor/prasugrel + HNF/HBPM (+ anti-GPIIb/IIIa) si haut risque}\footnotemark

\end{cases}$
\footnotetext{Anti-agrégant plaquettaire et anticoagulant respectivement}
  \item + $\beta$-bloquant + statine $\pm$ dérivé nitré $\pm$ inhibiteur
    calcique\footnote{Anti-ischémiques}
  \end{itemize}
\item si (risque élevé et Grace > 140) ou (risque faible mais élevé à +6/12h) :
  poursuite médic + coronarographie + angioplastie
\item sinon, tests non invasifs
\end{enumerate}

Notes :
\begin{itemize}
\item Doser troponine ssi suspicion !
\item Échocardiographie pour DD
\item Coronarographie suivant le risque :
\begin{itemize}
  \item très haut risque : en urgence !
  \item haut risque : < 24h (score GRACE > 140) ou < 72 (GRACE $\in [109, 140]$)
  item bas risque (GRACE < 109)  à discuter 
\end{itemize}
\end{itemize}

\subsection{IDM}

Ici, obstruction par thrombus

5 catégories : 1 à 5. Type 1 (spontané) =
\begin{itemize}
  \item sus-ST : désobstruer ASAP
  \item sans sus-ST : prévenir
\end{itemize}
\danger urgence ! \skull

Physiopatho : accident vasculaire coronaire athérothrombotique occlusif ou
occlusion coronaire aigüe (segmente : nécrose totale à 12h, akinésie)

\paragraph{Diagnostic}
Douleur précordiale : angineuse \textbf{au repos > 30min},
\underline{trinitrorésistante} (la douleur peut manquer !)

Examen clinique normal

ECG : sus-décalage ST sur $\ge 2$ dérivations contiguës. Donne la topographie
(antérieur/latéral, inférieur/postérieur).
Parfois en miroir

\fbox{(Douleur thoracique > 30min) \land{} ECG = IDM ST} 

\paragraph{Évolution}
Sd de reperfusion : \dec douleur, négativation ondes T, T = $38^{\circ}$
à +6h

Onde Q de nécrose (diagnostic a posteriori)

Marqueur = troponine (ASAP, +6h, +12h), éventuellement myoglobine (rapide++) ou
CPK-MB si récidive

\paragraph{DD} 
Douleur thoracique : péricardite aigüe, EP, dissection aortique, sous-diaphragme (cholécystite aigüe,
ulcère perforé, pancréatite aigüe).

Simule IDM : Penser à mycocardite aigüe (IRM), cardiomypoathie de stress
(coronarographie)

\paragraph{Complications précoces}
Rythme/conduction : 
\begin{itemize}
  \item rythme ventriculaire : extrasystole < tachycardie < fibrillation
    ventriculaire (FV = plupart des morts subites ! Besoin d'un choc électrique)
  \item supra-ventriculaire : décompensation hémodynamique, accidents emboliques
  \item \gls{BAV} (transitoire/définitif) ou
    hypervagotonie\footnote{Bradycardie, hypotension} (Ttt : atropine, remplissage
    macromoléculaire)
\end{itemize}

Hémodynamiques
\begin{itemize}
  \item insuf. ventriculaire G : grave, faire échocardio vite (4 stades)
  \item choc cardiogénique : diagnostic si hypotension artérielle mal tolérée,
    ne répond pas au rempilssage macromoléculaire. Souvent OCA + 24/48h.
    Mortalité > 70\%
  \item infarctus ventricule D : hypotension, champs pulmonaires clairs,
    turgescence jugulaire. Regarder dérivations droites (!) : sus-ST.
    Échocardiographie
\end{itemize}
Mécaniques :
\begin{itemize}
  \item rupture paroi libre ventricule G : rapidement fatal
  \item rupture septale : +24-48gh. Échocardiographie doppler. Forte mortalité
  \item insuf mitrale : fuite par prolapsus valvlaire. Ttt chir
\end{itemize}
Thrombotique : thrombus intra-VG, embolies systémique : échocardio. (Thrombose
veineuse, EP)

Péricardite : sd inflammatoire, souvent asymptomatique.

Récidive ischémique \thus récidive IDM. Épreuve d'effort à  +5 jours.

\paragraph{Complications tardives}
Péricardite à +3 semaines (sd de Dressler)

Dysfonction ventricule G : scint/échocardio de stress/IRM cardiaque. Évolue en
dilatation VG/anévrisme

Troubles rythmes ventriculaires sévères : défibrillateurs automatique
implantable (DAI)

\subsection{Traitement}
Reperfusion !!
\begin{itemize}
  \item ICP si < 120min, précédée d'\{aspirine, inhib récepteurs P2Y12, anticoagulant\}
  \item sinon fibrinolyse IV par TNK-tPA
  \end{itemize}
  \textit{Antiagrégants et anticoagulant pendant transport !}
  
Efficacité : reperfusion dans 90min (50\%). Sd reperfusion

Complications : AVC, réocclusion (surtout si ttt antiagrégant interrompu)

\paragraph{Associé}
\begin{itemize}
  \item antalgique
  \item antiagrégant : aspirine et inhib récepteur P2Y12 [\danger clopidogrel
    seulement si fibronolyse]
  \item anticoagulant : bivalirudine si ICP
  \item $\beta$bloquant (avec prudence)
  \item IEC dans 24h
  \item éplérénone précocement (si FEVG < 40\% ou insuf cardiaque)
\end{itemize}

\paragraph{Tttt des complications}
Troubles rythmes ventriculaire : amiodarone

Troubles rythmes supra-ventriculaire : AVK si mal toléré (hémodynamique)

BAV transitoire : atropine.

BAV après IDM antérieur : sonde d'entraînement électrosystoliques.

Insuf ventriculaire G : diurétique, IEC, épléronone

Choc cardiogénique : lutter contre {hypovolémie, troubles rythme}, sidération
(dobutamine). Assistance circulatoire/cardiaque/cardiocirculation,
revascularisation

Mécanique : rupture paroi libre = mortelle, septale = suture chir, mitrale =
remplacement valvulaire.

\subsection{Suivi}
\begin{itemize}
  \item antiagrégants plaquettaires : aspirine + clopidogrel (sauf si angor
    stable : aspirine)
  \item statines : si SCA/ango stable
  \item $\beta$bloquant : si infarctus
  \item EAC si coronarions post-infact
  \item épléronone : IDM étendu FEVG < 40à%
\end{itemize}
Éventuellement DAI

\section{228 : Douleur thoracique aigüe}%
\label{sec:228_douleur_thoracique_aigue}

\subsection{CAT}
Détresse vitale ?
\begin{itemize}
  \item respi : FR < 10 ou > 30/min, tirage, sueurs, cyanose, $SpO_2$
  \item hémodynamique : arrêt circulatoire, choc, c\oe{}ur pulmonaire, pouls
    paradoxal
  \item trouble conscience
\end{itemize}

4 urgences vasculaire : PIED (péricardite, infarctus, embolie pulmonaire,
dissection)

Examens : ECG 12 + 5 dérivations, radio poumon, troponinémie

Transfert USIC

\subsection{Urgences}

\paragraph{Sd coronarien aigü}
\begin{itemize}
  \item FR, ATCD
  \item douleur spontanée de repos > 20min : constriction, pesanteur, brûlure,
    rétrosternale, irradie  cou/épaule/avant-bras/tête. \danger présentation
    \textbf{atypique} possible
  \item examen clinique, radio normale
  \item ECG : sus/sous décalage ST
  \item doser myoglobine (< 6h) \lor{} troponine
\end{itemize}

\paragraph{Dissection aortique}
\begin{itemize}
  \item HTA, sd de Marfan, maladie de Turner
  \item Douleur aigüe, prolongée, intense, déchirement, irradie dans dos, descend
    vers lombes
  \item Clinique : $\Delta$PAS > 20mmHg (bras), abolition 1 pouls, souffle
    insuffisance aortique, déficit neuro
  \item ECG : normal \lor{} SCA
  \item Radio : élargissement médiastin
  \item \textit{Échocardio et ETO \lor{} scanner} 
  \item Chir en urgence, contrôle pression artérielle
\end{itemize}

\paragraph{Embolie pulmonaire} Y penser si douleur thoracique, dyspnée, radio
normale \skull
\begin{itemize}
  \item Terrain
  \item 2 tableaux
    \begin{itemize}
      \item infarctus pulmonaire : douleur basithoracique, hémoptysie noire
      \item c\oe{}ur pulmonaire aigü : dyspnée, défaillance ventriculaire
    \end{itemize}
  \item EC : parfois thrombose veineuse
  \item radio normale
  \item ECG : c\oe{}ur pulmonaire droit
  \item \textit{D-Ddimère \thus doppler veineux MI, angoscan ou scinti}. HBPM sans
    attendre !
\end{itemize}

\paragraph{Péricardite aigüe}
Tamponnade péricardite = urgence \skull
\begin{itemize}
  \item douleur thoracique, dyspnée, polypnée \thus orthopnée, toux
  \item turgescence jugulaire, reflux hépatojugulaire
  \item Choc : tachycardie, PAS < 90mmHg
  \item Pouls paradoxal
  \item ECG . microvoltage
  \item radio : cardiomégalie
  \item \textit{échocardio}  (compression VG par VD)
\end{itemize}

Péricardite non compliquée (plus bénin) :
\begin{itemize}
  \item terrain
  \item douleur thoracique augmente inspiration, decubitus. Calmée par
    antéflexion
  \item ECGA : sus-ST diffus, sous-PQ, microvoltage
  \item \textit{échocardio, troponine} 
\end{itemize}

\paragraph{Myopéricardite}
Douleur type péricardite mais \textbf{peut simuler SCA} .

Échocarido + (coronarographie normale)

\subsection{Chroniques cardiaque}
Angor stable

Douleur d'angor : d'effort du rétrécissement aortique serré, fonction des
tachycardies chroniques

Douleur d'effort de myoacardiopathie obstructives.

(HTA pulmonaire)

\subsection{Extra-cardiaques}
Urgences moyennes : 4 P = \{pneumothorax, pleurésie,
pneumonies, pancréatite\}, ulcère gastrique/duodénale, cholécystite, douleurs
radiculaires

\section{223 : Artériopathie oblitérante (aorte, MI)}%
\label{sec:223_arteriopathie_obliterante_aorte_mi_}
\subsection{AOMI\footnote{Artériopathie oblitérante des membres inférieurs}}

Épidémio : \male > \female. Pic = 60-75 (\male), 70-80 (female). Prévalence :
1-2\%

\paragraph{Clinique}
Classif de Rutherford : 
\begin{enumerate}[label=\Roman*]
  \item asymptomatique 
  \item claudication légère/modérée/sévère
  \item douleur ischémique de repos 
  \item perte de substance faible/majeur(ulcère/gangrène)
\end{enumerate}

Claudication intermittente : douleur "crampe" au mollet après $x$m de marche.
Disparaît en 5min. Sévère si $x < 200$m. \danger{} Sévère $\neq$ symptomatique

Puis au repos : 
\begin{itemize}
  \item douleurs de décubitus : brûlure orteils, avant-pied. Amélioré par
    déclivité
  \item trouble trophiques : peau mince, fragile, perte pilosité. Puis plaies,
    ulcères, gangrène
  \item ischémie permanente : douleur > 10 j, antalgique résisntant. Critique si
    PF\footnote{Pression de perfusion} < 50mmHg (cheville) ou 30mmHg (gros
    orteil) !
\end{itemize}
Physique : 
\begin{itemize}
  \item inspection : pâle, cyanosé. Interdigitaux++)
  \item palpation : froid, douleur à palpation musc si sévère), pouls, temps recoloration cutané, anévrisme
abdo, poplité
  \item auscult : souffle
\end{itemize}
AOMI si IPS\footnote{Index de pression systolique = pression systolique
cheville/bras} < 0.90, sévére si < 0.60

\paragraph{Paraclinique}
\begin{itemize}
  \item Test de marche (6min ou tapris roulant) : -30mmHG \lor{} -20\% évoque AOMI
  \item Transcutané de la $PO_2$ : hypoxie si < 35mmHg, critique si < 10mmHg
  \item écho-doppler artériel des MI
  \item Si revascularisation : angioscanner des MI, angiographie par RM,
    artériographie des MI
\end{itemize}

\paragraph{DD} 
\begin{itemize}
  \item Douleurs hanches : neuro, rhumato, veineuse, musc
  \item Douleurs de décubitus : neuropathie sensorielle, sd régionaux douloureux
    complexes, compression radiculaire
  \item Ulcères : veineux, microcirculatoire, neuropathie, trauma...
\end{itemize}

\paragraph{Étiologie} : atteinte athéromateuse = 95\%. Sinon : arthériopathies
inflammatoires, dysplasie fibromusculaire, coarctaation de l'aorte, atteinte
post-radique ou post-trauma, toxique, gelures, compressions extrinsèques,
atteinte de l'artère poplitée
    
\paragraph{Traitement}
Local : FR, antiagrégant plaquettaire (risque CV), statine (LDL), IEC (PA) $\pm$
$\beta$-bloquants si coronaire

Local : arrêt tabac, marche. Éventuellement statine (périmètre de marche),
prostaglandine (ischémie critique non revasc.)

Revascularisation si ischémie permanente : endovasc. (stent) ou chir (pontage).
Association possible. Parfois endartériectomie ou amputation

\paragraph{Pronostic} : grave, esperance de vie -10 ans

\subsection{Anévrismes}
Dilatation du \diameter{} > 50\%. Artères cérébrales, aorte, artères poplitées, iliaques

\paragraph{Aorte abdominale}
FR : tabac, ATCD familiaux, âge. Risque de rupture > \female. Haut risque CV

90\% des cas : si maladie athéromateuses. Associés à athérosclérose (90\%).
Formes familiales, évolution aortite.

Clinique : 
\begin{itemize}
  \item asymptomatique : dépister si FR
  \item symptomatique : douleur abdo/lobmaire $\pm$ choc hémorragique. Risque de
    rupture imminente \thus scanner en urgence \skull
  \item autre : complication embolique, compression, sd inflammatoire
\end{itemize}
Paraclinique : écho abdo (dépistage), scanner abdo-pelvien ou IRM = réf

\paragraph{PEC}
Asymptomatique : surveillance si \diameter < 50cm sinon chir (pontage) ou
endoprothèse (si haut risque chir)

Symptomatique : \danger anévrisme rompu = urgence chir \danger. Ne pas attendre
résultat

Suivi : écho-doppler si prothèse viasc, scanner/écho si endoprothèse.

\paragraph{Anévrisme poplité}
Asymptomatique : masse battante. Opéré si > 20mm
Au contraire, complication = embolie (ou ischémie)

\subsection{Ischémie aigüe des MI}
\danger urgence vasculaire !

Chronologie : +2h cellules nerveuses, +6h rhabdomyolyse, +24h nécrose. Sd des
loges.

Reperfusion : sd de reperfusion ou troubles métaboliques, insuf. rénale (ou
choc)

\paragraph{Diagnostic} Clinique, ne pas retarder la chirurgie \danger

Douleur brutale, intense, broiement, impotence fonctiennelle. 

Livie, douleur à palpation musc, pouls abolis en aval, anesthésie, paralysise

\paragraph{Étiologie} 2 tableaux (qui peuvent se mélanger) :
\begin{itemize}
  \item thrombose artérielle in situ (surtout AOMI)
  \item embolie sur artères saines (surtout cardiaque : fibrillation atriale)
\end{itemize}
Donc ascultation cardiaque, ECG, palpation abdo, bilan coagulation

Évualuer état général, fonction cardiaque, comorbidité

\paragraph{Traitement}
Médical : HNF, antalgique niveau 3, oxygène, soins locaux.

revascularisation : chirurgie (embolectomie sond Fogarty) voire fibrinolyse
$\pm$ angioplastie, aponévrotomie. Amputation possible

Surveiller acidose métabolique hyperK, insuf rénanel : diurèse, iono, urée,
créat.









\section{231 : Rétrécissement aortique}%
\label{sec:231_retrecissement_aortique}
Obstruction à l'éjection du VG\footnote{Ventricule gauche}, ici au niveau de la
valve aortique

\paragraph{Étiologies} 
\begin{itemize}
  \item bicuspidie < 65 ans
  \item dégénératif après (rarement post-rhumatismal)
\end{itemize}

Physiopatho: \inc pression \thus hypertrophie pariétale. Compense un temps
l'élévation de pression puis dysfonction systolique. Dysfonction diastolique

\paragraph{Complication}
Insuf cardiaque, fibrillation auriculaire, troubles conduction, mort subite++

\subsection{Clinique}

Pronostic vital mis en jeu si symptomes ! \skull

Dyspnée d'effort, angor d'effort, syncope (d'effort ?), (hémorragie digestive)

Examen : 
\begin{itemize}
  \item auscultation : \{souffle mésosystolique éjectionnel, rude, râpeux\},
    abolition B2 si calcifié
  \item frémissement palpatoire (foyer aortique), (choc de point dévié en bas à
    gauche 
\end{itemize}

\subsection{Explorations}
Radio thorax : éventullement dilatation VG, surcharge pulmonaire

ECG : souvent hypertrophie VG, auriculaire G, troubles conduction/rythme

Cathétérisme : pas habituellement mais coronarographie pour pré-op si \male > 40 ans, FR, angor d'effort ou insuf
cardiaque

Scanner cardiaque : pré-op si TAVI (cf supra)

\paragraph{Échocardiographie-doppler transthoracique} : examen clé. Critères
\begin{itemize}
  \item V max > 4m/s
  \item gradient moyen > 40 mmHg
  \item surface aoritque < 1 $\text{cm}^2$
\end{itemize}
Évalue conséquences sur VG, débit cardiaque, pressions droites

Examiner taille aort, valve mitrale, tricuspide

\subsection{Traitement}
\begin{itemize}
  \item valve chirurgical : mécanique si jeune mais AVK à vie, sinon prothèse bio (> 65
    ans)
  \item valvulopathie percutanée abandonnée
  \item implantation percutanée d'une valve aortique (TVAI)
\end{itemize}
Si symptomatique. Sinon test d'effort.

NB : si FE < 35\=, échocardio de stress sous dobutamine pour risque opératoire


\section{231 : Insuffisance mitrale}%
\label{sec:231_insuffisance_mitrale}
Reflux de sang depuis le VG\footnote{Ventricule gauche} vers l'OG\footnote{Oreillette gauche} pendant la systole.

Classif de Carpentier
\begin{enumerate}
  \item valves restent dans le plan de l'anneau (perforations)
  \item au moins une valve hors du plan de l'anneau (prolapsus)
  \item au moins une valve sous le plan de l'anneau
\end{enumerate}

\paragraph{Étiologies}
\begin{itemize}
  \item Rhumatismale (rare) : type III
  \item Dystrophique (fréquente++) : type II. Soit "dégénerescences myxoïdes"
    (trop de tissu, trop de mobilité), soit dégénerescences fibroélastiques
    (rupture de cordage)
  \item Sur endocardite : type I (perforations) ou II (rupture de cordage)
  \item Ischémique : soit aigüe (rupture de pilier, urgence \skull !), soit
    chronique (type III)
  \item Fonctionnelle : souvent une évolution de cardiopathie avec dilatation VG
    et atteinte systolique
\end{itemize}

Causes des insuf. mitrales aigües : rupture de portage ou de pliier, dysfonction
de pilier ischémique, perforation par endocardite.

Tableau hémodynamique \thus urgence vitale \danger 

Conséquences Hémodynamiques : altération contractilité VG (aval), HTAP pouvaint
être importante si aigü (amont)

\subsection{Diagnostic}
\danger{} peut être asymptomatique

Dyspnée : d'effort (lente et prgoressif), de repos, orthopnée, paroxystique
nocturne, OAP

Examen : 
\begin{itemize}
  \item palpation : frémissement systolique apex, (déviation et abissement choc
    de pointe)
  \item auscultation : souffle systolique de régurgitation, en "jet de vapeur",
    souffle holosystaque de B1 à post-B2, irradie vers l'aisselle ou la base
  \item (autres : galop B3, roulement mésodiastolique, éclat B2, souffle
    d'insuf. tricuspide)
  \item poumon : râles de stase
\end{itemize}

Para clinique
\begin{itemize}
  \item ECG longtemps normal. hypertrophie OG, VG, VD, fibrillation atriale, 
  \item Radio thorax : normale si $\le$ modérée. cardiomégalie, dilatation OG,
    HTAP
  \item \textit{ETT et ETO} : référence. Sévérité côté par SOR\footnote{surface
    de l'orifice régurgitant}, VR\footnote{Volume régurgité} (Vérifier la
    tricuspide)
  \item Cathétérisme : coronarographie seulement en pré-op si \male > 40 ans ou
    \female monopausée avec FR
  \item Épreuve d'effort, échocardio d'effort
\end{itemize}

Évolution : si constitué, bien toléré pendant longtemps. Si brutal : évolue vers
oedème pulmonaire. Complication : endocardite infectieuse, fibrillation/flutter
atrial, insuf cardiaque, complications thromboembolique

\subsection{Prolapsus valvulaire mitral}
Primitif ou associé. \female. Formes familiales.

Signes fonctionnels absents ou ceux IM.

Clinique : clic méso-/télosystoliques, souffle d'IM.

Examen : échocardio.

Évolution bénigne ou complication

\subsection{Traitement}
\begin{itemize}
  \item aigüe mal tolérée : chir urgence 
  \item chronique III ou IV symptomatique : chir
  \item chronique III ou IV asymptomatique : chir si retentissement VG ou
    surveillance échodoppler 6 mois (chir si symptômes, retentissement, troubles
    rythmes supraventriculaire)
\end{itemize}

Chirurgie : idéalement plastie reconstructicie, sinon remplacement valvulaire
(mécanique si jeune mais anticoag à vie, bioprothèse si > 65 ans)

Médical :
\begin{itemize}
  \item IM aigüe : ttt OAP ou choc, chir en urgence
  \item poussée insuf. cardiaque : diurétiques de l'anse, vasodilatateurs,
    digilatique (fibrillation atriale), anticoagulant oraux (fibrillation
    atriale)
\end{itemize}


\section{231 : Insuffisance aortique}%
\label{sec:231_insuffisance_aortique}
Régurgitation de sang vers VG en diastole.

\paragraph{Physiopatho} 
\begin{itemize}
  \item Chronique
surcharge volume et pression. Aorte : \inc
PA\footnote{Pression artérielle}
systolique, \dec PA diastoliques
Hypertrophie compense (parfois pendant des années !!) puis fibrèse
\item Aigüe : surtout EI\footnote{Endocardite infectieuse}, surcharge brutale,
  \inc pression puis oedème pulmonaire
\end{itemize}

\subsection{Étiologies}
Chronique :
\begin{itemize}
  \item dystrophique(freq++) : annulo-ectasiante (valves normale mais anneau
    dilaté), sd des valves flasques
  \item EI qui perfore les valvules
  \item malformative (bicuspidie aortique)
  \item rhumatismale (rare)
  \item inflammatoire, infectieuses, médicamenteuse
\end{itemize}
Aigü : EI, dissection aortique, rupture d'anévrisme d'un sinus de Valsalva,
traumatique

Prothèse : désinsertion partielle, dysfonction

\subsection{Clinique}
Fonctionnel : dyspnée d'effort, (angor d'effort ,), insuf cardiqaue (rare,
tardive)

Physique : 
\begin{itemize}
  \item ascult : souffle diastolique++, "doux, lointain, humé, aspiratif",
    souffle systolique éjectionnel d'accompagnement, roulement de Flint
    apexin/galop
  \item palpation : choc de pointe étalé, en bas à gauche
  \item hyperpulsatilité artérielle périphérique (pouls++), \dec PA
    diastolique++
\end{itemize}

ECG : normal ou HVG\footnote{Hypertrophie ventriculaire gauche} diastolique, (ou
HVG systolique)

Radio : \inc index cardiothoracique si volumineuse chronique

\textit{Échocardio-doppler (ETT)} = confirmer, quantifie dilatation VG 

Coronarographie : pré-op, \male > 40 ans ou \female monopausée, FR

IRM/scanner : dimension aorte, surveillance

\paragraph{DD} 
\begin{itemize}
  \item souffle diastolique : insuf pulmonaire
  \item double souffle (rupture sinus Valsalva), souffle continu, frottement
    péricardique
\end{itemize}
\paragraph{Évolution}
\begin{itemize}
  \item Chronique : si volumineuses, sévère dès les symptômes \thus surveillance
\danger, opération même si asympto. \danger dystrophique, bicuspidies
  \item Aigu : OAP, mort subite \thus chir précoce
\end{itemize}
Complications : EI ++, insuf cardiaque (tardive), rupture aortique, (mort subite)
   
\paragraph{Surveillance} 
Chronique : 1-2/an si fuite importante, sinon tous 2-3ans

Aigü : chir rapidement

\subsection{Traitement}
Médical : 
\begin{itemize}
  \item si volumineuse et IVG : IEC, diurétique et chir rapidement
  \item dilatation de l'aorte : beta-bloquant, losartan
\end{itemize}
Hygiène dentaire, examen tous 6 mois pour prévenir EI

Chirurgie : 
\begin{itemize}
  \item remplacement valvulaire si IA isolée
  \item valve + aorte si dystrophique ou (bicuspidie \land{} dilatation aortique)
\end{itemize}

Quand faire la chir ?
\begin{itemize}
  \item chronique volumineuse
    \begin{itemize}
      \item  symptomatique : urgent \danger
      \item asymptomatique : FEVG < 50\%, dilatation aorte $\ge 55$mm, diamètre
        VG télédiastolique > 70mm, télésystolique > 50mm
    \end{itemize}
  \item dystrophique et dilatation aorte asc : dès $\ge 55$mm
  \item aigüe volumineuse : urgence
\end{itemize}

\section{150 : Surveillance des porteurs de valves, prothèses vasculaires}%
\label{sec:150_surveillance_des_porteurs_de_valves_protheses_vasculaires}

\begin{itemize}
  \item Prothèses mécaniques : double ailette, à vie, anticoagulant à vie
    (risque thrombose)
  \item Biologiques : pas d'anticoagulant, 40\% à 15 ans.
\end{itemize}
Risque majeur d'EI $\forall$ prothèse !

\subsection{Complications}
\begin{itemize}
  \item Thromboemboliques (freq++) : surtout mécanique, surtout prothèses mitrale,
    anciennes, fibrillation atriale
    \begin{itemize}
      \item Embolie systémiques : souvent cérébrales
      \item Thromboses de prothèse mécanique : accidents brutaux (OAP, syncope,
        choc, mort subite). Diagnostic difficile : appartition d'un
        souffle/roulement. \textit{ETT, ETO} \\
        Chir d'urgence possible
        \danger DD avec EI parfois difficile
    \end{itemize}
  \item Désinsertions de prothèses (5\%) : spontané, EI. À évoquer si apparition d'un
    souffle, anémie hémolytique, insuf. cardiaque. Confirmé par ETT, ETO(++)

  \item Infectieuses
    \begin{itemize}
      \item médiastinie post-op (1\%)
      \item Endocardite infectieuses : \textbf{redoutable} \skull\\
        Précoce (50\%) ou tardive. Diagnostic : ETT, ETO++\\
        Prévention/traitement de tout foyer infectieux (ORL, dentaire)\\
        Hémocultures systémiques devant fièvre inexpliquée
    \end{itemize}
  \item Traitement anticoagulant : 1.2\% patiens-années risque hémorragique
  \item Dégéneresce bioprothèses
\end{itemize}

\subsection{Surveillance}
Post-op : AVK (à vie si mécanique, 3 mois si bio). ETT à +3mois (référence !)

Puis : 1/mois puis tous les 3 mois. Cardiologue à +3 mois puis 1-2/an.

Clinique : 
\begin{itemize}
  \item surveiller symptômes, dyspnée, insuf cardiaque
  \item ascult : attention à \dec intensité bruits (ou variables), \inc
    intensité d'un souffle, bruit diastolique surajouté
\end{itemize}
Radio, ECG mais surtout ETT, ETO

\textit{Biologie} ++ : équilibre AVK parfait, à vie \thus INR tous les mois $\in
[2.5, 4]$.

FR : valve non aortique, ATCD, fibrillation atriale, \diameter OG > 50mm,
contraste spontané dense OG, sténose mitrale, FE < 35\%, hypercoagulabilité

Ne pas interrompre AVK sauf pronostic vital !. Si chir extracardiaque : HNF
pendant l'arrêt AVK

\section{149 : Endocardite infectieuse}%
\label{sec:149_endocardite_infectieuse}
Infections des valves cardiaque ou de l'endocarde pariétal. Dominées par les
staphylocoques

\subsection{Physiopatho}
Bactéries adhèrent sur une lésion préexistante \thus
\begin{itemize}
  \item insuffisance valvulaire, souffle, risque de défaillance cardiaque
  \item végétations \thus embolies septiques, lésions de vasculairet, anévrisme
    "mycotique"
\end{itemize}

Cardiopathies à haut risque : prothèses valvilaire, cardiopathies congénitale
cyanogiène, ATCD EI

50\% des EI sur c\oe{}ur présumé sain !

Hémocultures positives (90\%)
\begin{itemize}
  \item streptocoques oraux, streptocoques du groupe D
  \item staphylocoques : blanc, coagulase négative
\end{itemize}
Hémocultures négatives :
\begin{itemize}
   \item ATB
   \item croissante lente : HACEK, Brucella, champignons
   \item intra-cellulaire : \bact{burnetii}, Chlamydia, Bartonella,
     \bact{whipplei}
\end{itemize}

\subsection{Clinique}
\danger Manif trompeuses. Y penser si souffle cardiaque et fébrile, AVC,
purpura, lombalgies féribles

\begin{itemize}
  \item Sd infectieux : fièvre, AEG, splénomégalie
  \item Apparition/modif souffle, insuf cardiaque
  \item cutané (nodosité d'Osler !), respi, ophtalmo, rhumato(freq), neuro,
    rénale
\end{itemize}

Diagnostic : hémoculture, échocardio

Autres : NFS, \{CRP, électrophorèse\}, complexes immuns circulants, \{urée,
créat\}, BNP

Classif de Duke : 2 majeurs ou (1 majeur et 3 mineurs) ou (5 mineurs)
\begin{itemize}
  \item majeurs
    \begin{itemize}
      \item Hémocultures : micro-org typique, HC $\ge 2$ \lor{} positive > 12h
        \lor{} positive à \bact{burnetii}
      \item Échocardio avec végétation, abcès, désinsertion prothétique \lor{} nouveau souffle de régurgitation valvulaire
    \end{itemize}
  \item mineurs
    \begin{itemize}
      \item cardiopathie à risque/toxicomanie
      \item $\ge 38^{\circ}$
      \item comilcation vasc
      \item immunologique
      \item hémoc/séro positive
    \end{itemize}
\end{itemize}

\paragraph{Évolution}
Complications : insuf cardiaque (num. 1 DC), neuro (num 2), embolies (septiques,
cérébrales, splénique, rénales, coronaires), infarctus spnélique, arythmies et
troubles de conduction

Penser à scanner cérébral et abdo-pelvien !

Pronostic : 
\begin{itemize}
  \item sur aortique : chir
  \item staph ou prothèse : mortalité++
  \item pneumocoque, bacciles Gram négatif : destruction valvulaire graves
  \item levure : grosse végétations
\end{itemize}

\subsection{Traitement}
Bithérapie IV

Fonction rénale pour aminosides et vancomycine !

\begin{itemize}
  \item Strepto oraux/groupe D : amoxicilline et gentamicine (2 semaine bi, 4
    semaines mono) [vancomycine + gentamicine si allergie]
  \item entérocoques : idem
  \item staph : si sensible : cloxacilline (+gentamicine + rifampicine si sur
    prothèse). Sinon vancomycine (+gentamicine + rifampicine)
  \item hémoc négative : en attendant amox + acide clavilanique + gentamicine
\end{itemize}

Chirurgie : valvie native si possible. Intervention si insuf cardiaque ou sd
infectieux non contrôlé

\paragraph{Prévention}
Hémoc avant antio \danger

ATBprophylaxie : amoxicilline (clindamycine si allergie) avant geste (région
apical/gingivale, perforation muqueuse orale ou (extraction dentaire et haut
risque))

\section{236 : Souffle cardiaque chez l'enfant}%
\label{sec:236_souffle_cardiaque_chez_l_enfant}
Très fréquent.

Malformation congénitale (1\%), souffle fonction, cardiomyopathie/myocardite
aigüe (rarement), acquises (exceptionnelles)

\paragraph{Auscultation chez l'enfant} Rythme rapide, irrégulier.

B2 dédoublé : anormal si large et fixe.

Éclat B2 : HTA pulmonaire, malposition des gros vaisseaux

B3 physiologique (apex)

Clic possible

\subsection{Clinique}
Fonctionnel : souvent absent, dyspnée d'effort. \danger douleur thoraciques =
rarement cardiaques !

Souffle :
\begin{itemize}
  \item varie en temps et position : innocent
  \item bruyant, irradiant largement : organique
  \item diastolique : organique
  \item frémissant : organique
  \item holosystolique, de régurgitation : organique
  \item cou et sus-sternal : aortique ; dos : pulmonaire ;kirradiant :
    communication intra-V
\end{itemize}
Associés : 
\begin{itemize}
  \item regarder $SaO_2$
  \item troubles alimentaires, dyspnée, sueur, retard staturopondéral : large
    shunt
  \item HTA, pas de pouls fémoraux : coarctation aortique
\end{itemize}

\paragraph{Complémentaire}
Radio thorax : cardiomégalie (\danger "fausses")
\begin{itemize}
  \item saillie arc moyen G : shunt gauche-droite
  \item arc moyen G conctave : hypoplasie voie pulmonaire
\end{itemize}

ECG : fréquence diminue avec l'âge. T < 0 en $V_1$ $V_4$

\textit{Échocardio} = examen clé

Autres : effort, holter ECG, IRM cardiaque, scanner multibarettes, cathétérisme
cardiaque (rare)

\subsection{Cardiopathies}
\paragraph{Naissance, +2 mois}
\begin{itemize}
  \item Souffle isolé : examen clinique, ECG, radio pulmonaire, échocardio
  \item Insuf cardiaque : coarctation préductale \thus chir urgente
  \item Cyanose : transposition des gros vaissaux \thus chir avant N+15 jours
\end{itemize}

\paragraph{N+2 mois, marche}
\begin{itemize}
  \item Insuf cardiaque : shunts gauche-droite surtout (\thus opérer avant 1 an
    si large !!), communication intra-V
    large, persistance canal artériel, canal atrioventriculaire
  \item cyanose : tétralogie de Fallot\footnote{Communication intra-V,
      hypertrophie ventriculaire d, sténose pulmonaire, dextroposition de
    l'aorte}
\end{itemize}
\paragraph{2 à 16 ans}
\begin{itemize}
  \item Malformatives : raraes, bien tolérées
  \item Souffles "innocents" (1/3) : asymptomatique, systolique, éjectionnels,
    faible intensité, (intensité varie avec position), doux. Ne rien faire
\end{itemize}




\section{337 : Malaise, perte de connaissance}%
\label{sec:337_malaise_perte_de_connaissance}
\begin{itemize}
  \item syncope : trouble de conscience, hypotonie, début brutal/rapide, souvent
    bref. Comportement, orientation normaux après retouru conscience
  \item lipoythmie : sensation de perte de connaissance
  \item Stokes-Adams : syncope à l'emporte-pièce
  \item autres : coma, confusion mentale, crise comitiale, AVC, cataplexie,
    narcolepsie
\end{itemize}

\paragraph{Physiopatho}
Hypoperfusion de la substance réticulée du tronc cérébral (< 60 mmHg ou arrêt >
6 secondes) \thus perte conscience, tonus, myclonies si > 30s

\subsection{Étiologies}
Cause cardiaques mécaniques
\begin{itemize}
  \item rétrécissement aortique : à l'effort
  \item cardiomyopathies hypertrophiques obstructives : génétique, à l'effort ou
    post-effort immédiat. Auscut : souffle systolique sternum gauche, ECG :
    hypertrophie VG
  \item EP massive
  \item Tamponnade brutale
\end{itemize}

Cause cardiaques électriques :
\begin{itemize}
  \item Tachycardie
  \item BAV
  \item dysfonction sinusal
  \item défaillance stimulateur cardiaque
\end{itemize}

Hypotension :
\begin{itemize}
  \item avec tachycardie sinusale : iatrogènes, orthostatique
  \item avec bradycardie sinusale : hypotension réflexe, vasovagale
\end{itemize}

\paragraph{DD} : 
\begin{itemize}
  \item métaboliques (hypoglycémie, hypoxie-hypercapnie,
encéphalopathie hépatique)
\item toxiques (toxico, médical, alcool++, CO++)
\item psy (trouble de conversion, attaque de panique, simulation)
\item neuro (vasc) : infarctus cérébraux, AIT, insuf. vertébrobasilaire,
  drop-attacks
\end{itemize}

\subsection{PEC}
\begin{algorithm}
   Perte de connaissance brève, pas de crise comitale ? Si non : \textit{épilepsie,
    AVC/AIT, coma, intoxication, céphalée, SAS} \footnote{Syndrome d'apnée du
    sommeil} \faHandStopO\;
   Syncope. Signe de gravité ? Si oui : urgence = SCA, EP... \faHandStopO\;
   Interrogatoire, cliinque, ECG ? Si cause évidente (méca, électrique,
    hypotension) \faHandStopO\;
   Cardiopathie sous-jacente ? Si oui : holter, électrophysio\;
   Sinon probablement neurocardiogénique
\end{algorithm}

Interrogatoire :
\begin{itemize}
  \item âge, ATCD : mort subite (famille), cardiopathie si âgé, médicaments
  \item prodrome, postures, activité
  \item mouvements anormaux, durée, réveil, courbature
\end{itemize}
Examen neuro (déficité), CV (pression artérielle)

ECG : diagnostic si bradycardie < 40/min, tachycardie (supra)ventriculaire, BAV
complet ou 2eme degré, défaillance stimulateur cardiaque

\paragraph{Paraclinique}
Éliminer cardiopathie sous-jacente : \textit{échochardio} , test d'effort, BNP,
troponine

Autres : Holter-ECG (dysfonction sinusale, trouble conduction AV). Sinon
électropélectrophysiologique endocavidate (déclenchement tachycardie
ventriculaire), test d'inclinaisant (déclencher syncope vasovagale),
hyperréflexie sinocartidienne, ECG implantable

\paragraph{Gravité}
\begin{itemize}
  \item Trouble du rythme ventriculaire/de conduction supposé
  \item syncope inexpliquée chez cardiaque
  \item suspicion maladie génétique chez jeune
  \item syncope et trauma grave
  \item syncope d'effort
  \item syncope de décubitus
\end{itemize}

\paragraph{Formes typiques}
\begin{itemize}
  \item syncope neurocardiogénique : vasovagale (debout, vue du sang,
    \textbf{jeune} ), réflexe
    (miction), hyperréflexie sinocarotidienne (rasage, \textbf{âgé} )
  \item hypotension artériel : âgé, iatrogène, debout prolongé
  \item troubles du rythme/conduction : plus souvent : tachycardie
    ventriculaire. Diagnostic = électrophysio.. Diagnostic = électrophysio..
    Diagnostic = électrophysio.
\end{itemize}

\section{230 : Fibrillation atriale}%
\label{sec:230_fibrillation_atriale}

Tachycardie irrégulière due à une activité anarchique des oreillettes
(400-600/min) > 30 secondes.

Noeud AV filtre à 130-180/min \thus tachy irrégulière \thus risque
d'insuffisance cardiaque et thromboembolique (stase). Évolue : fibrose
oreillettes, dilatation atriale

Fréquente chez âgé

\paragraph{Classification}
\begin{itemize}
  \item Premier épisode
  \item Paroxystique : retour en sinusal < 7 j
  \item Persistante :  retour en sinusal > 7 j ou après cardioversion
  \item Permanente : échec cardioversion/non tentée
\end{itemize}

\subsection{Diagnostic}
Signes usuels : palpitations, dypsnée d'effort, angor fonctionnel, asthénie
inexpliquée...

Auscul : bruits irréguliers, rythme $\pm$ rapide

\textit{ECG} : indispensable \danger
\begin{itemize}
  \item petites/grosse mailles
  \item QRS lents réguliers
  \item dysfonction sinusale à l'arrêt de FA\footnote{Fibrillation atriale}
    (brady-tachy)
\end{itemize}

Autres : \{iono, créat, TSHus, NFS\}, radio thorax, echocardio

\paragraph{Étiologies}
\begin{enumerate}
  \item HTA (âgé)
  \item valvulopathie (mitrale)
  \item autres : respi (SAS !), cardiomyopathies, SCA, hyperthyroïde (y penser
    !), péricardites, chir cardiaque récente, cardiopathies congénitales,
    phéochromocytome
\end{enumerate}

\paragraph{Tableaux cliniques}
\textit{FA isolé, c\oe{}ur normal} : quinqua, palpitation nocturnes $\pm$ angor
fonctionnel \lor{} dypsnée d'effort. Échocardio normale. Exclure SAS et HTA !\\
\thus seulement anti-arythmique (flécaïdine)

\textit{FA avec insuf cardiaque} : souvent séquelle infarctus sévère ou
cardiomyopathie dilatée à coronaires saines. OAP/ décompensation cardiaque
gloable.\\
\thus antiocoagulants oraux \arrow{} cardioversion (parfois urgence) \arrow{}
anticoag. au long cours, amiodarone\footnote{Maintien rythme sinusal}

\textit{FA valvulaire post-rhumatismale}  : persistante/permanente sur maladie
mitrale. \\
\thus à discuter, AVK au long cours

\textit{Embolie artérielle systémique} :souvent cérébrale. FA méconnue chez
\female âgée avec FR embolique (HTA, diabète). Écarter SCA (tropo, ECG)\\
\thus aigü : (thrombolyse), aspirine \arrow{} AVK, héparine

\textit{Maladie de l'oreillette}  : alternance FA paroxystique rapide-brady\\
\thus stimulateur cardiaque définitif

\subsection{Traitement}
\paragraph{Risque thromboembolique}
Cardioversion : à risque par défaut ! Donc héparine ou anticoag. oral. 
Si risque très élevée, vérifier l'absence de trombos atrial G

Chronique :
\begin{itemize}
  \item FA valvulaire : risque très élevé
  \item FA isolé sur c\oe{}ur sain : risque faible
  \item sinon score CHADS2 \danger{} pas si FA valvulaire !!! : \texttt{Congestion +1,
    Hypertension artérielle +1, Âge > 75 ans +1, Diabète +1, Stroke +2}\\
    anticoag si CHADS2 > 1
\end{itemize}

\paragraph{FA persistante ou premier accès < 7 j}
\begin{itemize}
  \item Prévention thromboembolique par HNF IV (AVK/nouveau anticoag
    directement si bien toléré et pas à haut risque)
  \item Cardioversion : antiocoag orale -3 semaine et + 4 semaines. Choc
    électrique sous anesthésie générale \lor{} médicament (amiodarone). Rarement en
    urgence.
\end{itemize}

\paragraph{Entretien}
\begin{itemize}
  \item Anticoagulant selon terrain : AVK (INR !), inhib trombine (dabigatran), inhib
facteur X (rivaroxaban, apixaban, edoxaban)\\
Si FA valvulaire : seuls AVK
\item respect FA et seul contrôle FC (oui) \lor{} contrôle FA (paroxystique,
  persistante)
\end{itemize}

\paragraph{Éducation du patient}
HTA, risque d'embole cérébrale, effets secondaire amiodarone (thyroïde,
photosensiblisitation, dépôts cornéens)






\section{234 : Troubles de la conduction intracardiaque}%
\label{sec:234_troubles_de_la_conduction_intracardiaque}
Fréquences d'échappement :
\begin{itemize}
  \item noeud AV : 40-50/min
  \item faisceau de His : 35-45/min
  \item branches et ventricules : < 30/min
\end{itemize}
Dysfonction sinale et BAV peuvent être symptomatiques. Les BAV isolé non.

\subsection{ECG}%
\paragraph{Dysfonction sinusale}
Arrêt par le noeud sinusal ou non-transmission à l'atrium.

\begin{itemize}
  \item Tracé plat sans P bloqué (!)
  \item BSA\footnote{Bloc sinuso-atrial} II : si pause après P = cycle normal $\times N$
  \item Si arrêt sinusal ou BSA complet : asystolie ou bradycardie
  \item Bradycardie sinusale inappropriée (éveil)
\end{itemize}

\paragraph{Blocs atrioventriculaires}
Dans le faisceau de His ou infra : rythme très lent donc grave \skull
\begin{itemize}
  \item BAV I : PR constant mais > 0.2s\footnote{1 mm = 0.02s}
  \item BAV II Wenckebach : allongement PR proressif jusque bloqué (souvent QRS
    < 0.12s)
  \item BAV II Möbitz : PR normal, multiple P bloqué\footnote{Wenckebach 2:1 ou
    Möbitz 2:1 ne peuvent être différenciés sur l'ECG}
  \item BAV III : aucun P ne passe, ventricule à leur rythme, plus lent.
    \danger{} DC possible (torsade de pointes)
  \item BAV III + FA\footnote{Fibrillation atriale} : bradycardie (!), rythme
    régulier (!)
\end{itemize}

\paragraph{Blocs de branches}
\danger{} BdB gauche gêne le diagnostic d'infarctus !!
\begin{itemize}
  \item Droit : QRS > 0.12s et RsR' en V1 
  \item Gauche : QRS > 0.12s et R exclusif en V6 
  \item Hémi-bloc\footnote{Hémibloc : forcément de la branche gauche qui se
    divise en faisceau antérieur et postérieur} antérieur : dévation axiale gauche, QRS < 0.12s, Q1S3 et S3 > S2
  \item Hémi-bloc postérieur : dévation axiale droite, QRS < 0.12s, S1Q3

\end{itemize}

\subsection{Clinique}
\paragraph{Dysfonction sinusale}
Asymptomatique, lipoythmie, syncopes...

Fréquent si âgé

Étiologies : \{médic (bradycardisant), vagal\}, cardiaque, maladies systémiques, neuromusculaire,
post-chir, HTIC, hypothermie, (septicémies), ictères rétentionnels sévères,
\{hypoxie,hypercapnie, acidose sévère\}, hypothyroïdie.

Diagnostic : \textit{ECG} ! 
\begin{itemize}
  \item bradycardie en éveil, pas d'accélération à l'effort
  \item pauses P sans ondes > 3 s
  \item BSA II
  \item bradycardie avec rythme d'échappement atrial/jonctionnel
  \item sd bradycardie-tachycardique
\end{itemize}

Cliniques usuelles :
\begin{itemize}
  \item dégénérative liée à l'âge : \female, multiple médicaments. Souvent + FA
    $\pm$ troubles conductif sur noeud AV. Traiter !
  \item hypervagotonie : sportif. ECG : brady < 50/min en éveil. Test
    atropine/d'effort normalise. Ne pas traiter.
\end{itemize}

\paragraph{BAV}
Cf dysfonction sinusale. Peut avoir fibrillation ventriculaire suite à torsade
de pointe. Fréquen si âgé

Étiologies : hyperkaliémie+++ \\
autres = fibrose, rétrécissement aortique dégénératif, causes ischémiques
SCA (mauvais pronostic si (infra)-hissien !), infectieux, \{médic, vagal\},
systémiques, neuromusculaire, post-chir, postcathétérisme, postradiothérapie,
néoplasiqe, congénital

Diagnostic : précisier degré, paroxistique/permanent, siège++ 
\begin{itemize}
  \item nodaux : souvent BAV I, BAV II Wenckebach, BAV III à QRS fins \thus
    Holter
  \item (infra)hissiens : sur des BdB ou BAV II Möbitz. \skull si complet DC
    possible !\\
    \thus étude endocavitaire
\end{itemize}

Cliniques usuelles :
\begin{itemize}
  \item BAV complet sur infarctus antérieur : régressif sous 15 j(sinon stimulateur ?),
    sensible à l'atropine
  \item BAV dégénératif
  \item BAV congénital
\end{itemize}

\paragraph{BdB}
Toujours asymptomatique si isolé. Grave si lipoythimie/syncope \danger \thus
étude endocavitaire

Étiologies :
\begin{itemize}
  \item Droit : peut être bénin. Surtout dans patho pulmonaires
  \item gauche : jamais bénin ! (dégénératif ou cardiopathie). SCA de cause
    ischémique possible \skull
\end{itemize}

Diagnostic : incomplète si QRS < 120ms, complet sinon. Droit/gauche/ bi- ou
trifasculaire. Chercher cardiopathie sous-jacente

\subsection{PEC}
\begin{itemize}
  \item 
Dysfonction sinusale : Confirmer l'ECG par Holter (à répéter éventuellement)
Si vagal possible, test d'inclinaison. Si âgé, on peut chercher une
hyperréflectivité sinocarotidienne.
\item BAV : médicament, SCA (territoire inférieur), myocardite ? \\
  Bloc permanent ? ECG suffit. Sinon enregistremeent Holter \\
  Si suspicien infra-hissien, etédue endocavitaire possible.\\
  Échocardio et troponine dans tous les cas
\item BdB : HTA ou cardiopathie ?\\
  Droit chez jeune asymptomatique $\approx$ variante normale\\
  si syncope sur cardiopathie : cherche tachycardie ventriculaire
\end{itemize}

\subsection{Traitement}
Bradycardie grave = urgence ! (rea) \skull

Brady avec BAV III plus grave que brady par DS

Médicaments tachycardisants (atropine, catécholamine), stimulation cardiaque
temporaire (percutanée, transthoracique)

Stimulateur pour 
\begin{itemize}
  \item dysfonction sinusale symptomatique seulement
  \item BAV III sis non curable
  \item BAV II si infra-hissiens ou symptomatiques
  \item BdB avec symptômes et BAV paroxistique (sinon non !)
\end{itemize}
Toujours traiter cause



\section{229 : ECG}%
\label{sec:229_ecg}
Normales : FC $\in [60, 100]$ battements/min, P < 120ms

\subsection{Hypertrophies}

Atriales
\begin{itemize}
  \item droite : P > 2.5mm en D2 ou > 2 mm en $V_1$ ou $V_2$
  \item gauche : P > 0.12s
\end{itemize}
Ventriculaires
\begin{itemize}
  \item gauche : Sokolov : S$V_1$ + r$V_5$ > 35mm. \danger{} QS ou sus-ST peut
    mimer un infarctus !
  \item droite : +110$^{\circ}$
\end{itemize}

\subsection{Troubles de conduction}
BdB :
\begin{itemize}
  \item droit : QRS > 0.12s, RsR' en $V_1$ 
  \item gauche : QRS > 0.12s et rS ou QS en $V_1$
\end{itemize}
Hémibloc : 
\begin{itemize}
  \item antérieur: -30$^{\circ}$, $S_3$ > $S_2$
  \item postérieur : +90 $^{\circ}$, $S_1 Q_3$
\end{itemize}
Bifasciculaire : BdB droit + (un des hémibloc)\\
Trifasciculaire \skull : (alterne BdB droit et gauche) ou (BdB droit \land{}
alternance hémiblocs)

BAV
\begin{itemize}
  \item I : PR constant > 200ms
  \item II : PR croissant jusque P bloqué \lor{} un seul P sur plusieurs
  \item III : aucun P, QRS réguliers lents
\end{itemize}

Dysfonction sinusale : asystole, bloc sino-atrial II

\subsection{Troubles du rythme supraventriculaire}
Man\oe{}uvres vagales : valsalva, (jeune : compression carotidienne unilat) sinon
vagomimétique

Fibrillation atriale : 
\begin{itemize}
  \item 100-200/min, QRS irréguliers, atriale = mailles amples ou fines.
  \item \danger{} : BAV III, brady-tachy ou BdB possibles !
\end{itemize}
Flutters atriaux : (souvent + FA)
\begin{itemize}
  \item 300/min avec "dents de scie" en $D_1$, $D_3$ aVF
  \item ventriculaire : rapide (pas toujours), régulières (pas toujours).
    Ralentie par man\oe{}uvre vagale !
\end{itemize}
Tachycardie atriale : (moins fréq)
\begin{itemize}
  \item 120-200/min
  \item tachy régulières à QRS fin, souvent coupés de retours en rythme sinusal
\end{itemize}
Tachycardies jonctionnelles (fréq++)
\begin{itemize}
  \item 130-260/min
  \item pas d'activité atriale, retour en sinusal à man\oe{}uvre vagale
\end{itemize}
Extrasystole (freq, physio). Si un battement sur 2, bigéminisme

\subsection{Troubles du rythme ventriculaire}
\fbox{Toute tachycardie à QRS est une tachycardie ventriculaire JPDC \skull}

Tachycardies ventriculaires
\begin{itemize}
  \item > 100/min
  \item QRS > 0.12s pendant $\ge 3$ battements
\end{itemize}
Fibrillation ventriculaire : \textbf{urgence absolue}  \skull

Torsade de pointe : si allongement QT, bradycardie

Extrasystoles ventriculaires : banales, sur c\oe{}ur sain, regarder étiologie

\subsection{Autres}
\begin{itemize}
  \item Hypokaliémie : T plates/négative, sous-ST, QRS normale, allongement QT
  \item Hyperkaliémie : T ample pointe, allongement PR, élargissement QRS
  \item Péricardites : (1. microvoltage, sus-ST, sous-PQ), (2. T plat), (3. T
    négative), (4. normale)
  \item Sd Wolf-Parkinson-White : PR < 0.12s, "empâtement" QRS
\end{itemize}

Maladie coronaires : sus-ST
\begin{itemize}
  \item chercher miroir, +2mm en précordial, +1 mm e frontal
  \item sur au moins 2 dérivations
  \item BdB gauche complet suffit !
\end{itemize}
Ondes Q de nécrose : +6h, 1/4 du QRS

\section{235 : Palpitations}%
\label{sec:235_palpitations}
Sensation que le c\oe{}ur bat trop fort/vite/irrégulièrement

Interrogatoire : 
\begin{itemize}
  \item fréquence, effort, durée
  \item \danger{} douleur thoracique, perte de connaissance, dyspnée
\end{itemize}
Gravité ?
\begin{itemize}
  \item ATCD personnels : post-infarctus, HTA, troubles du rythme, stimulateur,
    médic
  \item ATCD familiaux : mort subite < 35 ans
  \item clinique : pouls > 150 /min, hypotension artérielle, angor, insuf
    cardiaque, neuro
  \item ECG : tachy à QRS large = urgence absolue \skull\\
    autres : anomalie repolarisaton (SCA ?), BAV II ou III (rare), tachy à QRS
    fins + clinique
\end{itemize}
Diagnostic : chercher cardiopathie sous-jacente, ECG concomitant
\begin{itemize}
  \item interrogatoire : alcool++, fièvre++, déshydratation, SAS,
    hyperthyroïdie, grossesse
  \item ECG, echocardio, ECG d'effort
\end{itemize}

\paragraph{Étiologies fréquentes}
Extrasystoles : cherche (extra) cardiqaue :
\begin{itemize}
  \item alcool, électrocution, pneumopathie, hyperthyroïdie, anomalie
    électrolytique, anxiété, grossesse, SAS
  \item \danger{} obèse/diabétique : bien vérifier si fibrillation atriale !
\end{itemize}
Tachycardie sinusale :
\begin{itemize}
  \item cardio (avec dyspnée) : insuf cardiaque, EP, épanchement péricarde...
  \item extra : fièvre, anémie, hypoxie, hyperthyroïde, grossesse, alcool,
    hypotension artérielle, SAS...
\end{itemize}
Troubles supra-ventriculaires
\begin{itemize}
  \item flutters/tachy atriale
  \item tachycardie jonctionnelle : jeune, coeur normal, polyurie fin d'accès,
    arrêt par vagal
\end{itemize}
Troubles ventriculaires : rares, plutôt syncope. Sur cardiopathi et signes
gravité.

Névrose cardiaque = élimination


\section{232 : Insuffisance cardiaque}%
\label{sec:insuffisance_cardiaque}
Déf clinique : symptômes d'IC\footnote{Insuffisance cardiaque} (dyspnée, oedèmes
chevilles, fatigue...) \land{} signes d'IC (crépitant, turgescence jugulaire...)
\land{} anomalie de structure/fonction du coeur

Prévalence : 1-2\%, augment avec l'âge

Adaptation :
\begin{itemize}
  \item cardiaque : remodelage (dilatation ventriculaire, hypertrophie), \inc
    inotropie, tachycardie
  \item extra-cardiaque : vasoconstriction, rétention hydrosodée, activation
    neurohormonale
\end{itemize}
Si IC non réversible, curable, 2 mécanismes selon la fonction systolique :
\begin{itemize}
  \item diminuée (défaut contraction donc dilat)
  \item préservée (parois épaissies)
\end{itemize}

\subsection{Diagnostic}
Fonctionnels :
\begin{itemize}
  \item respi : dyspnée (cf classification NYHA [I à IV]), d'effort, orthopnée, dyspnée paroxystique nocturne
  \item \danger{} asthme, toux, hémoptysie aussi 
  \item autres : fatigue (repos/effort), faiblesse musculaire, palpitations
  \item si sévère : respi, neuro, digestif
  \item IC droite : hépatalgie
\end{itemize}
Physique : pauvre donc des signes sont facteur de gravité
\begin{itemize}
  \item cardiaque : palpation : choc de pointe en bas à gauche\\
    auscult : tachy, galop B3, éclat B2 en pulmonaire, souffle d'insuf
    mitrale/tricuspide, souffle de valvulopathie
  \item pulmonaire : râles (sous-)crépitants, épanchement pleural
  \item artériel : pouls rapide. si PAS < 100mHg, facteur de gravité
  \item \textbf{signes périphériques dIC droite} : turgescence jugulaire, reflux
    hépatojugulaire, hépatomégalie, oedèmes périph, ascite
\end{itemize}

\textit{ECG}  peu contributif.

\textit{RX thorax}  :
\begin{itemize}
  \item cardiomégalie (RCT > 0.5)
  \item stase pulmonaire :
    \begin{itemize}
      \item redistribution vasc de la base vers sommets
      \item oèdème interstitiel (ligne B de Kerley, gros vaisseux hilaires flou,
        réticulo-nodulaire aux base)
      \item oedème alvéolaire ("ailes de papillons")
    \end{itemize}
  \item épanchement pleural
\end{itemize}

Bio : Na+, K+, créat, héptaique, TSH us, NFS, fer

\textit{Dosage (NT-pro)BNP}  : intéressant si normaux ou très élevés

\textit{ETT} : indispensable ! Peut orienter : ischémique, valvulopathie,
hypertrophie

Autres :
\begin{itemize}
  \item coronarographie : important ! Revsc ou peut orienter vers cardiomyopathie
    dilatée
  \item IRM cardiaque : peut compléter échocardio (mesure FE)
  \item scintigraphie : mesure FE
  \item Holter : troubles du rythme V ou supra-V
  \item Épreuve d'effort
  \item Cathétérisme : mesure pression pulmonaires, débit cardiaque
\end{itemize}

\subsection{Étiologies}
Toute patho cardiaque peut donner une IC....

\begin{itemize}
  \item Cardiopathies ischémique : 1ere cause ! Souvent plusieurs IDM
  \item HTA : hypertrophie, développe coronaire
  \item Cardiomyopathies : dilatées (25\% familiale), hypertrophique (surtout
    familiable), restrictive (rare)
  \item Valvulopathies : gauche
  \item Troubles rythme (supra)-ventriculaire
  \item Péricarde
  \item IC droit : conséquence IC gauche ou siolé (patho pulmonaire, HTA
    pulmonaire...)
  \item À débit augmenté
\end{itemize}

\subsection{Formes cliniques}
\paragraph{ Insuf aigüe}
OAP sutout : détresse respi aigüe (inondation alvéolaire) 
\begin{itemize}
  \item polypnée, orthopnée
  \item sueurs, anxiété, cyanose, grésillement laryngé, toux + expect mousseuse
    rose saumon
  \item râles crépitant
\end{itemize}
\thus PEC immédiate \skull
   
Choc cardiogénique possible : < 85mmHg, extrémités froides, marbrures, oligurie

Toujours chercher facteurs favorisants : rupture traitement, surinfection
bronchique, troubles du rythme, anémie, EP, dysthyroïdes, iatrogène, poussée
hyppertensive

\paragraph{Autres}
Chronique

Fonction systolique préservée : 50\%

\paragraph{Complications}
\begin{itemize}
  \item DC : 50\% à ) ans
  \item IC aigǜe avec hospit
  \item troubles rythme (supra)ventriculaire
  \item thromboemboliques
  \item hypotension artérielle
  \item troubles hydroélectrolytiques, insuf rénale
  \item anémie, carence martiale
\end{itemize}

\subsection{Traitement IC chronique}
\paragraph{Étiologie}
FR, revasc

\paragraph{Hygiène}
< 5g de sel, conserver poids, pas de tabac, diminuer alcool, activité physique,
pas d'efforts importants au travail, vaccins : grippe (âgé), pneumocoque,
contraception pour éviter grossesse

\paragraph{Médicaments}
\begin{enumerate}
  \item Diurétiques (congestion) + IEC (diminue angiotensine II) + betabloquants
    (\skull pas immédiatement si crise aigüe \danger{}) $\pm$ antagonistes des
    récepteurs aux minéralo-corticoïdes (diurétiques)
  \item Si échec : + ivabradine (diminue FC)
  \item Si échec : défibrillateur automatique (+ resynchronisation si QRS >
    120ms)
  \item si échec : digoxine (n'améliore pas la survie) ou nitrés (vasodilatateur), voire greffe/assistance
\end{enumerate}

NB : IC à FE conservée : bloquer régine-angiotensine, ralentir FC, diuértiques

\subsection{Traitement IC aigüe}
OAP : 
\begin{itemize}
  \item domicile : assis, furosémide IV, dérivés nitrés, SAMU (?)
  \item hôpital : assis, apport IV G5\%, $O_2$, furosémide, dérivés nitrés si
    PAS < 100 mmHg, (morphine), HPBM systémique !
\end{itemize}
Poussée sans OAP france : (hôpital), diurétique VI, rééquilibration traitement,
cause ??

Choc cardiogénique : sonde urinaire++, inotropes (ex: dobutamine)

\section{222 : HTA pulmonaire artérielle}%
\label{sec:hta_pulmonaire_arterielle}

Classif des HT pulmonaires : 
\begin{enumerate}
  \item HTAP
  \item maladies cardiaques (seules post-capillaires)
  \item appareil respi \lor{} hypoxémiantes
  \item thromboemboliques chroniques
  \item mécanismes multiples/complèexes
\end{enumerate}
\subsection{Adulte}
Rare, Pas de cause connue. Plutôt àfemale, 40-50ans. Traitement à vue, pas curatif

Épaissisement des paroi artères/artérioles, perte vasodilatation \thus \inc résistance
Augmentation des résistances des art

Définition : 
\begin{itemize}
  \item PAPm\footnote{Pression pulmonaire moyenne} $\ge 25$ mmHg 
  \item PAPO\footnote{Pression artérielle pulmonaire d'occlusion, $\approx$ pression
oreillette G} $\ge 15$mmHg
\item débuit cardiaque normal/diminué
\end{itemize}

\paragraph{Clinique}
\begin{itemize}
  \item Dyspnée++, très progressive (attention sous-estimation). 
  \item puis asthénie, fatigabilité à l'effort, douleurs thoraciques angineuses,
    palpitations, lipothymies à l'effort
  \item insuf. cardiaque droite : hépatalgie, oedème membres inférieurs
\end{itemize}
Interrogatoire : ATCD, anorexigènes, toxiques, maladie (sclérodermie : sd de
Raynaud, dysphagie, dyspepsie)

Y penser si clinique normale !

CV : signes d'HT (non spécifiques), insuf cardiaque droite
\paragraph{Complémentaires}
\begin{itemize}
  \item Radio thorrax normale (DD: fibrose, bronchoemphysième chronique,
    sarcoïdose)
  \item EFR : normales
  \item gazométrie artérielle : normale (ou hypercapnie)
  \item ECG: normal ou \{grande onde P, dév. axiale droite, BdB droit, trouble
    repolarisation\}
\end{itemize}
\textit{Échocardio}  = examen clé, 

\textit{Cathétérisme cardiaque droit}  = confirmation

\subsection{Enfant}

Augmentation du débit ou des résistances

\paragraph{Clinique}
Nouveau-né : cyanose réfractaire $O_2$, détresse respi/circ \thus échocardio en
urgence \danger

Enfant : diagnostic seulement stade très avancé : dyspnée d'effort, syncope,
fatigue
\thus dépistage si cardiopathie congénitale, patho. respi. chronique, maladie de
systèmes, ATCD familiaux

\paragraph{Examens}
Radio thorax, ECG aspécifiques.

Échocardio : diagnosic et cherche cardioapathie. Cathétérisme confirme.

Scanner (parenchyme), écho hépatiques (shunt porto-cave)

\section{199 : Dysnée aigüe et chronique}%
\label{sec:199_dysnee_aigue_et_chronique}
Difficulté respi, fréquent++

Sémiologique : Terrain, rapidité, circonstance, intensité (classif NYAH), nombre d'oreillers

Clinique : \{inspi/expi/les 2\}, \{tachy (> 20), brady (<10)\}, rythme
(Kussmaul, Cheynes-Stokes), intensité (poly/hypo/oligopnée)

\subsection{Aigüe}
\textit{Examens}  : NFS, ECG, radiothorax, gazométrie artérielle, D-dimère, BNP

\textit{Orientation} 
\begin{itemize}
  \item Signes de choc ? Traiter
  \item Sinon, dyspnée inspiratoire ? Si oui, laryngé : 
    \begin{itemize}
      \item brady inspiratoire, cornage, tyrage
      \item (corps étranger, épiglotite, laryngite) chez l'enfant
      \item trachéal
    \end{itemize}
  \item Sinon  dyspnée expiratoire ? Si oui : 
    \begin{itemize}
      \item crise asthme : sibilants. À distinguere de l'asthme aiue grave
        (urgence \danger)
      \item exacerbation BPCO : ATCD, sibilants, hippocratisme digital (\danger
        EP possible)
      \item pseudo-asthme
    \end{itemize}
    cardiaque
  \item Sinon, crépitants ? Si oui, 
    \begin{itemize}
      \item OAP : orthopnée, wheezine, crépitants bilat, terrain, RX
        (cardiomégalie, oèdeme alvéolaire), ECG (Q nécrose, FA), BNP \inc
      \item pneumopathie infectieuses : sd infectieux, fièvre, toux, exet
        purulente, crépitants, RX : opacité parenchymateuses systématisée
    \end{itemize}
  \item Sinon, sd pleural ? Si oui, pneumothorax, épanchement pleural (asymétrie
    ascult, RX)
  \item Sinon : 
    \begin{itemize}
      \item EP : fréq++, brutale, auscultations normales, gazométrie : effet
        shunt
      \item sepsis sévère, acidose, anémie, neuromusc
    \end{itemize}
  \item Sinon psychogène
\end{itemize}

Autres :
\begin{itemize}
  \item cardiaque : tamponnade (orthopnée, tachy, assourdissement bruits, ascult
    pulmonaire normale, turgescene jugulaire, pouls paradoxal), troubles
    (supra)-ventriculaire, choc cardiogénique
  \item pulmonaire : SDRA, décompensation aigùe, atélectasies, trauma
\end{itemize}

\subsection{Chronique}
Examens :NFS, EFR, échocardio, radiothorax, effort, scanner thoracique,
cathétérisme

\begin{itemize}
  \item Cardiaque : ECG, radiothorax, BNP (VPP !)
    \begin{itemize}
      \item insuf cardiaque
      \item constriction péricardique
    \end{itemize}
  \item Pulmonaire :
    \begin{itemize}
      \item obstructif : BPCO, asthme à dyspnée continue
      \item restrictif : PID, pneumoconioses, post-tuberculose, paralysie phrénique,
        cyhoscoliose, obésité morbide
    \end{itemize}
  \item HTAP
  \item HT pulmonaire post-embolique
\end{itemize}





\section{224 : Thrombose veineuse profonde et embolie pulmonaire}%
\label{sec:224_thrombose_veineuse_profonde_et_embolie_pulmonaire}
Complications de la TVP\footnote{Thrombose veineuse profonde} :
EP\footnote{Embolie pulmonaire}(précoce), SPT\footnote{Syndrome post-thrombotique}
(tardif)

TVP : obstruction thrombotique d'un tronc veineux profond. EP : idem mais
artères pulmonaires ou leurs branches (secondaires TVP à 70\%)

3eme cause de DC en france

Facteurs de risques temporaires (chirurgie orthopédique, trauma, alitement > 3
jours), permaments (ATCD MTEV\footnote{Maladie thromboemolique veineuse}

\paragraph{Physiopatho} stase veineuse, lésions pariétales, anomalies de l'hémostase \thus
obstruction/lyse. Si migre dans les artères pulmonaires : symptômes si
obstruction à 30-50\% \thus anomalie hémodynamique \thus insuf respi

\paragraph{Évolution}
TVP distales asymptomatique (postop) : 20\% deviennent proximales\\
TVP distales symptomatique : récidive (9\% si ttt anticoagulant)\\
TVP proximal symptomatique : risque important !\\
EP : TVP + 3/7 jours, mortelle dans l'heure à 10\%

\subsection{Thrombose veineuse profonde}

\paragraph{Clinique} = orientation\\
Douleur du MI, oedème unilatéral, signes inflammatoires, dilatation veines
superficielles ou asymptomatique

\textit{DD}  : \{traumatisme, claquage musc\}, kyste synovial, \{SPT, insuf veineuse
primaire\}, \{sciatique, compression extrinsèque\}, \{érysipèle, lymphangite,
cellulite\}, lymphoedeme, insuf cardiaque droite/rénale/hépatique

Score de probablitié clinique : Wells (faible/interm/forte)

\begin{algorithm}
  Si âge < 80 ans, pas de cancer, pas de chirurgie < 30 jours, D-dimères.
    Si < 500$\mu$g/L, \faHandStopO. Sinon, goto 2.\;
  Contention élastiques (+HBPM si proba clinique $\ge$ intermediate)\;
  Puis écho-doppler veineux\;
  Si positif : ttt et contention. Si négatif mais suspicion forte :
    surveillance $pm$ ttt
\caption{Diagnostic TVP}
\end{algorithm}
NB : TVP va recherche le thrombus, l'incompressibilité de la veine à la
pression, \dec signal, remplissage partiel du throbus.

\paragraph{Étiologies}
\begin{itemize}
  \item FR transitoire (chir, fracture < 3 mois, immobilisation > 3 j) ? Sinon, "non
provoquée
\item Recherche de thrombophilie si (1er épisode (non provoqué < 60 ans) \lor{}
  femme âge procrééer)) \lor{} (récidive (MTEV proximale) \lor (TVP distale non
  provoquée)
\item Recherche cancer : > 40 ans ou bilan thrombophilie négatif
\end{itemize}

Formes particulières :
\begin{itemize}
  \item thromboses veineuse superficielles :x(sur trajet saphène, douloureux,
    rouge, inflammatoire, cordon induré \thus écho-doppler
  \item TVP pelvienne
  \item thrombose veine cave inférieure
  \item phlébite bleue : très rare mais grave
\end{itemize}

Si grossesse : bilan thrombophilie si ATCD familaux/personnels MTEV

Si cancer : HBPM long cours

\paragraph{Évolution} : favorablement si bien conduit mais récidive toujours
possible. Complications : 
\begin{itemize}
  \item SPT : lourdeur de jambes, oedeme de cheville, dilat veineuses
    superficielles, troubles trophiques sans ulcère, ulcères sus-malléolaires
  \item EP
\end{itemize}

\subsection{Embolie pulmonaires}

\paragraph{Clinique} pas spécifique : dypsnée, douleur thoracique, syncope,
crachats hémoptoïques, asymptomatique

Radiothorax : aspécfique. 

Gaz du sang : hypoxie-hypocapni

ECG : normale /tachy sinusale / souffrance VD

\thus score probablitié (Genève, Wells)

\begin{algorithm}
  \caption{EP supectée à haut risque}
  Si scanner non disponible, ETT\;
  Si positif et scanner disponible, goto 3. Si positif et scanner indisponible,
  traitement. Si négatif : \faHandStopO\;
  Si scanner positif : traitement spécifique. Sinon \faHandStopO\;
\end{algorithm}

\begin{algorithm}
  Proba clinique faible/intermédiaire : D-Dimère. Si positif, goto 2\;
  Si scanner ou écho MI positif, traitement\;
  \caption{EP supectée non haut risque}
\end{algorithm}

NB : 
\begin{itemize}
  \item si suspicion d'EP, TVP \textbf{proximale} à l'echo des MI suffit au
diagnostic.
  \item Scinti : si normale, pas d'EP
  \item ETT : en urgence, signe de surcharge peut suffire à poser le diagnostic
\end{itemize}

\textit{DD} :
\begin{itemize}
  \item douleur thoracique : IDM, péricardiite, dissection aortique,
    pneumothorax
  \item dyspnée aigüe : oedème aigu pulmonaire, crise d'asthme, décompensation
    BPCO, pneumopathie
\end{itemize}

\paragraph{Pronostic}
\begin{itemize}
  \item Grave si choc/hypotension (PAS < 90mmHg ou -40mmHg > 15 min sans cause
rythmique/hypovélimuqe/septiques) \thus DC > 15\%
\item intermédiaire : 3-15\%
\end{itemize}

\paragraph{Évolution} favorable. Complications : choc cardiogénique réfractaire
(DC), rédicive, HTAP chronique post-embolique (rare mais grave)

\subsection{Traitement curatif}
Confirmer diagnostic avant traitement antiocagulant, sauf si absence risque
hémorragique en attentant confirmation

\begin{itemize}
  \item HNF : 500UI/kg/jour puis selon test d'hémostase. Pour insuffisants
    rénaux sévères ou instables
  \item HBPM, fondaparinux : pas de surveillance bio
\end{itemize}
Surveillance des plaquettes \textbf{non si}  : HBPM hors contexte post-op,
fondaparinux 

Thrombolytique : seulement si EP grave !

Relai per os AVK précoce. Objectif INR = 2.5

Durée : $\ge$ 3 mois si TVP proximale ou EP

Compression élastique $\ge 2$ ans, ASAP. 

Hospitalisation ? 
\begin{itemize}
  \item EP
  \item TVP proximale chez insuf rénaux sévère, risuqe hémorragique, syndrome
    obstructif ou localisation iliocave
\end{itemize}

Filtre cave : non systématique

\paragraph{Cas particuliers}
\begin{itemize}
  \item 
TPV distale : symptomatique : anticoagulement 6 semaines seulement si premier épisode. Dans tous les cas,
compression $\ge 2$ ans
\item TVS : \textbf{pas} recommandé = AINS, antiocoagulants en curatif, chirurgie
\item cancer : HBPM, arrêt si plaquettes < 50g/L
\end{itemize}
Prevention : 
\begin{itemize}
  \item risque modéré : HBPM/HNF/fondaparinux, compression
  \item risque élevé : idem sans HNF
\end{itemize}

Nouveau anticoagulant oraux (rivaroxaban) :
\begin{itemize}
  \item prévention chir hanche-genou
  \item curatif TVP, EP
  \item pas si insuf rénale sévère ou insuf hépatique
  \item pasa de surveillance bio
\end{itemize}

\section{221 : Hypertension artérielle}%
\label{sec:221_hypertension_arterielle}
Grades (PA systolique, PA diastolique (PAS, PAD)):
\begin{itemize}
  \item 1 : 140/90 - 159/99
  \item 2 : 160/100 - 759/109
  \item 3 : > 180/110
  \item systolique isolé : > 140 et < 90
\end{itemize}

HTA modérée = plus fréquente\\
\inc avec l'âge. Plus fréquent chez femmes, noirs, obèse, consommation sel,
défavorisé. Génétique 30\%.

Risque : PAS, PAD sauf après 60 ans ! PAS, Pression pulsée = PAS - PAD

\paragraph{Physiopatho}
Régulation :
\begin{itemize}
  \item court : sympathique 
  \item moyen : rénine-angiotensine,-aldostérone et peptide natriurétique (ANP,
    BNP)
  \item long : natriurièse de pression, arginine-vasopressine
\end{itemize}

90\% sont essentielles.

\subsection{Évolution}
\paragraph{Complications}
\begin{itemize}
  \item 
\textit{Neuro}  : AVC ischémique, hémorragie cérébrale/méningée, encéphalopathie
hypertensive, lacune cérébrale, démence vasculaire, rétinopathie hypertensive
\item \textit{CV}  : insuf cardiaque systolique, insuf VG (anomalie de
  \textit{remplissage} , cardiopathie ischémique, fibrillation atriale, arythmie
  ventriculaire, complications artérielles. Mortalité CV x5 (\male) ou x3 (\female)
\item \textit{rénales} : évolution vers l'insuf rénale par néphroangioscélore,
  sténose athéromateuse de l'artère rénale, diurétiques, IEC
\item rénale
\end{itemize}

\paragraph{Urgences hypertensives} : HTA sévère \land{} atteinte aigüe des organes
\begin{itemize}
  \item Urgences : SCA, insuf VG, dissection aortique, encéphalopathie
    hypertensivee, hémorragie méningée/AVC, phéochromocytome, \{amphétamines, 
    LSD, cocaïne\}, péri-op, (pré)éclampise, sd hémolytique et urémique
  \item examens : bio, ECG, radiothorax, échocardio, FO\footnote{Fond d'oeil},
    scanner cérébrale, amagerie aortique
\end{itemize}

\paragraph{HTA maligne} rare. Hypovolémie (natriurèse). \\
Tableau : PAD > 130mmFg, oèdeme papillaire au FO, insuf VG, insuf rénale
aigüe++. \\
Évolution en quelques mois++

\subsection{Bilan inital}
Mesure de la pression : 
\begin{itemize}
  \item ascult (attention : effet bouse blanche, HTA masquée, rigidité des
    artères si âgé)
  \item MAPA (24h, toutes 15min), obj < 130/80 en 24h
  \item automesure 
\end{itemize}
Évaluation :
\begin{itemize}
  \item interrogatoire : \{ancienneté, FR, organes (cerveaux, yeux, coeur rein,
    artères)\}, \{secondaire (médica : contraceptifs oraux++, AINS++...)\}
  \item clinique : atteinte des organes, secondaire, obésité
  \item complémentaires : gylcémie, cholestérol (total, HDL, trygylcéride, LDL),
kaliémie, créat, BU, ECG repos)
\end{itemize}
Atteintes spécifiques :
\begin{itemize}
  \item coeur : ECG (HVG, hypertrophie atriale G), échocardio (HVG = masse VG >
    125g/$m^2$ (110 chez \female)
  \item écho carotide (AOMI : IPS < 0.9)
  \item rein: 
  \item FO : stade 3 (hémorragie, exsudats), 4 (oedème papillaire)
\end{itemize}

\paragraph{Calcul du risque}
Élevé si grade 3 \lor{} ((grade 1 ou grade 2) \land{} $\ge 3$ FR)\\
FR : âge, tabac, ATCD familaux d'accident CV précoce, diabète, dyslipidémie

\subsection{HAT secondaire}
Dépistage : point d'appel, grade 3, < 3* ans, HTA résistante

\begin{itemize}
  \item néphropathie parenchymateuse : palpation de masses abdo bilat \thus écho
    abdo et créat, protéinure, sédiment urinaire
  \item HTA rénovasculaire : clinique (souffle abdo lat, OAP récidivant sans
    explication), bio (hypoK et hyperaldostéronisme ou insuf rénanle). Diag
    écho, confirmé par angio-IRM. \\
    Ttt : hygiénodiététique, aspirine, statine, antihypertenseurs
  \item phéochromocytome : triade de Ménard (céphalée, sueurs, palpitations).
    Dosage urinaire métanéphrines, catécholamines\\
    Localisation tumeur (IRM), $\alpha$ et $\beta$ bloqueurs, exérèse chir
  \item Sd de Conn : dépisté par kaliémie. \inc aldostérone/rénine. Localisation
    tumeaur scanner/IRM\\
    Chir éventuelle
  \item Coarctation aortique (enfant, adultee jeune) : clinique, confirmée IRM.
    Ttt chir ou endoluminal
  \item SAS : obèse, si HTA résistante. Diag par polysomnographie
  \item Médicaments : cocaïne, amphétamines, AINS, corticoïdes, ciclosporine,
    contraceptifs oraux, réglisse
  \item HTA gravidique : 16-22 SA. Primipare, noires ou obèses.\\
    DD : HTA > 20 SA, grossesse chez hypertendue, prééclampsie\\
    Complications : éclampise++, rénale, cardiqaue, RCIU, HELLP\\
    \textbf{Pas} régime sans seil, ni IEC, ARA II, diurétiques
\end{itemize}

\subsection{Traitement}
Objectif : < 140/90\\
Si grade 3 ou (grade 1/2 et $\ge 3$FR)

Hygiénodiététique, traitement FR, éducation.

\paragraph{Médicaments}
\begin{itemize}
  \item inhibiteurs calciques
  \item IEC
  \item antagonistes récepteurs angiotensine II
  \item btea-bloqueurs
  \item diurétiques thiazidique
\end{itemize}
\danger : pas de beta-bloquers ni diurétique si FR métaboluqes.\\

En association la plupart du temps

Éventuellement :
\begin{itemize}
  \item antiagrégant : aspirine 75mg/j si ATCD CV ou créat ou risque CV élevé
  \item hypolipidémiant : diabétique 2 ou maladie CV ou haut risque CV
\end{itemize}

Autres :
\begin{itemize}
  \item HTA résistante si $\ge 3$ classes médicaments dont 1 diurétiques
  \item âgé : attentio hypotension orthostatique, \textbf{pas} régime sans seul,
    $\ge 3$ médic, objectif : PAS > 150\\
    ttt : diurétique, calcium bloqeurs
  \item Urgence : éviter baisse brutale tension et hypotension \skull
\end{itemize}
\section{225 : Insuffisance veineuse chronique}%
\label{sec:225_insuffisance_veineuse_chronique}

Rappel : réseau superificel = veine saphène interne, saphène externe\\
Insuffisance veineuse quand dysfonction du retour, soit par valvules, soit par
péristaltisme (contraction musculaire, écrasement voûte plantaire)

\begin{itemize}
  \item reflux dans le réseau superficiel = \textit{varices}  (>
    3mm\footnote{$\neq$ télangiectasies < 1mm}) : essentielles ou
    secondaires
  \item \textit{post-thrombotique} : reflux (destruction valvulaire) \lor{} obstruction par thrombose. 
  \item \textit{Insuf valvulaire profonde primitive} : rare
  \item \textit{Déficience pompe musculaire mollet} 
\end{itemize}

\paragraph{FR} Varices : âge, ATCD, obésité, grossesse, \female\\
MTEV : immobilisation, cancer, anomalies hémostase

\paragraph{Symptômes} jamboes foulders, aggravé par station debout prolongéé,
fatigue vespérale, chaleur. Calmée par le froid, marche, surélévation

\paragraph{Clinique}
\begin{itemize}
  \item dermite ocre
  \item télangiectasies, veines réticulaires (plantaire, malléole)
  \item oedème cheville
  \item lipodermatosclérose
  \item atrophie blanche
  \item varices
  \item ulcère veineux (stade ultime) : périmalléolaire, oval, peu algique, peu
    creusant, non nécrotique, exsudatif
\end{itemize}
NB : classif CEAP (clinique, étio, anat, physiopatho)

Examens : \textit{écho-doppler}  veineur des MI

\subsection{Traitement}
\begin{itemize}
  \item Compression élastique (bas, bandes) : à vie si chroniqu. Attention CI si AOMI
  \item Hygiène de vie
  \item Invasif : sclérothérape, traitement endoveinux, chirurgie (conservatrice/exérèse), (recanalisation)
\end{itemize}
NB : CI à la chir = sd obstructif veineux profond

\section{233 Péricardite aigüe}%
\label{sec:233_pericardite_aigue}

\subsection{Diagnostic}
2 parmi
\begin{itemize}
  \item douleur thoracique : résistant trinitrine, majorée décubits, calmée par
    l'antéflexion
  \item frottement péricardique
  \item ECG typique : 4 stades = \{ sus-ST, T plates, T négatives,
    normalisation\}. Aussi : sous-PQ, tachycardie sinusale, microvoltage
  \item épanchement péricardique
\end{itemize}
Plus (parfois) : fièvre modérée, dyspnée, épanchement pleural

Examens complémentaires (hors ECG) :
\begin{itemize}
  \item bio : inflammation, nécrose, iono urée, créat, hémoc si fièvre
  \item radiothorax normale (rectitude gauche ou cardiomégalie)
  \item échocardio : épanchement péricardique, masse péricardique
\end{itemize}
Parfois : IRM (2eme intention), ponction ou drainage

Hospit ? si étiologie ou risque (symptôme > plusieurs jours)

\paragraph{Étiologies} 90\% virale ou inconnue
\begin{itemize}
  \item aigùe virale : fréq++, sérologie inutile si typique sans gravité.
    Évolution favorable (tamponnade/constriction péricard rare).\\
    Pendant VIH
  \item purulente : rare mais grave. Immunodéprimé, infections sévère.\\
    Évolue vers tamponnade/constriction péricardique\\
    ATB
  \item tuberculeuse : AEG, fièvre modéré peristante. Sujet tuberculeux, âgé,
    greffé, VIH, alcoolique.\\
    BK, PCR\\
    Évolue vers tamponnade/constriction péricardique\\
    ABT
  \item néoplasique : souvent métastase. Échocardio\\
    Ponction/biopsie essentielle. Récidive fréquente
  \item systémique auto-immunes : lupus, polyarthrite rhumatoïde, sclérodermie,
    pérartérite noueuse, dermatomyosite
  \item IDM : précoce (5j) : favorable, tardive (2-16sem) : sd Dressler
  \item insuf rénale chronique : urémique ou dialyséa au long cours
  \item sd post-péricardotomie : post chir ou greff cardiaque.
\end{itemize}

\subsection{Complications}
\begin{itemize}
  \item Tamponnade : compression des cavités D par épanchement.\\
    Clinique : douleur thoracique, signes droits, choc, bruits coeur assourdis,
    pouls paradoxal. \\
    ECG, radiothorax, échocardio ("swining heart")
  \item Myocardite : insuf cardiaque fébrile. Échocardio, IRM+++
  \item Péricardite récidivante : colchicine ?
  \item Péricardite chronique > 3 mois : souvent népolasique
  \item Péricardite chronique constrictive : adiastolie avec égalisation
    pression télédiastoliques\\
    INsuf cardiaque droite et gauche. Diago écho. Chir ?
\end{itemize}

\subsection{Traitement}
Bénigne : hospit, repos, douleur, AINS ou aspirine, protection gastrique,
colchicine (CI si insuf rénale sévère)

Tamponnade : urgence \skull, remplissage, ponction/drainage

\section{327 : Arrêt cardiocirculatoire}%
\label{sec:327_arret_cardiocirculatoire}

Mort subite : après symptômes < 1h. Arrêt cardiorespi (ACR) : plus d'activité
mécanique cardiaque\\
"No flow" : intervalle sans rénimation = \textbf{important} 
"Low flow" : intervalle sans rétablissement HD

Chaîne de survi : alerte, réa (RCP), défibrillantio, RCP spéc

> 10 min de fibrillation/arrêt : < 5\% de récupérations

\subsection{Étiologies}
Fibrillation ventriculaire puis brady extrême puis asystolies
Autres :
\begin{itemize}
  \item SCA inauguraux (40-77\=)
  \item autres CV : troubles rythme sur ischémie ancienne, hypertrophique,
    trouble du rythme/conduction indépendant, tamponnade, dissection aortique...
  \item non vasc : toxique, traumatique, insu respi aigu€, noyades
\end{itemize}

\paragraph{Diagnostic} ne bouge plus, ne réagit plus, ne respire plus, plus de
pouls

\subsection{CAT}
ABCD : maintien voies Aériennes, assistance respi (B), Circulation (massage
cardique), Défibrillation et Drogue

\paragraph{Médicaments}
\begin{itemize}
  \item Adrénaline (vasoconstricteur) : 1mg/4min, avant 2eme choc
  \item Antiarythmique (après 2eme choc): amiodarone 300mg dans 30mL de sérum
    salé isotonique (lidocaïne sinon), (sulfate
    de magnésium)
  \item Bradycardie : atropine (isoprénaline)
  \item (bicarbonate de sodium équimolaire pour alcaliniser)
  \item (thrombolyse)
\end{itemize}

Survie -10\% par minute sans réa

Asystolie = marque arrêt ancien (FV peut évoluer en asystolie)

\paragraph{PEC}
3 phases :
\begin{itemize}
  \item < 12h : acidoses, radicaux libres, enzymes musc
  \item 12h - 3eme jour : atteinte organes
  \item  > 3 j : sd septique
\end{itemize}

Cardiaque : dobutamine (trouble contractilité), monitoring échocardio $\pm$
coronaro/angioplastie

Cerveau :
\begin{itemize}
  \item $O_2$, ventilation importants ! (éviter hypertension intracrânen
    fatable)
  \item sédation 24-49h
  \item éviter hyperglycémie !
  \item hypothermie à 34 degré
\end{itemize}

\section{264 : Diurétiques}%
\label{sec:264_diuretiques}

\subsection{Modes d'actions}
\paragraph{Diurétiques de l'anse} Furosémide

Inhibe réabsorption Na+, K+, Cl- branche ascendante anse de Henlé

Action rapide et courte. Vasodilatation veineuse (utile pour OAP)

\paragraph{Thiazidiques} Hydrochlorothiazide

Inhibe réabsorption NaCl au segment proximal du tube contourné distal. Augment
excrétion K, Cl mais diminue Ca+.

Inefficace si insuf rénale sévère

\paragraph{Épargnants le potassium} spironolactone

Diminue excrétation K+ , Hl au tube contourné distal. Effet inférieur au 2
autres

\subsection{Indications}
HTA : 
\begin{itemize}
  \item toujours un diurétique
  \item doses faibles
  \item spironolactone seulement si hyperaldostéronisme primaire
  \item si insuf rénale sévère seulement diurétique de l'anse
\end{itemize}
Insuf cardiaque :
\begin{itemize}
  \item furosémide++
  \item dose minimale efficace
  \item aigü : diurétique de l'anse 
\end{itemize}

\subsection{Prescription}
Toujours régime hydrosodé

\paragraph{Effets secondaires}
Hydroélectriques 
\begin{itemize}
  \item déshydratation : âgé, peut entrainer IR aigue
  \item hyponatrémie 
  \item hypokaliémie : fréquene, modérée, surveillance, apport alimentaires
    !
  \item hyperkaliémie : dangereuse \danger{} bradycardie sévères, torubles
    rythme ventriculaire. Favorisé par IEC, ARA II, insuf rénale
  \item hypovélime : parfois hypotension orthostatique
\end{itemize}
Autres : 
\begin{itemize}
  \item hyperglycémie, hyperuricémie (anse, thiazidiques)
  \item augmentation choléstérole (thiazidique)
  \item gynécomastie (spironolactone)
  \item ototoxicité (anse)
  \item interaction : lithium, hypokaliémiant, AINS
\end{itemize}







\section{326 : Antithrombotiques}%
\label{sec:326_antithrombotiques}


Cliniques usuelles : 
\begin{itemize}
  \item Bifasciculaire du sujet âgé avec perte de conaissance : BdB droite et
    hémibloc antérieur gauche. Chercher cardiopathie sous-jacente. souvent
    endocavitaire.
  \item BdB gauche de l'infarctus antérieur
  \item Bloc alternant : (BdB droit III \land{} gauche III) \lor{} (alternance
    entre les 2 hémibloc avec BdB droit III)
\end{itemize}
\subsection{Antiagrégants plaquettaires}
\paragraph{Aspirine}
Inhibe Cox1. Irréversible. Effets antalgique, anti-inflammatoires (doses plus
fortes), anticancéreux

Posologie : 300mg + 75mg/j.

Indications : 
\begin{itemize}
  \item prev. secondaire : coronaropathie, artériopathie des MI, AVC (à
vie)
\item  prev. primaire : coronaropathie, AVC
\end{itemize}

EI : saignements, intolérances gastriques (d'où IPP)

Situations à risque : attendre 6 semaines/3-6 mois tout acte invasif à risque
hémorragique. Si risque très important, arrêt 5 j seulement.

\paragraph{Thiéonpyridines, ticagrelor}
Bloque récepteur P2Y12
\begin{itemize}
  \item Clopidogrel : 300-600mg + 75mg/j. SCA et post-angioplastie coronaire
    (avec aspirine)
  \item Prasugrel : 60mg + 10mg/j. SCA post-angioplastie 
  \item Ticagrelor : 180mg + 90x2mg/j. SCA 
\end{itemize}

CI absolue : prasugrel si ATCD accident cérébral.

\paragraph{Autres}
\begin{itemize}
  \item Anti-GPIIb-IIIa : voie veineuse, courtes périodes, conditions très particulières
  \item Dipyridamole : plus utilisé
\end{itemize}

\subsection{Héparines}%
\begin{itemize}
  \item HNF : IV, effet immédiat. Antidote : sulfate de protamine. 80 UI/kg puis
    18 UI/kg/h \\
    \textbf{Surveiller TCA !} \danger
  \item HPBM : demi-vie plus longue. Attention au rein !!. 100 I/kg x2/j
  \item fondaparinux : demi-vie plus longue. Attention au rein !!
\end{itemize}

Indications : pour anticoag urgente = thromboses veineuses profondes, EP,
troubles du rythmes, SCA.

EI : complications hémorragiques, TIH

Apparentés :
\begin{itemize}
  \item danaparoïde : anticoag si passif de TIH
  \item bivalirudine : pour angioplastie coronaire
\end{itemize}

\subsection{Anti-vitamines K}
Agit sur facteurs dépendant de vitamine K. Oral, traitement longue durée.

Coumadine (ref), fluindione = demi-vie longue. Acénocoumarole = demi-vie courte.

Utiliser l'héparine en relais des AVK avec 4-5j de chevauchement et 2 INR
efficaces à 24h

\textbf{Surveiller INR}  1/mois \danger

Antidotes : PPSB, vit. K

Indications : fibrillation atriale, TVP, EP, valve cardiaque mécanique,
complications de l'IDM/insuf cardiaque

Situations à risque ? Si très important :
\begin{itemize}
  \item arrêt 3-4j avant
  \item ou arrêt 4-5 j avant et relais héparine (pour TVP, EP > 3 mois ou FA
    risque embolique élevée ou valves mécaniques
\end{itemize}

\subsection{Nouveau anticoagulants oraux}
Dabigatran, rivaroxaban, apixaban. Action rapide (2h) mais pas d'antidote

CI : dabigatran si fonction rénale altérée. Suivre rein pour tous !!!

\subsection{Thrombolytiques}
Active la fibrinolyse physio. Suffixes : -kinase, -téplase

Indications spécifiques (IV) : 
\begin{itemize}
  \item IDM < 6-12h
  \item AVC < 4h30
  \item EP grave
\end{itemize}

\subsection{Accident anticoagulants}
\paragraph{Héparines}
Hémorragique : 
\begin{itemize}
  \item 1-4\%.
  \item clinique : TCA > 3 témoin, asymptomatique, anémie microcytaire
    ferriprive, hématome
  \item respecter prescription (CI si insuf rénale sévère)
  \item si accident majeur : continuer ? Antidote ? Remplissage IV dans tous les
    cas
\end{itemize}
Thrombopénies induites par les héparines : 
\begin{itemize}
  \item type I = bénin, non immun. Type II = grave, immun, +7/10jours.
  \item mécanisme thrombotique (!)
  \item clinique : plaquettes < 100G/L, thromboses veineuses/artérielles,
    résistance héparine, thrombose/thrombopénie juste après arrêt héparine
  \item CAT : confirmer, éliminer causes infectieuses, médicam, test ELISA
  \item puis arrêt héparine, danaparoïde sodique à la place (éventuellement
    AVK), NFS tous les jours, déclaration \textbf{obligatoire} 
  \item prévention : relais AVK, remplacer par HBPM ou fondaparinux
\end{itemize}

\paragraph{AVK}
Hémorragiques : 
\begin{itemize}
  \item AVK = 1ere cause d'hospit iatrogène !
  \item hémorragie grave : extériorisé non contrôlable, instabilité
    hémodynamique, geste hémostatique, transfusion de culots globulaires,
    pronostic vital/fonctionnelle
  \item CAT (si grave) : arrêt AVK, INR urgence et vitamine K + CCP (antidote),
    surveillance biologique
\end{itemize}


\printglossary
\printglossary[type=\acronymtype]

\end{document}

%%% Local Variables:
%%% mode: latex
%%% TeX-master: t
%%% End:
