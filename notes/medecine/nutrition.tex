\documentclass{article}

\usepackage{hyperref}
\input header
\graphicspath{{../../pictures/medecine/}}
\usegdlibrary{trees, layered, force}

\title{Nutrition\\
\large Fiches}
\author{A. Praga}

\begin{document}
\maketitle

\section{Glucides}%
\label{sec:glucides}

\tikz \graph [
  % Labels at the middle 
  edge quotes mid,
  % Needed for multi-lines
  nodes={align=center},
  sibling distance=3cm,
  edges={nodes={fill=white}}, 
layered layout]
{
  Glycolyse[draw] // [layered layout] {
    glucose + ADP -> pyruvate + ATP;
  };
  Néoglucogenèse[draw] // [layered layout] {
    acide aminés -> pyruvate;
    "(triglycérides du foie)" -> glycérol -> "glucose-6-phosphate";
    lactate -> pyruvate -> "glucose-6-phosphate" -> glucose;
  };
};

\tikz \graph [
  % Needed for multi-lines
  nodes={align=center},
  level distance=2cm,
layered layout]
{
  "Glycolyse anaérobie"[draw] // [layered layout] {
    glucose ->["muscles"] lactate ->["foie"] pyruvate;
  };
};

\section{Lipides}%
\label{sec:lipides}
Lipides = triglycérides (95\%), choléstérol, phospholipides, ester vitamines
solubles


\tikz \graph [
  nodes={align=center},
  layered layout
  ]
{
  // [layered layout, grow=right, level distance=3cm] {
    intestin [draw, circle] -> chylomicrons ->["LPL"] "remnants de\\chylomicrons" 
    -> foie[draw, circle];
  };
  // [spring layout, node distance=2cm] {
    foie -> VLDL -> IDL -> LDL -> "c. périphériques" -> HDL -> foie;
  };
};


\end{document}
