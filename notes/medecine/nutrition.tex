% Created 2019-09-08 Sun 11:19
% Intended LaTeX compiler: pdflatex
\documentclass[11pt]{article}
\usepackage[utf8]{inputenc}
\usepackage[T1]{fontenc}
\usepackage{graphicx}
\usepackage{grffile}
\usepackage{longtable}
\usepackage{wrapfig}
\usepackage{rotating}
\usepackage[normalem]{ulem}
\usepackage{amsmath}
\usepackage{textcomp}
\usepackage{amssymb}
\usepackage{capt-of}
\usepackage{hyperref}
\date{\today}
\title{État de l'art sur la nutrition}
\hypersetup{
 pdfauthor={},
 pdftitle={État de l'art sur la nutrition},
 pdfkeywords={},
 pdfsubject={},
 pdfcreator={Emacs 26.3 (Org mode 9.1.9)}, 
 pdflang={English}}
\begin{document}

\maketitle
\tableofcontents


\section{Transfatty acids}
\label{sec:org0bba6d2}
\subsection{cite:brouwer2016effectBrouwer}
\label{sec:orgd592857}
\begin{itemize}
\item source : WHO
\item objectif : étudie le remplacement des acides transsaturés
\item hypothèses : acides trans élèvés => augment lDL => risque CV élevé.
\item scope : pays développés (USA, pays européens)
\item méthodes : méta-analyses des articles entre 1990 et 2014. Régression
sur les données des articles
\item limites : pas mal de renormalisation pour ajuster les données entre les études
\item niveau de preuve haut
\item résultat : remplacer les acides trans par des monosaturés/polysaturés/saturé
carbs augment le HDL, diminue le LDL

\url{https://www.who.int/nutrition/publications/nutrientrequirements/healthydiet\_factsheet/en/}
\url{https://www.who.int/nutrition/topics/guideline-development/nugag\_dietandhealth/en/}
\end{itemize}
\end{document}
