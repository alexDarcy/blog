\documentclass{article}
\usepackage[hidelinks]{hyperref}
\usepackage{longtable}
\usepackage{booktabs}
\usepackage[draft]{graphicx}
\usepackage[draft]{graphicx}
\graphicspath{{../../pictures/medecine/}}
% French
\usepackage[T1]{fontenc}
\usepackage[francais]{babel}
%-------------------------------------------------------------------------------
% For graphs
\usepackage{tikz}
\usetikzlibrary{graphs}
\usetikzlibrary{graphdrawing}
\usetikzlibrary{arrows,positioning,decorations.pathreplacing}
\usegdlibrary{trees, layered, force}
\usetikzlibrary{quotes}
\usepackage{biocon}
%-------------------------------------------------------------------------------
% No spacing in itemize
\usepackage{enumitem}
\setlist{nolistsep}
% tightlist from pandoc
\providecommand{\tightlist}{%
  \setlength{\itemsep}{0pt}\setlength{\parskip}{0pt}}
 % Danger symbol (need fourier package)
\newcommand*{\TakeFourierOrnament}[1]{{%
\fontencoding{U}\fontfamily{futs}\selectfont\char#1}}
\newcommand*{\danger}{\TakeFourierOrnament{66}}
% Skull (need Symbola font)
\usepackage{amsmath,fontspec,newunicodechar}
\newfontface{\skullfont}{Symbola}[Scale=MatchUppercase]
\NewDocumentCommand{\skull}{}{%
  \text{\skullfont\symbol{"1F571}}%
}
% Hospital sign
\usepackage{fontspec} % For fontawesome
\usepackage{fontawesome}
% Diameter
\usepackage{wasysym}
% Chemical compound
\usepackage{chemformula}
% Footnote in section
\usepackage[stable]{footmisc}
% Custom header/footers
\usepackage{fancyhdr}
% No numbering
\setcounter{secnumdepth}{0}
% Only sections in TOC
\setcounter{tocdepth}{1}
% Set header
\pagestyle{fancy}
\fancyhf{}
\fancyhead[L]{\leftmark}
\fancyhead[R]{\thepage}
%\renewcommand{\headrulewidth}{0.6pt}
% Custom header : no uper case
\renewcommand{\sectionmark}[1]{%
  \markboth{\textit{#1}}{}}


\title{Pneumologie\\
\large Fiches}
\author{A. Praga}

% Negate \implies
\usepackage{centernot} 
\begin{document}
\maketitle
\tableofcontents

\input bacteries-header.tex

\section{73 : Addiction au tabac}
3 fumées : courant primaire (inspiré), secondaire (tabagisme passif++), tertiaire (exhalé)\\
Produits :
\begin{itemize}
  \item nicotine = dépendance
  \item fumée de tabac = 0.3 $\mu$m, pénètre partout !
  \item goudrons = cancérigène
  \item CO = hypoxie, risque ischémie, marqueur tabagisme des dernières heures
\end{itemize}
16 millions de fumeurs en 2013 (France). Éducation et cat. sociale faible = plus
tabagiques
\subsection{Pathologies}
Cancéreuses :
\begin{itemize}
  \item 90\% des cancers broncho-pulmonaires en actif, 25\% en passif
  \item voies aérodig sup, vessie, pancréas, rein, col de l'utérus
\end{itemize}
Respi : BPCO, asthme\\
Cardio-vasc : 
\begin{itemize}
  \item cardiopathies ischémiques
  \item artérite
  \item HTA
  \item cérébro-vasc
  \item relation tabac-athérome, maladie coronaire
\end{itemize}
Autres : digestives, kératites, retard consolidation os, agueusie, anosmie\\
Passif :
\begin{itemize}
  \item +25\% risque cancer bronchique
  \item +25\% maladies cardio-vasc
  \item aggrave asthme, BPCO
  \item nourrisson : RCIU\footnote{retard croissance intra-utérin}, \(\nearrow\) risque
    infections respi, 1ere cause identifiée mort-subite
\end{itemize}
\subsection{Prise en charge}
Consommation (paquets/années), dépendance, autres (alcool = déclencheur,
cannbas), motivation, comorbidités (psy, cardio-vasc, respi)
\subsection{Traitement}
Conseil, motivationnell, traitement (substitut nicotinique, varénicline \danger
suivi, bupropion), thérapies cognitivo-comportementales, cigarette
électronique\\
Suivi : réussi si \(\ge 1\) an



\section{108 : Troubles du sommeil}
\subsection{Définitions}
\danger Sd d'apnées hyopnées obstructives du sommeil (SAOS) = profil "en peigne"\\
Sd d'apnées centrales = désat profondes et soutenus

Apnée : 
\begin{itemize}
  \item obstructive : arrêt débit \(\ge\) 10s + efforts ventilatoires
  \item centrale : idem, sans efforts ventilatoires
  \item mixte = obstructive et centrale
\end{itemize}
Hypopnée = diminution ventilation \(\ge\) 10 s \(\wedge\) (débit -50\% \(\vee\) désat
transcut \(\ge\) 3\% \textpm{} microéveil)

SAOS = (somnolence diurne non expliquée \(\vee\) (2 parmi : ronflements, étouffement,
éveils répétés, sommeil non réparateur, fatigue diurne, trble concentration,
nycturie)) \(\wedge{}\)  IAH \(\ge\) 5\\
Sévère si IAH \(\ge\) 30.
\subsection{Épidémiologie}
2 millions en France.
FR : obésite, homme, âge, anomalie voie aériennes supérieures
Comorbidités :
\begin{itemize}
  \item neuropsy + accidents x2
  \item CV : HTA, insuf. coronaire, cardiaque, ACFA
  \item sd métabolique = obésité abdo \(\wedge\) (2 parmi : HTA, glycémie \(\ge\) 5.6 mmol/L,
    HDL bas, hypertriglycéridémie
  \item diabète
  \item dyslipidémie
\end{itemize}

NB : obstruction VAS = réduction \diameter, \nearrow{} collapsibilité,
\searrow{} muscles dilatateurs
\subsection{Diagnostic}
\subsubsection{Clinique :}
\begin{itemize}
  \item nocturne = roflements, pause respi, étouffement, nycturie
  \item diurne = somnolence excessive (questionnaire d'Epworth)
\end{itemize}
DD : hypersomnie diurne (insomnie, sd dépressif, sédatif, hygiène de sommeil,
neuro)
\subsubsection{Examen}
\begin{itemize}
  \item IMC, obésité abdo (\(\ge\) 94 cm \male, 80cm \female)
  \item ORL
  \item CV, respi
\end{itemize}
\subsubsection{Complémentaire}
EFR, imagerie optionnels.
Saturation transcut en O\(_{\text{2}}\) possible.\\
Référence = enregistrement polygraphique ventilatoire (ou polysomnographique
mais cher++)
\subsection{Traitement}
Général : PEC\footnote{prise en charge} surpoids, éviction \{benzodiazépines,
myorelaxants, morphiniques\}, PEC CV

Spécifique :
\begin{itemize}
  \item Pression positive continue = 1ere intention
  \item Orthèse d'avancée mandibulaire = 2eme
  \item Autres : orthèse anti-décubitus dorsal, chirurgie = \{vélo-amygdalienne,
    avancée maxillo-mandibulaire (lourd++), nasale\}
\end{itemize}
\subsection{Autres}
\begin{itemize}
  \item Sd d'apnée de type central : dû à : insuf. cardiaque sévère (traitement !),
\end{itemize}
atteinte tronc cérébral, séjour ajtitude, morphiniques
\begin{itemize}
  \item Hypoventilation alvéolaire : aggravation de l'état de veille. Traitement =
    VNI\footnote{ventilation non invasive}
  \item Sd obésite hyooventilation :  PaCO\(_{\text{2}}\) > 45 mmHg \(\wedge\) PaO\(_{\text{2}}\) < 70 mmFg \(\wedge\)
    IMC > 30 kg/\(m^2\), pas d'autre cause. Traitement = VNI
\end{itemize}

\section{151 : Infections broncho-pulmonaires communautaires}
\subsection{Bronchite aigue}
Très fréquente, virale 90\%. \\
Diagnostic clinique : épidémie, toux sèche \(\to\) productive, expectorations,
pas de crépitants.\\
Ttt symptomatique seulement !

\subsection{EBPCO} Cf~\nameref{subsec:ebpco}

\subsection{Pneumonie aigue communautaire}
Clinique : \{toux, expect purulentes, dyspnée\} + \{fièvre, asthénie\} +
crépitants\\
Radio !\\
Si \faHospitalO : hémocultures, ECBC, antigenurie pneumocoque (PCR,
antigenurie légionelle)\\
\faHospitalO : signes de gravité (score CRB65 \(\ge 1\)) / incertitude / échec domicile / comorbidité / inobservance

\begin{table}[htpb]
  \centering
  \label{Orientation clinique (non discriminant !)}
  \begin{tabular}{llll}
    \toprule
    & Pneumocoque & Atypique & Légionellose\\
    \midrule
    Début & Brutal & Progressif & Rapidement progressif\\
    Signes & Thoraciques & Extra-thoraciques & Extra-thoraciques\\
    Biologie &  & Anémie hémolytique & Hyponatrémie, rhabdomyolyse\\
    & CG+ chainettes/Agurie &  & Antigénurie\\
    Radio \faHeart & condensation systématisée & Opacités multifocales & CS ou OM\\
    \bottomrule
  \end{tabular}
\end{table}

\begin{itemize}
  \item Pneumocoque : fréquent+++. Pas de transmission interhumaine
  \item Atypique : \bact{mpneumoniae}, \bact{cpneumoniae}, \bact{psitacci}
  \item Legionnella : pas d'isolement. DO
  \item Pneumonie virale : signes respi + sd grippal. Grippale = diagnostic PCR, traitement = inihbiteur neuramidase \danger \bact{dore}
\end{itemize}
\subsubsection{Traitement}
Oral, 7j (8-14 si légionellose, 21j si légionelles graves/ID)

Urgent, probabiliste, réévalué 48-72h

Ambulatoire :Amoxicilline ou macrolide \(\to\) switch

\faHospitalO  Amoxicilline \(\to\) réévaluation

Réa : C3G IV + macrolide IV / FQ pour pneumocoque

\subsubsection{Échecs}
\begin{itemize}
  \item Compliquée : épanchement pleural, abcès, obstacle
  \item Observance, pharmacocinétique, hors spectre
  \item Diagnostic :
    \begin{itemize}
      \item focalisé : embolie pulmonaire
      \item diffuse : cf pneumopathies interstitielles diffuses aigües
      \item excavée : cancer, tuberculose pulmonaires, infarctus pulmonaire, vascularite\ldots{}
    \end{itemize}
\end{itemize}

\subsubsection{Prévention}
Vaccin : 2 doses + rappel enfant, 2 doses adulte (ID, comorbidité)

\subsubsection{Immunodéprimé}
Pneumocystose pulmonaire
Radio : sd interstitiel diffus bilatéral symétrique. ATB cotrimoxazole 21j


\section{151 : Tuberculose}
\bact{tuberculose}
\subsection{Tuberculose maladie}
\subsubsection{Pulmonaire}
Classique : 
\begin{itemize}
\item \{asthénie, anorexie, amaigrissement\} + fébricule (nocturne) + sueurs nocturnes
\item Radio : nodules, infiltrats, cavernes (lobes sup, post.)
\end{itemize}
Miliaire (hématogène) :
\begin{itemize}
\item AEG marqué (fébrile)
\item Radio : interstitiel diffus micronodulaire
\end{itemize}
Autres : pleurésie (épanchement unilat, exsudative lymphocytaire), pneumonie
aigüe, extrapulmonaire
\subsubsection{Extra-pulmonaires}
Ganglionnaire, ostéo-articulaire, génito-urinaire, neuroméningé
\subsection{Diagnostic}
Mises en évidence du bacille :
\begin{itemize}
\item 3 prélèvement (ECBC/tubage). Si miliaire : +hémocultures, ECBU
\item direct, culture, identification et ATBgramme
\item Granulomes épithélioïdes gigantocellulaires avec nécrose caséeuses,
  coloration Ziehl Neelsen (> $10^3$bacilles/mL)
\end{itemize}
\subsection{Traitement}
Quadrithérapie (lutter MDR ou XDR) : 1/jour 1h avant petit déjeuner
\begin{itemize}
\item Isoniazide (polynévrite SM\footnote{sensitivomotrice}, surv. transaminases) : 6 mois
\item Rifampicine (inducteur enzymatique\footnote{contraception !}) : 6 mois
\item Ethambutol (névrite optique rétrobulbaire) : 2 premiers mois
\item Pyrazinamide (surv. transaminases)
\end{itemize}
Autres : aminosides, fluoroquinolones \\
Avant traitement : 
\begin{itemize}
\item hémogramme, \{transaminases, bilirubine, phosphatases alcalines, \$\(\gamma\)\$GT\}, créat,natrémie,uricémie.
\item proposer VIH, VHB, VHC
\item visions des couleurs
\end{itemize}

\subsection{ITL}
Diagnostic (\danger pas pour tuberculose active): 
\begin{itemize}
\item IDR (non spécifique, faux négatifs !) : +72h : \diameter > 10mm (15 si vaccination) ou +10mm en 3 mois : suspicion ITL
\item IGRA\footnote{Interferon Gamma Release Assay} : > 15 ans ou migrant  < 15 ans
\end{itemize}
Traitement : 
\begin{itemize}
\item primaire si < 2 ans, secondaire si < 18 ans ou ID ou ITL < 1 an ou séquellaire
\item INH 9 mois ou INF + RMP 3 mois
\end{itemize}
\subsection{Vaccination}
Sauf VIH, déficits immunitaires
\subsection{Cas particuliers}
Grossesse : INH (+vit. B6) + RMP (+vit. K1) + EMB\\
Insuf. rénale : [15,30]mL/min : diminuer EMB, PZA. si < 15mL/min : diminuer INH, EMB, PZA\\
Insuf. hépatique : [3,6N] : arrêt PZA. Si > 6N, arrêt INH, PZA\\
ID, VIH : penser tuberculose si fièvre\\
Si anti-TNF, bilan tuberculose
\subsection{DO}



\section{180 : Accidents du travail}
Maladie professionnelle = lente, prolongée (\(\neq\) accidend du travail)
\subsection{Maladies}
\subsubsection{Asthmes}
10-15\% des asthmes. On a
\begin{itemize}
  \item asthme professionnel : (avec latence [immunologique]) \(\vee\) (sans latence
    [exposition unique])
  \item asthme aggravé par le travail
\end{itemize}
Métiers : boulangers, santé, coiffeurs, peintres pistolets, bois, nettoyage\\
Diagnostic : de novo, profession, rythme
\subsubsection{BPCO}
TVO lente progressive + inflammation poumons. 10-20\% BPCO\\
Métiers : mines, BTP, fonderie, textile, agricole
\subsubsection{Cancers}
\begin{itemize}
  \item mésothéliome : 2/100 000habi/an, amiante++, DO\footnote{déclaration obligatoire}
  \item bronchique primitif
\end{itemize}
\subsubsection{PID}
\begin{itemize}
  \item Hypersensibilité X agricole
  \item Silicose (+cancer bronchique primitif)
  \item Bérylliose : adénopathies médiastinales, sd infiltrant parenchymateux
  \item Sidérose : fumées d'oxyde de fer, micronodule \textpm{} emphysème
  \item Asbestose : fibrose
\end{itemize}
\subsubsection{Liées à l'amiante}
\begin{itemize}
  \item Cancer : mésothéliome, cancer bronchique primitif
  \item Pleural (plaques, épaississements, pleurésies), parenchyme (fibrose)
\end{itemize}
\subsection{Reconnaissance}
Indemnité : 
\begin{itemize}
  \item 60\% salaire si < 28 jours, 80\% sinon
  \item si séquelles, taux incapacité permante < 10\% : capital, > 10\% rente, > 40\%
    rentre et autres
\end{itemize}

\section{182 : Hypersensibilités et allergies respiratoires}%
\label{sec:182_hypersensibilites_et_allergies_respiratoires}

\begin{figure}[htpb]
  \centering
  \resizebox{\linewidth}{!}{
    \tikz \graph [
  % Labels at the middle 
      edge quotes mid,
  % Needed for multi-lines
      nodes={align=center},
      sibling distance=3cm,
      edges={nodes={fill=white}}, 
    layered layout]
    {
      HS non allergique;
      HS allergique -> {
        Non IgE;
        IgE -> {
          atopique -> "insectes\\helminthes\\médicaments";
          non atopique -> "rhinite\\asthme\\alimentaire\\professionelles";
        };
      };
    };
  }
  \caption{Hypersensibilité}
\end{figure}
\subsection{Définitions}%
Atopie : prédisposition héréditaires à IgE face à des allergènes.\\
Allergie : réaction d'HS par mécanismes immunologiques\\
Sensibilisation : test cutané positif à un allergène\\
Allergène : capable d'induire une réaction d'HS = pneumallergènes (aéroportés),
trophallergènes (alimentaires), profesionnels, recombinants

Hypersensibilité :
\begin{itemize}
  \item type 1 (immédiate) : la plus fréquente, IgE. Rhinite, asthme
  \item type 2 : complément et phagocytose. Réactions médicament.
  \item type 3 : complexes immuns. Pneumonies d'HS
  \item type 4 : LT, cytotoxiques 48-72h. Granulome épithélioïde gigantocellulaire.
    Dermato.
\end{itemize}

\paragraph{Asthme et rhinite allergique}
Polygénique.\\
Environnement : infections virales, sensibilitations pneumallergènes, tabac dès
conception, pollution air intérieur. 

\paragraph{Anomalies voies aériennes (asthme)}
Remodelage bronchique : épaississement (membrane basale et muscle lisse $\wedge$ oedème
bronchique) et obstruction $\diameter$ (mucus)

\paragraph{Réaction IgE}
\begin{itemize}
  \item Sensibilisation : synthèse IgE spécifiques (lymphocytes B)
  \item effectrice : fixation de l'allergène $\implies$ activation (histamine,
    cytokines) $\implies$ cascade allergique
\end{itemize}

\subsection{Épidémiologie}
Atopie = 30-40\% population. HS médicament = 7\% pop. Allergie alimentaire : 2\%
des 9-11 ans

FR : 
\begin{itemize}
  \item génétique (enfants à risque)
  \item environnement : fréq $\nearrow$ temps, alimentaires, tabagisme passif
    maternel, allergènes, pollution atmosphérique \danger niveau de preuve 
\end{itemize}
Morbidité forte, mortalité encore forte pour l'asthme.

\subsection{Diagnostic}
Clinique : asthme, rhinite, conjonctivite

\paragraph{Diagnostic}
Unité de temps, lieu et action (perannuels [acariens,
blattes, phanères d'animaux, végéteaux d'intérieur, moisissures] ou saisonniers
[pollens]) $\wedge$ IgE spécifiques

Prick-test = référence.
\begin{itemize}
  \item $\diameter \ge 3$mm / témoin
  \item acariens, pollens, phanères d'animaux, blatte, moisissures chez l'adulte
  \item arachide, blanc d'oeuf, poisson, lait de vache si < 3 ans
  \item CI : antihistaminiques, $\beta$-bloquants, eczéma, grossesse si
      allergie médic
\end{itemize}

Autres tests : dosage IgE spécifique (moins sensible), umltiallergénique (sensible mais
    pas quantitatif)

\paragraph{Professionnelles} boulangers, santé, coiffeurs, peintres pistolets, bois, nettoyage\\

\paragraph{Autres}
\begin{itemize}
  \item Test de provocation = certitude mais dangereux
  \item Très sécifique : dosage IgE totale, dosage éosinophiles sanguin, dosage tryptage sérique
\end{itemize}

\subsection{Traitement}
\paragraph{Éviction allergènes} Toujours.

\paragraph{Symptomatiques}
\begin{itemize}
  \item antihistaminiques : rhinite, conjonctivite, prurit
  \item corticoïdes : systémitique si urgence (prednisone, prednisolone), local
    si traitement fond
\end{itemize}

\paragraph{Immunthérapie spécifique}
Faibles doses croissantes d'allergènes :
\begin{itemize}
  \item Sous-cutanées/sub-linguale : acariens, pollens, hyménoptères
  \item orale : pollen
\end{itemize}
CI : maladies allergiques, dysimmunités, grosses (induction), asthme sévère non
contrôlé, mastocytoses, $\beta$-bloquants

ES : 
\begin{itemize}
  \item syndromique (asthme, rhinite, urticaire) = alerte
  \item générale (hypotensions, bronchospasme, choc anaphylactique) =
    interruption
\end{itemize}


\section{184 : Asthme, rhinite}
Asthme 6\%, mortalité 1000 DC/an, en baisse. Morbidité en hausse

Rhinite allergique 24\%.
\subsection{Définitions}
Asthme : 
\begin{itemize}
\item inflammation chronique modifiant les VAS\footnote{voies aériennes
supérieures} avec symptômes respi \(\wedge\) obstruction voies aériennes réversible
\item interaction gènes-evironmt, déclencheurs : exercice, HS aspirine/AINS,
irritants inhalés
\end{itemize}
Symptômes d'asthme : 
\begin{itemize}
\item gêne respiratoire, dyspnée, sifflements, oppression thoracique,
toux
\item < 20 min, réversible, variable
\end{itemize}
Exacerbations : \nearrow symptômes > 2 jours, non calmée, sans retour état
habituel

TVO : 
\begin{itemize}
\item VEMS/CVF < 0.7
\item réversible si +200mL et +12\% après BDCA\footnote{Broncho-dilatateurs à courte durée
d'action}
\end{itemize}

Hyperréactivité bronchique : -20\%VEMS après métacholine/air sec (VPP = 100\%)

Débit expiratoire de pointe (pour urgence)

\subsection{Diagnostic}
\paragraph{Asymptomatique}
Diagnostic :
\begin{itemize}
\item symptômes caract : plusieurs symptômes (cf def), $\nearrow$ nuit/réveil, variable,
\end{itemize}
réversibles, déclenché par \{rire, virus, allergènes, irritants\}
\begin{itemize}
\item \(\wedge\) obstruction bronchique : sibilant, TVO réversible ou apparaît avec métacholine
\end{itemize}
Sévérité évaluée à 6 mois
\paragraph{Exacerbation}
Cf def. Signe de sévérité :
\begin{itemize}
\item mots, assis en avant, agité
\item FR > 30/min
\item muscles respi annexes
\item FC > 120/min, SpO\(_{\text{2}}\) < 90\%
\item DEP < 50\%
\item silence auscultatoire \skull
\item respi paradoxale \skull
\item troubles conscience, bradycardie, collapsus \skull
\end{itemize}
\paragraph{DD}
Sans TVO : cordes vocales, sd hyperventilation

TVO non réversible : \{BPCO, bronchectasies, mucoviscidose, bronchiolites
constrictives\}, autres (corps étranger, tumeur, insuf. cardiaque)
\paragraph{Bilan}
Facteurs favorisants, radio thorax, EFR (+test métacholine)

\subsection{Traitement}

\paragraph{Long cours}
Ttt de fond :
\begin{itemize}
\item BDCA à la demande
\item CSI\footnote{corticostéroïdes inhalés} faible < anti-leucotriènes < CSI moyen/fort
\(\vee\) (CSI faible et ALT\footnote{anti-leucotriènes} < triotropium \(\vee\) (CSI fort et
ALT) < CSO\footnote{corticostéroïdes oraux} faible
\end{itemize}
Autre : 
\begin{itemize}
\item activité physique (sauf plongée)
\item facteurs favorisants : rhinite, allergie, tabac, \(\beta\)-bloquant
(aspirine/AINS) reflux GO, comorbidités
\item vaccins grippe (pneumocoque si asthme sévère)
\end{itemize}
Efficace ?
\begin{itemize}
\item symptômes contrôlés (diurnes < 2/sem, pas de réveil nocture, BDCA < 2/sem, pas
limitation d'activité)
\item exacerbations < 2 corticostéroïdes systém./an
\item VEMS/CV > 0.7 et VEMS \(\ge\) 80 \%
\end{itemize}
Non contrôlé '
\begin{itemize}
\item CSI faible + BDLA\footnote{broncho-dilatateur à longue durée d'action} < CSI
moyen/fort + BDLA < centre spécialisé
\end{itemize}
Suivi : périodique, +3mois si changement ttt, mensuel si grossesse

\paragraph{Urgence}

\begin{figure}[htpb]
  \centering
  \tikz \graph [
  % Labels at the middle 
  edge quotes mid,
  % Needed for multi-lines
  nodes={align=center},
  sibling distance=3cm,
  edges={nodes={fill=white}}, 
  layered layout]
  {
    "Évaluation" -> {
      "Exacerbation\\
      sans signes de gravite" -> 
      "\(\beta_2\) mimétique forte doses\\
      chambre inhalation\\
      \textbf{Corticothérapie orale}  7j" [draw]
      -> Réévaluation 1H;
      "Exacerbation \\
      avec signes de gravite" -> 
      "\(\beta_2\)-mimétique forte dose" [level distance=2cm] -> {
        "O\(_2\)\\
        \(\beta_2\) mimétique nébulisés (5mg/20min)\\
        $\pm$ ipratropium\\
        \textbf{Corticothérapie orale} " [>"\faHospitalO", draw];
      };
      Perte de contrôle ->
      "\(\beta_2\) mimétique" [draw] -> Réévaluation 1H;
    };
  };
  \caption{Asthme : traitement d'urgence}
\end{figure}

Pas d'ATB sauf si suspicion bactérienne. Adrénaline seulement pour choc anaphylactique
\danger pas de BDLA 

\subsection{Rhinite allergique}
Diagnostic :
\begin{itemize}
\item PAREO : prurit, anosmie, rhinorrhée, éternuement, obstruction nasale
\item fosses nasales au speculum inflammées
\item allergique (argumenter !!)
\end{itemize}
Sévère si persistant (> 4 semaines/an) et retentissement qualité de vie

TDM

Ttt : laval nasal, allergie, antihistaminique/corticoïdes nasaux, tabac, stress


\section{Pathologies auto-immunes}%
\label{sec:pathologies_auto_immunes}

Manifestations respi. viennent de (par priorité décroissante) :
\begin{itemize}
  \item infectieuse (favorisé par ttt)
  \item toxicité médicamenteuse
  \item spécicifique
  \item indépendant
\end{itemize}

\subsection{Complications infectieuses}
Immunodépression :
\begin{itemize}
  \item corticoïdes forte dose > 1 mois
  \item méthotrexate
  \item cyclophosphamide
  \item anti-TNF$\alpha$ (infliximab)
\end{itemize}
Tuberculose : 
\begin{itemize}
  \item comme ID = 50\% extrapulmonaire, 25\% disséminé
  \item prévention si TNF-$\alpha$ : INH 9 mois $\vee$ INH-RMP 3 mois
\end{itemize}
Pneumocystose : si corticoïdes forte dose $\vee$ méthotrexate $\vee$
cyclophosphamide
\begin{itemize}
  \item clinique : début brutal, insuf. respi, mortalité élevée
  \item radio thorax : condensation alvéolaire/verre dépoli bilat
  \item penser co-infections
\end{itemize}
\subsection{Médicaments}
Méthotrexate : plus fréquent, pneumopathie d'HS (opacités diffuses), LBA
lymphocytaire. Évolution favorable à l'arrêt + corticothérapie

Inhibiteurs TNF-$\alpha$ : PID/granulomatoses, anaphylactique

\subsection{Connectivites}
Polyarthrite rhumatoide
\begin{itemize}
  \item PID\footnote{pneumopathie interstitielle diffuse} : radio =
    réticulations, rayons de miels, bronchectasies. Surveillance seulement
  \item pleurésie rhumatoïde : unilatérale, peu abondante, exsudative. Évolution
    favorable
  \item nodules pulmonaires rhumatoïdes. 
  \item bronchiolite oblitérante
\end{itemize}
Sclérodermie systémique : CREST (Calcinose, Raynaud, dyskinésie oEsophagienne,
Sclérodactyie, Télangiectasie), Ac anti-nucléraires
\begin{itemize}
  \item PID : semblable à PINS\footnote{pneumopathie interstitielle non
    spécifique} : opacités en verre dépoli, bronchectasies par traction. Survie
    85\% à 5 ans
  \item hypertension artérielle pulmonaire : dyspnée. Echo. cardiaque +
    cathéterisme cardiaque
\end{itemize}
Lupus érythémateux disséminé 
\begin{itemize}
  \item pleurésie lupique : peu abondant, svt bilatéral, svt + péricardite
  \item infectieux, sd hémorragie alvéolaire
\end{itemize}
Dermato-, polymyosite
\begin{itemize}
  \item PID chronique : 1ere cause DC, opacités verre dépoli, LBA lymphocytaire.
    Ttt : corticothérapie + IS
  \item PID (sub)aigüe
\end{itemize}
Sd de Gougerot-Sjögren
\begin{itemize}
  \item bronchite lymphocytaire chronique : toux sèche chronique
  \item PID, lymphome pulmonaire primitif
\end{itemize}

\subsection{Vascularitess}
Granumolatose avec polyangéite : 40-50ans, début ORL+ poumon (+ rein)
\begin{itemize}
  \item radio : nodules s'excavant, opacités verre dépoli
  \item Ttt urgent = corticothérapie, cyclophosphamide
\end{itemize}
Granumolatose éosinophilique avec polyangéite :  hyperéosinophilie, pneumopathie
éosino.

Polyangéite microscopique : sd hémorragique alvéolaire

\section{199 : Dyspnée aigüe et chronique}
\label{sec:199_dyspnee_aigue_et_chronique}

Aigüe = quelques heures/jours

Détresses respi aigüe :
\begin{itemize}
  \item cyanose
  \item sueur
  \item FR > 30/min ou < 10/min
  \item tirage, muscle respi accessoires
  \item respi abdo paradoxale
\end{itemize}
Hémodynamique :
\begin{itemize}
  \item FC > 110/min
  \item choc (marbrures, oligurie, angoisse, extr. froides)
  \item PAS < 80 mmHg
  \item insuf. ventriculaire droite (turgescence jug, OMI, signe Harzer)
\end{itemize}
Urgence !!

\begin{figure}[htpb]
  \centering
  \resizebox{\linewidth}{!}{
  \tikz \graph [
  % Labels at the middle 
  edge quotes mid,
  % Needed for multi-lines
  nodes={align=center},
  sibling distance=3cm,
  level distance=2cm,
  edges={nodes={fill=white}}, 
  layered layout]
{
  Anomalie respi ? -> 
  {
    "Inspiration\\(VAS)" -> {
      Enfant -> {
        "\textbf{Corps étranger}";
        "\textbf{Laryngite}";
      };
      Adulte -> {
        "\textbf{Oedeme de Quincke}\\
        Terrain allergie";
      };
    };
    "expiration\\(bronches)" -> {
      "\textbf{Asthme}\\
      Allergie\\
      Facteur\\
      Médic\\
      Hyposodé";
      "\textbf{OAP}\\
      Orthopnée\\
      Expect. mousseuses\\
      Médic\\
      Hyposodé";
      "\textbf{EBPCO}\\
      Tabac\\
      Bronchite aigüe";
    };
    non -> {
      douleur -> {
        "\textbf{Infectieux}\\
        Sd infectieux\\
        Toux\\
        Expect. purul";
        "\textbf{Pneumothorax}\\
        Sd pleural\\
        Brutal";
        "\textbf{EP/vasculaire}\\
        Alitement\\
        Voyage\\
        Phlébite";
      };
      sans douleur ->  
      "\textbf{OAP}\\
      Agé\\
      Orthopnée\\
      Expect. mousseuses\\
      Nocturne\\
      Crépitants";
    };
  };
};
}
\caption{Dyspnée aigüe}
\end{figure}
\begin{figure}[htpb]
  \centering
  \resizebox{\linewidth}{!}{
\tikz \graph [
   %Labels at the middle 
  edge quotes mid,
   %Needed for multi-lines
  nodes={align=center},
  sibling distance=3cm,
 level distance=2cm,
  edges={nodes={fill=white}}, 
layered layout]
{
  Sibilants -> {
    "\textbf{Asthme}\\
    Jeune\\
    Atopie\\
    Variable\\
    Nocturne";
    "\textbf{BPCO}\\
    Tabac\\
    Toux, expect.";
    "\textbf{Insuf cardiaque G}\\
    ATCD\\
    Orthopnée\\
    Toux\\
    Nocturne";
  };
  Crépitants -> {
    "\textbf{Insuf cardiaque G}\\
    ATCD\\
    Orthopnée\\
    Toux\\
    Nocturne";
    "\textbf{PID}\\
    Toux sèche\\
    Maladie systémique";
  };
  Normal -> {
    "\textbf{EP/vasc.}\\
    ATCD\\
    Phlébite";
    "\textbf{Neuromusc.}\\
    Orthopnée\\
    Respi. paradoxale\\
    Neuro";
    "\textbf{Sd hyperventil}\\
    Non lié effort\\
    Vertiges\\
    Paresthésies";
  };
};
}
\caption{Dyspnée chronique}
\end{figure}
Quantification : échelle Borg [0-10] (aigüe) ou MRC [0-4] (chronique)
\section{200 : Toux chronique}
\label{sec:200_toux_chronique}

\danger Éliminer toux post-infectieuse (< 3 semaines)

Signes de gravité : 
\begin{itemize}
  \item AEG, sd infectieux
  \item dyspnée d'effort, hémoptysie
  \item modification toux chez fumeur
  \item dysphonie, dysphagie, fausses routes
  \item Adénopathies cervicales suspectes
  \item Anomalies cardiopulmonaires
\end{itemize}

\begin{figure}[htpb]
  \centering
  \caption{PEC initiale}
  \resizebox{\linewidth}{!}{
\tikz \graph [
   %Labels at the middle 
  edge quotes mid,
   %Needed for multi-lines
  nodes={align=center},
  level distance=40pt,
  sibling distance=3cm,
  edges={nodes={fill=white}}, 
layered layout]
{
  Signe de gravité -> {
    Exploration "oui"];
    "Médicaments ?" "non"] -> {
      Test d'éviction "oui"];
      "Coqueluche ?" "non"] -> {
        Test diagnostique"oui"];
        "Radio thorax anormale ?" -> {
          Bilan spécialisé "oui"];
          "cf~\nameref{subsec:toux_orientation}" "non"];
        };
      };
    };
  };
};
}
\end{figure}

\subsection{Orientation diagnostique}
\label{subsec:toux_orientation}
\textbf{ORL}  
\begin{itemize}
  \item rhinosinusiens : sd rhinorrée postérieur++, obstruction nasale chronique
  \item carrefour aérodigestif : diverticule de Zenker, laryngite chronique
\end{itemize}

\textbf{Respiratoire} 
\begin{itemize}
  \item Asthme : TVO réversible/hyperréactivité bronchique
  \item BPCO : TVO non réversible
  \item Cancer bronchique, tumeurs, bronchectasise (cf
    \nameref{sub:bronchectasies}
\end{itemize}

\textbf{RGO}  : pyrosis. Endoscopie digestive si FR, pHmétrie des 24h

\textbf{Allergique} 

\textbf{Systémique}  :  sd Gougerot-Sjögren, polychondrite atrophiante, maladie
de Hortone, granulomatoes avec polyangéite, rectocolite hémorr., maladie de
Crohn.

\textbf{Comportement}  : dernière étiologie

\paragraph{Traitement d'épreuve} (ordre d'échec):
\begin{enumerate}
  \item RGO : bromphéniramine + pseudoéphédrine 3 sem
  \item asthme si TVO réversible, corticoïdes inhlalés BD inhalés si
    hyperréactivité bronchique
  \item avis spé
\end{enumerate}

Traitement symptomatique : arrêt tabagisme. Éviter si possible
\begin{itemize}
  \item sèche : opiacés, antihistaminique anticholinergique, non
    antihistaminique non opiacés
  \item produtive : mucomodificateurs, kiné
\end{itemize}

\subsection{Bronchectasies}
\label{sub:bronchectasies}
Types : bronchectasies (élargissement $\diameter$), bronchocèle (pus), "par
traction" (NB: pas des vraies bronchectasies)


Étiologie : 
\begin{itemize}
  \item infection respi sévères : coqueluche, tuberculose (virales respi enfant,
    pneumonie bact, suppuration suite sténose)
  \item mucoviscidose
  \item non infectieux (poumon radique, aspergillose allergique, SDRA,
    systémique, déficit immunitaire)
\end{itemize}

Évolution : colonisation bactérienne, hémoptysie, TVO (car dilatation seulement
proximale), insuffisance respi

Clinique :
\begin{itemize}
  \item toux productive quotidienne depuis l'enfance
  \item hémoptysie
  \item infections à \{\bact{influenzae}, \bact{pneumocoque}\} puis \{\bact{dore},
    \bact{aeruginosa}++\}
\end{itemize}

Diagnostic : TDM (certitude) = bronches de $\diameter$ > artère, lumière
bronchique > 1/3 parenchyme, pas de réduction du $\diameter$, grappes de kystes,
opacités tubulées

Traitement : ATB si exacerbation, complications parenchymateuses. Macrolides
pour l'inflammation. Chir si très local + compliqué.

\section{201 : Hémoptysie}%
\label{sec:201_hemoptysie}

Urgence \skull

\begin{enumerate}
  \item Est-ce une hémoptysie ? Hématémèse ou ORL possibles
  \item Gravité ? Suivant abondance, terrain, persistance $\implies$ risque = hématose et asphyxie
\end{enumerate}

\subsection{Étiologies}
\begin{itemize}
  \item Tumeurs bronchopulmonaires++
  \item Bronchectasies++
  \item Tuberculose (évolutive/séquelles)++
  \item Idiopathique++
  \item Infections : aspergillaires, pneumopathie infectieuses nécrosante
  \item Vasc : embolie pulmonaire, HT pulmonaire, anévrysmes/malformations
  \item Hémorragie alvéolaires : insuf. cardiaque gauches, médicaments/toxiques,
    vascularites, collagénose, sd Goodpasture
\end{itemize}

\subsection{Diagnostic}
Localisation (important !)

Interrogatoire : ATCD respi, cardiaque, histoire médicale

Examens 
\begin{itemize}
  \item clinique : \{$SpO_2$, tension, pouls\}, mauvaise tolérance respi, gêne
    latéralisée, hippocratisme digital
  \item radio thorax pour siège (verre dépoli/sd alvéolaire), lésion
  \item scanner plus précis : nature, localisation, carto. vasc.
  \item (endoscopie : hématémèse, multiples lésions, tumeur proximale)
  \item (artériographie bronchique : ttt par embolisation)
  \item autres : gaz du sang, dosage Hg, bilan coagulation, groupe sanguin, {BK,
    ECG} pour étiologie (OAP hémorragique pour ECG)
\end{itemize}

\danger BPCO $\centernot\implies$ hémoptysie

\subsection{Traitement}
$O_2$ + vasconstriction IV, protection voies aériennes (décubitus latéral,
ventilation mécanique

Embolisation artérielle bronchique

Chir si localiés, fonction respi et "à froid"




\section{202 : Épanchement pleural}%
\label{sec:epanchement_pleural}

\subsection{Diagnostic}
Suspicion clinique :
\begin{itemize}
  \item douleur thoracique, dyspnée, toux sèche, hyperthermie
  \item sd pleural liquidien (silence auscult, matité, $\emptyset$ transmission
    corde vocales, souffle pleurétique
\end{itemize}
Sigenes de gravité : détresse respi, choc septique, choc hémorragique

Confirmation imagerie
\begin{itemize}
  \item radio : opacité dense, homogène, non sytématisé, limité par ligne
    concave (DD atélectasie : EP = "repousse" vers côté sain). \\
    Cas difficiles : profil, sous pulmonaire, cloisonnés
  \item échographie : cloisonné, pleurésie VS collapus, guide ponction
  \item TDM en urgence si embolie pulmonaire ou hémothorax
\end{itemize}

\subsection{Causes}
\paragraph{Transsudats}
\begin{itemize}
  \item insuf. cardiaque G
  \item cirrhose
  \item sd néphrotique
  \item atélectasie
  \item embolie pulmonaire
\end{itemize}

\paragraph{Exsudats}
\textit{Néoplasiques} 
\begin{itemize}
  \item métastatique : liquide sérohématique/rosé/citrin, cytodiagnostic +
    biopsie (aveugle/vue) sauf si cancer connu
  \item mésothéliome 
    \begin{itemize}
      \item exposition amiante
      \item imagerie : épaississement pleural diffus, circonférentiel
      \item liquide citrin/sérohématique
      \item biopsie sous thorascopie++
    \end{itemize}
\end{itemize}

\textit{Infectieux} 
\begin{itemize}
  \item parapneumoniques (=bactérien) 
    \begin{itemize}
      \item non compliqués (liquide clair, pas de germe) : ATB
      \item compliqué : ATB + évacuation liquide
    \end{itemize}
  \item virale
  \item tuberculeuse : progressif, fièvre modérée, amaigrissement. Exsudat riche
    en proténie. Biopsie++ (aveugle/vue)
\end{itemize}

\textit{Autres}  :
\begin{itemize}
  \item embolie pulmonaire
  \item bénigne liée à l'amiante (exclusion++)
  \item post trauma, rupture oesophagienne, sous-diaphragme
  \item systémique : lupus, polyarthrite rhumatoide
\end{itemize}

\subsection{Ponction}
Qui ? Majorité sauf peu abondant et insuf cardiaque G (sauf si
unilat/asymétrique $\vee$ douleur pleurale/fièvre $\vee$ traitement insuffisant)

Quand ? Urgence si fébrile, hémothorax ou mauvaise tolérance

Tout le liquide ? Si étiologique ou non cloisonné

\paragraph{Biologie}
1ère intention :
\begin{itemize}
  \item biochimie
    \begin{itemize}
      \item transsudat si protide < 25g/L ou critère de Light faux
      \item exsudat si protide > 35g/L ou critère de Light vrais
    \end{itemize}
  \item cytologie (cf tableau)
  \item recherche germes pygènes, mycobactéries
\end{itemize}
NB : Light : LDH > 200 UI/L $\vee$ protiduse pleuraux/sérique > 0.5 $\vee$ LDH
pleuraux/sérique > 0.6

2eme intention : 
\begin{itemize}
  \item trauma $\wedge$ hématocrite pleural/sanguin > 0.5 $\implies$ probablmt
    hémothoraux (urgence \danger)
  \item amylase pleurale si pancréatique et sous-phrénique
  \item si triglycéride > 1.1g/L $\implies$ chylothorax
\end{itemize}

\begin{table}[htpb]
  \centering
  \caption{Épanchements pleuraux avec exsudats : étiologies}
  \label{tab:label}
  \begin{tabular}{lllll}
    \toprule
    & Cellules tumorales & Neutrophiles & Lymphocytes & Éosinophiles\\
    \midrule
    Néoplasique & Métastasique &  & Cancer & Cancer\\
                & Mésothéliome &  & Lymphome & \\
                & Hémopathies malignes &  &  & \\
    \midrule
    Infectieux &  & Parapneumonique & Tuberculose & Parasitose\\
    \midrule
    Autres &  & Embolie pulmonaire & Sarcoïdose & Hémothorax\\
           &  & Pancréatite & Chylothorax & Pneumothorax\\
           &  & Sous-phrénique & PR, lupus & Embolie pulmonaire\\
           &  & Oesophage &  & Asbestosique bénigne\\
           &  &  &  & Médicament\\
    \bottomrule
  \end{tabular}
\end{table}


\section{203 : Opacités et masses thoraciques}%
\label{sec:203_opacites_et_masses_thoraciques}

\begin{itemize}
  \item < 3mm : micronodules
  \item $[3,30]$mm : nodules
  \item > 30mm : masses
\end{itemize}

\subsection{Nodules}
Origine maligne probable si :
\begin{itemize}
  \item homme, > 50 ans, fumeur
  \item carcinogènes pro, > 1cm (> 3cm ++)
  \item contours spiculés, polylobés, irrégulier
  \item attire structures proches
  \item augmente de taille
  \item pas de calcifications
  \item fixe TED-FDG
\end{itemize}
Certitude = histologie

\textbf{Tumeurs malignes}  
\begin{itemize}
  \item cancers bronchopulmonaires primitifs : > 50 ans, fumeur, souvent nodule
    solitaire
  \item secondaires : opacités rondes régulières
\end{itemize}

\textbf{Tumeurs bénignes} 
\begin{itemize}
  \item Hamartochondrome (freq++) : "pop-corn", pathognomonique
  \item Tumeurs  carcinoïdes
\end{itemize}

\textbf{Non tumorales}
\begin{itemize}
  \item infectieux 
    \begin{itemize}
      \item abcès à pyogène (contexte aigü fébrile)
      \item bactérie filamenteuse crossance Tente
      \item tuberculome ($\implies$ prélèvements)
      \item kytes hydatiques ("membrane flottente")
      \item aspergillome (opacité ronde + croissant gazeux)
    \end{itemize}
  \item granumolatose avec polyangétie
  \item nodules rhumatoïdes
  \item atélectasies
  \item masses pseudo-tumorale silicotiques (micronodules, confluents ?)
  \item malformations artérioveineuses
\end{itemize}

\paragraph{Examens}
\begin{itemize}
  \item TDM et TEP
  \item Fibroscopie pronchique systémitaque
  \item Ponction transpariétal sous TDM (sauf insuffisance respi)
  \item Autres : thoracotomie, médiastinoscopie si ADP médiastinales fexant en TEP-FDG
\end{itemize}
Prélevement si solide, > 8mm, hypermétabolique. Sinon surveillance TDM (sauf non solide et image résolutive à 6 semaines)

\subsection{Médiastin}
Diagnostic : limite externe nette, raccord pente douce, limite interne non
visible, tonalité hydrique

DD: intraparenchymateux, pariétal $\implies$ TMD

\paragraph{Médiastin antérieur}
\begin{itemize}
  \item Supérieur : goître plongeant $\implies$ TMD : continuité glande thyroïde
  \item Moyen : 
    \begin{itemize}
      \item tumeurs thymiques : épithéliales (thymomes, carcinomes thymiques),
        lymphomes thymiques, kystes, tumeurs bénignes
      \item Tumeurs germinales : bénignes, séminomateuses, non séminomateuses
        (carcinomes embryonnaires, vitellines, choriocarcinomes)
    \end{itemize}
  \item Inférieur : kystes pleuropéricardiques
\end{itemize}

\paragraph{Médiastin moyen}
\begin{itemize}
  \item Tumoral : cancer bronchopulmonaires, lymphomes, LLC, cancers
    extra-thoraciques
  \item Non tumoral : sarcoïdose, tuberculose, silicose, infections
    parenchymateuses chroniques, histoplasmose (Amérique du Nord)
  \item Autres : insuf. cardiaque gauche
\end{itemize}

\paragraph{Médiastin postérieur} neurogènes

Diagnostic :
\begin{itemize}
  \item médiastin antérieur : 
    \begin{itemize}
      \item $\alpha$-foetoprotéine (tumeurs vitellines), HCG\footnote{Hormone gonadotrophine chorionique} (choriocarcinomes)
      \item ponction transpariétale
      \item médiastinotomie
      \item chir si complète et peu mutilante
    \end{itemize}
  \item médiastin moyen : médiastinoscopie $\vee$ ponction transbronchique
  \item médiastin postérieur : 
    \begin{itemize}
      \item ponction transpariétale, transoesophagienne
      \item chir si complète et peu mutilante
    \end{itemize}
\end{itemize}

NB : urgence si jeune et suspicion de tumeur germinale non séminomateuses

\section{204 : Insuffisance respiratoire chronique}
\label{sec:org6d633b6}
\subsection{Mécanismes}
\label{sec:org47f478d}
Hypoxémie
\begin{itemize}
\item Inadéquation ventilation/perfusion :
\begin{itemize}
\item effet shunt (mauvaise ventilation) => oxygénothérapie corrige
\item shunt vrai (communication anat. ou non ventilé) => oxygénothérapie ne corrige
\end{itemize}
pas
\item Hypoventilation alvéolaire : pure (commande, neuromusc) ou effet "espace mort"
\end{itemize}
(mauvaise perfusion)
\begin{itemize}
\item Atteinte de la surface d'échange
\end{itemize}
Hypercapnie : hypoventilation alvéolaire (pompe ventilatiorie/commande centrale
ou effet espace mort)

\subsection{Conséquences}
\label{sec:orgf6b3986}
Hypoxémie : Polyglobule, rétention hydrosodée (fréquente), hypertension pulmaire

Hypercapnie : compensée par le rein

\subsection{Étiologies}
\label{sec:org5310651}
Hypoxémie si PaO\(_{\text{2}}\) < 70mmHg (arbitraire). Selon EFR
\begin{center}
\begin{tabular}{llllll}
\toprule
TV & TVO & TVR T\(_{\text{LCO}}\)/V\(_{\text{A}}\) bas & TVR T\(_{\text{LCO}}\)/V\(_{\text{A}}\) normal & mixte & non\\
\midrule
 & BPCO, asthme & P. interstitiel & Sd obésité-hyperventil & DDB & HTP\\
 & bronchiolite &  & cage thoracique & Muscoviscidose & \\
\midrule
Méchanisme & V\(_{\text{A}}\)/Q & Surf d'échange & Hypoventilation & V\(_{\text{A}}\)/Q & Surf d'échange\\
\midrule
Atteinte & échangeur & échangeur & pompe/central & échangeur & vasculaire\\
\bottomrule
\end{tabular}
\end{center}

\subsection{Diagnostic}
\label{sec:orgf8776e2}
\subsubsection{Symptômes}
\label{sec:org35a8cf3}
IRC = dyspnée (sous-évaluée), neuropsy + patho initiale

Physique : 
\begin{itemize}
\item IRC : cyanose, insuf. cardiaque D (turgescence jugulaire, oedeme MI, reflux
hépato-jugulaire)
\item patho : 
\begin{itemize}
\item obstructive : distension thoracique, dimin. bilat murmure vésicul
\item restrictive : râle crépitant des bases, hippocratisme digital
\end{itemize}
\end{itemize}

\subsubsection{Diagnostic}
\label{sec:org7599c0f}
PaO\(_{\text{2}}\) < 70 mmHg (gaz du sang : hypercapnie)

Étiologie : EFR donne TVO (VEMS/CVF < 70\% => BPCO), TVR (échangeur/pompe) ou
mixte (DDB, mucov)
Radio thorax
Autres : NFS (polyglobulie), ECG (dextrorotation, BDB droite, repolarisation),
écho cardiaque systématique (éval. ventricule D, dépistage du G)

\subsection{Traitement}
\label{sec:org870a2d5}
Cause, arrêt tabac, vaccins (grippe, pneumocoque), réhabilitation respi.

Oxygénothérapie de longue durée : 2 mesures à 2 semaines avec
\begin{itemize}
\item obstructive = PaO\(_{\text{2}}\) < 55mmHG \(\vee\) ( \(\in\) [55, 60] mmHG \(\wedge\) hypoxie
tissulaire (Ht > 55\%, HTP, insuf. vetre D, SpO\(_{\text{2}}\) nocturne \(\le\) 88\%)
\item restrictivé = PaO\(_{\text{2}}\) < 60 mmHg
\end{itemize}
Efficace si  IRC après BPCO, 15h/jour, forme gazeuse ou liquide

Ventilation long cours : IRC restrictive, la nuit.
Chir : 200/an

\subsection{Pronostic}
\label{sec:org2ec66b4}
Irréversible, risque = insuf. respi aigüe de causes : IR basse, dysfonction
cardia G, EP


\section{205 : BPCO\footnote{Bronchopneumopathie chronique obstructive}}
\label{sec:205-bpco}

\subsection{Définitions}
BPCO = \{toux, dyspnée, expector., infections respi basses\} récurrentes et
TVO\footnote{Trouble ventilatoire obstructif} (VEMS/CVF < 0.7)
\textbf{persistant}\\
Exacerbation aigùe = aggravation \(\ge 2\) jours\\
Bronchite chronique (toux productive quotidienne \(\ge 3\) mois, \(\ge 2\) ans),
emphysème (élargissement espaces aériens distaux + destructions parois
alvéolaires) inclus dans BPCO\\
\begin{table}[htbp]
\caption{Différences asthme-BPCO}
\centering
\begin{tabular}{ll}
\toprule
Asthme & BPCO\\
\midrule
Obstructive non réversible & Obst. réversible\\
Jeune, atopique & Fumeur, > 40 ans\\
Survient \textasciitilde{}40 ans & Enfance\\
\bottomrule
\end{tabular}
\end{table}

Sévérité : 
\begin{itemize}
\item obstruction : GOLD (1 à 4, [80, 100] [50,80], [30,49], [0, 30])
\item dyspnée : MRC (0 à 4)
\item fréquences exacerbations (\$\(\ge\) 2\$/an = grave)
\end{itemize}

Épidémio: en augmentation dans le monde. \\
FR : \{tabac++, aérocontaminants professionnels\}, \(\alpha{}1\) antitrypsine

Évolution : perte fonction respi, exacerbations, handicap respi, risque
d'insuffisance respi, comorbidité CV = 1ere cause de mortalité\\
Score BODE\footnote{Body mass index, Obstruction, Dispnea, Exercice} pour la prédiction.

\subsection{Diagnostic}
Signes fonctionnels : dyspnée++, toux, expectorations.\\
Signes physiques : après TVO : \(\nearrow\) temps expiratoire, \(\searrow\) murmure vésiculaire et bruts coeur, distension thoracique

\subsubsection{EFR}
\begin{itemize}
\item spirométrie (bronchodilat. ou corticoïdes)
\item pléthysmographie (distension pulmonaire)
\item DLCO (patho si < 70\%)
\item si VEMS < 50\% ou $SpO_2< 90$\% : gaz du sang, test d'exercice
\end{itemize}

\subsubsection{Autres}
(TDM), ECG si VEMS < 50\%, NFS (polyglobulie, anémie), (dosage \(\alpha1\) antitrypsine

\subsection{Traitement}
\begin{figure}[htpb]
  \centering
\tikz \graph [
  % Labels at the middle 
  edge quotes mid,
  % Needed for multi-lines
  nodes={align=center},
  sibling distance=3cm,
 level distance=2cm,
  edges={nodes={fill=white}}, 
tree layout]
{
  Dyspnée/exacerbations -> {
    BD longue durée[>"oui", draw]-> 
    {
      2 BD [draw, >"Exacerb"] -> "Cortico + 2 BD"[draw, >"exacerb"];
      "Cortico. inhalé\\
      + $\beta2$ longue durée"[draw, >"dyspnée"]-> "Cortico + 2 BD"[draw,
      >"insuffisant"]
      -> "Réévaluation";
      2 BD [draw, >"Exacerb"] -> "Réévaluation"[>"dyspnée"];
      ;
    };
    BD courte durée [>"non", draw];
  }
};
\caption{Traitement BPCO}
\end{figure}


Arrêt tabac++, vaccins grippe et pneumocoques, réhabilitation respiratoire, oxygénothérapie, chirurgie possibles

\subsection{Exacerbations BPCO}
\label{subsec:ebpco}
\subsubsection{Diagnostic :}
\begin{itemize}
\item BPCO connu : \(\nearrow\) dyspnée, toux/expect.
\item sinon : cf détresse respi
\end{itemize}

Déclenchants : majorité = infectieux mais souvent pas de facteur précis
(\bact{influenzae}, \bact{pneumocoque}, \bact{catarrhalis})

DD : PAC, dysfonction cardiaque gauche, embolie pulmonaires, pneumothorax, médicaments CI, trauma/chir thoracique, insuffisance cardiaque gauche aigüe.

\subsubsection{Explorations}
Imagerie thorax, ECG, NFS, CRP, iono, créat, gazométrie

\subsubsection{Traitement}
Bronchodilatateurs $\beta2$ agonistes courte-durée.\\
ATB : majoration purulence : amox + acide clav si FR, sinon amox-acide
clav/pristinamycine/macrolides\\
Autres \faHospitalO : oxygénothérapie, kiné, HBPM, (assistance ventilatoire)

\section{206 : Pneumopathies infiltrantes diffuses}
\subsection{Présentation}
Clinique : dyspnée d'effort prgorsessive.\\
EFR : TVR\footnote{trouble ventilatoire restrictif} ( CPT < 80\% et VEMS/CVL >
70\% ) et DLCO < 70\%, hypoxémie, désaturation\\
Radio : opacités parenchymateuse non systématisées bilatérales

\subsection{PID aigüe}
\subsubsection{Étiologies}
Connues : lymphangite carcinomateuse, insuf. cardiaque gauche, médicamenteuse\\
Inconnues : sarcoïdose, fibrose plumonaire idiopathique
\subsubsection{Démarche}
\begin{itemize}
\item Contexte (ATCD, ID, exposition)
\item ECG, BNP, echo cardiaque
\item LBA si possible
\item PEC thérapeutique (réa si détresse respi, \(O_2\), ATB probab si fièvre, arrêt de médic. pneumotoxiques)
\end{itemize}

\subsection{PID subaigüe/chronique}
\subsubsection{Démarches}
Interrogatoire++ : terrain (sarcoïdose=25-45 ans, FPI si > 60 ans), tabac (histiocytose langerhansienne, DIP), toxico, médic, ATCD radio, exposition

Clinique : état général, signes de connectivite


\begin{table}[htbp]
\caption{Biologie PID subaigüe}
\centering
\begin{tabular}{ll}
\toprule
Examen & Maladie\\
\midrule
NFS, CRP & Sd inflammatoire\\
 & Hyperéosinophilie, lymphopénie\\
BNP & Insuf. cardiaque\\
Créat & Insuf. rénale\\
Précipitines sériques & Hypersensib. (si contexte)\\
CEA, calcémie, calciurie & Sarcoïdose\\
Facteur rhumatoïdes etc & Connectivites\\
ANCA & Vascularite\\
Séro VIH & Opportuniste\\
\bottomrule
\end{tabular}
\end{table}

\begin{table}[htbp]
\caption{LBA PID subaigüe}
\centering
\begin{tabular}{ll}
\toprule
Normal & 80\% macrophages\\
 & < 15\% lymphocytes\\
 & < 5\% PNN\\
 & < 2\% PNE\\
\midrule
Alvéolite & Hypercellularité totale\\
Histiocytose langerhansienne & Macrophage\\
Sarcoïdose, PHS & Lymphocytaire\\
P. à éosinophiles & Éosinophilique\\
POC\footnote{Pneuompathie organisée cryptogénique} & Panachées\\
Hémorragie alvéolaire & Rosé\\
Protéinose alvéolaire primitive & Laiteux\\
\bottomrule
\end{tabular}
\end{table}

Examens complémentaires : fibro + LBA > biopsie bronchique > biopsie pulmonaire chir, biopsie transbronchique

\subsubsection{Oedeme pulmonaire}
Mécanisme : Surcharge hémodynamique\\
Clinique : HTA, coronaropathie, valvulopathie mitrale\\
Diagnostic : ECG, BNP, écho coeur\\
Imagerie : Péri-hilaire

\subsubsection{Tuberculose}
Mécanisme : BK\\
Clinique : Contage, AEG, hémoptysie\\
Diagnostic : Expectorations : ED, culture, biopsie transbronchique\\
Imagerie : 
\begin{itemize}
\item pulmonaire = nodules, infiltrats, excavations
\item miliaire = micronodules diffus
\end{itemize}

\subsubsection{Médicaments}
Imagerie : condensations, verre dépoli, épanchement pleural

\subsubsection{Pneumopathies d'hypersensibilité}
Mécanisme : Ag organiques\\
Clinique : 
\begin{itemize}
\item aigüe : sd peudo-grippal quelques heurs
\item subaigüe : semaines/mois avec toux, fébricule, rales crépitants, squeaks
\item chronique : dyspnée, toux sèche
\end{itemize}
Diagnostic : Sérologie, LBA\\
Imagerie : Micronodules centrolobulaires flous, verre dépoli (lobes supérieurs)\\
Traitement : éviction Ag

\subsubsection{Pneumoconioses}
Mécanisme : Amiante, silice\\
Clinique : Exposition\\
Imagerie : 
\begin{itemize}
\item silicose : opacités micronodulaires diffuses \(\implies\) masses pseudotumorales. Peut donner un cancer bronchique
\item asbestose : opacités linéaires non septables des bases \(\parallel\) ou \(\bot\) plèvre, réticulations et rayons de miels comme FPI. Évolue vers insuf respi chronique.
\end{itemize}

\subsubsection{Sarcoïdose}
Mécanisme : Signes extra-respiratoires\\
Diagnostic : anapath : extra-pulmonaire, biopsie EP et TB. Adénopathies médiastinales \\
Imagerie : Nodules, micronodules (ditribution lymphatique), adénopathie, hyperdensités, distorsions bronchiques

\subsubsection{Fibrose pulmonaire idiopathique}
Clinique : Dyspnée d'effort progressive, toux sèche, hippocratisme digital, crépitants sec base\\
EFR : trouble ventilatoire restrictif, diminution DLCO\\
Imagerie : Réticulations, bronchectasies, rayons de miel. Domine sous pleur et bases

\subsubsection{Connectivites}
Mécanisme : Dysimmunitaire\\
Clinique : Extra-respi : polyarthrite rhumatoide, sclérodermie, lupus, vascularite\\
Diagnostic : Ac spécifiques\\
Imagerie : Réticulations, hyperdensités, bronchectasies

\subsubsection{Pneumopathie interstitielle non spécifique}
Origine : connectivite, médicaments (idiopathique)
Imagerie : verre dépoli, réticulations, bronchectasies (sauf extrème périphérie du poumon)

\subsubsection{Proliférations tumorales}
Lymphangite carcinomateuse : toux sèche, rebelle. Radio : épaississements nodulaires des septas intralobulaires. Diagnostic par biopsies des éperons\\
Carcinome lépidique : verre dépoli. 


\section{207 Sarcoidose}
Maladie : systémique, cause inconnue, hétérogène, ubiquitaire. Début 25-45ans
dans 2/3\\
Atteinte médiastino-pulmonaire 90\%

\subsection{Expression}
\label{sec:org39048da}
\subsubsection{Pulmonaire}
\label{sec:org4ec1d7e}
Toux (dyspnée)
Radio : 4 stades
\begin{itemize}
\item I : adénopathies hilaires bilatérales symétriques
\item II : + atteinte parenchyme (micronodulaire diffus, parties moyennes supérieures)
\item III : atteinte parenchyme isolée
\item IV : fibrose = opacités parenchymateuses rétractiles + ascension hiles, distorsion bronchovasc (sup et post)
\end{itemize}
TDM : atteinte parenchyme = micronodule selon lymphatiques. Utile pour : formes atypiques ou détection précoce (fibrose, complications [greffe aspergillaire])\\
EFR : sd restrictif, DLCO $\searrow$\\
Endoscopie bronchique (non system) : normal/en "fond d'oeil". Biopsie : \{éperons, LBA\} > \{ponction ganglions médiastinaux, transbronchique\} > médistanoscopie\\
Formes atypiques : TVO, cavitaires, pseudonodulaires/alvéolaires

\subsubsection{Extra-pulmonaire}
\label{sec:org316c190}
\begin{itemize}
\item Oeil : uvéite antérieure aigue (toujours cherche uvétie postérieure)
\item Peau : nodules cutanés, lupus pernio, érythème noueux
\item Adénopathies
\item Foie
\item Moins fréquentes : nerveux (sd méningé, paires craniennes), ORL (obstruction
nasale,
\end{itemize}
sd Mikulicz, sd Heerfordt), ostéo-articulaire (bi-arthrite cheville =
spécfique++), 
coeur (BAV, bloc branche droit), rein (\(\nearrow\) créatininémie)
\begin{itemize}
\item Généraux : asthénie (pas de fièvre sauf sd de Löfgren)
\end{itemize}
Sd de Löfgren = érythème noueux + adénopathies hilaires médiastinales (+ fièvre)
\subsubsection{Biologie}
\label{sec:orgbfe5d87}
\begin{itemize}
\item Hypercalciurie
\item Lymphopénie CD4
\item Hypergammaglobulinémie
\item Enzyme de conversion de l'angiotensine sérique (ECA)
\end{itemize}
\subsection{Diagnostic}
\label{sec:org6670330}
Clinique + radio + lésions granulomateuses tuberculoides sans nécrose caséeuse +
élimination DD
\subsection{Évolution}
\label{sec:org707f9ea}
< 2 ans : évolution favorable sans traitement.\\
Chronique > 2 ans : attention au vital/fonctionnel \\
Suivi : 3-6 mois\\
Pronostic : 80\% favorable sans traitement, 10\% séquelles, 5\% DC

\begin{table}[htbp]
\caption{Pronostic de la sarcoidose}
\centering
\begin{tabular}{ll}
\toprule
Négatif & Positif\\
\midrule
> 40 ans & Érythème noueux\\
Chronicité & Forme aigüe\\
Stade III, IV & Stade 1 asymptomatique\\
Extra-respi grave & \\
\bottomrule
\end{tabular}
\end{table}

Atteintes :
\begin{itemize}
\item pulmonaire : insuf. respir chronique, principace cause DC
\item extra-thoracique : attention fonctionnel/vital
\end{itemize}

\subsection{Traitement}
\label{sec:orgc05b9f6}
Atteinte respi : pas si sd de Löfgren ou stade I asymptomatique\\
Corticoïdes > 10 mois à 0.5mg/kg (décroissance par 6-12 semaines)\\
2eme intention : hydroxychloroquine, méthotreaxe, azathioprine\\
3eme intention : cyclophsamide, anti-TNF-\(\alpha\)

\section{222 : Hypertension artérielle pulmonaire}%
\label{sec:hypertension_arterielle_pulmonaire}

Circulation : (basse pression, faible résistante) $\implies$ forte résistance

Critère : PAPm\footnote{Pression de l'artèr pulmonaire moyenne} $\ge 25$ mmHG et 
\begin{itemize}
  \item PAPO\footnote{Pression artérielle pulmonaire occluse $\approx$ pression
    capillaire pulmonaire} $\le 15$ mmHG si précapillaire
  \item PAPO $> 15$ mmHG si postcapillaire
\end{itemize}
5 groupes : 
\begin{enumerate}
  \item hypertension artérielle pulmonaire (HTAP)\footnote{Idiopathique,
      héritable, médicaments, maladie veino-occlusive, hémangiomatose capillaire
    pulmonaire, HTP persistante du nouveu-né}: pré-capillaire
  \item \textbf{cardiopathie gauche}  : post-capillaire (fréq+++)
  \item HTP maladie \textbf{respiratoire chronique}  : pré-capillaire (freq++)
  \item HTP post-embolique chronique : pré-capillaire
\end{enumerate}

\paragraph{Pronostic}
6 cas/million (idiopathique), femme.

Survie avec ttt : 58\% à 3 ans

\subsection{Diagnostic}
Découverte : dyspnée, dépistage

Détection :
\begin{itemize}
  \item fonctionnel : dyspne d'effert+++ progressive (lipothymie à l'effert,
    syncope, asthénie, douleurs angineuses, palpilations, hémoptysies)
  \item physique : 
    HTP (signe de Carvallo\footnote{souffle holosystolique
    d'insuf. tricuspid majoré à l'inspiration}, éclat B2, souffle diastolytique
    d'insuf\\
    insuf cardiaque D (tachy, galop, turgescence jugulaire, reflux
    hépato-jugulaire, HMG\footnote{hépatomégalie}, OMI\footnote{oèdeme des
    membres inférieurs}, anasarque)
  \item imagerie thorax : dilatation artères plumon, élargissement coeur D
  \item ECG : hypertrophie D, troulbe rythme
\end{itemize}

Écho cardiaque transthoracique = non invasif de référence. Diagnostifg par
cathétérisme cardiaque D avec \textbf{PAPm $\ge 25$ mmHg}

\paragraph{Démarche}
Si écho cardiaque compatible avec HTP :
\begin{itemize}
  \item Cardiopathies G, maladies respi + exmans pour groupe 2 et 3. Si confirmé
    : \faHandStopO
  \item Sinon regarder signes thrombo-embolie chronique (scinti, angiosca) : Si
    groupe 4, \faHandStopO
  \item Sinon confirmer HTP précapillaire 
  \item Si confirmée, tester pour groupe 1 (connectivites, médicaments, VIH,
    cardiopathie conégin, HT portale, schistosomiase) ou groupe 5
\end{itemize}



\section{224 : Embolie pulmonaire et thrombose veineuse profonde}%
\label{sec:224_embolie_pulmonaire_et_thrombose_veineuse_profonde}

Maladie fréquente (1 cas /10 000 si < 40 ans, 1/100 si > 75 ans) et grave.

= {embolie pulmonaire (EP), thrombose veineuse profonde (TVP)}

Facteurs de risque :
\begin{itemize}
  \item acquis : majeurs = chir < 3mois, trauma MI\footnote{membres inférieurs},
    \faHospitalO{} aigü, cancer en cours  ttt, sd antiphospholipide, sd
    néphrotique
  \item constit : rare = déficit (antithrombien, prot. C, S), fréq = (mutation {Leiden,
    prothrombine}, facteur VIII > 150\%)
\end{itemize}
Complications : DC, récidive (mortelle ou non), séquelle (HTP thrombo-embolique
chronique si EP, sd post-phlébite si TVP)

Risque de récidive dépend de la clinique : élevé si (non provoqué par facteur
majeur ou modéré) $\vee$ facteur persistant)

Conséquence :
\begin{itemize}
  \item hémodynamique : $\nearrow$ pression artérielle pulmonaire, dilatation
    VD\footnote{ventricule droit}, compression VG
  \item respiratoire : hypoxémie (effet espace mort (non perfusé) $\implies$ effet shunt
    (ventil/perfusion diminué))
\end{itemize}

\subsection{Diagnostic de l'embolie pulmonaire}
\begin{figure}[htpb]
  \label{fig:ep-diag}
  \centering
  \resizebox{0.5\linewidth}{!}{
    \tikz \graph [
  % Labels at the middle 
      edge quotes mid,
  % Needed for multi-lines
      nodes={align=center},
      level distance=2cm,
      sibling distance=3cm,
      edges={nodes={fill=white}}, 
    layered layout]
    {
      "Proba. clinique" -> {
        "D-dimères" [>"faible", draw] -> {
          "Pas de ttt" [>"négatif"];
          "\texttt{\$examen}" [>"positif", draw];
        };
        "\texttt{\$examen}" [>"forte", draw] -> {
           Ttt [>"positif"];
        };
      };
    };
  }
  \caption{Diagnostic général pour l'EP, TVP}
\end{figure}


\paragraph{Suspicion}
Clinique : douleur thoracique (type pleurale) $\vee$ dyspnée isolée (ascult.
normale !) $\vee$ état de choc

Radio thoracique, ECG : élimine les DD

Probabilité clinique : important, suivant des scores (Wells, Genève)

\paragraph{Examens}
Voir la figure~\nameref{fig:ep-diag} avec \texttt{examen} = angioscanner.

D-dimère positifs = $\max(\text{âge}, 50) \times 10 \mu{}$g/L

CI à l'angioscan : insuf. rénale sévère (< 30ml/min) $\implies$ scintigraphie

Si angioscan ou scintigraphie négatif : pas d'EP !

\danger{} si EP grave (état de choc) : angioscan immédiatement ou échographie
cardiaque en attendant. Traitement après écho si pas d'accès à l'angioscan
\skull\\
Sur l'écho, chercher dilatation cavité D, HTP, septum paradoxal

\textit{Si grossesse}  : D-dimères $\implies$ écho. veineuse $\implies$ angioscanner
(prévenir le pédiatre \skull)

\subsection{Diagnostic de TVP}
Voir la figure~\nameref{fig:ep-diag} avec \texttt{examen} = échographie veineuse
des MI

\subsection{Traitement}
\paragraph{Principes}
Urgence \skull $\implies$ anticoagulant

CI : coagulopathie sévère, hémorragie intracrânienne spontanée, hémorragie
active difficilement contrôlable, chir récente, (thrombopénie à l'héparine)

\paragraph{Types de traitement}
Héparines + relais AVK dès injection IV
\begin{itemize}
  \item HBPM, fondaparinux > HNF, sauf si IR sévère (< 30ml/min)
  \item arrêt si 5 jours d'AVK et IV (ensemble) $\wedge$ INR $\in [2,3]$ à 24h
\end{itemize}
Anticoagulants oraux directs : rivaroxaban, apixaban (France)
\begin{itemize}
  \item rapide, demi-vie courte
  \item facteur X
  \item CI : IR sévère, grossesse, interfaction médic (cytochrome 3A4 ou
    P-glycoprroténie)
\end{itemize}
Éucation thérapeutique\\
Autres :
\begin{itemize}
  \item filtre cave (si CI absolue coagulant $\vee$ EP récidivant)
  \item fibrinolyse (si EP + choc, sauf si hémorragie active,
    AIC\footnote{Accident ischémique cérébral}< 2 mois,
    hémorragie intracrânienne)
  \item embolectomie (très rare)
  \item contention veineuse (sauf si EP sans TVP)
  \item lever +1h
\end{itemize}

\paragraph{Stratégie}
Score sPESI = "90 100 110 C C" $\implies$ sat < 90\%, PAS < 100 mmHg, FC >
110/min, Cancer, insuf Cardiaque chronique:

Risque faible (sPESI = 0) : \faHospitalO{} < 48h, anticoagulation
Risque intermédiaire (sPESI > 0)
\begin{itemize}
  \item dysfonction VD $\vee$ élévation biomarqueurs\footnote{BNP, NT-pro-BNP,
      troponine} : \faHospitalO{}
    médecine, anticoagulation
  \item dysfonction VD $\wedge$ élévation biomarqueurs : urgence \danger{}
    $\implies$ USI\\
    $O_2$, scope
    Anticoag : HNF/HBPM puis AVK/AOP à 48-72h\\
    (trombolysie si choc)
\end{itemize}
Haut risque (choc : PAS < 90mmHG $\vee$ -40mmHg) : urgence \skull $\implies$ réa
\begin{itemize}
  \item $O_2$ (ventilation méca.), scope
  \item anticoag HNF (!)
  \item thrombolyse avec arrêt HNF tant que TCA > 2x témoin
  \item (embolectomie)
\end{itemize}

\paragraph{Durée}
3 mois si 1ere EP/TVP provoque par facteur majeur transitoire $\vee$ risque
hémorragique élevée. 6 mois ou plus sinon

\paragraph{Étiologie}
Chercher cancer occulte dans tous les cas (clinique, radio poumon, NFS VS,
dépistage globux)

Bilan coga : {antithrombine, prot C, S}, mutation {Leiden, prothrombine}, sd
antiphospholipides

\paragraph{Prophylaxie}
\begin{itemize}
  \item Post-op : (chir et > 40 ans) $\vee$ (chir hanche/genou/caricinologique, anomalie
coag, (> 40 ans et ATCD MTEV))
  \item polytrauma $\vee$ ({rhumato, inflammatoire intestin, infectin} + 1 FR)

\end{itemize}




\appendix
\section{228 : Douleur thoracique aigüe et chronique}%
\label{sec:228_douleur_thoracique_aigue_et_chronique}

\subsection{Signes de gravité}
\begin{itemize}
  \item Respi : cyanose, tachypnée, lutte avec tirage, balancement thoraco-abdominal
  \item CV : pâleur, tachycardie, hypotension, choc (marbrures, extrémités
    froides)
  \item neuro : lipothymie/syncope, agitation/trble vigilance, général (sudation)
\end{itemize}

\danger arrêt cario-respi ! bradypnée, (bradycardie + choc + troubles vigilance

\subsection{Examens}
Fréquence respi, $SpO_2$, radio thorax, ECG.
Si {brady, tachy}pnée ou $SpO_2 < 95\%$, gaz du sang

\subsection{Urgences vitales}
\begin{itemize}
  \item Syndrome coronaire agiüe (fréquent++ 1/3) : ECG + troponines
  \item Embolie pulmonaire (fréquent) : suspicion si douleur thorax, pas
    d'anomalie ascult, RX thorax "normale" \textit{surtout}  si hypoxémie +
    facteurs de risque\\
    Dyspnée ou douleur thoracique aigüe chez TVP = EP
  \item dissection aortique (exceptionnelle) : échocardio + angioscanner
  \item Tamponnade (peu fréq) : suspicion (hypotension réfractaire, insuf.
    cardiaque D aigüe, microvoltage + alternance ECG) $\implies$ echo cardiaque
  \item Pneumothorax : ATCD (!), radio thorax \\
    pneumomédiastin (rare) : scanner
\end{itemize}

\subsection{Non urgent}
Rythmées par respiration :
\begin{itemize}
  \item post-traumatique
  \item pneumonie infectieuses : radio thorax
  \item épanchement pleural (douleur latéral-base, majorée inspiration, toux)
  \item infarctus pulmonaire (doelur basithoracique, faible hémoptysie)
  \item trachéobronchite aigüe
  \item musculosquelettique, nerveurse : tumeurs costales, lésivos vertèbres,
    névralgies cervicobrachiales
\end{itemize}
Non rythmées :
\begin{itemize}
  \item angor d'effort stable (calmée 2-5min post-effort)
  \item péricardituqe (viral si aigü, tuberculose/néoplasie sinon)
  \item cocaïne (fréquente) : SCA, myopéricardite,pneumothorax
  \item zona thoracique (brûlures, hyperesthésie 24h avant)
  \item digestives : reflux gastro-oesophagen, spasmes oesophagiens. Exclure SCA
    \skull
  \item psychogènes
\end{itemize}




\section{Gaz du sang}
\label{appendix:gds}
Le pH est déterminé par l'équilibre entre les bicarbonates (\ch{HCO_3-}) et
$PCO_2$ :
\begin{equation}
  pH = K_1 + log\frac{[\ch{HCO_3-]}}{K_2 p_{CO_2}}
\end{equation}
avec $K_1$, $K_2$ constante.

Un déséquilibre sur un terme induit une compensation sur l'autre. Si le
déséquilibre n'est pas compensé, on aboutit à une acidose ou une alcose.

\begin{table}[htpb]
  \centering
  \caption{Gaz du sang artériel}
  \label{tab:gds}
  \begin{tabular}{ll}
  \toprule
  \(PO_2\) & [80, 100] mmHg\\
  \(SaO_2\) & [95, 98] \%\\
  \(PCO_2\) & [35, 45] mmHg\\
  \ch{HCO_3^-} & [22, 29] mmol/L\\
  pH & [7.38, 7.24]\\
  \bottomrule
  \end{tabular}
\end{table}

Insuf. respi chonique : 
$$PCO_2 \nearrow \implies \ch{HCO_3-} \nearrow \implies \text{acidose compensée}$$
Hypoventilation alvéolaire : 
$$PCO_2 \nearrow \implies \ch{HCO_3-} \text{ N ou} \nearrow \implies \text{acidose}$$
IR, acidocétose... :
$$ \ch{HCO_3-} \searrow \implies PCO_2 \text{ N ou} \searrow \implies \text{acidose}$$
Vomissements, sonde NG :
$$ \ch{HCO_3-} \searrow \implies PCO_2 \text{ N ou} \nearrow \implies \text{alcalose}$$
Hyperventilation alvéolaire :
$$ PCO_2 \searrow \implies \ch{HCO_3-} \text{ N ou} \searrow \implies \text{alcalose}$$

\section{Physiologie}%

$P_{alv}$ = pression alvéolaire, $P_{ip}$ = pression interpleurale (dans la
plèvre), pression transpulmonaire = $P_{ip} - P_{alv}$.

\begin{figure}[htpb]
  \centering
  \caption{Inspiration (gauche), expiration (droite)}
  \tikz \graph [ nodes={align=center}, layered layout]
  {
    Contraction diaphragme -> Expansion thorax -> "$P_{ip} < P_{atm}$"
    -> Hausse pression transpulmonaire -> Expansion poumons 
    -> "$P_{alv} < P_{atm}$"
    -> Arrivée d'air dans les alvéoles;
  };
  \tikz \graph [ nodes={align=center}, layered layout]
  {
    "Arrêt contraction\\ diaphragme et intercostaux" -> Rétraction thorax 
    -> "Valeur initiale de $P_{ip}$"
    -> Valeur initiale pression transpulmonaire 
    -> Rétraction poumons 
    -> "$P_{alv} > P_{atm}$"
    -> Expulsion d'air depuis les alvéoles;
  };

\end{figure}
\end{document}
