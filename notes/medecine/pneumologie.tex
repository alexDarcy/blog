\documentclass{article}
\usepackage[hidelinks]{hyperref}
\usepackage{longtable}
\usepackage{booktabs}
\usepackage{graphicx}
\graphicspath{{../../pictures/medecine/}}
% French
\usepackage[T1]{fontenc}
\usepackage[francais]{babel}
%-------------------------------------------------------------------------------
% For graphs
\usepackage{tikz}
\usetikzlibrary{graphs}
\usetikzlibrary{graphdrawing}
\usetikzlibrary{arrows,positioning,decorations.pathreplacing}
\usegdlibrary{trees, layered, force}
\usetikzlibrary{quotes}
\usepackage{biocon}
%-------------------------------------------------------------------------------
% No spacing in itemize
\usepackage{enumitem}
\setlist{nolistsep}
% tightlist from pandoc
\providecommand{\tightlist}{%
  \setlength{\itemsep}{0pt}\setlength{\parskip}{0pt}}
 % Danger symbol (need fourier package)
\newcommand*{\TakeFourierOrnament}[1]{{%
\fontencoding{U}\fontfamily{futs}\selectfont\char#1}}
\newcommand*{\danger}{\TakeFourierOrnament{66}}
% Skull (need Symbola font)
\usepackage{amsmath,fontspec,newunicodechar}
\newfontface{\skullfont}{Symbola}[Scale=MatchUppercase]
\NewDocumentCommand{\skull}{}{%
  \text{\skullfont\symbol{"1F571}}%
}
% Hospital sign
\usepackage{fontspec} % For fontawesome
\usepackage{fontawesome}
% Diameter
\usepackage{wasysym}

\title{Pneumologie\\
\large Fiches}
\author{A. Praga}

\begin{document}
\maketitle

\input bacteries-header.tex

\section{151 : Infections broncho-pulmonaires communautaires}
\subsection{Bronchite aigue}
Très fréquente, virale 90\%. \\
Diagnostic clinique : épidémie, toux sèche \(\to\) productive, expectorations,
pas de crépitants.\\
Ttt symptomatique seulement !

\subsection{EBPCO} Cf Sec.

\subsection{Pneumonie aigue communautaire}
Clinique : \{toux, expet purulentes, dyspnée\} + \{fièvre, asthénie\} +
crépitants\\
Radio !\\
Si \faHospitalSign : hémocultures, ECBC, antigenurie pneumocoque (PCR,
antigenurie légionelle)\\
\faHospitalSign : signes de gravité (score CRB64 \(\ge 1\)) / incertitude / échec domicile / comorbidité / inobservance

\begin{table}[htpb]
  \centering
  \label{Orientation clinique (non discriminant !)}
  \begin{tabular}{llll}
    \toprule
    & Pneumocoque & Atypique & Légionellose\\
    \midrule
    Début & Brutal & Progressif & Rapidement progressif\\
    Signes & Thoraciques & Extra-thoraciques & Extra-thoraciques\\
    Biologie &  & Anémie hémolytique & Hyponatrémie, rhabdomyolyse\\
    & CG+ chainettes/Agurie &  & Antigénurie\\
    Radio & condensation systématisée & Opacités multifocales & CS ou OM\\
    \bottomrule
  \end{tabular}
\end{table}

\begin{itemize}
  \item Pneumocoque : fréquent+++. Pas de transmission interhumaine
  \item Atypique : \bact{mpneumoniae}, \bact{cpneumoniae}, \bact{psitacci}
  \item Legionnella : pas d'isolement. DO
  \item Pneumonie virale : signes respi + sd grippal. Grippale = diagnostic PCR, traitement = inihbiteur neuramidase \danger \bact{dore}
\end{itemize}
\subsubsection{Traitement}
\label{sec:org48d8e3b}
Oral, 7j (8-14 si légionellose, 21j si légionelles graves/ID)

Urgent, probabiliste, réévalué 48-72h

Ambulatoire :Amoxicilline ou macrolide \(\to\) switch

\faHospitalSign  Amoxicilline \(\to\) réévaluation

Réa : C3G IV + macrolide IV / FQ pour pneumocoque

\subsubsection{Échecs}
\label{sec:orgaaf4c70}
\begin{itemize}
  \item Compliquée : épanchement pleural, abcès, obstacle
  \item Observance, pharmacocinétique, hors spectre
  \item Diagnostic :
    \begin{itemize}
      \item focalisé : embolie pulmonaire
      \item diffuse : cf pneumopathies interstitielles diffuses aigües
      \item excavée : cancer, tuberculose pulmonaires, infarctus pulmonaire, vascularite\ldots{}
    \end{itemize}
\end{itemize}

\subsubsection{Prévention}
Vaccin : 2 doses + rappel enfant, 2 doses adulte (ID, comorbidité)

\subsubsection{Immunodéprimé}
Pneumocystose pulmonaire
Radio : sd interstitiel diffus bilatéral symétrique. ATB cotrimoxazole 21j


\section{151 : Tuberculose}
\label{sec:org4a79e2f}
\bact{tuberculose}
\subsection{Tuberculose maladie}
\label{sec:orgbb2ab4f}
\subsubsection{Pulmonaire}
\label{sec:orgf99718d}
Classique : 
\begin{itemize}
\item \{asthénie, anorexie, amaigrissement\} + fébricule (nocturne) + sueurs nocturnes
\item Radio : nodules, infiltrats, cavernes (lobes sup, post.)
\end{itemize}
Miliaire (hématogène) :
\begin{itemize}
\item AEG marqué (fébrile)
\item Radio : interstitiel diffus micronodulaire
\end{itemize}
Autres : pleurésie, pneumonie aigüe, extra
\subsubsection{Extra-pulmonaires}
\label{sec:orgd02575b}
Ganglionnaire, ostéo-articulaire, génito-urinaire, neuroméningé
\subsection{Diagnostic}
\label{sec:orge4cd17f}
Mises en évidence du bacille :
\begin{itemize}
\item 3 prélèvement (ECBC/tubage)
\item direct, culture, identification et ATBgramme
\item Granulomes épithélioïdes gigantocellulaires avec nécrose caséeuses, coloration Ziehl Neelsen
\end{itemize}
\subsection{Traitement}
\label{sec:org17cd56f}
Quadrithérapie (lutter MDR ou XDR) : 1/jour 1h avant petit déjeuner
\begin{itemize}
\item Isoniazide (polynévrite SM\footnote{sensitivomotrice}, surv. transaminases) : 6 mois
\item Rifampicine (inducteur enzymatique\footnote{contraception !}) : 6 mois
\item Ethambutol (névrite optique rétrobulbaire) : 2 premiers mois
\item Pyrazinamide (surv. transaminases)
\end{itemize}
Autres : aminosides, fluoroquinolones \\
Avant traitement : 
\begin{itemize}
\item hémogramme, \{transaminases, bilirubine, phosphatases alcalines, \$\(\gamma\)\$GT\}, créat,natrémie,uricémie.
\item proposer VIH, VHB, VHC
\item visions des couleurs
\end{itemize}

\subsection{ITL}
\label{sec:org8f8c4fe}
Diagnostic (\danger pas pour tuberculose active): 
\begin{itemize}
\item IDR (non spécifique, faux négatifs !) : +72h : \diameter > 10mm (15 si vaccination) ou +10mm en 3 mois : suspicion ITL
\item IGRA\footnote{Interferon Gamma Release Assay} : > 15 ans ou migrant  < 15 ans
\end{itemize}
Traitement : 
\begin{itemize}
\item primaire si < 2 ans, secondaire si < 18 ans ou ID ou ITL < 1 an ou séquellaire
\item INH 9 mois ou INF + RMP 3 mois
\end{itemize}
\subsection{Vaccination}
\label{sec:org8933781}
Sauf VIH, déficits immunitaires
\subsection{Cas particuliers}
\label{sec:orgbcd4ffe}
Grossesse : INH (+vit. B6) + RMP (+vit. K1) + EMB\\
Insuf. rénale : [15,30]Ml : diminuer EMB, PZA. si < 15Ml/min : diminuer INH, EMB, PZA\\
Insuf. hépatique : [3,6N] : arrêt PZA. Si > 6N, arrêt INH, PZA\\
ID, VIH : penser tuberculose si fièvre\\
Si anti-TNF, bilan tuberculose
\subsection{DO}
\label{sec:org19d663a}

\appendix
\section{Physiologie}%

$P_{alv}$ = pression alvéolaire, $P_{ip}$ = pression interpleurale (dans la
plèvre), pression transpulmonaire = $P_{ip} - P_{alv}$.

\begin{figure}[htpb]
  \centering
  \caption{Inspiration (gauche), expiration (droite)}
  \tikz \graph [ nodes={align=center}, layered layout]
  {
    Contraction diaphragme -> Expansion thorax -> "$P_{ip} < P_{atm}$"
    -> Hausse pression transpulmonaire -> Expansion poumons 
    -> "$P_{alv} < P_{atm}$"
    -> Arrivée d'air dans les alvéoles;
  };
  \tikz \graph [ nodes={align=center}, layered layout]
  {
    "Arrêt contraction\\ diaphragme et intercostaux" -> Rétraction thorax 
    -> "Valeur initiale de $P_{ip}$"
    -> Valeur initiale pression transpulmonaire 
    -> Rétraction poumons 
    -> "$P_{alv} > P_{atm}$"
    -> Expulsion d'air depuis les alvéoles;
  };
\end{figure}
\end{document}
