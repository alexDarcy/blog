\documentclass{article}
\usepackage[hidelinks]{hyperref}
\usepackage{longtable}
\usepackage{booktabs}
\usepackage{graphicx}
\graphicspath{{../../pictures/medecine/}}
% French
\usepackage[T1]{fontenc}
\usepackage[francais]{babel}
%-------------------------------------------------------------------------------
% For graphs
\usepackage{tikz}
\usetikzlibrary{graphs}
\usetikzlibrary{graphdrawing}
\usetikzlibrary{arrows,positioning,decorations.pathreplacing}
\usegdlibrary{trees, layered, force}
\usetikzlibrary{quotes}
%-------------------------------------------------------------------------------
% No spacing in itemize
\usepackage{enumitem}
\setlist{nolistsep}
% tightlist from pandoc
\providecommand{\tightlist}{%
  \setlength{\itemsep}{0pt}\setlength{\parskip}{0pt}}
 % Danger symbol (need fourier package)
\newcommand*{\TakeFourierOrnament}[1]{{%
\fontencoding{U}\fontfamily{futs}\selectfont\char#1}}
\newcommand*{\danger}{\TakeFourierOrnament{66}}
% Skull (need Symbola font)
\usepackage{amsmath,fontspec,newunicodechar}
\newfontface{\skullfont}{Symbola}[Scale=MatchUppercase]
\NewDocumentCommand{\skull}{}{%
  \text{\skullfont\symbol{"1F571}}%
}
% Hospital sign
\usepackage{fontspec} % For fontawesome
\usepackage{fontawesome}

\title{Pneumologie\\
\large Fiches}
\author{A. Praga}

\begin{document}
\maketitle

\section{Physiologie}%

$P_{alv}$ = pression alvéolaire, $P_{ip}$ = pression interpleurale (dans la
plèvre), pression transpulmonaire = $P_{ip} - P_{alv}$.

\begin{figure}[htpb]
  \centering
  \caption{Inspiration (gauche), expiration (droite)}
  \tikz \graph [ nodes={align=center}, layered layout]
  {
    Contraction diaphragme -> Expansion thorax -> "$P_{ip} < P_{atm}$"
    -> Hausse pression transpulmonaire -> Expansion poumons 
    -> "$P_{alv} < P_{atm}$"
    -> Arrivée d'air dans les alvéoles;
  };
  \tikz \graph [ nodes={align=center}, layered layout]
  {
    "Arrêt contraction\\ diaphragme et intercostaux" -> Rétraction thorax 
    -> "Valeur initiale de $P_{ip}$"
    -> Valeur initiale pression transpulmonaire 
    -> Rétraction poumons 
    -> "$P_{alv} > P_{atm}$"
    -> Expulsion d'air depuis les alvéoles;
  };
\end{figure}
\end{document}
