%\documentclass{article}
%\usepackage{hyperref}
%\input header
%\usepackage{biocon}

\newacronym{ADP}{ADP}{Adénopathies}
\newacronym{BD}{BD}{Bronchodilatateur}
\newacronym{DIP}{DIP}{Pneumopathie Interstitielle Desquamante}
\newacronym{EP}{EP}{Embolie pulmonaire}
\newacronym{FPI}{FPI}{Fibrose Pulmonaire Idiopathique}
\newacronym{HMG}{HMG}{Hépatomégalie}
\newacronym{ITK}{ITK}{Inhibiteur de tyrosine kinase}
\newacronym{LBA}{LBA}{Lavage Broncho-Alvéolaire}
\newacronym{MTEV}{MTEV}{Maladie Thrombo-Embolique Veineuse}
\newacronym{OMI}{OMI}{Oedème des membres inférieurs}
\newacronym{PAPm}{PAPm}{Pression de l'artère pulmonaire moyenne}
\newacronym{PINS}{PINS}{Pneumonpathie Interstitielle Non Spécifique}
\newacronym{POC}{POC}{Pneuompathie organisée cryptogénique}
\newacronym{TVO}{TVO}{Troubles Ventilatoires Obstructifs}
\newacronym{TVP}{TVP}{Thrombose veineuse profonde}
\newacronym{TVR}{TVR}{Troubles Ventilatoires Restrictif}
\newacronym{VAS}{VAS}{Voies Aériennes Supérieures}
\newacronym{FIVA}{FIVA}{Fonds d'indemnisation des victimes de l'amiante}

\newglossaryentry{VEMS}
{
  name={VEMS}, 
  description={volume expiratoire maximal en 1s (après inspiration maximale)}
}
\newglossaryentry{CV}
{ name = Capacité Vitale,
  description = volume total mobilisable maximal = VC + VRI + VRE
}
\newglossaryentry{VC}
{ name=Volume courant,
  description={volume mobilisé pendant une respiration normale}
}
\newglossaryentry{VRI}
{ name = Volume de réserve inspiratoire,
  description = volume supplémentaire (par rapport au VC) avec
    une inspiration forcé
}
\newglossaryentry{VRE}
{ name = Volume de réserve expiratoire,
  description = idem VRI mais en expiration forcée
}
\newglossaryentry{VR}
{ name = Volume résiduel,
  description = volume restant (impossible à expirer)
}
\newglossaryentry{CVF}
{ name = Capacité Vitale Forcée,
  description = volume expulsé avec force (CPT - VR)
}
\newglossaryentry{CVL}
{ name = Capacité Vitale Lente,
  description = idem CVF mais lentement
}  
\newglossaryentry{CPT}
{ name = {Capacité Pulmonaire Totale},
  description = {Capacité Vitale + volume résiduel}
}
\newglossaryentry{PAPO}{
  name = PAPO,
  description = Pression artérielle pulmonaire occluse $\approx$ pression
  capillaire pulmonaire
}

\glsaddall % Add all entries
\glsunsetall % Acronyms always in short form


\section{73 - Addiction au tabac}
3 fumées : courant primaire (inspiré), secondaire (tabagisme passif++), tertiaire (exhalé)\\
Produits :
\begin{itemize}
\item nicotine = dépendance
\item fumée de tabac = 0.3 $\mu$m, pénètre partout !
\item goudrons = cancérigène
\item CO = hypoxie, risque ischémie, marqueur tabagisme des dernières heures
\end{itemize}
16 millions de fumeurs en 2013 (France). Éducation et cat. sociale faible = plus
tabagiques
\subsection{Pathologies liées au tabac}
K :
\begin{itemize}
\item 90\% des cancers broncho-pulmonaires (tabagisme actif), 25\% (passif) 
\item voies aérodigestives sup, vessie, pancréas, rein, col de l'utérus
\end{itemize}
Respiratoires : BPCO, asthme\\
Cardio-vasc : 
\begin{itemize}
\item cardiopathies ischémiques
\item artérite
\item HTA
\item cérébro-vasculaires
\item relation tabac-athérome, maladie coronaire
\end{itemize}
Autres : digestives, kératites, retard consolidation os, agueusie, anosmie\\
Passif :
\begin{itemize}
\item +25\% risque cancer bronchique
\item +25\% maladies CV
\item aggrave asthme, BPCO
\item nourrisson : RCIU\footnote{Retard croissance intra-utérin}, \(\nearrow\) risque
  infections respi, mort-subite (1ere cause identifiée)
\end{itemize}
\subsection{Prise en charge}
Évaluer : consommation (paquets/années), dépendance, autres (alcool = déclencheur,
cannabis), motivation, comorbidités (psy, CV, respi)
\subsection{Traitement}
Conseil, motivationnel, traitement (substitut nicotinique, varénicline [\danger
suivi], bupropion), thérapies cognitivo-comportementales, cigarette
électronique\\
Sevrage : réussi si \(\ge 1\) an


\section{108 - Troubles du sommeil}
\subsection{Définitions}
\danger Sd d'apnées hypopnées obstructives du sommeil (SAOS) = profil "en peigne"\\
Sd d'apnées centrales = désat profondes et soutenus

Apnée : 
\begin{itemize}
\item obstructive : arrêt débit \(\ge\) 10s + efforts ventilatoires
\item centrale : idem, sans efforts ventilatoires
\item mixte = obstructive et centrale
\end{itemize}
Hypopnée = (diminution ventilation \(\ge\) 10 s) et ((débit -50\%) ou (désat
transcut \(\ge\) 3\%) \textpm{} microéveil)

SAOS = (somnolence diurne non expliquée ou (2 parmi : ronflements, étouffement,
éveils répétés, sommeil non réparateur, fatigue diurne, trble concentration,
nycturie)) \(\wedge{}\)  IAH\footnote{Index d'Apnées et Hypopnées (nb par heure)} \(\ge\) 5\\
Sévère si IAH \(\ge\) 30.
\subsection{Épidémiologie}
2 millions en France.
FR : obésité, homme, âge, anomalie voie aériennes supérieures. 
Comorbidités :
\begin{itemize}
\item neuropsy + accidents (risque x2)
\item CV : HTA, insuf. coronaire, cardiaque, ACFA
\item sd métabolique\footnote{Obésité abdo et (2 parmi : HTA, glycémie \(\ge\) 5.6 mmol/L,
    HDL bas, hypertriglycéridémie)}
\item diabète
\item dyslipidémie
\end{itemize}

NB : obstruction \gls{VAS} dûe à {réduction \diameter, \nearrow{} collapsibilité,
  \searrow{} efficacité muscles dilatateurs}
\subsection{Diagnostic}
\subsubsection{Clinique :}
\begin{itemize}
\item nocturne = ronflements, pause respi, étouffement, nycturie
\item diurne = somnolence excessive (questionnaire d'Epworth)
\end{itemize}
DD : hypersomnie diurne (insomnie, sd dépressif, sédatif, hygiène de sommeil,
neuro)
\subsubsection{Examen}
\begin{itemize}
\item IMC, obésité abdo (\(\ge\) 94 cm \male, 80cm \female)
\item ORL
\item CV, respi
\end{itemize}
\subsubsection{Complémentaire}
(EFR et imagerie optionnels)
Saturation transcut en O\(_{\text{2}}\) possible.\\
Référence = enregistrement polygraphique ventilatoire (ou polysomnographique
mais cher++)
\subsection{Traitement}
Général : PEC\footnote{prise en charge} surpoids, éviction \{benzodiazépines,
myorelaxants, morphiniques\}, PEC CV

Spécifique :
\begin{itemize}
\item Pression positive continue = 1ere intention
\item Orthèse d'avancée mandibulaire = 2eme
\item Autres : orthèse anti-décubitus dorsal, chirurgie = \{vélo-amygdalienne,
  avancée maxillo-mandibulaire (lourd++), nasale\}
\end{itemize}
\subsection{Autres}
\begin{itemize}
\item Sd d'apnée de type central : cause = {insuf. cardiaque sévère (traitement !),
    atteinte tronc cérébral, séjour altitude, morphiniques}
\item Hypoventilation alvéolaire : aggravation de l'état de veille. Traitement =
  VNI\footnote{ventilation non invasive}
\item Sd obésité hypoventilation : (PaCO\(_{\text{2}}\) > 45 mmHg) et (PaO\(_{\text{2}}\) < 70 mmFg) et
  (IMC > 30 kg/\(m^2\)), pas d'autre cause. Traitement = VNI
\end{itemize}

\section{151 - Infections broncho-pulmonaires communautaires}
\subsection{Bronchite aigue}
Très fréquente, virale 90\%. \\
Diagnostic clinique : épidémie, toux sèche \(\to\) productive, expectorations,
pas de crépitants.\\
Ttt symptomatique seulement !

\subsection{EBPCO} Cf~\nameref{subsec:ebpco}

\subsection{Pneumonie aigue communautaire}
Clinique : \{toux, expect purulentes, dyspnée\} + \{fièvre, asthénie\} +
crépitants\\
Radio !\\
Si \faHospitalO : hémocultures, ECBC, antigenurie pneumocoque (\pm PCR,
antigenurie légionelle)\\
\faHospitalO : signes de gravité (score CRB65\footnote{Confusion, fréquence
  Respiratoire \ge 30min, (Blood) PAs < 90mmHg ou PAd \ge 60mmg, Age \ge 65 ans}
\(\ge 1\)) / incertitude / échec domicile / comorbidité / inobservance

\begin{table}[htpb]
  \centering
  \label{Orientation clinique (non discriminant !)}
  \begin{tabular}{llll}
    \toprule
    & Pneumocoque & Atypique & Légionellose\\
    \midrule
    Début & Brutal & Progressif & Rapidement progressif\\
    Signes & Thoraciques & Extra-thoraciques & Myalgie, digestif\\
    Biologie &  & Anémie hémolytique & Hyponatrémie, rhabdomyolyse\\
    Microbio & CG+ chainettes/Ag pneumocoque + &  & Ag légionelle +\\
    RX thorax & condensation systématisée & Opacités multifocales & CS ou OM\\
    \bottomrule
  \end{tabular}
\end{table}

\begin{itemize}
\item Pneumocoque : fréquent+++. Pas de transmission interhumaine
\item Atypique : \bact{mpneumoniae}, \bact{cpneumoniae}, \bact{psitacci}
\item Legionnella : pas d'isolement. DO
\item Pneumonie virale : signes respi + sd grippal. Grippale = diagnostic PCR, traitement = inihbiteur neuramidase \danger \bact{dore}
\end{itemize}
\subsubsection{Traitement}
Oral, 7j (8-14 si légionellose, 21j si légionelles graves/ID)

Urgent, probabiliste, réévalué 48-72h

Ambulatoire :Amoxicilline ou macrolide \(\to\) switch

\faHospitalO  Amoxicilline \(\to\) réévaluation

Réa : C3G IV + macrolide IV / FQ pour pneumocoque

\subsubsection{Échecs}
\begin{itemize}
\item Compliquée : épanchement pleural, abcès, obstacle
\item Observance, pharmacocinétique, hors spectre
\item Diagnostic :
  \begin{itemize}
  \item focalisé : embolie pulmonaire
  \item diffuse : cf pneumopathies interstitielles diffuses aigües
  \item excavée : cancer, tuberculose pulmonaires, infarctus pulmonaire, vascularite\ldots{}
  \end{itemize}
\end{itemize}

\subsubsection{Prévention}
Vaccin : 2 doses + rappel enfant, 2 doses adulte (ID, comorbidité)

\subsubsection{Immunodéprimé}
Pneumocystose pulmonaire : 
\begin{itemize}
\item RX : sd interstitiel diffus bilatéral symétrique.
\item ATB cotrimoxazole 21j
\end{itemize}


\section{180 - Accidents du travail}
Maladie professionnelle = lente, prolongée (\(\neq\) accident du travail)
\subsection{Maladies}
\subsubsection{Asthmes}
10-15\% des asthmes. On a
\begin{itemize}
\item asthme professionnel : (avec latence [immunologique]) ou (sans latence
  [exposition unique])
\item asthme aggravé par le travail
\end{itemize}
Métiers : boulangers, santé, coiffeurs, peintres pistolets, bois, nettoyage\\
Diagnostic : de novo, profession, rythme
\subsubsection{BPCO}
\gls{TVO} lente progressive + inflammation
poumons. Facteurs professionnels = 10-20\% BPCO\\
Métiers : mines, BTP, fonderie, textile, agricole
\subsubsection{Cancers}
\begin{itemize}
\item mésothéliome : 2/100 000habi/an, amiante++, DO
\item K bronchique primitif
\end{itemize}
\subsubsection{Pneumopathie interstitielle diffuse (PID)}
\begin{itemize}
\item Pneumopathie d'hypersensibilité : agricole
\item Silicose (+cancer bronchique primitif)
\item Bérylliose : adénopathies médiastinales, sd infiltrant parenchymateux
\item Sidérose : fumées d'oxyde de fer, micronodule \textpm{} emphysème
\item Asbestose : fibrose
\end{itemize}
\subsubsection{Liées à l'amiante}
\begin{itemize}
\item Cancer : mésothéliome, cancer bronchique primitif
\item Pleural (plaques, épaississements, pleurésies), parenchyme (fibrose)
\end{itemize}
\subsection{Reconnaissance}
3 conditions : médicale, administrative, professionnelle. 

Déclaration à la CPAM (employeur si AT, sinon victime)

Indemnité\footnote{Pour l'amiante : caisse d'assurance malaide ou FIVA}
\begin{itemize}
\item 60\% salaire si < 28 jours, 80\% sinon
\item si séquelles, selon taux incapacité permanente : capital (< 10\%), rente
  (> 10\%), rentre et autres (> 40\%)
\end{itemize}

\section{182 - Hypersensibilités et allergies respiratoires}%
\label{sec:182_hypersensibilites_et_allergies_respiratoires}

\begin{figure}[htpb]
  \centering
  \resizebox{0.5\linewidth}{!}{
    \tikz \graph [
    % Labels at the middle 
    edge quotes mid,
    % Needed for multi-lines
    nodes={align=center},
    sibling distance=3cm,
    edges={nodes={fill=white}}, 
    layered layout]
    {
      HS non allergique;
      HS allergique -> {
        Non IgE;
        IgE -> {
          atopique -> "insectes\\helminthes\\médicaments";
          non atopique -> "rhinite\\asthme\\alimentaire\\professionelles";
        };
      };
    };
  }
  \caption{Hypersensibilité}
\end{figure}
\subsection{Définitions}%
Atopie : prédisposition héréditaires à IgE face à des allergènes.\\
Allergie : réaction d'HS par mécanismes immunologiques\\
Sensibilisation : test cutané positif à un allergène\\
Allergène : capable d'induire une réaction d'HS = pneumallergènes (aéroportés),
trophallergènes (alimentaires), professionnels, recombinants

Hypersensibilité :
\begin{itemize}
\item type 1 (immédiate) : la plus fréquente, IgE. Rhinite, asthme
\item type 2 : complément et phagocytose. Réactions médicament.
\item type 3 : complexes immuns. Pneumonies d'HS
\item type 4 : LT, cytotoxiques 48-72h. Granulome épithélioïde gigantocellulaire.
  Dermato.
\end{itemize}

\paragraph{Asthme et rhinite allergique}
Polygénique.\\
Environnement : infections virales, sensibilitations pneumallergènes, tabac dès
conception, pollution air intérieur. 

\paragraph{Anomalies voies aériennes (asthme)}
Remodelage bronchique : épaississement (membrane basale et muscle lisse $\wedge$ oedème
bronchique) et obstruction $\diameter$ (mucus)

\paragraph{Réaction IgE}
\begin{itemize}
\item Sensibilisation : synthèse IgE spécifiques (lymphocytes B)
\item effectrice : fixation de l'allergène $\implies$ activation (histamine,
  cytokines) $\implies$ cascade allergique
\end{itemize}

\subsection{Épidémiologie}
Atopie = 30-40\% population. HS médicament = 7\% pop. Allergie alimentaire : 2\%
des 9-11 ans

FR : 
\begin{itemize}
\item génétique (enfants à risque)
\item environnement : fréq $\nearrow$ temps, alimentaires, tabagisme passif
  maternel, allergènes, pollution atmosphérique \danger niveau de preuve 
\end{itemize}
Morbidité forte, mortalité encore forte pour l'asthme.

\subsection{Diagnostic}
Clinique : asthme, rhinite, conjonctivite

\paragraph{Diagnostic}
Unité de temps, lieu et action (perannuels [acariens,
blattes, phanères d'animaux, végétaux d'intérieur, moisissures] ou saisonniers
[pollens]) $\wedge$ IgE spécifiques

Prick-test = référence.
\begin{itemize}
\item $\diameter \ge 3$mm / témoin
\item acariens, pollens, phanères d'animaux, blatte, moisissures chez l'adulte
\item arachide, blanc d'oeuf, poisson, lait de vache si < 3 ans
\item CI : antihistaminiques, $\beta$-bloquants, eczéma, grossesse si
  allergie médic
\end{itemize}

Autres tests : dosage IgE spécifique (moins sensible), multiallergénique (sensible mais
pas quantitatif)

\paragraph{Professionnelles} boulangers, santé, coiffeurs, peintres pistolets, bois, nettoyage\\

\paragraph{Autres}
\begin{itemize}
\item Test de provocation = certitude mais dangereux
\item Très sécifique : dosage IgE totale, dosage éosinophiles sanguin, dosage tryptage sérique
\end{itemize}

\subsection{Traitement}
\paragraph{Éviction allergènes} Toujours.

\paragraph{Symptomatiques}
\begin{itemize}
\item antihistaminiques : rhinite, conjonctivite, prurit
\item corticoïdes : systématique si urgence (prednisone, prednisolone), local
  si traitement fond
\end{itemize}

\paragraph{Immunthérapie spécifique}
Faibles doses croissantes d'allergènes :
\begin{itemize}
\item Sous-cutanées/sub-linguale : acariens, pollens, hyménoptères
\item orale : pollen
\end{itemize}
CI : maladies allergiques, dysimmunités, grosses (induction), asthme sévère non
contrôlé, mastocytoses, $\beta$-bloquants

ES : 
\begin{itemize}
\item syndromique (asthme, rhinite, urticaire) = alerte
\item générale (hypotensions, bronchospasme, choc anaphylactique) =
  interruption
\end{itemize}


\section{184 - Asthme, rhinite}
Asthme 6\%, mortalité 1000 DC/an, en baisse. Morbidité en hausse

Rhinite allergique 24\%.
\subsection{Définitions}
Asthme : 
\begin{itemize}
\item inflammation chronique modifiant les VAS avec symptômes respi et obstruction voies aériennes réversible
\item interaction gènes-environement, déclencheurs : exercice, HS aspirine/AINS,
  irritants inhalés
\end{itemize}

TVO : 
\begin{itemize}
\item \gls{VEMS}/CVF < 0.7
\item réversible si +200mL et +12\% après BDCA\footnote{Broncho-dilatateurs à courte durée
    d'action}
\end{itemize}

Hyperréactivité bronchique : -20\%VEMS après métacholine/air sec (VPP = 100\%)

Débit expiratoire de pointe (pour urgence)

\subsection{Diagnostic}
\paragraph{Asymptomatique}
Diagnostic :
\begin{itemize}
\item symptômes caractéristiques (20 min, réversible, variable)
  \begin{itemize}
    \item plusieurs parmi \{gêne respiratoire, dyspnée,
        sifflements, oppression thoraciques\}
    \item $\nearrow$ nuit/réveil, variable,réversibles, déclenché par \{rire, exercice
      virus, allergènes, irritants\}
    \end{itemize}
\item \textbf{et} obstruction bronchique : sibilant, TVO réversible ou apparaît avec métacholine
\end{itemize}
Sévérité évaluée à 6 mois
\paragraph{Exacerbation}
\nearrow{} symptômes > 2 jours, non calmée, sans retour état habituel

Signes de sévérité :
\begin{itemize}
\item mots (au lieu de phrases), assis en avant, agité
\item FR > 30/min
\item muscles respi annexes
\item FC > 120/min, SpO\(_{\text{2}}\) < 90\%
\item DEP < 50\%
\item silence auscultatoire \skull
\item respi paradoxale \skull
\item troubles conscience, bradycardie, collapsus \skull
\end{itemize}
\paragraph{DD}
Sans TVO : cordes vocales, sd hyperventilation

TVO non réversible : \{BPCO, bronchectasies, mucoviscidose, bronchiolites
constrictives\}, autres (corps étranger, tumeur, insuf. cardiaque)
\paragraph{Bilan}
Facteurs favorisants, radio thorax, EFR (+test métacholine)

\subsection{Traitement}

\paragraph{Long cours}
Ttt de fond : cf Table~\ref{tab:ttt_asthme}

\begin{table}
  \centering
  \begin{tabular}{ccccc}
    \toprule
    Palier 1 & Palier 2 & Palier 3 & Palier 4 & Palier 5 \\
    \midrule
    CSI faible & ALT & CSI moyen/fort & triotropium & CSO faible\\
             & & ou (CSI faible et ALT) & ou (CSI fort et ALT) &\\
  \bottomrule                                                              
  \end{tabular}
  \caption{Ttt de fond de l'asthme (\textbf{BDCA à la demande}).\\
    CSI = Corticostéroïdes inhalés. ALT =
    anti-leucotriènes. CSO = Corticostéroïdes oraux}
  \label{tab:ttt_asthme}
\end{table}

Autre : 
\begin{itemize}
\item activité physique (sauf plongée)
\item facteurs favorisants : rhinite, allergie, tabac, \(\beta\)-bloquant
  (aspirine/AINS), RGO, comorbidités
\item vaccins grippe (pneumocoque si asthme sévère)
\end{itemize}
Efficace ?
\begin{itemize}
\item symptômes contrôlés (diurnes < 2/sem, pas de réveil nocture, BDCA < 2/sem, pas
  limitation d'activité)
\item exacerbations < 2 corticostéroïdes systém./an
\item VEMS/CV > 0.7 et VEMS \(\ge\) 80 \%
\end{itemize}
Si non contrôlé
\begin{itemize}
\item CSI faible + BDLA\footnote{Broncho-dilatateur à longue durée d'action} < CSI
  moyen/fort + BDLA < centre spécialisé
\end{itemize}
Suivi : périodique, +3mois si changement ttt, mensuel si grossesse

\paragraph{Urgence} cf figure~\ref{fig:asthme_urgence}.

\begin{figure}[htpb]
  \centering
  \tikz \graph [
  % Labels at the middle 
  edge quotes mid,
  % Needed for multi-lines
  nodes={align=center},
  sibling distance=3cm,
  edges={nodes={fill=white}}, 
  layered layout]
  {
    "Évaluation" -> {
      "Exacerbation\\
      sans signes de gravite" -> 
      "\(\beta_2\) mimétique forte doses\\
      chambre inhalation\\
      \textbf{Corticothérapie orale}  7j" [draw]
      -> Réévaluation 1H;
      "Exacerbation \\
      avec signes de gravite" -> 
      "\(\beta_2\)-mimétique forte dose" [level distance=2cm] -> {
        "O\(_2\)\\
        \(\beta_2\) mimétique nébulisés (5mg/20min)\\
        $\pm$ ipratropium\\
        \textbf{Corticothérapie orale} " [>"\faHospitalO", draw];
      };
      Perte de contrôle ->
      "\(\beta_2\) mimétique" [draw] -> Réévaluation 1H;
    };
  };
  \caption{Asthme : traitement d'urgence}
\end{figure}

Pas d'ATB sauf si suspicion bactérienne. Adrénaline seulement pour choc anaphylactique
\danger pas de BDLA 

\subsection{Rhinite allergique}
Diagnostic :
\begin{itemize}
\item PAREO : prurit, anosmie, rhinorrhée, éternuement, obstruction nasale
\item fosses nasales au speculum inflammées
\item allergique (argumenter !!)
\end{itemize}
Sévère si persistant (> 4 semaines/an) et retentissement qualité de vie

TDM

Ttt : laval nasal, allergie, antihistaminique/corticoïdes nasaux, tabac, stress


\section{188, 189 : Pathologies auto-immunes}%
\label{sec:pathologies_auto_immunes}

Manifestations respi. viennent de (par priorité décroissante) :
\begin{itemize}
\item infectieuse (favorisé par ttt)
\item toxicité médicamenteuse
\item spécicifique
\item indépendant
\end{itemize}

\subsection{Complications infectieuses}
Immunodépression :
\begin{itemize}
\item corticoïdes forte dose > 1 mois
\item méthotrexate
\item cyclophosphamide
\item anti-TNF$\alpha$ (infliximab)
\end{itemize}
Tuberculose : 
\begin{itemize}
\item clinique semblable à l'ID = 50\% extrapulmonaire, 25\% disséminé
\item prévention si TNF-$\alpha$ : INH 9 mois ou INH-RMP 3 mois
\end{itemize}
Pneumocystose : si corticoïdes forte dose ou méthotrexate ou
cyclophosphamide
\begin{itemize}
\item clinique : début brutal, insuf. respi, mortalité élevée
\item radio thorax : condensation alvéolaire/verre dépoli bilat
\item penser co-infections
\end{itemize}
\subsection{Médicaments}
Méthotrexate : plus fréquent, pneumopathie d'HS (opacités diffuses), \gls{LBA}
lymphocytaire. Évolution favorable à l'arrêt + corticothérapie

Inhibiteurs TNF-$\alpha$ : PID/granulomatoses, anaphylactique

\subsection{Connectivites}
Polyarthrite rhumatoide
\begin{itemize}
\item PID\footnote{Pneumopathie interstitielle diffuse} : radio =
  réticulations, rayons de miels, bronchectasies. Surveillance seulement
\item pleurésie rhumatoïde : unilatérale, peu abondante, exsudative. Évolution
  favorable
\item nodules pulmonaires rhumatoïdes. 
\item bronchiolite oblitérante
\end{itemize}
Sclérodermie systémique : CREST (Calcinose, Raynaud, dyskinésie oEsophagienne,
Sclérodactyie, Télangiectasie), Ac anti-nucléraires
\begin{itemize}
\item PID : semblable à PINS\footnote{Pneumopathie interstitielle non
    spécifique} : opacités en verre dépoli, bronchectasies par traction. Survie
  85\% à 5 ans
\item HTA pulmonaire : dyspnée. Echo. cardiaque +
  cathéterisme cardiaque
\end{itemize}
Lupus érythémateux disséminé 
\begin{itemize}
\item pleurésie lupique : peu abondant, svt bilatéral, svt + péricardite
\item infectieux, sd hémorragie alvéolaire
\end{itemize}
Dermato-, polymyosite
\begin{itemize}
\item PID chronique : 1ere cause DC, opacités verre dépoli, \gls{LBA} lymphocytaire.
  Ttt : corticothérapie + IS
\item PID (sub)aigüe
\end{itemize}
Sd de Gougerot-Sjögren
\begin{itemize}
\item bronchite lymphocytaire chronique : toux sèche chronique
\item PID, lymphome pulmonaire primitif
\end{itemize}

\subsection{Vascularites}
Granulomatose avec polyangéite : 40-50ans, début ORL+ poumon (+ rein)
\begin{itemize}
\item radio : nodules s'excavant, opacités verre dépoli
\item Ttt urgent = corticothérapie, cyclophosphamide
\end{itemize}
Granumolatose éosinophilique avec polyangéite :  hyperéosinophilie, pneumopathie
éosino.

Polyangéite microscopique : sd hémorragique alvéolaire

\section{199 : Dyspnée aigüe et chroniques}
\label{sec:199_dyspnee_aigue_et_chronique}
Examens : ECG, RX thorax, gaz du sang, D-dimère, BNP, NFS a minima

\subsection{Aigüe}
= quelques heures/jours

Détresses respi aigüe :
\begin{itemize}
\item cyanose
\item sueur
\item FR > 30/min ou < 10/min
\item tirage, muscle respi accessoires
\item respi abdo paradoxale
\end{itemize}
Hémodynamique :
\begin{itemize}
\item FC > 110/min
\item choc (marbrures, oligurie, angoisse, extr. froides)
\item PAS < 80 mmHg
\item insuf. ventriculaire droite (turgescence jug, OMI, signe Harzer)
\end{itemize}
Urgence !!

Voir table~\ref{tab:dyspnee_aigue}. Autres :
\begin{itemize}
\item cardiaque : tamponnade\footnote{orthopnée, tachy, assourdissement bruits,
  ascult pulmonaire normale, turgescence jugulaire, pouls paradoxal}, troubles
(supra)-ventriculaire, choc cardiogénique
\item pulmonaire : SDRA, décompensation aigüe, atélectasies, trauma
\end{itemize}

\begin{table}[htbp]
\caption{Étiologies de dyspnée aigüe}
\label{tab:dyspnee_aigue}
\centering
\begin{adjustbox}{max width=\textwidth}
\begin{tabular}{lll}
\toprule
Inspiratoire & Expiratoire & Sinon\\
\midrule
corps étranger (enfant) & asthme (jeune, allergie, sibilants) & EP
                                                                (ascul. normale,
                                                                douleur
                                                                thoracique, phlébite)\\
épiglottite (enfant) & BPCO (tabac, bronchite aigüe, sibilants) & pneumothorax,
                                                                  épanchement
                                                                  pleural (sd
                                                                  pleural,
                                                                  douleur thoracique)\\
laryngite (enfant) & OAP (âgée, crépitants, expector mousseuse) & pneumopathie
                                                                  infectieuse
                                                                  (sd
                                                                  infectieux,
                                                                  douleur thoracique)\\
\oe{}dème de Quincke (terrain)&  & OAP (âgé, orthopnée, crépitants, expector mousseuse)\\
 &  & \\
\bottomrule
\end{tabular}
\end{adjustbox}
\end{table}

\subsection{Chronique}
cf table~\ref{tab:dyspnee_chronique}.
Autres :
\begin{itemize}
\item Cardiaque:constriction péricardique
\item Pulmonaire : (restrictif) pneumoconioses, post-tuberculose, paralysie phrénique, cyphoscoliose, obésité morbide
\item HTAP
\item HT pulmonaire post-embolique
\end{itemize}

\begin{table}[htbp]
\caption{Étiologies de dyspnée chroniques}
\label{tab:dyspnee_chronique}
\centering
\begin{adjustbox}{max width=\textwidth}
\begin{tabular}{lll}
\toprule
Sibilants & Crépitants & Auscult normale\\
\midrule
BPCO& PID (toux sèche, maladie systémique) & EP/maladie vasculaire pulmonaire \\
asthme& Insuf cardiaque gauche & Neuromusc \tablefootnote{signe neuro, orthopnée, respi abbdo paradoxale} \\
Insuf cardiaque gauche (ATCD cardiaque, orthopnée, toux) &  & Parétiale
                                                              (obésité, scoliose)\\
          && Hyperventilation\tablefootnote{C normal, vertige, $\ne$ effort, paresthésie}\\
\bottomrule
\end{tabular}
\end{adjustbox}
\end{table}
Quantification : échelle Borg [0-10] (aigüe) ou MRC [0-4] (chronique)
\section{200 - Toux chronique}
\label{sec:200_toux_chronique}

\danger Éliminer toux post-infectieuse (< 3 semaines)

Signes de gravité : 
\begin{itemize}
\item AEG, sd infectieux
\item dyspnée d'effort, hémoptysie
\item modification toux chez fumeur
\item dysphonie, dysphagie, fausses routes
\item adénopathies cervicales suspectes
\item anomalies cardiopulmonaires
\end{itemize}

\begin{figure}[htpb]
  \centering
  \caption{PEC initiale d'une toux chronique}
  \resizebox{0.6\linewidth}{!}{
    \tikz \graph [
    % Labels at the middle 
    edge quotes mid,
    % Needed for multi-lines
    nodes={align=center},
    level distance=40pt,
    sibling distance=3cm,
    edges={nodes={fill=white}}, 
    layered layout]
    {
      Signe de gravité -> {
        Exploration ["oui"];
        "Médicaments ?" ["non"] -> {
          Test d'éviction ["oui"];
          "Coqueluche ?" ["non"] -> {
            Test diagnostique ["oui"];
            "Radio thorax anormale ?" -> {
              Bilan spécialisé ["oui"];
              "cf~\nameref{subsec:toux_orientation}" ["non"];
            };
          };
        };
      };
    };
  }
\end{figure}

\subsection{Orientation diagnostique}
\label{subsec:toux_orientation}
\textbf{ORL}  
\begin{itemize}
\item rhinosinusiens : sd rhinorrée postérieur++, obstruction nasale chronique
\item carrefour aérodigestif : diverticule de Zenker, laryngite chronique
\end{itemize}

\textbf{Respiratoire} 
\begin{itemize}
\item Asthme : TVO réversible/hyperréactivité bronchique
\item BPCO : TVO non réversible
\item Cancer bronchique, tumeurs, bronchectasise (cf
  \nameref{sub:bronchectasies})
\end{itemize}

\textbf{RGO} : pyrosis. Endoscopie digestive si FR, pHmétrie des 24h

\textbf{Allergique} 

\textbf{Systémique} : sd Gougerot-Sjögren, polychondrite atrophiante, maladie
de Horton, granulomatose avec polyangéite, rectocolite hémorr., maladie de
Crohn.

\textbf{Comportement} : dernière étiologie

\paragraph{Traitement d'épreuve} (ordre d'échec):
\begin{enumerate}
\item RGO : bromphéniramine + pseudoéphédrine 3 sem
\item asthme si TVO réversible, corticoïdes inhalés ou bronchodilatateurs inhalés si
  hyperréactivité bronchique
\item avis spé
\end{enumerate}

Traitement symptomatique : arrêt tabagisme. Éviter si possible
\begin{itemize}
\item si toux sèche : opiacés, antihistaminique anticholinergique, non
  antihistaminique non opiacés
\item si toux productive : mucomodificateurs, kiné
\end{itemize}

\subsection{Bronchectasies}
\label{sub:bronchectasies}
Types : bronchectasies (élargissement $\diameter$), bronchocèle (pus), "par
traction" (NB: pas des vraies bronchectasies)


Étiologie : 
\begin{itemize}
\item infection respi sévères : coqueluche, tuberculose (virales respi enfant,
  pneumonie bact, suppuration suite sténose)
\item mucoviscidose
\item non infectieux (poumon radique, aspergillose allergique, SDRA,
  systémique, déficit immunitaire)
\end{itemize}

Évolution : colonisation bactérienne, hémoptysie, TVO (car dilatation seulement
proximale), insuffisance respi

Clinique :
\begin{itemize}
\item toux productive quotidienne depuis l'enfance
\item hémoptysie
\item infections à \{\bact{influenzae}, \bact{pneumocoque}\} puis \{\bact{dore},
  \bact{aeruginosa}++\}
\end{itemize}

Diagnostic : TDM (certitude) = $\diameter_\text{bronche}$ > $\diameter_\text{artère}$, lumière
bronchique > 1/3 parenchyme, pas de réduction du $\diameter$, grappes de kystes,
opacités tubulées

Traitement : ATB si exacerbation, complications parenchymateuses. Macrolides
pour l'inflammation. Chir si très local + compliqué.

\section{201 - Hémoptysie}%
\label{sec:201_hemoptysie}

Urgence \skull

\begin{enumerate}
\item Est-ce une hémoptysie ? Hématémèse ou ORL possibles
\item Gravité ? Suivant abondance, terrain, persistance $\implies$ risque = hématose et asphyxie
\end{enumerate}

\subsection{Étiologies}
\begin{itemize}
\item Tumeurs bronchopulmonaires++
\item Bronchectasies++
\item Tuberculose (évolutive/séquelles)++
\item Idiopathique++
\item Infections : aspergillaires, pneumopathie infectieuses nécrosante
\item Vasc : embolie pulmonaire, HT pulmonaire, anévrysmes/malformations
\item Hémorragie alvéolaires : insuf. cardiaque gauches, médicaments/toxiques,
  vascularites, collagénose, sd Goodpasture
\end{itemize}

\subsection{Diagnostic}
Localisation (important !)

Interrogatoire : ATCD respi, cardiaque, histoire médicale

Examens 
\begin{itemize}
\item clinique : \{$SpO_2$, tension, pouls\}, mauvaise tolérance respi, gêne
  latéralisée, hippocratisme digital
\item radio thorax pour siège (verre dépoli/sd alvéolaire), lésion
\item scanner plus précis : nature, localisation, carto. vasc.
\item (endoscopie : hématémèse, multiples lésions, tumeur proximale)
\item (artériographie bronchique : ttt par embolisation)
\item autres : gaz du sang, dosage Hb, bilan coagulation, groupe sanguin, {BK,
    ECG} si suspicion OAP hémorragique
\end{itemize}

\danger BPCO $\centernot\implies$ hémoptysie

\subsection{Traitement}
$O_2$ + vasconstriction IV, protection voies aériennes (décubitus latéral,
ventilation mécanique)

Embolisation artérielle bronchique

Chir si localisé, fonction respi OK et "à froid"

\section{202 : Épanchement pleural}%
\label{sec:epanchement_pleural}

\subsection{Diagnostic}
Suspicion clinique :
\begin{itemize}
\item douleur thoracique, dyspnée, toux sèche, hyperthermie
\item sd pleural liquidien (silence auscult, matité, $\emptyset$ transmission
  corde vocales, souffle pleurétique)
\end{itemize}
Signes de gravité : détresse respi, choc septique, choc hémorragique

Confirmation imagerie
\begin{itemize}
\item radio : opacité dense, homogène, non sytématisé, limité par ligne
  concave (DD atélectasie : médiastin dévié vers l'opacité). \\
  Cas difficiles : profil, sous pulmonaire, cloisonnés
\item échographie : cloisonné, différencie pleurésie et collapus, guide ponction
\item TDM en urgence si embolie pulmonaire ou hémothorax
\end{itemize}

\subsection{Causes}
\paragraph{Transsudats (protides < 25g/L)}
\begin{itemize}
\item insuf. cardiaque G
\item cirrhose
\item sd néphrotique
\item atélectasie
\item embolie pulmonaire
\end{itemize}

\paragraph{Exsudats (protides > 35 g/L)}\mbox{}\\
\textit{Néoplasiques} 
\begin{itemize}
\item métastatique : liquide sérohématique/rosé/citrin, cytodiagnostic +
  biopsie (aveugle/vue) sauf si cancer connu
\item mésothéliome 
  \begin{itemize}
  \item exposition amiante
  \item imagerie : épaississement pleural diffus, circonférentiel
  \item liquide citrin/sérohématique
  \item biopsie sous thorascopie++
  \end{itemize}
\end{itemize}

\textit{Infectieux} 
\begin{itemize}
\item parapneumoniques (=bactérien) 
  \begin{itemize}
  \item non compliqués (liquide clair, pas de germe) : ATB
  \item compliqué : ATB + évacuation liquide
  \end{itemize}
\item virale
\item tuberculeuse : progressif, fièvre modérée, amaigrissement. Exsudat riche
  en proténie. Biopsie++ (aveugle/vue)
\end{itemize}

\textit{Autres}  :
\begin{itemize}
\item embolie pulmonaire
\item bénigne liée à l'amiante (exclusion++)
\item post trauma, rupture oesophagienne, sous-diaphragme
\item systémique : lupus, polyarthrite rhumatoide
\end{itemize}

\subsection{Ponction}
Qui ? Majorité sauf si peu abondant et insuf cardiaque G (sauf si
unilat/asymétrique ou douleur pleurale/fièvre ou traitement insuffisant)

Quand ? Urgence si fébrile, hémothorax ou mauvaise tolérance

Tout le liquide ? Si étiologique ou non cloisonné

\paragraph{Biologie}
1ère intention :
\begin{itemize}
\item biochimie
  \begin{itemize}
  \item transsudat si protide < 25g/L ou critère de Light\footnote{
(LDH > 200 UI/L) ou (protides pleuraux/sérique > 0.5) ou (LDH
pleuraux/sérique > 0.6)} faux
  \item exsudat si protide > 35g/L ou critère de Light vrais
  \end{itemize}
\item cytologie (cf tableau)
\item recherche germes pygènes, mycobactéries
\end{itemize}

2eme intention :
\begin{itemize}
\item trauma $\wedge$ hématocrite pleural/sanguin > 0.5 $\implies$ probable
  hémothorax (urgence \danger)
\item amylase pleurale si pancréatique et sous-phrénique
\item si triglycéride > 1.1g/L $\implies$ chylothorax
\end{itemize}

\begin{table}[htpb]
  \centering
  \caption{Épanchements pleuraux avec exsudats : étiologies}
  \label{tab:label}
  \begin{tabular}{lllll}
    \toprule
    & Cellules tumorales & Neutrophiles & Lymphocytes & Éosinophiles\\
    \midrule
    Néoplasique & Métastasique &  & Cancer & Cancer\\
    & Mésothéliome &  & Lymphome & \\
    & Hémopathies malignes &  &  & \\
    \midrule
    Infectieux &  & Parapneumonique & Tuberculose & Parasitose\\
    \midrule
    Autres &  & Embolie pulmonaire & Sarcoïdose & Hémothorax\\
    &  & Pancréatite & Chylothorax & Pneumothorax\\
    &  & Sous-phrénique & PR, lupus & Embolie pulmonaire\\
    &  & Oesophage &  & Asbestosique bénigne\\
    &  &  &  & Médicament\\
    \bottomrule
  \end{tabular}
\end{table}


\section{203 - Opacités et masses thoraciques}%
\label{sec:203_opacites_et_masses_thoraciques}

\begin{itemize}
\item < 3mm : micronodules
\item $[3,30]$mm : nodules
\item > 30mm : masses
\end{itemize}

\subsection{Nodules}
Origine maligne probable si :
\begin{itemize}
\item homme, > 50 ans, fumeur
\item carcinogènes professionnels, > 1cm (> 3cm ++)
\item contours spiculés, polylobés, irrégulier
\item attire structures proches
\item augmente de taille
\item pas de calcifications
\item fixe TED-FDG
\end{itemize}
Certitude = histologie

\textbf{Tumeurs malignes}  
\begin{itemize}
\item cancers bronchopulmonaires primitifs : > 50 ans, fumeur, souvent nodule
  solitaire
\item secondaires : opacités rondes régulières
\end{itemize}

\textbf{Tumeurs bénignes} 
\begin{itemize}
\item Hamartochondrome (freq++) : "pop-corn", pathognomonique
\item Tumeurs  carcinoïdes
\end{itemize}

\textbf{Non tumorales}
\begin{itemize}
\item infectieux 
  \begin{itemize}
  \item abcès à pyogène (contexte aigü fébrile)
  \item bactérie filamenteuse crossance lente
  \item tuberculome ($\implies$ prélèvements)
  \item kytes hydatiques ("membrane flottante")
  \item aspergillome (opacité ronde + croissant gazeux)
  \end{itemize}
\item granumolatose avec polyangéite
\item nodules rhumatoïdes
\item atélectasies
\item masses pseudo-tumorale silicotiques (micronodules, confluents ?)
\item malformations artérioveineuses
\end{itemize}

\paragraph{Examens}
\begin{itemize}
\item TDM et TEP
\item Fibroscopie pronchique systématique
\item Ponction transpariétale sous TDM (sauf insuffisance respi)
\item Autres : thoracotomie, médiastinoscopie si ADP médiastinales fixant en TEP-FDG
\end{itemize}
Prélevement si solide, > 8mm, hypermétabolique. Sinon surveillance TDM (sauf non solide et image résolutive à 6 semaines)

\subsection{Masses/tumeurs du médiastin}
Diagnostic : limite externe nette, raccord pente douce, limite interne non
visible, tonalité hydrique

DD: intraparenchymateux, pariétal $\implies$ TMD

\paragraph{Médiastin antérieur}
\begin{itemize}
\item Supérieur : goître plongeant $\implies$ TMD : continuité glande thyroïde
\item Moyen : 
  \begin{itemize}
  \item tumeurs thymiques : épithéliales (thymomes, carcinomes thymiques),
    lymphomes thymiques, kystes, tumeurs bénignes
  \item Tumeurs germinales : bénignes, séminomateuses, non séminomateuses
    (carcinomes embryonnaires, vitellines, choriocarcinomes)
  \end{itemize}
\item Inférieur : kystes pleuropéricardiques
\end{itemize}

\paragraph{Médiastin moyen}
\begin{itemize}
\item Tumoral : cancer bronchopulmonaires, lymphomes, LLC, cancers
  extra-thoraciques
\item Non tumoral : sarcoïdose, tuberculose, silicose, infections
  parenchymateuses chroniques, histoplasmose (Amérique du Nord)
\item Autres : insuf. cardiaque gauche
\end{itemize}

\paragraph{Médiastin postérieur} "neurogènes"

Diagnostic :
\begin{itemize}
\item médiastin antérieur : 
  \begin{itemize}
  \item $\alpha$-foetoprotéine (tumeurs vitellines), HCG\footnote{Hormone gonadotrophine chorionique} (choriocarcinomes)
  \item ponction transpariétale
  \item médiastinotomie
  \item chir si complète et peu mutilante
  \end{itemize}
\item médiastin moyen : médiastinoscopie ou ponction transbronchique
\item médiastin postérieur : 
  \begin{itemize}
  \item ponction transpariétale, transoesophagienne
  \item chir si complète et peu mutilante
  \end{itemize}
\end{itemize}

NB : urgence si jeune et suspicion de tumeur germinale non séminomateuses

\section{204 - Insuffisance respiratoire chronique}
\label{sec:org6d633b6}
\subsection{Mécanismes}
\label{sec:org47f478d}
Hypoxémie
\begin{itemize}
\item Inadéquation ventilation/perfusion :
  \begin{itemize}
  \item effet shunt (mauvaise ventilation) => $O_2$ corrige
  \item shunt vrai (communication anat. ou non ventilé) => $O_2$ ne corrige
  pas
  \end{itemize}
\item Hypoventilation alvéolaire : pure (commande, neuromusc) ou effet "espace mort"
\end{itemize}
(mauvaise perfusion)
\begin{itemize}
\item Atteinte de la surface d'échange
\end{itemize}
Hypercapnie : hypoventilation alvéolaire (pompe ventilatoire/commande centrale
ou effet espace mort)

\subsection{Conséquences}
\label{sec:orgf6b3986}
Hypoxémie : Polyglobulie, rétention hydrosodée (fréquente), hypertension pulmonaire

Hypercapnie : compensée par le rein

\subsection{Étiologies}
\label{sec:org5310651}
Hypoxémie si PaO\(_{\text{2}}\) < 70mmHg (arbitraire). Voir table~\ref{tab:etio_irc}.
\begin{table}
\begin{center}
  \begin{tabular}{llllll}
    \toprule
    TV ? & Obstructif & Restrictif & Restrictif & Mixte & Non\\
         &            &  $\frac{T_{LCO}}{V_a}$ bas & $\frac{T_{LCO}}{V_a}$ normal & & \\
    \midrule
    Patho. & BPCO & Interstitielles & Sd obésité-hypoventil & DDB & HTP\\
       & asthme &  & Atteinte cage thoracique & Muscoviscidose & \\
       & bronchiolite &  & &  & \\
    Mécanisme & $\frac{V_a}{Q}$ & Surf d'échange & Hypoventilation & $\frac{V_a}{Q}$ & Surf d'échange\\
    Atteinte & échangeur & échangeur & pompe/central & échangeur & vasculaire\\
    \bottomrule
  \end{tabular}
\end{center}
\caption{Insuffisance respiratoire chronique : diagnostic simplifié (selon
  EFR). $V_a$ ventilation alvéolaire, $Q$ débit sanguin, $T_{LCO}$ capacité de
transfert du CO}
\label{tab:etio_irc}
\end{table}


\subsection{Diagnostic}
\label{sec:orgf8776e2}
\subsubsection{Symptômes}
\label{sec:org35a8cf3}
IRC = dyspnée (sous-évaluée), neuropsy + patho initiale

Physique : 
\begin{itemize}
\item IRC : cyanose, insuf. cardiaque D (turgescence jugulaire, oedeme MI, reflux
  hépato-jugulaire)
\item patho : 
  \begin{itemize}
  \item obstructive : distension thoracique, dimin. bilat murmure vésicul
  \item restrictive : râle crépitant des bases, hippocratisme digital
  \end{itemize}
\end{itemize}

\subsubsection{Diagnostic}
\label{sec:org7599c0f}
PaO\(_{\text{2}}\) < 70 mmHg (gaz du sang : hypercapnie)

Étiologie :
\begin{itemize}
\item EFR donne TVO (VEMS/CVF < 70\% => BPCO), TVR (échangeur/pompe) ou
mixte (DDB, mucov)
\item  Radio thorax
\item Autres : NFS (polyglobulie), ECG (dextrorotation, BDB droite, repolarisation),
écho cardiaque systématique (éval. ventricule D, dépistage du G)
\end{itemize}

\subsection{Traitement}
\label{sec:org870a2d5}
Cause, arrêt tabac, vaccins (grippe, pneumocoque), réhabilitation respi.

Oxygénothérapie de longue durée : indiquée si 2 mesures à 2 semaines avec
\begin{itemize}
\item obstructive = PaO\(_{\text{2}}\) < 55mmHG ou ( \(\in\) [55, 60] mmHG et hypoxie
  tissulaire\footnote{Ht > 55\%, HTP, insuf. ventricule D, SpO\(_{\text{2}}\) nocturne \(\le\) 88\%})
\item restrictive = PaO\(_{\text{2}}\) < 60 mmHg
\end{itemize}
Efficace si  IRC après BPCO, 15h/jour, $O_2$ gazeux ou liquide

Ventilation long cours : IRC restrictive, la nuit.

Chir rare (200/an)

\subsection{Pronostic}
\label{sec:org2ec66b4}
Irréversible, risque = insuf. respi aigüe (surtout causée par insuf. respi. basse, dysfonction
cardiaque G, EP)


\section{205 - BPCO\footnote{Bronchopneumopathie chronique obstructive}}
\label{sec:205-bpco}

\subsection{Définitions}
BPCO = \{toux, dyspnée, expectorations, infections respi basses\} récurrentes et
TVO (VEMS/CVF < 0.7)
\textbf{persistant}

Exacerbation aigùe = aggravation \(\ge 2\) jours

Inclus dans BPCO \textbf{si TVO} : 
\begin{itemize}
\item bronchite chronique (toux productive quotidienne \(\ge 3\) mois/an et \(\ge 2\) ans),
\item emphysème (élargissement espaces aériens distaux + destructions parois
alvéolaires) inclus dans BPCO
\end{itemize}

\begin{table}[htbp]
  \caption{Différences asthme-BPCO}
  \centering
  \begin{tabular}{ll}
    \toprule
    Asthme & BPCO\\
    \midrule
    Obstructive non réversible & Obstructive réversible\\
    Jeune, atopique & Fumeur, > 40 ans\\
    Survient \textasciitilde{}40 ans & Enfance\\
    \bottomrule
  \end{tabular}
\end{table}

Sévérité : 
\begin{itemize}
\item obstruction : GOLD (stade 1 (VEMS $\ge 80\%$) à 4 (VEMS < 30\%))
\item dyspnée : échelle MRC (0 (effort important) à 4 (habiller))
\item fréquences exacerbations ($\ge 2$/an = grave)
\end{itemize}

Épidémio: en augmentation dans le monde. \\
FR : \{tabac++, aérocontaminants professionnels\}, \(\alpha{}1\) antitrypsine

Évolution : perte fonction respi, exacerbations, handicap respi, risque
d'insuffisance respi, comorbidité CV = 1ere cause de mortalité\\
Score BODE\footnote{Body mass index, Obstruction, Dispnea, Exercice} pour la prédiction.

\subsection{Diagnostic}
Signes fonctionnels : dyspnée++, toux, expectorations.\\
Signes physiques (après apparition TVO) : \(\nearrow\) temps expiratoire,
\(\searrow\) murmure vésiculaire, $\searrow$ bruits coeur, distension thoracique

\subsubsection{EFR}
\begin{itemize}
\item spirométrie : TVO (VEMS/CVF < 0.7) persistante après bronchodilateur
\item pléthysmographie (identifie distension pulmonaire)
\item $T_{LCO} < 70\%$ = pathologique
\item si VEMS < 50\% ou $SpO_2< 90$\% : gaz du sang, test d'exercice
\end{itemize}

\subsubsection{Autres}
(TDM), ECG si VEMS < 50\%, NFS (polyglobulie, anémie), (dosage \(\alpha1\) antitrypsine)

\subsection{Traitement}
\begin{figure}[htpb]
  \centering
 \resizebox{0.5\linewidth}{!}{
  \tikz \graph [
  % Labels at the middle 
  edge quotes mid,
  % Needed for multi-lines
  nodes={align=center},
  sibling distance=4cm,
  level distance=2cm,
  edges={nodes={fill=white}}, 
  layered layout]
  {
    Dyspnée/exacerbations -> {
      BD longue durée[>"oui", draw]-> 
      {
        "2 BD longue durée" [draw, >"dyspnée"] -> "Cortico. inhalé\\+ 2 BD longue durée"[draw, >"exacerb"];
        "Cortico. inhalé\\
        + BD longue durée"[draw, >"exacerbation"]-> "Cortico. inhalé\\+ 2 BD longue durée"[draw,
        >"insuffisant"]
        -> "Réévaluation";
        "2 BD longue durée" [draw] -> "Réévaluation"[>"dyspnée"];
        ;
      };
      BD courte durée [>"non", draw];
    }
  };
  }
  \caption{Traitement BPCO}
\end{figure}

Arrêt tabac++, vaccins grippe et pneumocoques, réhabilitation respiratoire, oxygénothérapie, chirurgie possibles

\subsection{Exacerbations BPCO}
\label{subsec:ebpco}
\subsubsection{Diagnostic :}
\begin{itemize}
\item BPCO connu : \(\nearrow\) dyspnée, toux/expect.
\item sinon : cf détresse respi
\end{itemize}

Déclenchants : majorité = infectieux mais souvent pas de facteur précis
(\bact{influenzae}, \bact{pneumocoque}, \bact{catarrhalis})

DD : PAC, dysfonction cardiaque gauche, embolie pulmonaires, pneumothorax, médicaments CI, trauma/chir thoracique, insuffisance cardiaque gauche aigüe.

\subsubsection{Explorations}
Imagerie thorax, ECG, NFS, CRP, iono, créat, gazométrie

\subsubsection{Traitement}
Bronchodilatateurs $\beta2$ agonistes courte-durée.\\
ATB : majoration purulence : amox + acide clav si FR, sinon amox-acide
clav/pristinamycine/macrolides\\
Autres \faHospitalO : oxygénothérapie, kiné, HBPM, (assistance ventilatoire)

\section{206 - Pneumopathies infiltrantes diffuses}
\subsection{Présentation}
Clinique : dyspnée d'effort prgorsessive.\\
EFR : \gls{TVR} ( CPT < 80\% et VEMS/CVL >
70\% ) et TLCO < 70\%, hypoxémie, désaturation\\
Radio : opacités parenchymateuse non systématisées bilatérales

\subsection{PID aigüe}
\subsubsection{Étiologies}
Connues : lymphangite carcinomateuse, insuf. cardiaque gauche, médicamenteuse\\
Inconnues : sarcoïdose, fibrose plumonaire idiopathique
\subsubsection{Démarche}
\begin{itemize}
\item Contexte (ATCD, ID, exposition)
\item ECG, BNP, echo cardiaque
\item LBA si possible
\item PEC thérapeutique (réa si détresse respi, \(O_2\), ATB probabiliste si fièvre, arrêt de médic. pneumotoxiques)
\end{itemize}

\subsection{PID subaigüe/chronique}
\subsubsection{Démarches}
Interrogatoire++ : terrain (sarcoïdose=25-45 ans, \gls{FPI} si > 60 ans), tabac (histiocytose langerhansienne, \gls{DIP}), toxico, médic, ATCD radio, exposition

Clinique : état général, signes de connectivite

\begin{table}[htbp]
  \caption{Biologie PID subaigüe}
  \centering
  \begin{tabular}{ll}
    \toprule
    Examen & Maladie\\
    \midrule
    NFS, CRP & Sd inflammatoire\\
           & Hyperéosinophilie, lymphopénie\\
    BNP & Insuf. cardiaque\\
    Créat & Insuf. rénale\\
    Précipitines sériques & Hypersensib. (si contexte)\\
    CEA, calcémie, calciurie & Sarcoïdose\\
    Facteur rhumatoïdes etc & Connectivites\\
    ANCA & Vascularite\\
    Séro VIH & Opportuniste\\
    \bottomrule
  \end{tabular}
\end{table}

\begin{table}[htbp]
  \caption{LBA PID subaigüe}
  \centering
  \begin{tabular}{ll}
    \toprule
    Normal & 80\% macrophages\\
           & < 15\% lymphocytes\\
           & < 5\% PNN\\
           & < 2\% PNE\\
    \midrule
    Alvéolite & Hypercellularité totale\\
    Histiocytose langerhansienne & Macrophage\\
    Sarcoïdose, PHS & Lymphocytaire\\
    P. à éosinophiles & Éosinophilique\\
    \gls{POC} & Panachées\\
    Hémorragie alvéolaire & Rosé\\
    Protéinose alvéolaire primitive & Laiteux\\
    \bottomrule
  \end{tabular}
\end{table}

Examens complémentaires :
\begin{enumerate}
\item fibro et LBA (+ biopsie bronchique)
\item soit biopsie pulmonaire chir (pas de diagnostic), soit biopsie
  transbronchique et ADP médiastinales
\end{enumerate}

\subsubsection{Oedème pulmonaire}
Mécanisme : Surcharge hémodynamique\\
Clinique : HTA, coronaropathie, valvulopathie mitrale\\
Diagnostic : ECG, BNP, écho coeur\\
Imagerie : Péri-hilaire

\subsubsection{Tuberculose}
Mécanisme : BK\\
Clinique : Contage, AEG, hémoptysie\\
Diagnostic : Expectorations (ED, culture, biopsie transbronchique)\\
Imagerie : 
\begin{itemize}
\item pulmonaire = nodules, infiltrats, excavations
\item miliaire = micronodules diffus
\end{itemize}

\subsubsection{Médicaments}
Imagerie : condensations, verre dépoli, épanchement pleural

\subsubsection{Pneumopathies d'hypersensibilité}
Mécanisme : Ag organiques\\
Clinique : 
\begin{itemize}
\item aigüe : sd peudo-grippal quelques heurs
\item subaigüe : semaines/mois avec toux, fébricule, râles crépitants, squeaks
\item chronique : dyspnée, toux sèche
\end{itemize}
Diagnostic : Sérologie, LBA\\
Imagerie : Micronodules centrolobulaires flous, verre dépoli (lobes supérieurs)\\
Traitement : éviction Ag

\subsubsection{Pneumoconioses}
Mécanisme : Amiante, silice\\
Clinique : Exposition\\
Imagerie : 
\begin{itemize}
\item silicose : opacités micronodulaires diffuses \(\implies\) masses pseudotumorales. Peut donner un cancer bronchique
\item asbestose : opacités linéaires non septales des bases \(\parallel\) ou \(\bot\) plèvre, réticulations et rayons de miels comme FPI. Évolue vers insuf respi chronique.
\end{itemize}

\subsubsection{Sarcoïdose}
Mécanisme : Signes extra-respiratoires\\
Diagnostic : anapath : extra-pulmonaire, biopsie éperons bronchiques et transbronchique. ADP médiastinales \\
Imagerie : Nodules, micronodules (ditribution lymphatique), adénopathie, hyperdensités, distorsions bronchiques

\subsubsection{Fibrose pulmonaire idiopathique}
Clinique : Dyspnée d'effort progressive, toux sèche, hippocratisme digital, crépitants sec base\\
EFR : TVR, diminution TLCO\\
Imagerie : Réticulations, bronchectasies, rayons de miel. Domine sous-pleural et bases

\subsubsection{Connectivites}
Mécanisme : Dysimmunitaire\\
Clinique : Extra-respi (polyarthrite rhumatoide, sclérodermie, lupus, vascularite)\\
Diagnostic : Ac spécifiques\\
Imagerie : Réticulations, hyperdensités, bronchectasies

\subsubsection{Pneumopathie interstitielle non spécifique}
Origine : connectivite, médicaments (idiopathique)\\
Imagerie : verre dépoli, réticulations, bronchectasies (sauf extrême périphérie du poumon)

\subsubsection{Proliférations tumorales}
Lymphangite carcinomateuse : toux sèche, rebelle. \\
Radio : épaississements nodulaires des septas intralobulaires.\\
Diagnostic : biopsies des éperons\\
Carcinome lépidique : verre dépoli. 


\section{207 - Sarcoidose}
Maladie : systémique, cause inconnue, hétérogène, ubiquitaire. Début 25-45ans
dans 2/3\\
Atteinte médiastino-pulmonaire 90\%

\subsection{Expression}
\label{sec:org39048da}
\subsubsection{Pulmonaire}
\label{sec:org4ec1d7e}
Toux (dyspnée)
Radio : 4 stades
\begin{itemize}
\item I : adénopathies hilaires bilatérales symétriques
\item II : + atteinte parenchyme (micronodulaire diffus, parties moyennes supérieures)
\item III : atteinte parenchyme isolée
\item IV : fibrose = opacités parenchymateuses rétractiles + ascension hiles, distorsion bronchovasc (sup et post)
\end{itemize}
TDM : atteinte parenchyme = micronodule selon lymphatiques. Utile pour : formes atypiques ou détection précoce (fibrose, complications [greffe aspergillaire])\\
EFR : sd restrictif, DLCO $\searrow$\\
(Endoscopie bronchique : normal/muqueuse en "fond d'oeil".)\\
Biopsie : \{éperons, LBA\} > \{ponction ganglions médiastinaux, transbronchique\} > médistanoscopie\\
Formes atypiques : TVO, cavitaires, pseudonodulaires/alvéolaires

\subsubsection{Extra-pulmonaire}
\label{sec:org316c190}
\begin{itemize}
\item Oeil : uvéite antérieure aigue (toujours cherche uvéite postérieure)
\item Peau : nodules cutanés, lupus pernio, érythème noueux
\item ADP
\item Foie
\item Moins fréquentes : nerveux (sd méningé, paires craniennes), ORL (obstruction
  nasale,
\end{itemize}
sd Mikulicz, sd Heerfordt), ostéo-articulaire (bi-arthrite cheville =
spécfique++), 
coeur (BAV, bloc branche droit), rein (\(\nearrow\) créatininémie)
\begin{itemize}
\item Généraux : asthénie (pas de fièvre sauf sd de Löfgren)
\end{itemize}
Sd de Löfgren = érythème noueux + ADP hilaires médiastinales (+ fièvre)
\subsubsection{Biologie}
\label{sec:orgbfe5d87}
\begin{itemize}
\item Hypercalciurie
\item Lymphopénie CD4
\item Hypergammaglobulinémie
\item Enzyme de conversion de l'angiotensine sérique (ECA)
\end{itemize}
\subsection{Diagnostic}
\label{sec:org6670330}
Clinique + radio + lésions granulomateuses tuberculoides sans nécrose caséeuse +
élimination DD
\subsection{Évolution}
\label{sec:org707f9ea}
< 2 ans : évolution favorable sans traitement.\\
Chronique > 2 ans : attention au vital/fonctionnel \\
Suivi : 3-6 mois\\
Pronostic : 80\% favorable sans traitement, 10\% séquelles, 5\% DC

\begin{table}[htbp]
  \caption{Pronostic de la sarcoidose}
  \centering
  \begin{tabular}{ll}
    \toprule
    Négatif & Positif\\
    \midrule
    > 40 ans & Érythème noueux\\
    Chronicité & Forme aigüe\\
    Stade III, IV & Stade 1 asymptomatique\\
    Extra-respi grave & \\
    \bottomrule
  \end{tabular}
\end{table}

Atteintes :
\begin{itemize}
\item pulmonaire : insuf. respir chronique, principace cause DC
\item extra-thoracique : attention fonctionnel/vital
\end{itemize}

\subsection{Traitement}
\label{sec:orgc05b9f6}
Atteinte respi : pas de ttt si sd de Löfgren ou stade I asymptomatique\\
\begin{itemize}
\item 1ère intention : Corticoïdes > 12 mois à 0.5mg/kg (décroissance par 6-12 semaines)
\item 2eme intention : hydroxychloroquine, méthotrexate, azathioprine
\item 3eme intention : cyclophsamide, anti-TNF-\(\alpha\)
\end{itemize}

\section{222 - Hypertension artérielle pulmonaire}%
\label{sec:hypertension_arterielle_pulmonaire}

Circulation : (basse pression, faible résistante) $\implies$ forte résistance

Critère : \gls{PAPm} $\ge 25$ mmHG et
$\begin{cases}
  \text{\gls{PAPO}} \le 15 & \text{mmHG si précapillaire} \\
  \text{PAPO} > 15 & \text{mmHG si postcapillaire}
  \end{cases}  $

Classification internationale : 5 groupes
\begin{enumerate}
\item hypertension artérielle pulmonaire (HTAP)\footnote{Idiopathique,
    héritable, médicaments, maladie veino-occlusive, hémangiomatose capillaire
    pulmonaire, HTP persistante du nouveu-né}: pré-capillaire
\item \textbf{cardiopathie gauche}  : post-capillaire (fréq+++)
\item HTP maladie \textbf{respiratoire chronique}  : pré-capillaire (freq++)
\item HTP post-embolique chronique : pré-capillaire
\item HTP multi-factorielles : pré-capillaire
\end{enumerate}

\paragraph{Pronostic}
6 cas/million (idiopathique), femme.

Survie avec ttt : 58\% à 3 ans

\subsection{Diagnostic}
Découverte : dyspnée, dépistage

Détection :
\begin{itemize}
\item fonctionnel : dyspnée d'effort+++ progressive (lipothymie à l'effort,
  syncope, asthénie, douleurs angineuses, palpitations, hémoptysies)
\item physique : 
  \begin{itemize}
  \item HTP (signe de Carvallo\footnote{Souffle holosystolique
    d'insuf. tricuspid majoré à l'inspiration}, éclat B2, souffle diastolytique
  d'insuf pulmonaire)
  \item insuf cardiaque D (tachy, galop, turgescence jugulaire, reflux
  hépato-jugulaire, HMG, OMI, anasarque)
  \end{itemize}
\item imagerie thorax : dilatation artères pulmonaires, élargissement coeur D
\item ECG : hypertrophie D, trouble rythme
\end{itemize}

Écho cardiaque transthoracique = non invasif de référence. \\
Diagnostig par cathétérisme cardiaque D avec \textbf{PAPm $\ge 25$ mmHg}

\paragraph{Démarche}
Si écho cardiaque compatible avec HTP :
\begin{itemize}
\item Cardiopathies G, maladies respi + exams pour groupe 2 et 3. Si confirmé
  : \faHandStopO
\item Sinon regarder signes thrombo-embolie chronique (scinti, angioscan) : Si
  groupe 4, \faHandStopO
\item Sinon confirmer HTP précapillaire 
\item Si confirmée, tester pour groupe 1 (connectivites, médicaments, VIH,
  cardiopathie congénitale, HT portale, schistosomiase) ou groupe 5
\end{itemize}

\section{224 - Embolie pulmonaire et thrombose veineuse profonde (x2)}%
\label{sec:224_embolie_pulmonaire_et_thrombose_veineuse_profonde}
\gls{MTEV} = \{\gls{EP}, \gls{TVP}\}

Maladie fréquente (1 cas /10 000 si < 40 ans, 1/100 si > 75 ans) et grave.

Facteurs de risque :
\begin{itemize}
\item acquis : majeurs = chir < 3mois, trauma MI\footnote{Membres inférieurs},
  \faHospitalO{} aigü, cancer en cours  ttt, sd antiphospholipide, sd
  néphrotique
\item constit : rare = déficit (antithrombine, prot. C, S), fréq = (mutation {Leiden,
    prothrombine}, facteur VIII > 150\%)
\end{itemize}
Complications : DC, récidive (mortelle ou non), séquelle (HTP thrombo-embolique
chronique si EP, sd post-phlébite si TVP)

Risque de récidive dépend de la clinique : élevé si (non provoqué par facteur
majeur ou modéré) ou (\og facteur persistant)

Conséquence :
\begin{itemize}
\item hémodynamique : $\nearrow$ pression artérielle pulmonaire, dilatation
  VD\footnote{Ventricule droit}, compression VG
\item respiratoire : hypoxémie (effet espace mort (non perfusé) $\implies$ effet shunt
  (ventil/perfusion diminué))
\end{itemize}

\subsection{Diagnostic de l'embolie pulmonaire}
\begin{figure}[htpb]
  \centering
  \resizebox{0.3\linewidth}{!}{
    \tikz \graph [
    % Labels at the middle 
    edge quotes mid,
    % Needed for multi-lines
    nodes={align=center},
    level distance=2cm,
    sibling distance=3cm,
    edges={nodes={fill=white}}, 
    layered layout]
    {
      "Proba. clinique" -> {
        "D-dimères" [>"faible", draw] -> {
          "Pas de ttt" [>"négatif"];
          "\texttt{\$examen}" [>"positif", draw];
        };
        "\texttt{\$examen}" [>"forte", draw] -> {
          Ttt [>"positif"];
        };
      };
    };
  }
  \caption{Diagnostic général pour l'EP, TVP}
  \label{fig:ep-diag}
\end{figure}


\paragraph{Suspicion}
Clinique : douleur thoracique (type pleurale) ou dyspnée isolée (ascult.
normale !) ou état de choc

Radio thoracique, ECG : élimine les DD

Probabilité clinique : important, suivant des scores (Wells, Genève)

\paragraph{Examens}
Voir la figure~\ref{fig:ep-diag} avec \texttt{examen} = angioscanner.

D-dimère positifs = $\max(\text{âge}, 50) \times 10 \mu{}$g/L

CI à l'angioscan : insuf. rénale sévère (< 30ml/min) $\implies$ scintigraphie

Si angioscan ou scintigraphie négatif : pas d'EP !

\danger{} si EP grave (état de choc) : angioscan immédiatement ou échographie
cardiaque en attendant. Traitement après écho si pas d'accès à l'angioscan
\skull\\
Sur l'écho, chercher dilatation cavité D, HTP, septum paradoxal

\textit{Si grossesse}  : D-dimères $\implies$ écho. veineuse $\implies$ angioscanner
(prévenir le pédiatre \skull)

\subsection{Diagnostic de TVP}
Voir la figure~\ref{fig:ep-diag} avec \texttt{examen} = échographie veineuse
des MI

\subsection{Traitement}
\paragraph{Principes}
Urgence \skull $\implies$ anticoagulant

CI : coagulopathie sévère, hémorragie intracrânienne spontanée, hémorragie
active difficilement contrôlable, chir récente, (thrombopénie à l'héparine)

q\paragraph{Types de traitement}\mbox{}\\
Option 1 : Héparines + relais AVK dès injection IV
\begin{itemize}
\item HBPM, fondaparinux > HNF, sauf si IR sévère (< 30ml/min)
\item arrêt si 5 jours d'AVK et IV (ensemble) $\wedge$ INR $\in [2,3]$ à 24h
\end{itemize}
Option 2 : Anticoagulants oraux directs : rivaroxaban, apixaban (France)
\begin{itemize}
\item rapide, demi-vie courte
\item facteur X
\item CI : IR sévère, grossesse, interaction médic (cytochrome 3A4 ou
  P-glycoprroténie)
\end{itemize}
Éucation thérapeutique\\
Autres :
\begin{itemize}
\item filtre cave (si CI absolu aux anti-coagulant ou EP récidivant)
\item fibrinolyse (si EP + choc, sauf si hémorragie active,
  AIC\footnote{Accident ischémique cérébral}< 2 mois,
  hémorragie intracrânienne)
\item embolectomie (très rare)
\item contention veineuse (sauf si EP sans TVP)
\item lever +1h
\end{itemize}

\paragraph{Stratégie}
Score sPESI = \texttt{90 100 110 C C}\footnote{sat < 90\%, PAS < 100 mmHg, FC >
110/min, Cancer, insuf Cardiaque chronique}:

Risque faible (sPESI = 0) : \faHospitalO{} < 48h, anticoagulation\\
Risque intermédiaire (sPESI > 0)
\begin{itemize}
\item dysfonction VD ou élévation biomarqueurs\footnote{BNP, NT-pro-BNP,
    troponine} : \faHospitalO{}
  médecine, anticoagulation
\item dysfonction VD $\wedge$ élévation biomarqueurs : urgence \danger{}
  $\implies$ USI
  \begin{itemize}
  \item $O_2$, scope
  \item anticoag : HNF/HBPM puis AVK/AOP à 48-72h\\
  (trombolyse si choc)
    \end{itemize}

\end{itemize}
Haut risque (choc : PAS < 90mmHG ou -40mmHg) : urgence \skull $\implies$ réa
\begin{itemize}
\item $O_2$ (ventilation méca.), scope
\item anticoag HNF (!)
\item thrombolyse avec arrêt HNF tant que TCA > 2x témoin
\item (embolectomie)
\end{itemize}

\paragraph{Durée}
3 mois si 1ere EP/TVP provoqué par facteur majeur transitoire ou risque
hémorragique élevée. 6 mois ou plus sinon

\paragraph{Étiologie}
Chercher cancer occulte dans tous les cas (clinique, radio poumon, NFS VS,
dépistage globux)

Bilan coga : {antithrombine, prot C, S}, mutation {Leiden, prothrombine}, sd
antiphospholipides

\paragraph{Prophylaxie}
\begin{itemize}
\item Post-op : (chir et > 40 ans) ou (chir hanche/genou/caricinologique, anomalie
  coag, (> 40 ans et ATCD MTEV))
\item polytrauma ou ({rhumato, inflammatoire intestin, infectin} + 1 FR)

\end{itemize}


\section{228 - Douleur thoracique aigüe et chronique}%
\label{sec:228_douleur_thoracique_aigue_et_chronique}

\subsection{Signes de gravité}
\begin{itemize}
\item Respi : cyanose, tachypnée, lutte avec tirage, balancement thoraco-abdominal
\item CV : pâleur, tachycardie, hypotension, choc (marbrures, extrémités
  froides)
\item neuro : lipothymie/syncope, agitation/trble vigilance, général (sudation)
\end{itemize}

\danger arrêt cario-respi ! Y penser si bradypnée, (bradycardie et choc et troubles vigilance\}

\subsection{Examens}
Fréquence respi, $SpO_2$, radio thorax, ECG.
Si {brady, tachy}pnée ou $SpO_2 < 95\%$, gaz du sang

\subsection{Urgences vitales}
\begin{itemize}
\item Syndrome coronaire aigü (fréquent++ 1/3) : ECG + troponines
\item Embolie pulmonaire (fréquent) : suspicion si douleur thorax, pas
  d'anomalie ascult, RX thorax "normale" \textit{surtout}  si hypoxémie +
  facteurs de risque\\
  Dyspnée ou douleur thoracique aigüe chez TVP = EP
\item dissection aortique (exceptionnelle) : échocardio + angioscanner
\item Tamponnade (peu fréq) : suspicion (hypotension réfractaire, insuf.
  cardiaque D aigüe, microvoltage + alternance ECG) $\implies$ echo cardiaque
\item Pneumothorax : ATCD (!), radio thorax \\
  pneumomédiastin (rare) : scanner
\end{itemize}

\subsection{Non urgent}
Rythmées par respiration :
\begin{itemize}
\item post-traumatique
\item pneumonie infectieuses : radio thorax
\item épanchement pleural (douleur latéral-base, majorée inspiration, toux)
\item infarctus pulmonaire (douleur basithoracique, faible hémoptysie)
\item trachéobronchite aigüe
\item musculosquelettique, nerfs : tumeurs costales, lésions vertèbres,
  névralgies cervicobrachiales
\end{itemize}
Non rythmées :
\begin{itemize}
\item angor d'effort stable (calmée 2-5min post-effort)
\item péricardite (viral si aigü, tuberculose/néoplasie sinon)
\item cocaïne (fréquente) : SCA, myopéricardite,pneumothorax
\item zona thoracique (brûlures, hyperesthésie 24h avant)
\item digestives : reflux gastro-oesophagen, spasmes oesophagiens. Exclure SCA
  \skull
\item psychogènes
\end{itemize}

\section{306 - Tumeurs du poumon}%
\label{sec:306_tumeurs_du_poumon}

\subsection{Épidémiologie, étiologies}
1ere cause de mortalité par cancer en France (\male : constante, \female
$\nearrow$)

Tabac : 90\% (âge de début et durée) pour actif, 25\% passif

Professionel : 15\%

\paragraph{Histologie}
\begin{itemize}
\item "Non à petites cellules" (CBNPC) (80\%) : adénocarcinomes (périphérie),
  carcinomes épidermoïdes (proximal), indifférenciés à grandes cellules
\item neuro-endocrine "à petites cellules" (CBPC) (proximal, médiastin)
\end{itemize}

\subsection{Manifestations}
\textbf{Y penser si (AEG et tabagique) ou (symptôme fonctionnel respiratoire
  chez tabagique > 40A)}

\textit{Respiratoires}  : toux (souvent révélatrice, sèche, quinteuse), hémoptysie (<
10\%), bronchorrée (propre), dyspnée, pneumonie/bronchite (\danger{} si récidive
dans même territoire), douleur thoracique 

\textit{Locorégionales}  : pleurésies, dysphonies, sd cave supérieur, douleurs thoraciques
(fixes, tenaces), sd de Pancoast-Tobias\footnote{Névralgie cervicobrachiale avec
  douleurs radiculaire C8-D1, sd Claude-Bernard-horner}, paralysie phrénique

\textit{Extrathoracique}  : thromboses inexpliquées (y penser !), métastase (SNC,
foie, os, surrénales)

\textit{Sd paranéoplasiques} (à distance, réapparition $\implies$ rechute) :
hippocratisme digital, hypercalcémie, hyponatrémie, sd de Cushing, neurologique
(pseudomyasthénie de Lamert Eaton, neuropathies périph, encéphalopathies)

\subsection{Imagerie}
Initial = radio thorax chez fumeur > 40 ans
\paragraph{Radiothorax}
\begin{itemize}
\item projection (juxta-)hilaires
\item atélectasies
\item opacités arrondies intraparenchymateuses
\item cavitaires néoplasiques
\end{itemize}
% Normal $\centernot\implies$ pas de cancer

\paragraph{TDM injection}
Stade N0 sans adénopathie, N1 si hilaire, N2 si médiastin homolatéral, N3 si médiastin
controlatéral

\paragraph{TEP au 18-FDG}
Pas dans le cerveau !

Faux positifs : ganglions inflammatoires, infectieux

Faux négatifs : < 1 cm, non solide (verre dépoli)

\subsection{Diagnostic histologique}
\begin{itemize}
\item lésion centrale : fibroscopie bronchique
\item périphérique : ponction transpariétale
\item entre les 2 : fibro ou ponction ou thoracotomie
\end{itemize}
Chercher mutations EGFR, réarrangements ALK, ROS1

\subsection{Bilan préthérapeutique}
\textit{Extension}  : T(tumeur) = locale, N (node) = ganglionnaire, M
(métastase) = à distance. \\
si M : TDM abdo, IRM cérébrale, TEP-TDM

État général (Performance Status de 0 à 4), nutritionnel (-5\% en 1 mois ou
-10\% en 6 mois = \frownie{})

\textit{Cardiorespiratoire} : ECG, écho cardio, épreuve d'effort, doppler artériel des MI/cou,
coronarographie.\\
Règle : 1 lobe = -25\% de VEMS. Opération ssi VEMS post-op > 30\%
\subsection{Traitement}
\paragraph{CBPNC}
\begin{itemize}
\item localisé (stade I, II) : local (lobectomie ou radiothérapie)
\item localement avancé (stade III) : systémique (chimio) + local (radio ou
  chir)
\item disséminé (stade IV) : systémique = 
  $\left \{
    \begin{array}{lr}
      \text{\gls{ITK} si altération moléculaire ciblable} & \text{Médiane > 2 ans}\\
      \text{chimio IV et 3e génération sinon} & \text{Médiane 12 mois}\\
    \end{array}
  \right.$
\end{itemize}
\paragraph{CBPC\footnote{Chimiosensible rechutes fréquentes et rapides}}
Pas de chir !
$
\left \{
  \begin{array}{l}
    \text{radio + chimio (+ irradiation encéphale) si limitée}\\
    \text{chimio sinon}
  \end{array}
\right.
$

\paragraph{Symptomatique}
\textit{Douleur}  : Antalgiques, radiothérapie, chir, AINS, vertébroplastie

\textit{Dyspnée}  : lymphangite carcinomateuses (difficile), obstruction bronchique
(bronchoscopie), pleurésie exsudative (pleuroscopie), sd cave supérieur
(anticoag, corticoïdes)

\paragraph{Plan cancer}
Réunion de concertation pluridisciplinaire, consultation d'annonce, plan
personnalisé de soin
\paragraph{Suivi}
CBNPC : arrêt tabac, TDM 2-3 mois, clinique, bio (rein, NFS, hypercalcémie,
hyopnatrémie)

CBPC : médiane de survie = 16-20 mois si limitée, 8-12 mois si métastatique
\subsection{Cancers secondaires}
Clinique : pauvre (dyspnée, toux, douleur thoracique, généraux), chercher extension ganglionnaire

Imagerie : 
\begin{itemize}
\item lâcher de ballons ou miliaire
\item épanchement pleural
\item interstitiel (lymphangite carcinomateuses)
\end{itemize}

Diagnostic : clinique, 
\begin{itemize}
\item inconnu : chercher accessibles à ttt spécifique, TEP. Sinon :
  \begin{itemize}
  \item épidermoïde, adénocarcinome
  \item sinon : \female{} = {gynéco, mammo}, \male{} = {PSA, TR, écho prostate}
  \end{itemize}
\item connu : radio peut suffir
\item ancien : métastases 10 ans plus tard possibles
\end{itemize}

\section{333 - Oedème de Quincke et anaphylaxie}%
\label{sec:333_oedeme_de_quincke_et_anaphylaxie}

\subsection{Définition, épidémiologie}
HS systémique immédiate sévère. 2 types :
\begin{itemize}
\item allergique : production IgE spécifiques. Libération de médiateurs par
  mastocytes et basophiles $\implies$ vasodilatation, bronchoconstriction
\item non-allergique : pas d'exposition préalable, moins sévère 
\end{itemize}
Déf : symptômes CV, respi, cutané/digestif mettant en
jeu le pronostic vital immédiatement après contact

\paragraph{Épidémiologie}
35 DC (alimentaire), 40 DC (piqûre) / an\\
Agents :
\begin{itemize}
\item aliments (90\% enfant) : arachide++, protéine oeuf/lait, fruits exotiques
\item médicaments (15-20\%)
\item venin hyménoptères (15-20\%)
\item autres : latex, effort
\end{itemize}

\subsection{Clinique}
Délai < 1h (5min si médic. IV, 15 si piqûre, 30 si alimentaire).
Manifestations :
\begin{itemize}
\item CV : PAS < 100mmFg (ou -30\%), tachycardie, pâleur, hypotonie, malaise,
  perte connaissance, trouble rythme/conduction, ischémie myocardique
\item respi : de la toux à l'asthme (mortalité $\nearrow$ si asthme mal
  contrôlé)
\item cutanées, muqueuses : prurit, rash cutané érythémateux, urticaire et
  surtout 
  \begin{itemize}
  \item angioedeme : gonflement mal limité ferme, non érythémateux, sans
    prurit
  \item oedème de Quincke : angioedeme larynx et cou : gêne respi haute
    (dysphonie++, dysphagie++). \\
    Clinique : gonflement langue, luette, paupière lèvre ou face\\
    Possiblement létal \skull
  \end{itemize}
\item digestives
\end{itemize}
\textbf{Hautement probable}  si 
\begin{itemize}
\item gêne respi haute ou asthme ou choc
  mettent en jeu le pronostic vital
\item cutanéomuqueux
\item début brutal, progression rapide
\end{itemize}
Sévérité
\begin{enumerate}[label=\Roman*]
\item cutanéomuqeux généralisé
\item multiviscérale modérée
\item multiviscérale sévère
\item arrêt respiratoire
\item DC
\end{enumerate}

\paragraph{Diagnostic}
Clinique + contexte.

Doser systématiquement tryptase sérique (confirmation) : H+0, H+1, H+24

Étiologie : interrogatoire, {prick-tests, IDR}, dosage IgE

DD : 
\begin{itemize}
\item choc anaphylactique : choc {vagal, septique, cardiogénique}, hypoglycémie, mastocytose
\item Oedeme de Quincke : sd cave sup, érysipèle du visage, angi-eodeme à
  bradykinine, inhalation de corps étranger (toujours y penser !!)
\end{itemize}

\subsection{Traitement}
\paragraph{Urgence}
Adrénaline : 0.01mg/kg (< 0.5mg) adulte à \faHospitalO{} ou  0.3mg
auto-injection $\rightarrow$ répéter /5 min jusque stabilisation\\
Voie IM (pas de SC ou d'IV \danger{})

Remplissage vasc : 50-100mL adulte

$O_2$, libérer voie aériennes, bronchodilatateurs (courte durée)

Glucagon si adrénaline ne fonctionne pas. 

\paragraph{Hors urgence}
Anithistaminique, corticoïdes, arrêt agent.

NB : alerter, allongé jambes relevées (pas vertical \danger{} \skull), PLS si
inconscient

\paragraph{Préventif}
Trousse avec adrénaline si (absolument) :
\begin{itemize}
\item anaphylaxie antérieure (aliment, insecte ou latex), induite par l'effort
  ou idiopathique
\item allergie alimentaire et asthme persistant modéré non contrôlé
\item allergie hyménoptères avec réaction systémique antérieure (adulte) ou
  plus sévère que cutanéomuqueux (enfant)
\end{itemize}

\section{354 - Corps étranger des voies aériennes}%
\label{sec:354_corps_etranger_des_voies_aeriennes}

Pics = 
\begin{itemize}
\item < 3 ans : graines d'oléagineux avec symétrie G-D
\item âgé si troubles déglutition, mauvaise dentition. 2 tableaux : asphyxie
  aigüe (viande) ou pneumonie à répétition/suppuration bronchique
\item rarement chez l'adulte/ado : trauma facial (dents), bricolage, trouble
  de conscience. Plutôt à droite
\end{itemize}

\subsection{Diagnostic}
\paragraph{Clinique et radio}
Dans les heures post-inhalation :
\begin{itemize}
\item CE mobile (sd de pénétration) : toux quinteuse, suffocation avec tirage,
  cornage, cyanose. Résolutif en qq secondes
\item soit CE expulsé (pétéchies visage/tronc ou bouche/conjonctive)
\item plus rarement enclavement 
  \begin{itemize}
  \item proximal (enfant) : diminution murmure vésiculaire, wheezing
  \item distal (adulte) : asymptomatique
  \item exceptionnellement : oropharynx, larynx, lumière trachéale
  \end{itemize}
\end{itemize}
Radio : le plus souvent normale. Sinon : atélectasie, hyperclarté $\nearrow$
expiration

Mois/années post-inhalation : à évoquer si respi chronique/récidivante ne
répondant pas au tt ou anomalies radio persistantes dans même territoire

\paragraph{DD}
(sub)aigü : 
\begin{itemize}
\item détresse respi aigüe à début brutal, tirage, cornage : épiglottite aigüe
  (fièvre, modif voix, hypersaliv)
\item infection respi basse : PAC, bronchite sifflante (< 3 ans)
\end{itemize}
Chronique/récidive :
\begin{itemize}
\item trouble ventilation persistant : tumeur bronchique obstructive, sténose
  bronchique
\item infection respi même territoire : tumeur bronchique obstructive, foyer
  bronchectasies
\end{itemize}

\subsection{CAT}
\paragraph{Asphyxie aigüe}
Toux ou Heimlich (si conscient) ou réanimation. Si < 2 ans, tapes dos puis
pressions setrnum.
\paragraph{Sd pénétration no régressif}
Supposer que CE toujours présent !! (sauf confirmation entourage). 

Clinique : dimination unilat murmure, wheezing, persistance {toux, dyspnée,
  cornage, tirage}

RX  : piégage (expiration) $\wedge$ diminution unilatérale murmure $\implies$ CE
endobronchique (90\%)

\paragraph{Extraction}
Bronchoscopie rigide (ou souple). Chez l'enfant, centre spécialisé !

\section{354 - Détresse respiratoire aigüe}%
\label{sec:354_detresse_respiratoire_aigue}

Définition : Inadéquation charge - capacité de l'appareil respiratoire

\subsection{Dianostic}
Signes de luttes
\begin{itemize}
\item polypnée superficielle
\item recrutemente muscles (scalènes $\wedge$ intercostaux), abdominaux,
  dilatateurs des voies aériennes supérieures (très petit enfant)
\end{itemize}
Signes de faillite
\begin{itemize}
\item respi abdo paradoxale \skull{} défaillance court terme
\item cyanose (hypoxémie profonde) $\implies$ $O_2$ ASAP
\item neuro : astérixis, altérations comportement/vigilance. Glasgow < 9
  $\implies$ assitance ventilatoire
\end{itemize}
Appareil circulatoire
\begin{itemize}
\item coeur pulmonaire aigu\footnote{Turgescence jugulaire, reflux
    hépato-jugulaire, hépatomégalie douleureuse, signe de Harzer}
\item pouls paradoxal
\item hypercapnie : {céphalées, hypervasc des conjonctives}, {tremblements
    sueurs, tachy, hypotension}
\end{itemize}
État de choc 
\begin{itemize}
\item peau froide, marbrures, $\nearrow$ temps recoloration cutanée
\item PAs < 90 mmHg ou -50mmHg
\item tachy > 120min, FR > 25-30/min
\item oligurie
\item confusion, altération vigilance
\end{itemize}

\subsection{PEC}
Surveillance, $O_2$, voie veineuse gros calibre, assistance ventilatoire,
ventilation (non)invasive

Pour l'étiologie : radiothorax, ECG, prélèvement (gaz du sang, NFS, iono, urée,
créat, acide lactique)

\subsection{Diagnostic}

\begin{figure}[htpb]
  \centering
  \resizebox{0.8\linewidth}{!}{
    \tikz \graph [
    % Labels at the middle 
    edge quotes mid,
    % Needed for multi-lines
    nodes={align=center},
    sibling distance=3cm,
    edges={nodes={fill=white}}, 
    layered layout]
    {
      "Obstruction VAS ?" ->["non"]  Radiothorax -> {
        Anormale -> {
          Atélectasies [draw];
          "Anomalies\\pleurales" ->
          "Épanchement\\
          pleural compressif\\
          Pneumothorax" [draw];
          "Opacités\\parenchyme" -> 
          "Pneumonie infect\\
          OAP\\
          Patho. infiltrative" [draw];
        };
        Normale -> {
          "Asthme\\
          aigu grave\\
          EP" [draw];
          "BPCO\\
          Anomalie paroi\\
          Neuromusc" [draw];
        }
      };
    };
  }
  \caption{Diagnostic détresse respiratoire aigǜe}
  \label{fig:diag_detresse_respi}
\end{figure}

Précision de la figure~\ref{fig:diag_detresse_respi}
\begin{enumerate}
\item \textit{Obstruction}  : corps étranger, laryngite, oedème de Quincke, sténose
  trachée, tumeur laryngées
\item \textit{Anomalie radio} Priorité = pneumonie inf, OAP,
  PTX\footnote{Pneumothorax} sous
  tension\\
  Indication =
  \begin{itemize}
  \item OAP : {ATCD = insuf. cardiaque}, {expectoration mousseuse, orthopnée},
    {opacités bilat diffuses, périhilaire}
  \item Pneumonie infectieuse : {début brutal}, {expector. purulente}, {sd
      inflammatoire}
  \end{itemize}
\item \textit{Radio normale} de novo = EP, PTX compressif, asthme. Si
  chronique, dans l'ordre BPCO, paroi thoracique (obésité, déformation),
  neuromusc\\
  Orientation :
  \begin{itemize}
  \item EP : fébricule, turgescence jugul, radio normale
  \item Pneumothorax : {longiligne}, {pas de vibration vocale}, murmure vésicul,
    {hyperclarté}
  \item asthme : ATDS, toux purulente, râles ou silence
  \end{itemize}
\end{enumerate}

\subsection{SDRA}
Détresse respi < 7 jours sans défaillance cardiaque, sans surcharge volémique
avec opacités radio bilat diffuses

Mécanisme = oedème lésionnel.

Étiologies :
\begin{itemize}
\item exogène = infectieuse, toxique
\item endogènes : réponse inflammatoire systémique
\end{itemize}

\section{356 - Pneumothorax}%

\subsection{Définitions}
Air dans l'espace pleural $\implies$ collapsus 

\paragraph{Spontané}
Primaire = poumon sain, sujet jeune, non grave

Secondaire = patho, > 40 ans, peut décompenser

\begin{itemize}
\item Blebs (< 1 cm, plutôt primaires)
\item bulles d'emphysème : à la corticalité, tabagismes
\item lésions kystiques
\item cataménial
\end{itemize}

\paragraph{Traumatique}
Femé (côte fracturée) ou ouverts

\paragraph{Épidémiologie}
Spontané primaire : < 35 ans, masculin, longiligne et de grande taille, fumeur !

Spontané secondaire : BPCO (plus rarement asthme, mucoviscidose)

facteurs favorisants : $\Delta P_{atm}$, vols aérien, plongée subaquatique,
tabagisme actif (PAS
efforts physiques)

\subsection{Diagnostic}
Clinique : 
\begin{itemize}
\item fonctionnel : douleur thoracique (brutale, homolatérale, latérale,
  augmente toux), toux sèche irrit
\item physique : diminution murmure vésicul, pas de vibration vocales,
  tympanisme percusion
\end{itemize}
Radio : pas en expiration ! 3 catégorise : apical, axillaire, complet

Grave si dyspnée sévère ou collapsus tensionnel (! pas déviation médiastin) 

$\implies$ le plus souvent (pneumothorax compressif secondaire à fistule à
soupape) ou (PTX avec réserve ventilatoire réduite)

\paragraph{Atypique}
Récidive : 30\% (premier épisode), 50\% (second épisode)

Rarement avec pneumomédiastin

Sous ventilation mécanique : y penser si $\nearrow$ brutal pression
d'insufflation, collapsus brutal, plaie plèvre

\paragraph{DD}
Facile : douleur thoracique "respiro-dépendante".

Difficile : Dyspnée aigüe sans sd pleural 

\subsection{Traitement}
\begin{itemize}
\item Rien si décollement < 2cm
\item PTX compressif = urgence $\implies$ aiguille simple + immédiat
\item Pneumotthorax spontané primaire (bien ou mal toléré ) :
  exsufflation(petit cathéter sur voie thoracique antérieur) puis drain
  si échec
\item Pour tous les autres : drain
\end{itemize}

Prévention récidive par pleurodèse (accolement des feuillets) : si récidive
homolat, PTX persistant 5 jours, compliqué ou bilat

Sevrage tabac !!

Attention altitude si PNO existant (seulement existant)

VIh: traitement aussi pneumocystose


\section{Gaz du sang}
\label{appendix:gds}
Le pH est déterminé par l'équilibre entre les bicarbonates (\ch{HCO_3-}) et
$PCO_2$ :
\begin{equation}
  pH = K_1 + log\frac{[\ch{HCO_3-]}}{K_2 p_{CO_2}}
\end{equation}
avec $K_1$, $K_2$ constante.

Un déséquilibre sur un terme induit une compensation sur l'autre. Si le
déséquilibre n'est pas compensé, on aboutit à une acidose ou une alcose.

\begin{table}[htpb]
  \centering
  \caption{Gaz du sang artériel}
  \label{tab:gds}
  \begin{tabular}{ll}
    \toprule
    \(PO_2\) & [80, 100] mmHg\\
    \(SaO_2\) & [95, 98] \%\\
    \(PCO_2\) & [35, 45] mmHg\\
    \ch{HCO_3^-} & [22, 29] mmol/L\\
    pH & [7.38, 7.42]\\
    \bottomrule
  \end{tabular}
\end{table}

Interprétation :
\begin{enumerate}
\item Déterminer si acidose ou alcolose selon le pH
\item Respiratoire ou métabolique ? Si $PCO_2$ essaie de compenser (en sens
  inverse du pH), c'est respiratoire. Sinon métabolique.
\end{enumerate}

NB : la compensation ne normalise pas le pH!
NB : 
\begin{itemize}
\item acidose pulmonaire : hypoventilation alvéolaire
\item alcalose pulmonaire : hyperventilation alvéolaire
\item alcalose métabolique : pertes digestives [vomissements]
\end{itemize}

\section{Physiologie}%

$P_{alv}$ = pression alvéolaire, $P_{ip}$ = pression interpleurale (dans la
plèvre), pression transpulmonaire = $P_{ip} - P_{alv}$.

\begin{figure}[htpb]
  \centering
  \caption{Inspiration (gauche), expiration (droite)}
  \tikz \graph [ nodes={align=center}, layered layout]
  {
    Contraction diaphragme -> Expansion thorax -> "$P_{ip} < P_{atm}$"
    -> Hausse pression transpulmonaire -> Expansion poumons 
    -> "$P_{alv} < P_{atm}$"
    -> Arrivée d'air dans les alvéoles;
  };
  \tikz \graph [ nodes={align=center}, layered layout]
  {
    "Arrêt contraction\\ diaphragme et intercostaux" -> Rétraction thorax 
    -> "Valeur initiale de $P_{ip}$"
    -> Valeur initiale pression transpulmonaire 
    -> Rétraction poumons 
    -> "$P_{alv} > P_{atm}$"
    -> Expulsion d'air depuis les alvéoles;
  };

\end{figure}


%%% Local Variables:
%%% mode: latex
%%% TeX-master: t
%%% End:
