\documentclass{article}
\usepackage[hidelinks]{hyperref}
\usepackage{longtable}
\usepackage{booktabs}
\usepackage{graphicx}
\graphicspath{{../../pictures/medecine/}}
% French
\usepackage[T1]{fontenc}
\usepackage[francais]{babel}
%-------------------------------------------------------------------------------
% For graphs
\usepackage{tikz}
\usetikzlibrary{graphs}
\usetikzlibrary{graphdrawing}
\usetikzlibrary{arrows,positioning,decorations.pathreplacing}
\usegdlibrary{trees, layered, force}
\usetikzlibrary{quotes}
\usepackage{biocon}
%-------------------------------------------------------------------------------
% No spacing in itemize
\usepackage{enumitem}
\setlist{nolistsep}
% tightlist from pandoc
\providecommand{\tightlist}{%
  \setlength{\itemsep}{0pt}\setlength{\parskip}{0pt}}
 % Danger symbol (need fourier package)
\newcommand*{\TakeFourierOrnament}[1]{{%
\fontencoding{U}\fontfamily{futs}\selectfont\char#1}}
\newcommand*{\danger}{\TakeFourierOrnament{66}}
% Skull (need Symbola font)
\usepackage{amsmath,fontspec,newunicodechar}
\newfontface{\skullfont}{Symbola}[Scale=MatchUppercase]
\NewDocumentCommand{\skull}{}{%
  \text{\skullfont\symbol{"1F571}}%
}
% Hospital sign
\usepackage{fontspec} % For fontawesome
\usepackage{fontawesome}
% Diameter
\usepackage{wasysym}
% Chemical compound
\usepackage{chemformula}
% Footnote in section
\usepackage[stable]{footmisc}
% Custom header/footers
\usepackage{fancyhdr}
% No numbering
\setcounter{secnumdepth}{0}
% Only sections in TOC
\setcounter{tocdepth}{1}
% Set header
\pagestyle{fancy}
\fancyhf{}
\fancyhead[L]{\leftmark}
\fancyhead[R]{\thepage}
%\renewcommand{\headrulewidth}{0.6pt}
% Custom header : no uper case
\renewcommand{\sectionmark}[1]{%
  \markboth{\textit{#1}}{}}


\title{Pneumologie\\
\large Fiches}
\author{A. Praga}

\begin{document}
\maketitle
\tableofcontents

\input bacteries-header.tex

\section{151 : Infections broncho-pulmonaires communautaires}
\subsection{Bronchite aigue}
Très fréquente, virale 90\%. \\
Diagnostic clinique : épidémie, toux sèche \(\to\) productive, expectorations,
pas de crépitants.\\
Ttt symptomatique seulement !

\subsection{EBPCO} Cf~\nameref{subsec:ebpco}

\subsection{Pneumonie aigue communautaire}
Clinique : \{toux, expect purulentes, dyspnée\} + \{fièvre, asthénie\} +
crépitants\\
Radio !\\
Si \faHospitalSign : hémocultures, ECBC, antigenurie pneumocoque (PCR,
antigenurie légionelle)\\
\faHospitalSign : signes de gravité (score CRB65 \(\ge 1\)) / incertitude / échec domicile / comorbidité / inobservance

\begin{table}[htpb]
  \centering
  \label{Orientation clinique (non discriminant !)}
  \begin{tabular}{llll}
    \toprule
    & Pneumocoque & Atypique & Légionellose\\
    \midrule
    Début & Brutal & Progressif & Rapidement progressif\\
    Signes & Thoraciques & Extra-thoraciques & Extra-thoraciques\\
    Biologie &  & Anémie hémolytique & Hyponatrémie, rhabdomyolyse\\
    & CG+ chainettes/Agurie &  & Antigénurie\\
    Radio \faHeart & condensation systématisée & Opacités multifocales & CS ou OM\\
    \bottomrule
  \end{tabular}
\end{table}

\begin{itemize}
  \item Pneumocoque : fréquent+++. Pas de transmission interhumaine
  \item Atypique : \bact{mpneumoniae}, \bact{cpneumoniae}, \bact{psitacci}
  \item Legionnella : pas d'isolement. DO
  \item Pneumonie virale : signes respi + sd grippal. Grippale = diagnostic PCR, traitement = inihbiteur neuramidase \danger \bact{dore}
\end{itemize}
\subsubsection{Traitement}
Oral, 7j (8-14 si légionellose, 21j si légionelles graves/ID)

Urgent, probabiliste, réévalué 48-72h

Ambulatoire :Amoxicilline ou macrolide \(\to\) switch

\faHospitalSign  Amoxicilline \(\to\) réévaluation

Réa : C3G IV + macrolide IV / FQ pour pneumocoque

\subsubsection{Échecs}
\begin{itemize}
  \item Compliquée : épanchement pleural, abcès, obstacle
  \item Observance, pharmacocinétique, hors spectre
  \item Diagnostic :
    \begin{itemize}
      \item focalisé : embolie pulmonaire
      \item diffuse : cf pneumopathies interstitielles diffuses aigües
      \item excavée : cancer, tuberculose pulmonaires, infarctus pulmonaire, vascularite\ldots{}
    \end{itemize}
\end{itemize}

\subsubsection{Prévention}
Vaccin : 2 doses + rappel enfant, 2 doses adulte (ID, comorbidité)

\subsubsection{Immunodéprimé}
Pneumocystose pulmonaire
Radio : sd interstitiel diffus bilatéral symétrique. ATB cotrimoxazole 21j


\section{151 : Tuberculose}
\bact{tuberculose}
\subsection{Tuberculose maladie}
\subsubsection{Pulmonaire}
Classique : 
\begin{itemize}
\item \{asthénie, anorexie, amaigrissement\} + fébricule (nocturne) + sueurs nocturnes
\item Radio : nodules, infiltrats, cavernes (lobes sup, post.)
\end{itemize}
Miliaire (hématogène) :
\begin{itemize}
\item AEG marqué (fébrile)
\item Radio : interstitiel diffus micronodulaire
\end{itemize}
Autres : pleurésie (épanchement unilat, exsudative lymphocytaire), pneumonie
aigüe, extrapulmonaire
\subsubsection{Extra-pulmonaires}
Ganglionnaire, ostéo-articulaire, génito-urinaire, neuroméningé
\subsection{Diagnostic}
Mises en évidence du bacille :
\begin{itemize}
\item 3 prélèvement (ECBC/tubage). Si miliaire : +hémocultures, ECBU
\item direct, culture, identification et ATBgramme
\item Granulomes épithélioïdes gigantocellulaires avec nécrose caséeuses,
  coloration Ziehl Neelsen (> $10^3$bacilles/mL)
\end{itemize}
\subsection{Traitement}
Quadrithérapie (lutter MDR ou XDR) : 1/jour 1h avant petit déjeuner
\begin{itemize}
\item Isoniazide (polynévrite SM\footnote{sensitivomotrice}, surv. transaminases) : 6 mois
\item Rifampicine (inducteur enzymatique\footnote{contraception !}) : 6 mois
\item Ethambutol (névrite optique rétrobulbaire) : 2 premiers mois
\item Pyrazinamide (surv. transaminases)
\end{itemize}
Autres : aminosides, fluoroquinolones \\
Avant traitement : 
\begin{itemize}
\item hémogramme, \{transaminases, bilirubine, phosphatases alcalines, \$\(\gamma\)\$GT\}, créat,natrémie,uricémie.
\item proposer VIH, VHB, VHC
\item visions des couleurs
\end{itemize}

\subsection{ITL}
Diagnostic (\danger pas pour tuberculose active): 
\begin{itemize}
\item IDR (non spécifique, faux négatifs !) : +72h : \diameter > 10mm (15 si vaccination) ou +10mm en 3 mois : suspicion ITL
\item IGRA\footnote{Interferon Gamma Release Assay} : > 15 ans ou migrant  < 15 ans
\end{itemize}
Traitement : 
\begin{itemize}
\item primaire si < 2 ans, secondaire si < 18 ans ou ID ou ITL < 1 an ou séquellaire
\item INH 9 mois ou INF + RMP 3 mois
\end{itemize}
\subsection{Vaccination}
Sauf VIH, déficits immunitaires
\subsection{Cas particuliers}
Grossesse : INH (+vit. B6) + RMP (+vit. K1) + EMB\\
Insuf. rénale : [15,30]mL/min : diminuer EMB, PZA. si < 15mL/min : diminuer INH, EMB, PZA\\
Insuf. hépatique : [3,6N] : arrêt PZA. Si > 6N, arrêt INH, PZA\\
ID, VIH : penser tuberculose si fièvre\\
Si anti-TNF, bilan tuberculose
\subsection{DO}

\section{205 : BPCO\footnote{Bronchopneumopathie chronique obstructive}}
\label{sec:205-bpco}

\subsection{Définitions}
BPCO = \{toux, dyspnée, expector., infections respi basses\} récurrentes et
TVO\footnote{Trouble ventilatoire obstructif} (VEMS/CVF < 0.7)
\textbf{persistant}\\
Exacerbation aigùe = aggravation \(\ge 2\) jours\\
Bronchite chronique (toux productive quotidienne \(\ge 3\) mois, \(\ge 2\) ans),
emphysème (élargissement espaces aériens distaux + destructions parois
alvéolaires) inclus dans BPCO\\
\begin{table}[htbp]
\caption{Différences asthme-BPCO}
\centering
\begin{tabular}{ll}
\toprule
Asthme & BPCO\\
\midrule
Obstructive non réversible & Obst. réversible\\
Jeune, atopique & Fumeur, > 40 ans\\
Survient \textasciitilde{}40 ans & Enfance\\
\bottomrule
\end{tabular}
\end{table}

Sévérité : 
\begin{itemize}
\item obstruction : GOLD (1 à 4, [80, 100] [50,80], [30,49], [0, 30])
\item dyspnée : MRC (0 à 4)
\item fréquences exacerbations (\$\(\ge\) 2\$/an = grave)
\end{itemize}

Épidémio: en augmentation dans le monde. \\
FR : \{tabac++, aérocontaminants professionnels\}, \(\alpha{}1\) antitrypsine

Évolution : perte fonction respi, exacerbations, handicap respi, risque
d'insuffisance respi, comorbidité CV = 1ere cause de mortalité\\
Score BODE\footnote{Body mass index, Obstruction, Dispnea, Exercice} pour la prédiction.

\subsection{Diagnostic}
Signes fonctionnels : dyspnée++, toux, expectorations.\\
Signes physiques : après TVO : \(\nearrow\) temps expiratoire, \(\searrow\) murmure vésiculaire et bruts coeur, distension thoracique

\subsubsection{EFR}
\begin{itemize}
\item spirométrie (bronchodilat. ou corticoïdes)
\item pléthysmographie (distension pulmonaire)
\item DLCO (patho si < 70\%)
\item si VEMS < 50\% ou $SpO_2< 90$\% : gaz du sang, test d'exercice
\end{itemize}

\subsubsection{Autres}
(TDM), ECG si VEMS < 50\%, NFS (polyglobulie, anémie), (dosage \(\alpha1\) antitrypsine

\subsection{Traitement}
\begin{figure}[htpb]
  \centering
\tikz \graph [
  % Labels at the middle 
  edge quotes mid,
  % Needed for multi-lines
  nodes={align=center},
  sibling distance=3cm,
 level distance=2cm,
  edges={nodes={fill=white}}, 
tree layout]
{
  Dyspnée/exacerbations -> {
    BD longue durée[>"oui", draw]-> 
    {
      2 BD [draw, >"Exacerb"] -> "Cortico + 2 BD"[draw, >"exacerb"];
      "Cortico. inhalé\\
      + $\beta2$ longue durée"[draw, >"dyspnée"]-> "Cortico + 2 BD"[draw,
      >"insuffisant"]
      -> "Réévaluation";
      2 BD [draw, >"Exacerb"] -> "Réévaluation"[>"dyspnée"];
      ;
    };
    BD courte durée [>"non", draw];
  }
};
\caption{Traitement BPCO}
\end{figure}


Arrêt tabac++, vaccins grippe et pneumocoques, réhabilitation respiratoire, oxygénothérapie, chirurgie possibles

\subsection{Exacerbations BPCO}
\label{subsec:ebpco}
\subsubsection{Diagnostic :}
\begin{itemize}
\item BPCO connu : \(\nearrow\) dyspnée, toux/expect.
\item sinon : cf détresse respi
\end{itemize}

Déclenchants : majorité = infectieux mais souvent pas de facteur précis
(\bact{influenzae}, \bact{pneumocoque}, \bact{catarrhalis})

DD : PAC, dysfonction cardiaque gauche, embolie pulmonaires, pneumothorax, médicaments CI, trauma/chir thoracique, insuffisance cardiaque gauche aigüe.

\subsubsection{Explorations}
Imagerie thorax, ECG, NFS, CRP, iono, créat, gazométrie

\subsubsection{Traitement}
Bronchodilatateurs $\beta2$ agonistes courte-durée.\\
ATB : majoration purulence : amox + acide clav si FR, sinon amox-acide
clav/pristinamycine/macrolides\\
Autres \faHospitalSign : oxygénothérapie, kiné, HBPM, (assistance ventilatoire)

\section{206 : Pneumopathies infiltrantes diffuses}
\subsection{Présentation}
Clinique : dyspnée d'effort prgorsessive.\\
EFR : TVR\footnote{trouble ventilatoire restrictif} ( CPT < 80\% et VEMS/CVL >
70\% ) et DLCO < 70\%, hypoxémie, désaturation\\
Radio : opacités parenchymateuse non systématisées bilatérales

\subsection{PID aigüe}
\subsubsection{Étiologies}
Connues : lymphangite carcinomateuse, insuf. cardiaque gauche, médicamenteuse\\
Inconnues : sarcoïdose, fibrose plumonaire idiopathique
\subsubsection{Démarche}
\begin{itemize}
\item Contexte (ATCD, ID, exposition)
\item ECG, BNP, echo cardiaque
\item LBA si possible
\item PEC thérapeutique (réa si détresse respi, \(O_2\), ATB probab si fièvre, arrêt de médic. pneumotoxiques)
\end{itemize}

\subsection{PID subaigüe/chronique}
\subsubsection{Démarches}
Interrogatoire++ : terrain (sarcoïdose=25-45 ans, FPI si > 60 ans), tabac (histiocytose langerhansienne, DIP), toxico, médic, ATCD radio, exposition

Clinique : état général, signes de connectivite


\begin{table}[htbp]
\caption{Biologie PID subaigüe}
\centering
\begin{tabular}{ll}
\toprule
Examen & Maladie\\
\midrule
NFS, CRP & Sd inflammatoire\\
 & Hyperéosinophilie, lymphopénie\\
BNP & Insuf. cardiaque\\
Créat & Insuf. rénale\\
Précipitines sériques & Hypersensib. (si contexte)\\
CEA, calcémie, calciurie & Sarcoïdose\\
Facteur rhumatoïdes etc & Connectivites\\
ANCA & Vascularite\\
Séro VIH & Opportuniste\\
\bottomrule
\end{tabular}
\end{table}

\begin{table}[htbp]
\caption{LBA PID subaigüe}
\centering
\begin{tabular}{ll}
\toprule
Normal & 80\% macrophages\\
 & < 15\% lymphocytes\\
 & < 5\% PNN\\
 & < 2\% PNE\\
\midrule
Alvéolite & Hypercellularité totale\\
Histiocytose langerhansienne & Macrophage\\
Sarcoïdose, PHS & Lymphocytaire\\
P. à éosinophiles & Éosinophilique\\
POC\footnote{Pneuompathie organisée cryptogénique} & Panachées\\
Hémorragie alvéolaire & Rosé\\
Protéinose alvéolaire primitive & Laiteux\\
\bottomrule
\end{tabular}
\end{table}

Examens complémentaires : fibro + LBA > biopsie bronchique > biopsie pulmonaire chir, biopsie transbronchique

\subsubsection{Oedeme pulmonaire}
Mécanisme : Surcharge hémodynamique\\
Clinique : HTA, coronaropathie, valvulopathie mitrale\\
Diagnostic : ECG, BNP, écho coeur\\
Imagerie : Péri-hilaire

\subsubsection{Tuberculose}
Mécanisme : BK\\
Clinique : Contage, AEG, hémoptysie\\
Diagnostic : Expectorations : ED, culture, biopsie transbronchique\\
Imagerie : 
\begin{itemize}
\item pulmonaire = nodules, infiltrats, excavations
\item miliaire = micronodules diffus
\end{itemize}

\subsubsection{Médicaments}
Imagerie : condensations, verre dépoli, épanchement pleural

\subsubsection{Pneumopathies d'hypersensibilité}
Mécanisme : Ag organiques\\
Clinique : 
\begin{itemize}
\item aigüe : sd peudo-grippal quelques heurs
\item subaigüe : semaines/mois avec toux, fébricule, rales crépitants, squeaks
\item chronique : dyspnée, toux sèche
\end{itemize}
Diagnostic : Sérologie, LBA\\
Imagerie : Micronodules centrolobulaires flous, verre dépoli (lobes supérieurs)\\
Traitement : éviction Ag

\subsubsection{Pneumoconioses}
Mécanisme : Amiante, silice\\
Clinique : Exposition\\
Imagerie : 
\begin{itemize}
\item silicose : opacités micronodulaires diffuses \(\implies\) masses pseudotumorales. Peut donner un cancer bronchique
\item asbestose : opacités linéaires non septables des bases \(\parallel\) ou \(\bot\) plèvre, réticulations et rayons de miels comme FPI. Évolue vers insuf respi chronique.
\end{itemize}

\subsubsection{Sarcoïdose}
Mécanisme : Signes extra-respiratoires\\
Diagnostic : anapath : extra-pulmonaire, biopsie EP et TB. Adénopathies médiastinales \\
Imagerie : Nodules, micronodules (ditribution lymphatique), adénopathie, hyperdensités, distorsions bronchiques

\subsubsection{Fibrose pulmonaire idiopathique}
Clinique : Dyspnée d'effort progressive, toux sèche, hippocratisme digital, crépitants sec base\\
EFR : trouble ventilatoire restrictif, diminution DLCO\\
Imagerie : Réticulations, bronchectasies, rayons de miel. Domine sous pleur et bases

\subsubsection{Connectivites}
Mécanisme : Dysimmunitaire\\
Clinique : Extra-respi : polyarthrite rhumatoide, sclérodermie, lupus, vascularite\\
Diagnostic : Ac spécifiques\\
Imagerie : Réticulations, hyperdensités, bronchectasies

\subsubsection{Pneumopathie interstitielle non spécifique}
Origine : connectivite, médicaments (idiopathique)
Imagerie : verre dépoli, réticulations, bronchectasies (sauf extrème périphérie du poumon)

\subsubsection{Proliférations tumorales}
Lymphangite carcinomateuse : toux sèche, rebelle. Radio : épaississements nodulaires des septas intralobulaires. Diagnostic par biopsies des éperons\\
Carcinome lépidique : verre dépoli. 

\appendix



\section{207 Sarcoidose}
Maladie : systémique, cause inconnue, hétérogène, ubiquitaire. Début 25-45ans
dans 2/3\\
Atteinte médiastino-pulmonaire 90\%

\subsection{Expression}
\label{sec:org39048da}
\subsubsection{Pulmonaire}
\label{sec:org4ec1d7e}
Toux (dyspnée)
Radio : 4 stades
\begin{itemize}
\item I : adénopathies hilaires bilatérales symétriques
\item II : + atteinte parenchyme (micronodulaire diffus, parties moyennes supérieures)
\item III : atteinte parenchyme isolée
\item IV : fibrose = opacités parenchymateuses rétractiles + ascension hiles, distorsion bronchovasc (sup et post)
\end{itemize}
TDM : atteinte parenchyme = micronodule selon lymphatiques. Utile pour : formes atypiques ou détection précoce (fibrose, complications [greffe aspergillaire])\\
EFR : sd restrictif, DLCO $\searrow$\\
Endoscopie bronchique (non system) : normal/en "fond d'oeil". Biopsie : \{éperons, LBA\} > \{ponction ganglions médiastinaux, transbronchique\} > médistanoscopie\\
Formes atypiques : TVO, cavitaires, pseudonodulaires/alvéolaires

\subsubsection{Extra-pulmonaire}
\label{sec:org316c190}
\begin{itemize}
\item Oeil : uvéite antérieure aigue (toujours cherche uvétie postérieure)
\item Peau : nodules cutanés, lupus pernio, érythème noueux
\item Adénopathies
\item Foie
\item Moins fréquentes : nerveux (sd méningé, paires craniennes), ORL (obstruction
nasale,
\end{itemize}
sd Mikulicz, sd Heerfordt), ostéo-articulaire (bi-arthrite cheville =
spécfique++), 
coeur (BAV, bloc branche droit), rein (\(\nearrow\) créatininémie)
\begin{itemize}
\item Généraux : asthénie (pas de fièvre sauf sd de Löfgren)
\end{itemize}
Sd de Löfgren = érythème noueux + adénopathies hilaires médiastinales (+ fièvre)
\subsubsection{Biologie}
\label{sec:orgbfe5d87}
\begin{itemize}
\item Hypercalciurie
\item Lymphopénie CD4
\item Hypergammaglobulinémie
\item Enzyme de conversion de l'angiotensine sérique (ECA)
\end{itemize}
\subsection{Diagnostic}
\label{sec:org6670330}
Clinique + radio + lésions granulomateuses tuberculoides sans nécrose caséeuse +
élimination DD
\subsection{Évolution}
\label{sec:org707f9ea}
< 2 ans : évolution favorable sans traitement.\\
Chronique > 2 ans : attention au vital/fonctionnel \\
Suivi : 3-6 mois\\
Pronostic : 80\% favorable sans traitement, 10\% séquelles, 5\% DC

\begin{table}[htbp]
\caption{Pronostic de la sarcoidose}
\centering
\begin{tabular}{ll}
\toprule
Négatif & Positif\\
\midrule
> 40 ans & Érythème noueux\\
Chronicité & Forme aigüe\\
Stade III, IV & Stade 1 asymptomatique\\
Extra-respi grave & \\
\bottomrule
\end{tabular}
\end{table}

Atteintes :
\begin{itemize}
\item pulmonaire : insuf. respir chronique, principace cause DC
\item extra-thoracique : attention fonctionnel/vital
\end{itemize}

\subsection{Traitement}
\label{sec:orgc05b9f6}
Atteinte respi : pas si sd de Löfgren ou stade I asymptomatique\\
Corticoïdes > 10 mois à 0.5mg/kg (décroissance par 6-12 semaines)\\
2eme intention : hydroxychloroquine, méthotreaxe, azathioprine\\
3eme intention : cyclophsamide, anti-TNF-\(\alpha\)
\section{Gaz du sang}
\label{appendix:gds}
Le pH est déterminé par l'équilibre entre les bicarbonates (\ch{HCO_3-}) et
$PCO_2$ :
\begin{equation}
  pH = K_1 + log\frac{[\ch{HCO_3-]}}{K_2 p_{CO_2}}
\end{equation}
avec $K_1$, $K_2$ constante.

Un déséquilibre sur un terme induit une compensation sur l'autre. Si le
déséquilibre n'est pas compensé, on aboutit à une acidose ou une alcose.

\begin{table}[htpb]
  \centering
  \caption{Gaz du sang artériel}
  \label{tab:gds}
  \begin{tabular}{ll}
  \toprule
  \(PO_2\) & [80, 100] mmHg\\
  \(SaO_2\) & [95, 98] \%\\
  \(PCO_2\) & [35, 45] mmHg\\
  \ch{HCO_3^-} & [22, 29] mmol/L\\
  pH & [7.38, 7.24]\\
  \bottomrule
  \end{tabular}
\end{table}

Insuf. respi chonique : 
$$PCO_2 \nearrow \implies \ch{HCO_3-} \nearrow \implies \text{acidose compensée}$$
Hypoventilation alvéolaire : 
$$PCO_2 \nearrow \implies \ch{HCO_3-} \text{ N ou} \nearrow \implies \text{acidose}$$
IR, acidocétose... :
$$ \ch{HCO_3-} \searrow \implies PCO_2 \text{ N ou} \searrow \implies \text{acidose}$$
Vomissements, sonde NG :
$$ \ch{HCO_3-} \searrow \implies PCO_2 \text{ N ou} \nearrow \implies \text{alcalose}$$
Hyperventilation alvéolaire :
$$ PCO_2 \searrow \implies \ch{HCO_3-} \text{ N ou} \searrow \implies \text{alcalose}$$

\section{Physiologie}%

$P_{alv}$ = pression alvéolaire, $P_{ip}$ = pression interpleurale (dans la
plèvre), pression transpulmonaire = $P_{ip} - P_{alv}$.

\begin{figure}[htpb]
  \centering
  \caption{Inspiration (gauche), expiration (droite)}
  \tikz \graph [ nodes={align=center}, layered layout]
  {
    Contraction diaphragme -> Expansion thorax -> "$P_{ip} < P_{atm}$"
    -> Hausse pression transpulmonaire -> Expansion poumons 
    -> "$P_{alv} < P_{atm}$"
    -> Arrivée d'air dans les alvéoles;
  };
  \tikz \graph [ nodes={align=center}, layered layout]
  {
    "Arrêt contraction\\ diaphragme et intercostaux" -> Rétraction thorax 
    -> "Valeur initiale de $P_{ip}$"
    -> Valeur initiale pression transpulmonaire 
    -> Rétraction poumons 
    -> "$P_{alv} > P_{atm}$"
    -> Expulsion d'air depuis les alvéoles;
  };

\end{figure}
\end{document}
