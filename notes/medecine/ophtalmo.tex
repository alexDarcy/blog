% Created 2018-11-25 Sun 16:44
% Intended LaTeX compiler: lualatex
\documentclass[11pt]{article}
\usepackage[hidelinks]{hyperref}
\usepackage{booktabs}
\documentclass{article}
\usepackage[hidelinks]{hyperref}
\usepackage{longtable}
\usepackage{booktabs}
%\usepackage[draft]{graphicx}
\usepackage{graphicx}
\usepackage{fancyhdr}
% French
\usepackage[T1]{fontenc}
\usepackage[francais]{babel}
\usepackage{caption}
\usepackage[nointegrals]{wasysym} % Male-female symbol
% Smaller marign
\usepackage[margin=2.5cm]{geometry}
\usepackage{latexsym}
\usepackage{subcaption}
%-------------------------------------------------------------------------------
% For graphs
\usepackage{tikz}
\usepackage{tikzscale}
\usetikzlibrary{graphs}
\usetikzlibrary{graphdrawing}
\usetikzlibrary{arrows,positioning,decorations.pathreplacing}
\usetikzlibrary{calc}
\usegdlibrary{trees, layered}
\usetikzlibrary{quotes}
%-------------------------------------------------------------------------------
% No spacing in itemize
\usepackage{enumitem}
\setlist{nolistsep}
% tightlist from pandoc
\providecommand{\tightlist}{%
  \setlength{\itemsep}{0pt}\setlength{\parskip}{0pt}}
 % Danger symbol (need fourier package)
\newcommand*{\TakeFourierOrnament}[1]{{%
\fontencoding{U}\fontfamily{futs}\selectfont\char#1}}
\newcommand*{\danger}{\TakeFourierOrnament{66}}
% Skull (need Symbola font)
\usepackage{amsmath,fontspec,newunicodechar}
\newfontface{\skullfont}{Symbola}[Scale=MatchUppercase]
\NewDocumentCommand{\skull}{}{%
  \text{\skullfont\symbol{"1F571}}%
}
% Hospital sign
\usepackage{fontspec} % For fontawesome
\usepackage{fontawesome}
% Itemize in tabular
\newcommand{\tabitem}{~~\llap{\textbullet}~~}
% No numbering
\setcounter{secnumdepth}{0}
% Set header
\pagestyle{fancy}
\fancyhf{}
\fancyhead[L]{\leftmark}
\fancyhead[R]{\thepage}
%\renewcommand{\headrulewidth}{0.6pt}
% Custom header : no uper case
\renewcommand{\sectionmark}[1]{%
  \markboth{\textit{#1}}{}}

\usepackage[linesnumbered,ruled,vlined]{algorithm2e}
\author{Alexis Praga}
\date{\today}
\title{Ophtalmologie}
\hypersetup{
 pdfauthor={Alexis Praga},
 pdftitle={Ophtalmologie},
 pdfkeywords={},
 pdfsubject={},
 pdfcreator={Emacs 26.1 (Org mode 9.1.9)}, 
 pdflang={English}}
\begin{document}

\maketitle
\tableofcontents

\input{bacteries-header}
\section{Sémiologie oculaire}
\label{sec:org87d343e}
\subsection{Globe}
\label{sec:orgaa7159a}
Membranes (d'avant en arrière)
\begin{itemize}
\item externe = \{cornée (avant), conjonctive (sur sclère antérieure), sclère\}
\item intermédiaire (uvée) = \{iris, corps ciliaires, choroïde\}
\item interne (rétine) = \{épithelium pigmentaire, neurosensorielle (photorécepteurs,
fibres optiques\}\}. NB : phototransduction = pigment visuel
(épithelium) \(\rightarrow\) photorécepteurs (bâtonnets = \{vision périphérique,
nocturne\}, cône = \{détails, couleurs\} \(\in\) macula)
\end{itemize}
Contenu : 
\begin{itemize}
\item humeur aqueuse \(\in\) chambre antérieure, évacuée dans l'angle iridocornéen. P
normale si \(\le\) 22 mmHg.
\item cristallin : biconvexe, convergente, déformation par zonule \thus accomodation
\item corps vitré : 4/5e cavité oculaire \thus segment antérieur vs postérieur
\end{itemize}
Voies optiques : 
\begin{itemize}
\item nerf optique - chiasma - bandelettes optiques - corps genouillés
\end{itemize}
externes - radiations optiques - cortex occipital
\begin{itemize}
\item mydriase "sensorielle"\footnote{Plus de voie afférente  \thus RPM (réflexe photomoteur) direct et
consensuel aboli pour l'oeil atteint} vs "paralytique"\footnote{Plus de voie efférente \thus oeil atteint : RPM direct aboli. Oeil sain : RPM consensuel aboli}
\item voie efférente sympathique : si atteinte, sd Claude-Bernard-Horner (myosis, ptosis)
\end{itemize}
Annexes
\begin{itemize}
\item oculomoteur : nerf IV = \{oblique supérieur\}, VI = \{droit externe\} et III =
\{droit sup, droit inf, droit médial, oblique inf\}
\item protection : paupière (muscle orbiculaire, tarse), conjonctive (face interne
paupière, globe antérieur), film lacrymal
\end{itemize}

\subsection{Examen}
\label{sec:orgcb05282}
Interrogatoire : 
\begin{itemize}
\item baisse d'acuité visuelle (BAV), fatigue visuelle, myodésopsise,
\end{itemize}
métamorphopsies, héméralopie, scotome/amputation champ visuel, 
\begin{itemize}
\item mode d'installation : urgente si brutal
\item douleur superficielles/profondes
\item diplopie mono/bi-noculaire
\item évolution : aggration rapide = grave
\end{itemize}
Acuité visuelle : échelle de Monoyer (à 5m), de Parinaud (33cm)

Segment antérieur : "lampe à fente"
\begin{itemize}
\item conjonctive : rougeur (diffuse/localisée/culs-de-sac inférieure/avec
sécrétions/cercle périkératique/paupière), chemosis\footnote{Oedème conjonctival}
\item cornée (transparence)
\item iris : myosis/mydriase
\item chambre antérieure : inflammation (Tyndall\footnote{Cellules inflammatoires, protéines dans l'humeur aqueuse}, précipités rétro-cornéens,
synéchies iridocristaliniennes), hypopion (pus), hyphéma (sang)
\end{itemize}
Pression intraoculaire par tonomètre à air pulsé. Hypertonie si \(\ge\) 22 mmHg

Gonioscopie : angle iridocornéen

Fond d'oeil
\begin{itemize}
\item direct, indirect (apprentissage++), biomicroscopie
\item aspect : pôle postérieur = \{papille, vaisseaux rétitiens, macula\}, rétine
périphérique (si besoin)
\item lésions : microanévrisme (points rouges), hémorragie (intravitréennes,
prérétiniennes, sous-rétiniennes, intra-rétiniennes)
\end{itemize}
\end{document}
