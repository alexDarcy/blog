% Created 2019-01-10 Thu 21:32
% Intended LaTeX compiler: lualatex
\documentclass[11pt]{article}
\usepackage[hidelinks]{hyperref}
\usepackage{booktabs}
\documentclass{article}
\usepackage[hidelinks]{hyperref}
\usepackage{longtable}
\usepackage{booktabs}
%\usepackage[draft]{graphicx}
\usepackage{graphicx}
\usepackage{fancyhdr}
% French
\usepackage[T1]{fontenc}
\usepackage[francais]{babel}
\usepackage{caption}
\usepackage[nointegrals]{wasysym} % Male-female symbol
% Smaller marign
\usepackage[margin=2.5cm]{geometry}
\usepackage{latexsym}
\usepackage{subcaption}
%-------------------------------------------------------------------------------
% For graphs
\usepackage{tikz}
\usepackage{tikzscale}
\usetikzlibrary{graphs}
\usetikzlibrary{graphdrawing}
\usetikzlibrary{arrows,positioning,decorations.pathreplacing}
\usetikzlibrary{calc}
\usegdlibrary{trees, layered}
\usetikzlibrary{quotes}
%-------------------------------------------------------------------------------
% No spacing in itemize
\usepackage{enumitem}
\setlist{nolistsep}
% tightlist from pandoc
\providecommand{\tightlist}{%
  \setlength{\itemsep}{0pt}\setlength{\parskip}{0pt}}
 % Danger symbol (need fourier package)
\newcommand*{\TakeFourierOrnament}[1]{{%
\fontencoding{U}\fontfamily{futs}\selectfont\char#1}}
\newcommand*{\danger}{\TakeFourierOrnament{66}}
% Skull (need Symbola font)
\usepackage{amsmath,fontspec,newunicodechar}
\newfontface{\skullfont}{Symbola}[Scale=MatchUppercase]
\NewDocumentCommand{\skull}{}{%
  \text{\skullfont\symbol{"1F571}}%
}
% Hospital sign
\usepackage{fontspec} % For fontawesome
\usepackage{fontawesome}
% Itemize in tabular
\newcommand{\tabitem}{~~\llap{\textbullet}~~}
% No numbering
\setcounter{secnumdepth}{0}
% In TOC, only section
\setcounter{tocdepth}{1}
% Set header
\pagestyle{fancy}
\fancyhf{}
\fancyhead[L]{\leftmark}
\fancyhead[R]{\thepage}
%\renewcommand{\headrulewidth}{0.6pt}
% Custom header : no uper case
\renewcommand{\sectionmark}[1]{%
  \markboth{\textit{#1}}{}}
% Footnote in section
\usepackage[stable]{footmisc}
% Chemical compound
\usepackage{chemformula}

% Negate \implies
\usepackage{centernot} 

%-------------------------------------------------------------------------------
% Custom commands
%-------------------------------------------------------------------------------
% Logical and, or
\def\land{$\wedge{}$}
\def\lor{$\vee{}$}
\def\dec{$\searrow{}$}
\def\inc{$\nearrow{}$}


\usepackage[linesnumbered,ruled,vlined]{algorithm2e}
\usepackage[acronym, nonumberlist]{glossaries}
\makeglossaries
\newacronym{RPM}{RPM}{Réflexe photomoteur}
\newacronym{DMLA}{DMLA}{Dégénérescence maculaire liée à l'âge}
\newacronym{PEV}{PEV}{Potentiels évoqués visuels}
\def\ttt{\hspace*{1cm}Ttt: }
\newacronym{NOIA}{NOIA}{Neuropathie optique ischémique antérieure}
\author{Alexis Praga}
\date{\today}
\title{Ophtalmologie}
\hypersetup{
 pdfauthor={Alexis Praga},
 pdftitle={Ophtalmologie},
 pdfkeywords={},
 pdfsubject={},
 pdfcreator={Emacs 26.1 (Org mode 9.1.9)}, 
 pdflang={English}}
\begin{document}

\maketitle
\tableofcontents

% Printing bacteria with biocon package
\newbact{chlamydia}{genus=Chlamydia, epithet=trachomatis}
\newbact{botulisme}{genus=Clostridium, epithet=botulinum}
\newbact{burnetii}{genus=Coxiella, epithet=burnetii}
\newbact{charbon}{genus=Bacillus, epithet=anthracis}
%
\newbact{tranchees}{genus=Bartonella, epithet=quintana}
%
\newbact{recurrente}{genus=Borreila, epithet=recurrentis}
\newbact{ecoli}{genus=Escherichia, epithet=coli}
%
\newbact{faecalis}{genus=Enteroccocus, epithet=faecalis}
%
\newbact{gardnerella}{genus=Gardnerella, epithet=vaginalis}
\newbact{ducreyi}{genus=Haemophilus, epithet=ducreyi}
\newbact{influenzae}{genus=Haemophilus, epithet=influenzae}
\newbact{granulomatis}{genus=Klebsiella, epithet=granulomatis}
\newbact{aeruginosa}{genus=Pseudomonas, epithet=aeruginosa}
\newbact{tuberculose}{genus=Mycobacterium, epithet=tuberculosis}
\newbact{genitalium}{genus=Mycoplasma, epithet=genitalium}
\newbact{gonocoque}{genus=Neisseria, epithet=gonorrhoeae}
\newbact{typhus}{genus=Rickettsia, epithet=prowazekii}
\newbact{conorii}{genus=Rickettsia, epithet=conorii}
%
\newbact{dore}{genus=Staphylococcus, epithet=aureus}
\newbact{gallolyticus}{genus=Staphylococcus, epithet=gallolyticus}
\newbact{saprophyte}{genus=Staphylococcus, epithet=saprophyticus}
%
\newbact{pneumocoque}{genus=Streptococcus, epithet=pneumoniae}
\newbact{pyogenes}{genus=Streptococcus, epithet=pyogenes}
\newbact{toxoplasmose}{genus=Toxoplasma, epithet=gondii}
\newbact{syphilis}{genus=Treponema, epithet=pallidum}
\newbact{trichomonose}{genus=Trichomonas, epithet=vaginalis}
%-------------------------------------------------------------------------------
%% Parasits
%-------------------------------------------------------------------------------
\newbact{saginata}{genus=Taenia, epithet=saginata}
\newbact{solium}{genus=Taenia, epithet=solium}

\section{1 Sémiologie oculaire}
\label{sec:org746d448}
\subsection{Globe}
\label{sec:org6898fce}
Membranes (d'avant en arrière)
\begin{itemize}
\item externe = \{cornée (avant), conjonctive (sur sclère antérieure), sclère\}
\item intermédiaire (uvée) = \{iris, corps ciliaires, choroïde\}
\item interne (rétine) = \{épithelium pigmentaire, neurosensorielle (photorécepteurs,
fibres optiques\}\}. NB : phototransduction = pigment visuel
(épithelium) \(\rightarrow\) photorécepteurs (bâtonnets = \{vision périphérique,
nocturne\}, cône = \{détails, couleurs\} \(\in\) macula)
\end{itemize}
Contenu : 
\begin{itemize}
\item humeur aqueuse \(\in\) chambre antérieure, évacuée dans l'angle iridocornéen. P
normale si \(\le\) 22 mmHg.
\item cristallin : biconvexe, convergente, déformation par zonule \thus accomodation
\item corps vitré : 4/5e cavité oculaire \thus segment antérieur vs postérieur
\end{itemize}
Voies optiques : 
\begin{itemize}
\item nerf optique - chiasma - bandelettes optiques - corps genouillés externes - radiations optiques - cortex occipital
\item mydriase "sensorielle"\footnote{Plus de voie afférente  \thus \gls{RPM} direct et consensuel aboli pour l'\oe{}il atteint} vs "paralytique"\footnote{Plus de voie efférente \thus \oe{}il atteint : RPM direct aboli. \OE{}il sain : RPM consensuel aboli}
\item voie efférente sympathique : si atteinte, sd Claude-Bernard-Horner (myosis, ptosis)
\end{itemize}
Annexes
\begin{itemize}
\item oculomoteur : nerf IV = \{oblique supérieur\}, VI = \{droit externe\} et III =
\{droit sup, droit inf, droit médial, oblique inf\}
\item protection : paupière (muscle orbiculaire, tarse), conjonctive (face interne
paupière, globe antérieur), film lacrymal
\end{itemize}

\subsection{Examen}
\label{sec:org0f8baf4}
Interrogatoire : 
\begin{itemize}
\item baisse d'acuité visuelle (BAV), fatigue visuelle, myodésopsies, métamorphopsies, héméralopie, scotome/amputation champ visuel,
\item mode d'installation : urgence si brutal \skull
\item douleur superficielles/profondes
\item diplopie mono/bi-noculaire
\item évolution : aggration rapide = grave
\end{itemize}
Acuité visuelle : échelle de Monoyer (à 5m), de Parinaud (33cm)

Segment antérieur : "lampe à fente"
\begin{itemize}
\item conjonctive : rougeur (diffuse/localisée/culs-de-sac inférieure/avec
sécrétions/cercle périkératique/paupière), chemosis\footnote{\OE{}dème conjonctival}
\item cornée (transparence)
\item iris : myosis/mydriase
\item chambre antérieure : inflammation (Tyndall\footnote{Cellules inflammatoires, protéines dans l'humeur aqueuse}, précipités rétro-cornéens,
synéchies iridocristaliniennes), hypopion (pus), hyphéma (sang)
\end{itemize}
Pression intraoculaire par tonomètre à air pulsé. Hypertonie si \(\ge\) 22 mmHg

Gonioscopie : angle iridocornéen

Fond d'\oe{}il
\begin{itemize}
\item direct, indirect (apprentissage++), biomicroscopie
\item aspect : pôle postérieur = \{papille, vaisseaux rétitiens, macula\}, rétine
périphérique (si besoin)
\item lésions : microanévrisme (points rouges), hémorragie (intravitréennes,
prérétiniennes, sous-rétiniennes, intra-rétiniennes\footnote{Punctiformes, flammèches, profondes}), nodules cotonneux,
exsudats profonds, \oe{}dème papillaire\footnote{Si BAV, vasculaire probable. Sinon HTIC}
\end{itemize}

Oculomotricité

\subsection{Complémentaires}
\label{sec:orge0e2dac}
Fonctions visuelles
\begin{itemize}
\item Champ visuel :
\begin{itemize}
\item sensibilité lumineuse \dec périphérie, papille = zone aveugle
\item périmétrie cinétique (Goldman : trace isoptères) ou statique (dépistage glaucome)
\end{itemize}
\item Vision des couleurs : si dépistage anomalie congénitale (planches colorées) ou
\end{itemize}
affection acquise (Farnsworth = classer couleurs). Utile pour antipaludéens de
synthèse ou \{éthambutal, isoniazide\}

Angiographie du fond d'\oe{}ils : fluorescine ou vert d'indocyanine (\gls{DMLA}++)

Électrophysio : électrorétinogramme (si lésion rétiniennes étendues), \gls{PEV} (cortex occipital) (SEP++), électro-oculogramme (épithelium pigmentaire)

Echographie : mode A (longueur globe oculaire) ou B (décollement rétine, corps
étranger intraoculaire, tumeur)

Tomographie en cohérence optique (OCT) : affection maculaire (trou, DMLA),
dépistage glaucome chronique

\section{2 Réfraction}
\label{sec:org1018c7d}
\OE{}il \(\approx\) 60 dioptries (cornée = 42, cristallin = 20)
Emmétrope = \oe{}il normal. Amétrope = anomalie de réfraction

Punctum remotum (PR) : point le plus éloigné visible \emph{sans} accomoder. Punctum
proximum (PP) = point le plus proche visible \emph{en} accomodant.

Acuité visuelle = \(\frac{1}{\alpha}\) où \(\alpha\) = pouvoir séparateur de
l'\oe{}il. Mesurée par l'échelle de Monoyer (de loin) en 10eme (!) et de Parinaud
(de près)

\subsection{Accomodation}
\label{sec:org577813d}
Amplitude \dec jusque 0 (79 ans) \thus presbytie (BAV vision de près). Compensée
par verres sphériques convexes progressifs ou implant cristallinien

Sinon, \dec vision de près par : médicaments, paralysie III, maladie générale,
spasmes de l'accomodation

\subsection{Anomalies de la réfraction}
\label{sec:org8648e24}
Différencier maladie de l'\oe{}il/voie optique et anomalie réfraction !

Examen : réfractomètres automatiques \thus réfraction, kératométrie\footnote{Courbure de la cornée}. \footnote{Chez l'enfant, cycloplégique}

\subsubsection{Myopie}
\label{sec:orgf3dd81b}
\OE{}il trop convergent. \(\approx\) 20\% population occidentale. PR à distance finie,
PP plus proche.

Types : myopie d'indice (\inc indice de réfraction), de courbure (courbure
cornée \inc) ou axile (longueur axiale \inc)

Myopie
\begin{itemize}
\item faible : < 6 dioptrie
\item forte : > 6 dioptries ou longueur axiale \(\ge\) 26mm. Héréditaire, \(\in\) [1/10,
5/10] \emph{après} correction. Complications : glaucome chronique à angle ouvert,
cataracte, décollement de la rétine++
\end{itemize}

Correction : verres sphériques concaves.

Chirurgie par photoablation au laser excited dimer\footnote{Abrasion épithelium cornéen ou volet dans cornée.}. Chirurgie du cristallin
possible.

\subsubsection{Hypermétropie}
\label{sec:org0e37682}
Fréquent (enfant++). Pas assez convergent. PR = virtuel à l'arrière. Correction
par verres sphériques convexes, lentilles ou chir.

\subsubsection{Astigmatisme}
\label{sec:org473e40c}
Rayons de courbures différents pour les méridiens. Régulier si 2 méridiens
principaux \bot{}.

1 point à l'infini = 2 droite perpendiculaire (= focales) \thus myopique,
hypermétropique ou mixtes\footnote{Focales en avant/arrière/des 2 côtes de la rétine resp.}

Correction par verres cylindriques convexes/concaves, lentilles ou chir

\section{3 Suivi d'un nourisson}
\label{sec:org8b9a17a}
Déficits mineurs (amétropie, strabisme) ou sévères (grave !) (milieux transparents,
  malformation, rétinopathie, atteintes neuro centrales)

20\% d'enfant < 6 ans avec anomalie visuelle. Si non traitée, amblyopie\footnote{BAV par altération précoce de l'expérience visuelle.} ,
définitive après 6ans !

Développement :
\begin{itemize}
\item 1ere semaine : réflexe lumière, RPM
\item 2-4e semaine : reflexe de poursuite
\item 4-12e semaine : reflexe de poursuite
\item 3e mois : vision des formes
\item 4-5e mois : coordinatio \oe{}il-tête-main
\item > 2 ans : AV mesurable
\end{itemize}
Examens obligatoires pour \{strabisme, nystagmus, anomalie organique, trouble
comportement visuel\}: \emph{dépistage anténal - S1 - 4M - 9M - 2A - 3 à 6A}

Dépistage : leucocorie, glaucome congénital, malformations, infections
maternelles, maladies enfants secoués, rétinopathie des prématurés
\section{4 Strabisme de l'enfant}
\label{sec:org8a9c9a4}

Position de l'\oe{}il anormale et altération vision binoculaire. Due à une perturbation de la fusion.

Conséquences :
\begin{itemize}
\item Si aigu : diplopie possible. Si ancien : corrigé par cerveau \thus vision binoculaire non acquise si strabisme dans premiers mois de vie !. Amblyopie possible
\item Perturbation vision stéréoscopie (3D)
\end{itemize}

Souvent dans l'enfance. 4\% population. \textbf{Dépistage avant 2 ans}

\subsection{Dépistage}
\label{sec:org2dfb41a}
Jamais normal, toujours symptôme

Interrogatoire :
\begin{itemize}
\item date d'apparition
\item horizontal : \emph{eso-} si convergent, \emph{exo-} si divergent. Vertical : \emph{hyper-},
\emph{hypo-}. Si divergent < 9 mois, examen neuroradio
\item intermittent ?
\item \oe{}il dominé
\end{itemize}

Examen : 
\begin{itemize}
\item motilité : strabisme paralytique ?
\item segment antérieur et FO (fond d'\oe{}il) : perte transparence, patho rétinienne ? Si nystagmus : électrorétinogramme, PEV, IRM
\item mesure de réfraction sous collyre cycloplégique : amyotropie ? Hypermétropie fréquente
\item acuité visuelle : amblyopie ? (> 2/10 entre 2 yeux)
\item mesure de l'angle de déviation (si chir), vision binoculaire (pronostique)
\end{itemize}

\subsection{Traitement}
\label{sec:org24501a7}
Correction optique. Si amblyopie, occlusion de l'\oe{}il dominant (jusque 6-8ans)

Chir si angle résiduel avec correction. Correction optique après opération
\section{5 Diplopie (binoculaire)}
\label{sec:orgf89d5b7}
Binoculaire : disparaît à l'occlusion d'un \oe{}il\footnote{Monoculaire : cause cornéenne, irienne, cristalinienne \(\ne\) urgence.}. Souvent une urgence
\danger

Noyaux des nefs oculomoteurs \(\in\) tronc cérébral - racine - troncs -
muscle. Voies supranucléaires (latéralité, verticalité), internucléraires.

\begin{table}[htbp]
\caption{Champ d'action (\danger \(\ne\) action) avec l'innervation}
\centering
\begin{tabular}{ll}
\toprule
oblique inférieur (III) & droit supérieur (III)\\
\midrule
droit médial (III) & droit latéral (VI)\\
\midrule
oblique supérieur (IV) & droit inférieur (III)\\
\bottomrule
\end{tabular}
\end{table}

Mouvement bilatéraux : synergie des muscles

Vision binoculaire :
\begin{itemize}
\item loi de Hering : même influx nerveux pour muscles antagonistes. Loi de
Sherrington : muscle antagonistes se relâchent quand muscles synergistes se
contractent.
\item si correspondance rétinienne anormale (yeux non \(\parallel\)) : diplopie
\end{itemize}

\subsection{Diagnostic}
\label{sec:org1ca9f17}
\begin{itemize}
\item Signes fonctionnels : dédoublement toujours même direction\footnote{\danger méconnu si ptosis/\oe{}dème palpébral}, disparaît à
\end{itemize}
l'occlusion
\begin{itemize}
\item Interrogatoire : terrain, circonstance, brutal/progressif, \{douleurs, vertiges,
\end{itemize}
céphalées, nausées\}, \{horizontale, verticale, oblique\}, moment journée
\begin{itemize}
\item Attitude vicieuse ? Chercher déviation en position primaire par reflets
cornéens
\end{itemize}

Examens :
\begin{itemize}
\item motilité
\item cover-test (\oe{}il dévié puis se redresse)
\item au verre rouge (dissociation point rouge et blanc)
\item test de Lancaster (superposer flèches de couleurs différentes) \thus diagnostic
paralysie oculomotrie
\item RPM, inégalité pupillaire
\end{itemize}

\subsection{Sémiologie}
\label{sec:org5fa70c5}
\begin{itemize}
\item Paralysie du III : totale (ptosis total, mydriase aréflective, 0 accomodation) ou partielle
\item Paralysie du IV : diplopie verticale oblique
\item Paralysie du VI : convergence \oe{}il atteint, déficit abduction
\item Formes particulières :
\begin{itemize}
\item paralysie supranucléaire : sd Foville (latéralité), sd Parinaud (verticalité
et cv \thus pinéalome++)
\item paralysie intranucléaire : ophtalmoplégie intranucléaire (parallélisme OK mais
déficit adduction) \thus SEP
\item paralysie intraxiale : (atteinte fonction et diplopie) ou (diplopie et signe neuro
controlatéraux)
\end{itemize}
\end{itemize}
\subsection{DD}
\label{sec:org01f00b0}
Diplopie monoculaire, simulation, hystérie

\subsection{Étiologie}
\label{sec:orgc4bdfdf}
\begin{itemize}
\item Traumatique : fracture du plancher de l'orbite (élévation globe douleureuse),
hémorragie méningé traumatique
\item Tumeurs : HTIC, de la base du crâne
\item Vasculaires : AVC, insuf vertébrobasilaire, \textbf{anévrisme
intrâcranien} (Y penser si atteinte partielle, signes
pupillaires, sujet jeune, 0 FR vasc, céphalée \thus angioscanner urgence
\skull), fistule carotidocaverneuse
\item Avec exophtalmie : Basedow, tumeurs de l'orbite
\item douleureuse : penser \{anévrisme intracrânnien, dissection carotidienne, fistule
carotidocaverneuse\} = urgences \skull. Maladie de Horton. Sd Tolosa-Hunt
\item SEP : paralysie VI, ophtalmoplégie internucléaire
\item Myasthénie : diag = \{test Prostigmine, Ac anti récepteur acétylcholine, électromyographie\}
\end{itemize}
\section{6 \OE{}il rouge/douloureux}
\label{sec:org690defa}
\subsection{Examen}
\label{sec:org0dc3a9e}
Interrogatoire : mode d'apparition, douleur superficielle/profonde, BAV
   ?, ATCD ophtalmo et généraux, signes locaux associés

Examen à lampe à fente (bilatéral) :
\begin{itemize}
\item acuité visuelle
\item conjonctive : rougeur en nappe hémorragique (hémorragie
sous-conjonctivale\footnote{Cherche plaie conjonctivale !}, diffuse (conjonctivite), secteur (épisclérite),
cercle (kératite aigüe, uvéite antérieure)
périkératique
\item cornée : perte de transparence, dépôts rétro-cornéens
\item collyre à fluorescine pour ulcération cornéenne : unique (trauma), localisée
avec zone blanche (kératite bactérienne), dendritique (kératite herpétique),
petites et disséminées (kératite à adénoviruse, sd sec oculaire, corps
étranger)
\item iris et pupille : synéchie iridocristallinienne (uvéite), atrophie iris
(herpès), myosis (kératite aigüe, uvéite aigüe), semi-mydriase
aréflexique (glaucome aigü)
\item chambre antérieure : étroite, plate (glaucome aigü, plaie perforante), signes inflammatoires
\item tonus oculaire : hypertonie (glaucome aigu par fermeture de l'angle, glaucome
néovasculaire), hypotonie (plaie oculaire transfixiante)
\item conjonctive palpébrale : follicules (conjonctivite virale), papilles
(conjonctivite allergique), corps étranger
\item FO
\end{itemize}

\subsection{Étiologies}
\label{sec:orga4d921f}

\begin{table}[htbp]
\caption{Étiologies résumées d'\oe{}il rouge}
\centering
\begin{tabular}{ll}
\toprule
Non douloureux sans BAV & Douloureux avec BAV\\
\midrule
hémorragie ss-conjonctivale & kératite aigüe\\
conjonctivite & uvéite antérieure aigüe sclérite\\
(épi)sclérite & crise aigüe de glaucome par fermeture d'angle\\
 & glaucome néovasculaire\\
\bottomrule
\end{tabular}
\end{table}

\subsubsection{\OE{}il rouge, non douloureux, sans BAV}
\label{sec:orgea00195}
\begin{itemize}
\item \emph{hémorragie sous-conjonctivale spontanée} : chercher HTA, trouble
coagulation. Penser corps étranger, plaie sclérale \danger
\item \emph{conjonctivite} : sensation de "grain de sable", prurit
\item \emph{conjonctivite bactérienne} : sécrétions mucopurulentes (paupières collées le
matin). Germe Gram+.
\end{itemize}
\ttt hygiène des mains, lavage sérum phy, collyre antiseptique (pas forcément ATB !!)

\OE{}il rouge unilatéral, douloureux, sans BAV
\begin{itemize}
\item \emph{épisclérite} (sous conjonctive) : rougeur disparaissant avec collyre
vasoconstricteur.
\end{itemize}
\ttt corticothérapie locale
\begin{itemize}
\item \emph{sclérite} : douleur \inc mobilisation du globe. Ne disparait pas au
collyre. Cherche maladie de système (articulaire, vasc, granulomateuse,
infectieuse).
\end{itemize}
\ttt AINS générale et cause.

\subsubsection{Yeux rouges bilatéraux, douloureux, sans BAV}
\label{sec:orga290dcd}
\begin{itemize}
\item \emph{conjonctivite virale} : fréq++, contagieux. Sécrétion claires, ADP prétragienne
douloureuses à palpation.
\end{itemize}
\ttt inutile
\begin{itemize}
\item \emph{conjonctivite allergique} : terrain, prurit, chemosis\footnote{\OE{}dème conjonctival}, sécrétion claire,
volumineuses papilles.
\end{itemize}
\ttt bilan allergique, éviction, collyre antiallergique
\begin{itemize}
\item \emph{conjonctivite à \bact{chlamydia}} : tiers-monde++
\item \emph{sd sec oculaire} : fréq++. Diag = test Schirmer (quantité sécrétion lacrymale),
qualité film lacrymal, surfarce cornéenne, surface conjonctivale\}. Cause :
involution (âge), sd Gougerot-Sjögren.
\end{itemize}
\ttt substituts lacrymaux, évictions
  facteurs irritants, occlusion points lacrymaux
\begin{itemize}
\item autres : Basedow, malpositions palpébrale, conjonctivite d'irritation
\end{itemize}

\subsubsection{\OE{}il rouge, douloureux, BAV}
\label{sec:org238cec7}
\begin{itemize}
\item \emph{Kératite aigüe} : BAV, douleurs
superficielles importantes, larmoiement, photophobie,
blépharospasme. Érosion/ulcérationsc cornée, \dec transparence cornée,
cercle périkératique
\begin{itemize}
\item \emph{kératite à adénovirus} : petite ulcérations disséminées. Évolution favorable
(?)
\item \emph{kératite herpétique} : ulcération en "feuille de fougère".
\end{itemize}
\end{itemize}
\ttt
    valaciclovir \textpm{} aciclovir en pommade. Jamais de corticothérapie locale
    sans avoir éliminé une ulcération cornéenne \skull
\begin{itemize}
\item \emph{kératite zostérienne} : zona ophtalmique \thus (?) kératite superficielle ou
neuroparalytique (grave).
\end{itemize}
\ttt valaciclovir et protecteurs locaux
    cornéens
\begin{itemize}
\item \emph{kératite bactérienne, parasitaire, mycosique} : plage blanchâtre. Prélèver
sur l'abcès.
\end{itemize}
\ttt collyre ATB (si important : "collyre fortifiés"). \danger
    évolution : endophtalmie, perforation cornéenne, taie cornéenne cicatricielle
\begin{itemize}
\item \emph{kératite sur sd sec}
\item \emph{kératite d'exposition} (paralysie faciale) :
\end{itemize}
\ttt protecteurs cornéens en prévention, tarsorraphie (=suture)
\begin{itemize}
\item \emph{Uvéite antérieure} : cercle périkératique, transparence cornée OK, myosis,
(synéchies iridocristalliniennes ou iridocornéennes), Tyndall, précipités
rétro-cornéens. FO systématique ! 
\begin{itemize}
\item Causes : inconnue, spondylarthrite ankylosante (diag = \{sacro-iléite, rachis,
Ag HLA B27), uvéite herpétique, arthrite juvénile idiopathique, sarcoïdose,
Behçet, lupus erythémateux disséminé.
\end{itemize}
\end{itemize}
\ttt collyre mydriatique, collyres corticoïdes
\begin{itemize}
\item \emph{Glaucome aigu par fermeture de l'angle} : rare, pronostic sévère.
\begin{itemize}
\item prédisposition anatomique, pendant une mydriase
\item humeur aqueuse ne peut plus passer dans la chambre antérieure, s'accumule
dans chambre postérieure et bloque le trabeculum.
\item signes fonctionnels .: douleurs profondes++ irradiant dans trijumeau,
souvent nausées, vomissement, BAV
\item examen : douleurs intenses, \oe{}il rouge, transparence cornée \dec diffuse,
semi-mydriase aréflexique, angle iridocornéen fermé, hypertonie oculaire++
\item cécité en qq jours sans ttt \danger
\end{itemize}
\end{itemize}
\ttt antalgique, inhibiteurs de l'anhydrase en urgence \skull
    carbonique \footnote{\dec production d'humeur aqueuse.}, solutés hyperosmolaires\footnote{Déshydratation du vitré}, collyre hypotonisants,
    collyres myotiques
\begin{itemize}
\item post-crise : iridotomie périphérique sur \textbf{les deux yeux} (laser ou chir)
\end{itemize}
\begin{itemize}
\item \emph{Glaucome néovasculaire} : VEGF crée néovaisseux qui empêche la résorption de
l'humeur aqueuse
\begin{itemize}
\item néovaisseaux sur l'iris
\end{itemize}
\end{itemize}
\ttt hypotonisants locaux et généraux, photocoagulation ou injection anti-VEGF
\begin{itemize}
\item \emph{Endophtalmie post-opératoire} : douleur intense, \oe{}dème palpébral, hyalite
\end{itemize}
\section{7 Altération de la fonction visuelle}
\label{sec:org8c49ce6}
\subsection{Examen}
\label{sec:orgc5a6481}
Interrogatoire : BAV objective, altération CV  (myodésopsies, phosphènes,
métamorphopsies, éclipse visuelle (qq secondes) ou cécité monoculaire
transitoire (qq min-heures), aura visuelle), installation, unilatéral ?, douleur
? ATCD, ttt, traumatisme ?

Examen ophtalmo : AV (avec correction), RPM, segment antérieur, tonus oculaire,
\{cristallin, vitré, rétine, vaisseaux, nerf optique\}

\subsection{BAV progressive \footnote{Cf chapitre suivant pour BAV brutale}}
\label{sec:org5e47ad6}
Si améliorée par correction optique, trouble de réfraction. Sinon :

\subsubsection{Transparence anormale}
\label{sec:org1144803}
\emph{Cataracte} : BAV bilatérale, photophobie, myopie d'indice, diplopie
monoculaire. Perte de transparence du cristallin (opalescent). Étiologie : âge
surtout.\\
\ttt chir

Autres : \emph{cornée, vitré} (hyalite des uvéite)

\subsubsection{Transparence normale: atteinte nerf optique}
\label{sec:orge683d03}
\emph{Glaucome chronique à angle ouvert} (GCAO) : longtemps asymptomatique. Diag =
\{\inc tonus oculaire, altération CV, excavation glaucomateuse de la
papille\}. \\
\ttt collyre hypotonisants, trabéculoplastie (laser/chir)

Autres : \emph{neuropathie toxiques} (alcool-tabac, médic), \emph{héréditaires, compressive}

\subsubsection{Transparence normale: atteinte de la rétine/macula}
\label{sec:orgb2f4b7d}
Dystrophies rétiniennes héréditaires :
\begin{itemize}
\item \emph{maculopathies héréditaires} : maladie de Stargardt = débute vers 7-12 ans,
1/10e en fin d'évolution (!), en "\oe{}il de b\oe{}uf"
\item \emph{rétinopathies pigmentaires} : héméralopie, rétrécissement progressif du CV
(débute dans l'enfance), aspect réticulé "en ostéoblastes"
\end{itemize}
"Interface vitréomaculaire" = \emph{séparation vitré-région maculaire}
\begin{itemize}
\item membranes épi-/pré-maculaire (par tissu fibreux). "reflet" cellophane". OCT
maculaire. Chir possible
\item trous maculaires : OCT
\end{itemize}
\emph{Dégénérescence maculaire liée à l'âge}

\emph{\OE{}dème maculaire} : en "pétales de fleur" si important. Causes : 
\begin{itemize}
\item rétinopathie diabétique : \emph{ttt} : injection IV anti-VEGF ou corticoïdes
\item occlusion veine centrale de la rétine : \emph{ttt} idem
\item chir cataracte
\item uvéite postérieures : \emph{ttt} : cause ou corticoïdes retard
\end{itemize}

\emph{Maculopathies toxiques aux antipaludéens de synthèse} : potentiellement cécité
irréversible. Pas avant 5 ans ? Commence par périfovéolopathie \thus arrêt
immédiat du ttt \danger

\subsection{Altération du CV}
\label{sec:org48c6c8c}
= altération vision périphérique

\subsubsection{Affections rétiniennes}
\label{sec:org5fd72ca}
Scotomes (para)centraux , déficits périphériques

\subsubsection{Atteinte des voies optiques}
\label{sec:org7a5e628}
Atteinte nerf optique : cécité unilatérale (trauma, tumeur) 
\begin{itemize}
\item scotome central unilatéral ou caecocentral uni-/bi-latéral
\item déficicit fasciculaire possible
\item déficit altitudinal si \gls{NOIA}
\item étiologies :
\begin{itemize}
\item SEP (névrite optique rétrobulbaire)
\item NOIA
\item toxiques et métabolique : bilatéral, progressive. \{alcool-tabac, médicament,
professionnelle, métabolique\}, tumoral (tumeurs intraorbitaires, étage
antérieur du crâne)\}
\end{itemize}
\end{itemize}

Lésion du chiasma optique : hémianopsie/quadranopsie bitemporale (décussation
!). Étiologies : adénome hypophyse surtout

Lésions rétrochiasmatique : hémianopsie latérale homonyme. Si atteinte des
radiaton optiques, quadranopsie latérale homonymes. Étiologies vasc, tumoral,
trauma

Cécité corticale : bilatérale, brutale. Examen ophtalmo OK, RPM OK,
désorientation, hallucination visuelles, anosognosie
\section{8 Anomalies de la vision d'apparition brutale}
\label{sec:org72ba9b8}
\subsection{Diagnostic}
\label{sec:org3e5d0ef}
Interrogatoire : BAV ? altération CV ? myodésopsies,  phosphènes,
métamorphopsies ? rapidité, latéralité, type de douleurs, ATCD, ttt, trauma ?

Examen ophtalmo ( 2 yeux) : AV, RPM, segment antérieur, tonus oculaire, FO

\subsection{Étiologie}
\label{sec:org8a274f1}
\subsubsection{BAV, \oe{}il rouge et douloureux}
\label{sec:org72247a3}
\begin{itemize}
\item \emph{kératite aigüe} : douleur superficielles importantes, photophobie,
blépharospasme. \dec transparence cornée, cercle périkératique, ulcération(s)
cornéenne(s)
\item \emph{glaucome aigü par fermeture de l'angle} : douleurs profondes, intense,
irradiant dans trijumeaux. \inc\inc tonus oculaire ("bille de bois" à la
palpation)
\item \emph{uvéites}
\begin{itemize}
\item \emph{antérieure aigüe} : BAV, douleurs, cercle périkératique, myosis par synéchies
iridocristalliniennes. Tyndall, chercher uvéite postérieure
\item \emph{postérieure} : toxoplasmose oculaire le plus souvent. Myodésopsies, BAV. FO =
foyer blanchâtre puis cicatrice. \\
\end{itemize}
\end{itemize}
\ttt antiparasitaire si AV menacée
\begin{itemize}
\item autres  : glaucome néovasculaire, endophtalmie (contexte post-op)
\end{itemize}

\subsubsection{BAV, \oe{}il blanc indolore}
\label{sec:org5a620e4}
\begin{itemize}
\item FO non visible 
\begin{itemize}
\item \emph{hémorragie intravitréenne} : précédée d'une "pluie de cendre", BAV
variable. Écho. B pour éliminer un décollement de la rétine. Causes :
rétinopathie diabétique proliférante, occlusions ischémique de la veine
centrale de la rétine, déchirure rétinienne, sd de Terson\footnote{Hémorragie intravitréenne uni-/bi-latérale et hémorragie méningée par
rupture d'anévrisme intracrânien}, plaie
perforante
\item \emph{uvéite intermédiaire} (dans le vitré) : cellules inflammatoires
\end{itemize}
\item FO visible anormal
\begin{itemize}
\item \emph{occlusion de l'artère centrale de la rétine} : BAV brutale, mydriase
aréflexique et RPM direct aboli, \dec diffuse du calibre artériel, \oe{}dème
blanc rétinien ischémique de la rétine (macula rouge cerise)
\item \emph{occlusion de la veine centrale de la rétine} : BAV variable, \{\oe{}dème
papillaire, hémorragie rétiniennes disséminées, nodules cotonneux,
tortuosités veineuse\}. Préciser si ischémique
\item \emph{DMLA} : BAV et métamorphopsies brutales, décollement exsudatif de la rétine
maculaire \textpm{} hémorragies, exsudats profonds
\item \emph{décollement de la rétine rhegmatogène} : après une déhiscence, le vitré va
sous la rétine. 
\begin{itemize}
\item causes : idiopathique (âgé), myopie (forte), chir cataracte
\item évolution spontanée = cécité
\item \emph{ttt} chir (semi-urgence \danger)
\item clinique : myodésopsies, phosphènes, amputation CV périphérique, BAV
\item diag par FO : rétine en relief, mobile, avec des plis
\item toujours examiner \oe{}il controlatéral (ttt préventif par photocoagulation)
\danger
\end{itemize}
\item \emph{neuropathie optique ischémique antérieure} : BAV unilatérale brutale, \dec
RPM direct, \oe{}dème papillaire, déficit fasciculaire pour CV. Cause : surtout
artériosclérose mais penser à Horton (urgence \skull)
\end{itemize}
\item FO visible normal
\begin{itemize}
\item \emph{névrite optique rétrobulbaire} : BAV unilatérale progresse en qq jours (!),
douleur \inc mouvement oculaires, RPM direct \dec, FO normal, scotome
(caeco)-central
\item \emph{atteinte des voies (rétro)chiasmatiques} : tumorale (si progressive), vasc (si brutale)
\end{itemize}
\end{itemize}

Anomalies transitoire 
\begin{itemize}
\item \emph{cécité monoculaire transitoire} (qq minute) = amarause. FO pour embole
rétinien. Urgence \skull. Chercher athérome carotidien, cardiopathie embolinogène
\item insuf vertébrobasilaire, éclipses visuelle, scotome scintillant
\end{itemize}
\section{9 Prélèvement de cornée}
\label{sec:org22bcc12}
Le médecine prélèveur \(\ne\) celui qui a fait le constat de mort

Sérologies à faire : HIV, HTLV, hépatite B et C, syphilis

CI :
\begin{itemize}
\item locale : chir sur segment antérieure, uvéite, conjonctivite, tumeur,
rétinoblastome, mélanome choroïdien
\item infectieuses (Sida, rage, Creutzfeld-Jakob, hépatite..),
\item neuro inexpliquée, démence
\end{itemize}

Prélèvement \textbf{in situ}

\section{10 Greffe de cornée}
\label{sec:orgeaa0668}
Couches (depuis l'extérieur) = épithelium, couche de Bowman, stroma, membrane
  de Descemet, endothelium

Techniques
\begin{itemize}
\item kératoplastie transfixiante = toute les couches
\item kératoplastie lamellaire antérieure = épithelium, Bowman et stroma seulement
\item kératoplastie endothéliale : membrane de Descemet et endotheliale seulement.
\end{itemize}

Indications : trauma perforant de la cornée, brûlures chimiques, dégénérescence
cornéenne (kératocône), kératite (herpétique), dystrophie bulleuse chez âgé

Bon pronostic dans \(\frac{2}{3}\)

Complications rares (retard d'épithélialisation, \oe{}dème cornéen précoce, rejet
immunitaire, récivide de maladie causale, hypertonie oculaire, astigmatisme post-op)
\section{11 Traumatismes oculaires}
\label{sec:org4b9235f}
\subsection{Globe fermé (contusions)}
\label{sec:org5072932}
Dangerosité inversement \(\propto\) taille agent.

Interrogatoire : douleurs, AV, heure dernier repas, lésions associées.

Examen :
\begin{itemize}
\item contusions du segment antérieur 
\begin{itemize}
\item cornée : \emph{ttt} lubrifiant cicatrisant
\item conjonctive : plaie ou hémorragie sous-conjonctivale. Toujours chercher plaie
sclérale, corps étranger
\item chambre antérieure : hyphéma. Résorption spontanée
\item iris : iridodyalise \footnote{Désinsertion de la base de l'iris}, rupture sphincter irien, mydriase post-trauma
\item cristallin : (sub)luxation, cataracte contusive (plusieurs mois après\ldots{})
\item hypertonie oculaire : si lésions de l'angle iridocornée, hyphéma, luxation
antérieure du cristallin
\end{itemize}
\item contusions du segment postérieur
\begin{itemize}
\item \oe{}dème rétinien du pôle supérieur : guérison spontanée
\item hémorragie intravitréenne : résorption spontanée. Écho B si décollement de
rétine (DR) suspecté
\item déchirure rétiniennes périph. Photocoagulation prophylactique possible
(contre DR)
\item rupture de la choroïde : BAV définite si macula
\end{itemize}
\end{itemize}

\subsection{Globe ouverts}
\label{sec:orgae9182e}
Rupture du globe : pronostic plus péjoratif si postérieur

Trauma perforant : 
\begin{itemize}
\item AVP, accidents domestique, bricolage, agression.
\item plaies : larges (mauvais pronostic) ou petites (cornée ou sclère)
\item risque : méconnaître plaie, corps étranger. Si doute, scanner !
\end{itemize}

\subsection{Corps étrangers}
\label{sec:org7c99bd3}
Diagnosic évident : 
\begin{itemize}
\item superficiel 
\begin{itemize}
\item circonstance, symptôme de conjonctivite ou kératite superficielle. Corps étranger souvent visible
\item bon pronostic, ttt lubrifiant et antiseptique/ATB local
\end{itemize}
\item intraoculaire : circonstance, porte d'entrée et trajet visible
\end{itemize}
Délicat si trauma non remarqué, pas de porte d'entrée, trajet et corps non visible

Examens : TDM si doute. Pas d'écho B si transfixiante. Pas d'IRM \danger

Complications (si intraoculaire) : endophtalmie (grave++), DR, cataracte traumatique

Complications tardives : ophtalmie sympathique \footnote{Uvéite auto-immune sévère}, sidérose (fer), chalcose (cuivre)
\section{12 Brûlures oculaires}
\label{sec:org1ab22fa}
Accident industriels (graves), domestiques, aggression

Brûlures 
\begin{itemize}
\item thermique : peu grave (brûlure par cigarette). Cicatrisation rapide, sans séquelles
\item acides : gravité modérée-moyenne, grave si très concentré
\item basiques : grave !
\end{itemize}

Classification de Roper-Hall
\begin{enumerate}
\item bon pronostic
\item bon pronostic : opacité cornéenne mais détails iris, ischémie < 1/3 \diameter
\item pronostic réservé : désépithélialisation cornéenne totale, opacité cornéenne
, ischémie \(\in\) [1/3, 1/2] \diameter
\item pronostic péjoratif
\end{enumerate}

Ttt d'urgence par lavage (20-30min) par sérum phy (ou eau), puis collyre
corticoïde ASAP

Autres :
\begin{itemize}
\item brûlures des UV (ski, UV)
\item soudure à l'arc sans lunette
\item phototraumatisme (éclipse) : BAV définitive !
\end{itemize}

\section{13 Cataracte}
\label{sec:orgaa75fe2}
Déf: opacification (partielle ou non) du cristallin. 

\subsection{Diagnostic}
\label{sec:orgb9ce19c}
Découvert sur BAV (progressive, vision de loin), photophobie, (diplopie
monoculaire), jaunissement

Examen clinique :
\begin{itemize}
\item interrogatoire : âge, profession, ATCD (diabète, corticoïdes), myodésopsies,
métamorphopsies
\item AV (\oe{}il par \oe{}il, bionculaire)
\item lampe à fente : 
\begin{itemize}
\item cristallin (caracte \{nucléaire, sous-capsulaire postérieure, corticale, totale\})
\item éliminer autre patho (cornée, iris, vitré, rétine (DMLA, glaucome))
\item mesure tonus oculaire (hypertonie, glaucome)
\end{itemize}
\end{itemize}

Diagnostic clinique mais en complémentaire :
\begin{itemize}
\item écho en mode B si décollement de rétine/tumeur intraoculaire
\item pour le cristallin artificiel : kératométrie, longeur axiale
\end{itemize}

Étiologies :
\begin{itemize}
\item âge++
\item traumatique (contusion violent, trauma perforant)
\item secondaire à diabète, hypoparathyroïdie
\item secondaire aux corticoïdes locaux/généraux au long cours, à radiothérapie
orbitaire
\item secondaire à dystrophie myotonique de Steinert, trisomie 21
\item congénitales
\end{itemize}

\subsection{Traitement = chir}
\label{sec:org158602b}
Anesthésique topique/locolrégionale\footnote{Voire générale}, en ambulatoire. Sous dilatation
pupillaire. 

Technique = extraction extracapsulaire automatique par phacoémulsification. Puis
collyre anti-inflammatoire+ATB (1 mois).

Correction optique : implant de chambre postérieure (rarement antérieure)
\begin{itemize}
\item monofocal = sphérique (hypermétropie/myopie) ou torique (idem + astigmatisme)
\item multifocaux : corrige vision de loin et de près
\item si implant impossible : lentilles de contact/ ou lunettes (exceptionnel)
\end{itemize}

Indication : quand BAV = 5/10.

Complications :
\begin{itemize}
\item opacification de la capsule postérieure
\item endophtalmie\footnote{Infection intraoculaire sévère.} : à J2-J7. Ttt : ATB local, intravitréenne et générale
\item décollement de la rétine
\item \oe{}dème maculaire : semaines/mois. ttt : anti-inflammatoire
(local/locorégional)
\item kératite bulleuse
\end{itemize}

\section{14 Glaucome chronique}
\label{sec:org2cbc647}
= Glaucome primite à angle ouvert (GPAO) Neuropathie optique progressive altération
fonction/structure.
FR : âge, hypertonie oculaire (\(\ne\) cause !!), ATCD familaux directs, noirs
d'origine africaine, myopie

Physio : perte accélére des fibres optiques liée à l'âge

\subsection{Formes cliniques :}
\label{sec:org2bf8bfe}
GPAO à préssion élevée (> 21 mmHg) (70\% des pop. occidentales)
\begin{itemize}
\item anomalies de structures visibles cliniquement : papille = \{\dec surface,
hémorragies péripapillaires en flammèches, atrophie péripapillaire \(\beta\), \inc
excavation papillaire\}. Rapport de taille entre papilles des 2 yeux > 0.2
\thus suspect
\item autres anomalies de structures : OCT
\item anomalies de foncton : champ visuel mais AV touchée très tardivement
\end{itemize}

GPAO à pression normale (70\% pop. asiatique) : plutôt femmes, migraine, acrosyndromes

DD : 
\begin{itemize}
\item hypertonie oculaire : pression intraoculaire > 21mmHg, angle ouvert (gonio),
\(\emptyset\) neuropathie optique. Pas forcément de ttt.
\item glaucomes à angles ouvert secondaires
\item glaucomes par fermeture de l'angle
\item crise aigüe de fermeture de l'angle : douleur, urgence \skull
\item neuropathie optiques non glaucomateuses
\end{itemize}

\subsection{Traitement}
\label{sec:orgf6d3366}
Dépistage : seulement ATCD familaux de GPAO, myopie, > 40 ans

Ttt : \dec pression intra oculaire

Médicaments :
\begin{itemize}
\item à vie
\item 1ere intention : collyre à base de prostaglandines \footnote{EI : rougeur, irrit. oculaire. Puis iris plus sombre, \inc pousse cils.}(\inc élimination humeur
aqueuse) ou collyre betabloquant
(\dec sécrétion humeur aqueuse)
\item association possible mais \(\le\) 3
\item rarement : acétazolamide par voie générale
\end{itemize}

Trabéculoplastie au laser = photocoagulation de l'angle. Effet modeste, non durable.

Chir : trabéculectomie. Complications (rare) = cataracte, hyoptonie précoce
avec décollement choroïdien, infection globe oculaire. Principal échec :
fermeture de la voie par fibrose sous-conjonctivale.

\section{15 Dégénérescence maculaire liée à l'âge}
\label{sec:org6625cf6}
Atteinte de la macula chez > 50 A. Débutante/intermédaire : drusent, altération
pigmentaire, AV normale ou peu \dec. Évoluée : atrophique ou exsudative

Prévalence : 18\% après 50 ans, 37\% à 85 ans.

FR : âge, pop européennes, polymorphisme facteur H du complément, tabac, régime
pauvre en anti-oxydant/riches en (acide gras saturés et cholestérol)

\subsection{Diagnostic}
\label{sec:orgd46032e}
\begin{itemize}
\item Mesure AV (loin et près, recherche scotome central ou métamorphopsies (grille
\end{itemize}
d'Amsler)
\begin{itemize}
\item FO (drusen, altération pigmentaires, atrophie épithelium pigmentaire,
\end{itemize}
forme exsudative [cf /infra/].
\begin{itemize}
\item OCT : suivi, ou diagnostic si + FO
\end{itemize}

\subsection{Formes cliniques}
\label{sec:org471cc42}
\begin{itemize}
\item débutante : \emph{drusen} = résidus de phagocytose des photorécepteurs. Petites
lésions profondes jaunâtre. OCT
\item atrophique : disparition de l'épithelium pigmentaire. FO : atrophie de
l'épithelium pigmentaire et choroïde. BAV sévère avec scotome centrale
\item exsudative (néovaisseaux sous rétine) : \oe{}dème intrarétinien, hémorragies,
décollement maculaire exsudatif (BAV, métamorphopsies brutales). Sans ttt :
BAV sévère, scotome central définitif. BAV chez drusen = urgence \skull
\end{itemize}

\subsection{Ttt}
\label{sec:org3cfa575}
\begin{itemize}
\item débutante : vit E, C, zinc, lutéine, zéaxantine
\item atrophique : \(\emptyset\)
\item exsudative : injection d'anti-VEGF : ranibizumab,
aflibercet. bévacizumab. Stoppent néovaisseux, font régresser l'\oe{}dème. 40\%
ont amélioration visuelle à 2 ans.
\end{itemize}
Laser possible mais risque thrombose
\section{16 Occlusions artérielles rétiniennes}
\label{sec:orgbc060c2}
Artères ciliaires postérieure alimente les couches profondes (épithelium
pigmentaire de la rétine, photorécepteurs). 
Artère centrale de la rétine = couche internes (cellules bipolaires,
ganglionnaires, fibres optiques).

Arrêt circulatoire \thus lésions définitives en 90min \danger

\subsection{Occlusion de l'artère centrale de la rétine}
\label{sec:org9812f76}
Diagnostic : 
\begin{itemize}
\item BAV brutale (amaurose transitoires précédentes possibles)
\item Oeil blanc indolore, AV effondrée, mydriase aréflective (RPM direct aboli)
\item FO : dans les heures : oedème ischémique rétinien blanchâtre, tache "rouge
cerise de la macula")
\end{itemize}

Étiologie : 
\begin{itemize}
\item embolies : athérome carotidien++, cardiopathie embolinogène, ( fractures des os
lng ou emboles tumoraux)
\item thromboses : maladie de Horton (urgence \skull, à rechercher), (maladies de systèmes)
\item troubles coagulation : anomalie primitive, sd antiphospholipides, hyperhomocystéinémie
\end{itemize}

Évolution spontanée quasiment toujours défavorable

CAT : urgence \danger \skull (fonction visuelle et patho sous-jacente)
\begin{itemize}
\item bilan étio : athérome, carotidien, cardiopathie embolinogène, dissection
carotidienne (si jeune), horton
\item ttt décevant : hypotonisant (acétazolamide IV/per os), vasodilatateur,
anticoagulant si  bespoin fibrinolytique
\item bilan cardio
\item Aspirine dans tous les cas. Si jeune, bon état général, ttt max.
\item selon étio : antiagrégant plaquettaire (athérome carotidien), anti-vit K (emboles
cardiaques), endartériectomie
\end{itemize}

\subsection{Occlusion de branche l'artère centrale de la rétine}
\label{sec:orgc0d79ad}
Amputation du CV. BAV possible. FO : \oe{}dème rétinie ischémique en secteur

Évolution : \inc AV en qq semaines, pronostic visuel bon mais amputation
perisiste.

Même étio, sausf Horton. Même ttt.

\subsection{Nodules cotonneux}
\label{sec:org9356690}
Occlusion d'artérioles rétiniennes précapillaire \thus nodules cotonneux =
petites lésions blanches superficielles d'aspect duveteux.

Étio : HTA, diabète, occlusions veineuses rétiniennes, sida, lupus érythémateux
disséminé, périartérite noueuses, embolies graisseuses, pancréatite aigüe, sd Purtscher
\section{17 Occlusions veineuses rétiniennes}
\label{sec:org1afdc9f}
\subsection{Occlusion de la veine centrale de la rétine}
\label{sec:org1f43a64}
SF : vision trouble brutalement, BAV variable, \oe{}il blanc indolore
FO : 
\begin{itemize}
\item diagnostic = \{\oe{}dème papillaire, veines rétitiennes tortueuses et dilatées,
hémorragies sur la surface rétitienne, nodules cotonneux\}
\item formes non ischémiques (freq) : AV > 2/10, hémorragies en flammèches, ischémie
peu étendue
\item formes ischémiques\footnote{Différence avec l'angiographie fluorescinique} : AV < 1/10, réflexe pupillaire direct diminué, hémorragies
plus volumineuses, en tache
\end{itemize}

Suivi par OCT

Étiologie inconnue 
\begin{itemize}
\item mais > 50 ans avec FR vasc \thus recherche \{tabac, HTA,
\end{itemize}
diabète, hypercholestérolémie, SAS\}, hypertonie oculaire++
\begin{itemize}
\item si < 50 ans, 0 FR ou OVCR bilatérale, chercher anomalie primitive de la
coagulation, sd antiphospholipides, hyperhomocystéinémie, contraception, hyperviscosité
\end{itemize}

Évolution :
\begin{itemize}
\item formes non ischémiques : normalisation AV en 3-6 mois. Sinon : conversion en
forme ischémique, peristance d'un \oe{}dème maculaire cystoïde (BAV permanente !)
\item formes ischémiques : 
\begin{itemize}
\item pas de récupération fonctionnelle.
\item Pire complication = néovascularisation irienne \thus progression rapide vers glaucome
\end{itemize}
néovasculaire (3 mois) \thus prévention par photocoagulation panrétinienne
\begin{itemize}
\item néovascularisation prérétiniennes ou précapillaire (hémorragie intravitréenne)
\end{itemize}
\end{itemize}

Ttt 
\begin{itemize}
\item formes non ischémique : injection d'anti-VEGF/triamcinolone si \oe{}dème
maculaire cystoïde avec BAV persistante. Surveillance tous les mois
\item formes ischémiques : photocoagulation panrétinienne (PPR) pour éviter glaucome
néovasculaire
\item glaucome néovasculaire : PPR en urgence après hypotonisant local \danger
\end{itemize}


\subsection{Occlusion de branche veineuse rétinienne}
\label{sec:orgbc04b2d}
Identique à OVCR mais territoire plus limité.
Signe du croisement (cf chap 23) \thus > 60 ans, FR d'artériosclérose

Clinique : BAV variable, FO identique OVCR

Évolution favorable en majorité. Défavorable si maculopathie ischémique,
\oe{}dème maculaire persistante, néovaisseaux prérétiniennes \thus hémorragie du
vitré mais \textbf{pas} de GNV

Ttt similaire. 
\begin{itemize}
\item Si \oe{}dème maculaire peristant : injection intravitréenne
\item photocoagulation maculaire en grille (si \(\ge\) 3 mois, \oe{}dème maculaire
perisistante, AV ɇ 5/10)
\item photocoagulation sectorielle
\end{itemize}
\section{18 Pathologies des paupières}
\label{sec:orgb3b9d65}
Anatomie : plan antérieur cutanéomusculaire, plan postérieur : tarse,
conjonctive (papébrale, cul-de-sac, bulbauire). Sécrétion des larmes par la
glande lacrymale principale, évacuées par les voies lacrymales excrétrices

Fermeture paupière : nerf VII, ouverture par nerf III

\subsection{Pathologies}
\label{sec:orgaeb9ad2}
Orgelet : furoncle du bord libre, centré sur follicule du cil.
\begin{itemize}
\item infection bactérienne (\bact{dore}, tuméfaction rouge autour point blanc
\item ttt : ATB 8 jour
\end{itemize}
Chalazion : granulome inflammatoire sur glande de Meibomius (tarse)
\begin{itemize}
\item tuméfaction douloureuses
\item Ttt : pommade corticoïde locale, massage des paupières. Éventuellement
incision
\end{itemize}

\subsection{Autres}
\label{sec:org6a3a6b3}
Malformation palpébrales : 
\begin{itemize}
\item entropion : bascule vers l'intérieur. Sénile, cicatriciel)
\item ectropion . sénile, cicatriciel, paralytique
\item ptosis : neurogène, myogène, sénile, traumatique
\item lagophtalmie (inoclusion palpébrale) : anesthésie générale, coma prolongé,
paralysie faciale
\end{itemize}
Tumeurs palpébrales :
\begin{itemize}
\item bénignes : papillome, hydocystome (kyste lacrymal), xanthélesmas (dépôts
cholestérol). TTt : chir
\item malignes : 
\begin{itemize}
\item épithéliale : carcinome basocellulaire surtout (nodule perlé,
télangiectasise; peut menacer globe oculaire, pas de métastases), carcinome
épidermoide (métastase)
\item mélaniques : pronostic peut être effroyable. suspecter si tuméfaction
(pigmentée) des paupières
\item carcinomes sébacés, lymphomes MALT
\end{itemize}
\end{itemize}
Trauma : vérifier septum orbitaire, globe oculaire, canalicules lacrymaux
arrachés (urgence \skull)

\section{19 SEP}
\label{sec:org9cb9f18}
Névrite optique rétrobulbaire \(\in\) SEP (20\% = inaugurale). Atteint adulte \(\in\)
[20,40]ans, surtout \female

Neuropathie optique :
\begin{itemize}
\item BAV variable sur qq jours, douleurs rétro oculaires (80\%)
\item pupille de Marcus gunn (dilatation paradoxale). Si inflammation intraoculaire,
penser syphilis, sarcoïdose
\item Complémentaire :
\begin{itemize}
\item examen du CV : scotome (caeco)central.
\item dyschromatopsie d'axe rouge-vert
\item IRM encéphalique systématique (SEP)
\end{itemize}
\item régression en 3-6 mois. Risque de SEP = 50\% à 15 ans (75\% si lésions
encéphaliques à l'IRM). Phénomène d'Uhthoff possible\footnote{BAV transitoire quand température \inc}
\item ne pas confondre avec neuromyélite de Devic ! Inflammation (\(\ne\) aut-immune) du
SNC, BAV profonde rapide. IRM : pas d'inflammation cérébrale mais médullaire
\item ttt : corticothérapie  parentérale 3-5j (1g/j) puis oral court (11 j). Retarde
un nouvel épisde. Si SEP, ttt de fond (immunomodulateur [interférons, acétate
de glatiramère] ou natalizumab, fingolimod)
\item pronostic fonctionnel souvent favorable
\item DD : autres neuropathies optiques
\end{itemize}

Autres :
\begin{itemize}
\item oculomotrice : paralysie du VI, internucléaire (penser SEP si bilatérale chez jeune)\footnote{\OE{}il atteint = pas d'adduction mais adduction des yeux OK à la
convergence. Nystagmus pendulaire en abduction de l'\oe{}il sain !}
\item nystagmus : 1/3 patient avec SEP > 5 ans
\item périphlibite rétiniennes (5\%)
\end{itemize}
\section{20 Neuropathie optique ischémique antérieure}
\label{sec:org3968795}
Ischémie de la tête du nerf optique (artères ciliaire postiérieures)

Diagnostic clinique :
\begin{itemize}
\item BAV unilatérale brutale indolore
\item AV variable, \dec RPM direct, FO = \oe{}dème papillaire++, papille souvent
pâle, hémorrage en flammèche sur le bord papillaire
\item complémentaire : examen CV++ = \{déficit fasciculaire, altitudinal\},
angiographique du fond d'\oe{}il
\end{itemize}

Étiologie : 
\begin{itemize}
\item maladie de Horton = urgence \skull (cécité bilatérale définitive) \thus 
\begin{itemize}
\item signes systémique, modif artères temporale
\item signes oculaire : amaurose fugace, défaut de remplissage choroïdien (angiographie)
\item bio : VS \inc[fn:24] , CRP \inc
\item biopsie de l'artère temporale (ne doit pas retarder les soins !)
\end{itemize}
\item artériosclérose (fréq) : FR (tabac, HTA, diabète, hypercholestérolémie)
\end{itemize}

Évolution : résorbtion de l'\oe{}dème en 6-8 semainer (atrophie). Pas de récup
visuelle. Bilatéralisation possible en qq jours \danger

DD : causes d'\oe{}dème papillaire (HTIC)

Ttt : 
\begin{itemize}
\item artéritique = urgence ! Corticothérapie générale à forte dose puis per os
1mg/kg/j
\item non artéritique : pas de ttt efficace
\end{itemize}

\section{21 Rétinopathie diabétique}
\label{sec:org6cbdec2}
30\% diabétique en ont une. Diabète 1 : 90\% après 20 ans. Diabète 2 : 60\% à 15
ans

Physiopatho : hyperglycémie \thus \{accumulation sorbitol, glycation, stress
oxydatif\} \thus \{inflammation, activation rénine-angiotensine, modif flux
sanguin rétinien, production VEGF\} puis 
\begin{itemize}
\item occlusion capillaire et néovascularisation
\item \oe{}dème maculaire
\end{itemize}

\subsection{Diagnostic}
\label{sec:orgaa03d26}
Photographie du FO = référence et dépistage
\begin{itemize}
\item microanévrisme rétitiens : dilatations punctiformes rouges (postérieure)
\item hémorragie rétiniennes punctiformes
\item nodules contonneux
\item signes d'ischémie : hémorragies intrarétiniennes "en taches" ou en flammèche, anomalies
microvasculaire intrarétinienne, dilatations veineuses "en chapelet",
néovaisseux prérétiniens et précapillaires, hémorragie prérétiniennes/intravitréennes
\item complications : hémorragie intravitréennek, décollement de la rétine \emph{par
traction}, néovascularisation irien (\thus glaucome néovasculaire !)
\item signes maculaires : \oe{}dème maculaire (cystoïde), exsudats lipidiques
(souvent en couronne)
\end{itemize}

Complémentaire : OCT pour diagnostic et suivi \oe{}dème maculaire, echo en mode B
pour décollement de rétine par traction

\subsection{Dépistage RD}
\label{sec:orgd894dfc}
Diabète 1 : photo FO à la découverte puis surveillance annuelle
\begin{itemize}
\item enfant: pas avant 10 ans
\item grossesse : avant puis tous les 3 mois (tous les mois si RD !)
\end{itemize}
Diabète 2 : dès découverte

Surveillance :
\begin{itemize}
\item \(\emptyset\) RD on non proliférante minime : annuelle
\item sonn tous 4 à 6 mois
\item renforcé si puberté, adolescende, si équilibrage trop rapide de la glycémie,
chir bariatrique, diabète ancien mal équilibré, chir de la cataracte,
\oe{}dème maculaire
\end{itemize}

Classification :
\begin{itemize}
\item \(\emptyset\) RD
\item RD non proliférante : minime, modérée, sévère (hémorragie rétiniennes dans 4
quadrants ou dilatations veineuses 2 quadrants ou AMIR 1 quadrant)
\item RD proliférante : minime, sévère, compliquée
\end{itemize}

Progression lente, aggravations rapides possibles. \danger prolifération néovasc
peut donner cécité 

\subsection{Traitement}
\label{sec:orgac56bad}
Médical :
\begin{itemize}
\item équilibre glycémique et hypertension pour 2 diabètes !
\item pas de ttt médicamenteux
\end{itemize}
RD proliférante 
\begin{itemize}
\item photocoagulation panrétinienne : régression dans 90\%. INdication : RD
proliférante ou (RD non proliférante sévère si grossesse, sujet jeune
diabétique 1 avec normalisation rapide glycémie, chir cataracte)
\item injection intravitréenne anti-VEGF (pour néovasc. iridienne, glaucome
néovasculaire)
\item chir si RD proliférante avec hémorragie intravitréenne persistante/décollement
de rétine tractionnel
\end{itemize}
\OE{}dème maculaire 
\begin{itemize}
\item photocoagulationau lasar si exsudats lipidique ou liquide
\item injection anti-VEGF mensuelle. Dexaméthasone retard possible mais cataracte,
risque d'hypertonie oculaire (30\%)
\end{itemize}
\section{22 Orbitopathie dysthyroïdienne}
\label{sec:orga9f8f19}
Surtout maladie de Basedow (ou thyroïdite auto-immune d'Hashimoto

Thyrotoxicose : \{palpitations, tachycardie\}, \{nervosité, tremblement, insomnie\},
thermophobie, \{amaigrissement, fatigue\}

Classification :
\begin{itemize}
\item gravité = NOSPECS : No sign, Only lid retraction, Soft tissues, Proptosis \(\ge\)
3mm, Extraocular muscle, Corneal
\item inflammataion = CAS
(clinical activity score)
\end{itemize}

Manifestation :
\begin{itemize}
\item exophtalmie : bilatérale (75\%), axile, non pulsatile, réductible, indolore,
> 21 mm
\item rétraction paupière, asynergie oculopalpébrale vers le bas, \dec fréquence
clignement
\item trouble oculomoteur : myosite \thus diplopie
\end{itemize}

Complications :
\begin{itemize}
\item cornée : kératite, peforation
\item neuropathie optique compressive (3\%) \thus méthylprednisolone forte dose ou
décompression rapide \skull !
\end{itemize}

Complémentaire : TSH effondrée (hyperthyroïdie), scanner et IRM pour conforter,
\{pupille, champ visuel, couleurs, PEV\}

DD : infection orbitaire, fistules artériocaverneuses, tumeurs, orbitopathies
inflammatoires

Traitement 
\begin{itemize}
\item médical : 
\begin{itemize}
\item traiter thyrotoxicose (\danger iode radioactif peut aggraver orbitopathie) arrêt tabac, ttt locaux (symptômes), sélénium si forme modérée
\item CAS \(\ge\) 3 \thus anti-inflammatoire (méthylprednisolone)
\end{itemize}
\item chir : décompression orbitaire, muscles oculomoteurs, paupières
\end{itemize}
\section{23 Rétinopathie et choroïdopathie hypertensive}
\label{sec:org1e5bf95}
Bien différencier modifications liée à l'HTA (réversible par ttt de l'HTA) et
celles irréversibles liées à l'artériosclérose 

\inc Pa \thus vasoconstriction artérieur active (autorégulation)

\subsection{Rétinopathie hypertensive}
\label{sec:org84e5110}
AV normale. Signes oculaires seulement si sévère :
\begin{itemize}
\item barrière hématorétinienne rompue \thus hémorragies rétiniennes superficielles,
\oe{}dème maculaire et exsudat secs, \oe{}dème papillaire
\item occlusion artérioles précapillaires \thus nodules cotonneux, hémorragies
rétiniennes profondes
\item hémorragies en flammèches péripapillaire (superficielles) ou rondes sur toute
la rétine (profondes)
\end{itemize}
\thus non spécifiques mais évocateurs d'HTA si associés. Pas de BAV

\subsection{Choroïdopathie hypertensive}
\label{sec:orgdf35914}
Pas d'autorégulation pour les vaisseaux choroïdiens. Occlusion \thus ischémie et
nécrose de l'épithelium pigmentaire. cicatrisens en petite taches pigmentées
("d'Elschnig").

Formes plusu sévère : décollement de rétine exsudatif (post) avec
BAV. Normalisation avec ttt

\subsection{Artériosclérose}
\label{sec:org0af9bd2}
Lésions chroniques irréversibles mais asymptomatiques :
\begin{itemize}
\item accentuation du reflet artériolaire
\item signe du croisement\footnote{Veine "écrasée" sur le croisement avec l'artère, dilatée en amont}
\end{itemize}


Classification de Kirkendall 
\begin{itemize}
\item rétinopathie hypertensive : I (rétrécissement artériel sévère et disséminé),
II (idem + hémorragies rétiniennes, exsudats secs, nodules cotonneux), III
(idem + \oe{}dème papillaire
\item artériosclérose : I (croisemenrt), II (idem plus rétrécissement marqué), III
(idem plus occlusion de branche)
\end{itemize}

\printglossaries
\end{document}