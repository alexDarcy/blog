% Created 2018-12-01 Sat 13:24
% Intended LaTeX compiler: lualatex
\documentclass[11pt]{article}
\usepackage[hidelinks]{hyperref}
\usepackage{booktabs}
\documentclass{article}
\usepackage[hidelinks]{hyperref}
\usepackage{longtable}
\usepackage{booktabs}
%\usepackage[draft]{graphicx}
\usepackage{graphicx}
\usepackage{fancyhdr}
% French
\usepackage[T1]{fontenc}
\usepackage[francais]{babel}
\usepackage{caption}
\usepackage[nointegrals]{wasysym} % Male-female symbol
% Smaller marign
\usepackage[margin=2.5cm]{geometry}
\usepackage{latexsym}
\usepackage{subcaption}
%-------------------------------------------------------------------------------
% For graphs
\usepackage{tikz}
\usepackage{tikzscale}
\usetikzlibrary{graphs}
\usetikzlibrary{graphdrawing}
\usetikzlibrary{arrows,positioning,decorations.pathreplacing}
\usetikzlibrary{calc}
\usegdlibrary{trees, layered}
\usetikzlibrary{quotes}
%-------------------------------------------------------------------------------
% No spacing in itemize
\usepackage{enumitem}
\setlist{nolistsep}
% tightlist from pandoc
\providecommand{\tightlist}{%
  \setlength{\itemsep}{0pt}\setlength{\parskip}{0pt}}
 % Danger symbol (need fourier package)
\newcommand*{\TakeFourierOrnament}[1]{{%
\fontencoding{U}\fontfamily{futs}\selectfont\char#1}}
\newcommand*{\danger}{\TakeFourierOrnament{66}}
% Skull (need Symbola font)
\usepackage{amsmath,fontspec,newunicodechar}
\newfontface{\skullfont}{Symbola}[Scale=MatchUppercase]
\NewDocumentCommand{\skull}{}{%
  \text{\skullfont\symbol{"1F571}}%
}
% Hospital sign
\usepackage{fontspec} % For fontawesome
\usepackage{fontawesome}
% Itemize in tabular
\newcommand{\tabitem}{~~\llap{\textbullet}~~}
% No numbering
\setcounter{secnumdepth}{0}
% Set header
\pagestyle{fancy}
\fancyhf{}
\fancyhead[L]{\leftmark}
\fancyhead[R]{\thepage}
%\renewcommand{\headrulewidth}{0.6pt}
% Custom header : no uper case
\renewcommand{\sectionmark}[1]{%
  \markboth{\textit{#1}}{}}

\usepackage[linesnumbered,ruled,vlined]{algorithm2e}
\author{Alexis Praga}
\date{\today}
\title{Ophtalmologie}
\hypersetup{
 pdfauthor={Alexis Praga},
 pdftitle={Ophtalmologie},
 pdfkeywords={},
 pdfsubject={},
 pdfcreator={Emacs 26.1 (Org mode 9.1.9)}, 
 pdflang={English}}
\begin{document}

\maketitle
\tableofcontents

\input{bacteries-header}
\section{1 Sémiologie oculaire}
\label{sec:org573a0fc}
\subsection{Globe}
\label{sec:org6382993}
Membranes (d'avant en arrière)
\begin{itemize}
\item externe = \{cornée (avant), conjonctive (sur sclère antérieure), sclère\}
\item intermédiaire (uvée) = \{iris, corps ciliaires, choroïde\}
\item interne (rétine) = \{épithelium pigmentaire, neurosensorielle (photorécepteurs,
fibres optiques\}\}. NB : phototransduction = pigment visuel
(épithelium) \(\rightarrow\) photorécepteurs (bâtonnets = \{vision périphérique,
nocturne\}, cône = \{détails, couleurs\} \(\in\) macula)
\end{itemize}
Contenu : 
\begin{itemize}
\item humeur aqueuse \(\in\) chambre antérieure, évacuée dans l'angle iridocornéen. P
normale si \(\le\) 22 mmHg.
\item cristallin : biconvexe, convergente, déformation par zonule \thus accomodation
\item corps vitré : 4/5e cavité oculaire \thus segment antérieur vs postérieur
\end{itemize}
Voies optiques : 
\begin{itemize}
\item nerf optique - chiasma - bandelettes optiques - corps genouillés externes - radiations optiques - cortex occipital
\item mydriase "sensorielle"\footnote{Plus de voie afférente  \thus RPM (réflexe photomoteur) direct et
consensuel aboli pour l'\oe{}il atteint} vs "paralytique"\footnote{Plus de voie efférente \thus \oe{}il atteint : RPM direct aboli. \OE{}il sain : RPM consensuel aboli}
\item voie efférente sympathique : si atteinte, sd Claude-Bernard-Horner (myosis, ptosis)
\end{itemize}
Annexes
\begin{itemize}
\item oculomoteur : nerf IV = \{oblique supérieur\}, VI = \{droit externe\} et III =
\{droit sup, droit inf, droit médial, oblique inf\}
\item protection : paupière (muscle orbiculaire, tarse), conjonctive (face interne
paupière, globe antérieur), film lacrymal
\end{itemize}

\subsection{Examen}
\label{sec:orgc9a7a46}
Interrogatoire : 
\begin{itemize}
\item baisse d'acuité visuelle (BAV), fatigue visuelle, myodésopsies,
\end{itemize}
métamorphopsies, héméralopie, scotome/amputation champ visuel, 
\begin{itemize}
\item mode d'installation : urgente si brutal
\item douleur superficielles/profondes
\item diplopie mono/bi-noculaire
\item évolution : aggration rapide = grave
\end{itemize}
Acuité visuelle : échelle de Monoyer (à 5m), de Parinaud (33cm)

Segment antérieur : "lampe à fente"
\begin{itemize}
\item conjonctive : rougeur (diffuse/localisée/culs-de-sac inférieure/avec
sécrétions/cercle périkératique/paupière), chemosis\footnote{\OE{}dème conjonctival}
\item cornée (transparence)
\item iris : myosis/mydriase
\item chambre antérieure : inflammation (Tyndall\footnote{Cellules inflammatoires, protéines dans l'humeur aqueuse}, précipités rétro-cornéens,
synéchies iridocristaliniennes), hypopion (pus), hyphéma (sang)
\end{itemize}
Pression intraoculaire par tonomètre à air pulsé. Hypertonie si \(\ge\) 22 mmHg

Gonioscopie : angle iridocornéen

Fond d'\oe{}il
\begin{itemize}
\item direct, indirect (apprentissage++), biomicroscopie
\item aspect : pôle postérieur = \{papille, vaisseaux rétitiens, macula\}, rétine
périphérique (si besoin)
\item lésions : microanévrisme (points rouges), hémorragie (intravitréennes,
prérétiniennes, sous-rétiniennes, intra-rétiniennes\footnote{Punctiformes, flammèches, profondes}), nodules cotonneux,
exsudats profonds, \oe{}dème papillaire\footnote{Si BAV, vasculaire probable. Sinon HTIC}
\end{itemize}

Oculomotricité

\subsection{Complémentaires}
\label{sec:orgcdb7c61}
Fonctions visuelles
\begin{itemize}
\item Champ visuel :
\begin{itemize}
\item sensibilité lumineuse \dec périphérie, papille = zone aveugle
\item périmétrie cinétique (Goldman : trace isoptères) ou statique (dépistage
\end{itemize}
glaucome)
\item Vision des couleurs : si dépistage anomalie congénitale (planches colorées) ou
\end{itemize}
affection acquise (Farnsworth = classer couleurs). Utile pour antipaludéens de
synthèse ou \{éthambutal, isoniazide\}
Angiographie du fond d'\oe{}ils : fluorescine ou vert d'indocyanine (DMLA++)

Électrophysio : électrorétinogramme (si lésion rétiniennes étendues), potentiels
évoqués visuels (cortex occipital) (SEP++), électro-oculogramme (épithelium pigmentaire)

Echographie : mode A (longueur globe oculaire) ou B (décollement rétine, corps
étranger intraoculaire, tumeur)

Tomographie en cohérence optique (OCT) : affection maculaire (trou, DMLA),
dépistage glaucome chronique

\section{2 Réfraction}
\label{sec:org0c422d3}
\OE{}il \(\approx\) 60 dioptries (cornée = 42, cristallin = 20)
Emmétrope = \oe{}il normal. Amétrope = anomalie de réfraction

Punctum remotum (PR) : point le plus éloigné visible \emph{sans} accomoder. Punctum
proximum (PP) = point le plus proche visible \emph{en} accomodant.

Acuité visuelle = \(\frac{1}{\alpha}\) où \(\alpha\) = pouvoir séparateur de
l'\oe{}il. Mesurée par l'échelle de Monoyer (de loin) en 10eme (!) et de Parinaud
(de près)

\subsection{Accomodation}
\label{sec:org1c04ece}
Amplitude \dec jusque 0 (79 ans) \thus presbytie (BAV vision de près). Compensée
par verres sphériques convexes progressifs ou implant cristallinien

Sinon, \dec vision de près par : médicaments, paralysie III, maladie générale,
spasmes de l'accomodation

\subsection{Anomalies de la réfraction}
\label{sec:orgfa38e44}
Différencier maladie de l'\oe{}il/voie optique et anomalie réfraction !

Examen : réfractomètres automatiques \thus réfraction, kératométrie\footnote{Courbure de la cornée}. \footnote{Chez l'enfant, cycloplégique}

\subsubsection{Myopie}
\label{sec:orgd32cb9d}
\OE{}il trop convergent. \(\approx\) 20\% population occidentale. PR à distance finie,
PP plus proche.

Types : myopie d'indice (\inc indice de réfraction), de courbure (courbure
cornée \inc) ou axile (longueur axiale \inc)

Myopie
\begin{itemize}
\item faible : < 6 dioptrie
\item forte : > 6 dioptries ou longueur axiale \(\ge\) 26mm. Héréditaire, \(\in\) [1/10,
5/10] \emph{après} correction. Complications : glaucome chronique à angle ouvert,
cataracte, décollement de la rétine++
\end{itemize}

Correction : verres sphériques concaves.

Chirurgie par photoablation au laser ecited dimer\footnote{Abrasion épithelium cornéen ou volet dans cornée.}. Chirugie du cristallin
possible.

\subsubsection{Hypermétropie}
\label{sec:orgf1bba36}
Fréquent (enfant++). Pas assez convergent. PR = virtuel à l'arrière. Correction
par verres sphériques convexes, lentilles ou chir.

\subsubsection{Astigmatisme}
\label{sec:org826084f}
Rayons de courbures différents pour les méridiens. Régulier si 2 méridiens
principaux \bot.

1 point à l'infini = 2 droite perpendiculaire (= focales) \thus myopique,
hypermétropiique ou mixtes

Correction par verres cylindriques convexes/concaves, lentilles ou chir

\section{3 Suivi d'un nourisson}
\label{sec:org546339c}
Déficits mineurs (amétropie, strabisme) ou sévères (grave !) (milieux transparents,
  malforamtion, rétinopathie, atteintes neuro centrales)

20\% d'enfant < 6 ans avec anomalie visuelle. Si non traitée, amblyopie (BAV),
définitive > 6ans !

Développement :
\begin{itemize}
\item 1ere semaine : réflexe lumière, RPM
\item 2-4e semaine : reflexe de poursuite
\item 4-12e semaine : reflexe de poursuite
\item 3e mois : vision des formes
\item 4-5e mois : coordinatio \oe{}il-tête-main
\item > 2 ans : AV mesurable
\end{itemize}
Examens obligatoires pour \{strabisme, nystagmus, anomalie organique, trouble
comportement visuel\}:
\begin{itemize}
\item dépistage anténal (écho)
\item 1ere semaine
\item 4eme mois
\item 9eme mois
\item 2 ans
\item 3-6 ans
\end{itemize}

Dépistage : leucocorie, glaucome congénital, malformations, infections
maternelles, maladies enfants secoués, rétinopathie des prématurés
\section{4 Strabisme de l'enfant}
\label{sec:org835bde2}

Position de l'\oe{}il anormale et altération vision binoculaire. Provient d'une
perturbation de la fusion.

Conséquences :
\begin{itemize}
\item Si aigu : diplopie possible. Si ancien : corrigé par cerveaux \thus vision
\end{itemize}
binoculaire non acquise si strabisme dans premiers mois de vie !. Amblyopie
possible
\begin{itemize}
\item Perturbation vision stéréoscopie (3D)
\end{itemize}

Souvent dans l'enfance. 4\% population. \textbf{Dépistage avant 2 ans}

\subsection{Dépistage}
\label{sec:orgafc9a88}
Jamais normal, toujours symptôme

Interrogatoire :
\begin{itemize}
\item date d'apparition
\item horizontal : \emph{eso-} si convergent, \emph{exo-} si divergent. Vertical : \emph{hyper-},
\emph{hypo-}. Si divergent < 9 mois, examen neuroradio
\item intermittent ?
\item \oe{}il dominé
\end{itemize}

Examen : 
\begin{itemize}
\item motilité : strabisme paralytique ?
\item segment antérieur et FO (fond d'\oe{}il) : perte transparence, patho rétinienne ? Si nystagmus : électrorétinogramme, PEV, IRM
\item réfraction sous cycloplégique : amyotropie ? Hypermétropie fréuente
\item acuité visuelle : amblyopie ? (> 2/10 entre 2 yeux)
\item mesure de l'angle de déviation (si chir), vision binoculaire (pronostique)
\end{itemize}

\subsection{Traitement}
\label{sec:org3319bc0}
Correction optique. Si amblyopie, occlusion de l'\oe{}il dominant (jusque 6-8ans)

Chir si angle résident avec correction. Correction optique après opération
\section{5 Diplopie (binoculaire)}
\label{sec:orge586fb4}
Binoculaire : disparaît à l'occlusion d'un \oe{}il\footnote{Monoculaire : cause cornéenne, irienne, cristalinienne \thus pas une urgence.}. Souvent une urgence
\danger

Noyaux des nefs oculomoteurs \(\in\) tronc cérébral - racine - troncs -
muscle. Voies supranucléaire (latéralité, verticalité), internucléraires.

\begin{table}[htbp]
\caption{Champ d'action (\danger \(\ne\) action)}
\centering
\begin{tabular}{ll}
\toprule
oblique inférieur & droit supérieur\\
\midrule
droit médial & droit latéral\\
\midrule
oblique supérieur & droit inférieur\\
\bottomrule
\end{tabular}
\end{table}

Mouvement bilatéraux : synergie des muscles

Vision binoculaire :
\begin{itemize}
\item loi de Hering : même influx nerveux pour muscles antagonistes. Loi de
Sherrington : muscle antagonistes se relachent quand muscles synergistes se
contractent.
\item si correspondance rétinienne anormale (\oe{}ux non \(\parallel\)) : diplopie
\end{itemize}

\subsection{Diagnostic}
\label{sec:org2cf1ad0}
\begin{itemize}
\item Signes fonctionnels : dédoublement toujours même direction\footnote{\danger méconnu si ptosis/\oe{}dème palpébral}, disparaît à
\end{itemize}
l'occlusion
\begin{itemize}
\item Interrogatoire : terrain, circonstance, brutal/progressif, \{douleurs, vertiges,
\end{itemize}
céphalées, nausées\}, \{horizontale, verticale, oblique\}, moment journée
\begin{itemize}
\item Attitude vicieuse ? Chercher déviation en position primaire par reflets
cornéens
\end{itemize}

Examens :
\begin{itemize}
\item motilité
\item cover-test (\oe{}il dévié puis se redresse)
\item au verre rouge (dissociation point rouge et blanc)
\item test de Lancaster (superposer flèche de couleurs différente) \thus diagnostic
paralysie oculomotrie
\item RPM, inégalité pupillaire
\end{itemize}

\subsection{Sémiologie}
\label{sec:org9139943}
\begin{itemize}
\item Paralysie du III : totale (ptosis total, mydriase aréflective, 0 accomodation)
\end{itemize}
ou partielle
\begin{itemize}
\item Paralysie du IV : diplopie verticale oblique
\item Paralysie du VI : convergence \oe{}il atteint, déficit abduction
\item Formes particulières :
\begin{itemize}
\item paralysie supranucléaire : sd Foville (latéralité), sd Parinaud (verticalité
et cv \thus pinéalome++)
\item paralysie intranucléaire : ophtalmoplégie intranucléaire (parallélisme OK mais
déficie adduction) \thus SEP
\item paralysie intraxile : \{fonction et diplopie, diplopie et signe neuro
controlatéraux\}
\end{itemize}
\end{itemize}

\subsection{DD}
\label{sec:org15fcf33}
Diplopie monoculaire, simulation, hystérie

\subsection{Étiologie}
\label{sec:orgb94a861}
\begin{itemize}
\item Traumatique : fracture du plancher de l'orbite (élévation globe douleureuse),
hémorragie méningé traumatique
\item Tumeurs : HTIC, de la base du crâne
\item Vasculaires : AVC, insuf vertébrobasilaire, \textbf{anévrime
intracranien} (Y penser si atteinte partielle, signes
pupillaires, sujet jeune, 0 FR vasc, céphalée \thus angioscanner urgence
\skull)A, fistule carotidocaverneuse
\item Avec exophtalmie : Basedow, tumeurs de l'orbite
\item douleureuse : penser anévrisme intracrânnien, dissection carotidienne, fistule
carotidocaverneuse = urgence \skull. Maladie de Horton. Sd Tolosa-Hunt
\item SEP : paralysie VI, ophtalmoplégie internucléaire
\item Myasthénie : diag = \{test Prostigmine, Ac anti récepteur acétylcholine, électromyographie\}
\end{itemize}
\section{6 \OE{}il rouge/douloureux}
\label{sec:org9302011}
\subsection{Examen}
\label{sec:orgc3a3a6e}
Interrogatoire : mode d'apparition, douleur superficielle/profonde, BAV
   ?, ATCD ophtalmo et généraux, signes locaux associés

Examen à lampe à fente (bilatéral) :
\begin{itemize}
\item acuité visuelle
\item conjonctive : rougeur en nappe hémorragique (hémorragie
sous-conjonctivale\footnote{Cherche plaie conjonctivale !}, diffuse (conjonctivite), secteur (épisclérite),
cercle (kératite aigüe, uvéite antérieure)
périkératique
\item cornée : perte de transparence, dépôts rétro-cornéens
\item collyre à fluorescine pour ulcération cornéenne : unique (trauma), localisée
avec zone blanche (kératite bactérienne), dendritique (kératite herpétique),
petites ed disséminées (kératite à adénoviruse, sd sec oculaire, corps
étranger)
\item iris et pupille : synéchie iridocristallinienne (uvéite), atrophie iris
(herpès), myosis (kératite aigùe, uvéite aigǜe), semi-mydriase
aréflexique (glaucome aigü)
\item chambre antérieure : étroite, plate (glaucome aigù, plaie perforante), signes inflammatoires
\item tonus oculaire : hypertonie (glaucome aigu par fermeture de l'angle, glaucome
néovasculaire), hypotonie (plaie oculaire transfixiante)
\item conjonctive palpébrale : follicules (conjonctivite virale), papilles
(conjonctivite allergique), corps étranger
\item FO
\end{itemize}

\subsection{Étiologies}
\label{sec:org6b1ef96}
\OE{}il rouge, non douloureux, sans BAV
\begin{itemize}
\item hémorragie sous-conjonctivale spontanée : chercher HTA, trouble
coagulation. Penser corps étranger, plaie sclérale \danger
\item conjonctivite : "grain de sable", prurit
\item conjonctivite  bactérienne : sécrétions mucopurulentes (paupières collées le
\end{itemize}
matin). Germe Gram+. \emph{Ttt} : hygiène des mains, lavage sérum phy, collyre
antiseptique (pas forcément ATB !!)

\OE{}il rouge unilatéral, douloureux, sans BAV
\begin{itemize}
\item épisclérite (sous conjonctive) : rougeur disparaissant avec collyre
vasoconstricteur. \emph{Ttt} corticothérapie locale
\item sclérite : douleur \inc mobilisation du globe. Ne disparait pas au
collyre. Cherche maladie de système (articulaire, vasc, granulomateuse,
infectieuse). \emph{Ttt} AINS générale et cause.
\end{itemize}

Yeux rouges bilatéraux, douloureux, sans BAV
\begin{itemize}
\item conjonctivite virale : fréq++, contagieux. Sécrétion claires, ADP prétragienne
douloureuses à palpation. \emph{Ttt} inutile
\item conjonctivite allergique : terrain, prurit, chemosis\footnote{\OE{}dème conjonctival}, sécrétion claire,
volumineuses papilles. \emph{Ttt} bilan allergique, évition, collyre antiallergique
\item conjonctivite à \bact{chlamydia} : tiers-monde++
\item sd sec oculaire : fréq++. Diag = test Schirmer (quantité sécrétion lacrymale),
qualité film lacrymal, surfarce cornéenne, surface conjonctivale\}. Cause :
involution (âge), sd Gougerot-Sjögren. \emph{Ttt} substituts lacrymaux, évictions
facteurs irritants, occlusion points lacrymaux
\item autres : Basedow, malpositions palpébrale, conjonctivite d'irritation
\end{itemize}

\OE{}il rouge, douloureux, BAV
\begin{itemize}
\item Kératite aigüe : BAV, douleurs
superficielles importantes, larmoiement, photophobie,
blépharospasme. Érosion/ulcérationsc cornée, \dec transparence cornée,
cercle périkératique
\begin{itemize}
\item kératite à adénovirus : petite ulcérations disséminées. Évolution favorable
(?)
\item kératite herpétique : ulcération en "feuille de fougère". \emph{Ttt} :
valaciclovir \textpm{} aciclovir en pommadex. Jamais de corticothérapie locale
sans avoir élimine une ulcération cornéenne \skull
\item kératite zostérienne : zona ophtalmique \thus (?) kératite superficielle ou
neuroparalytique (grave). 9Ttt/ : valaciclovir et protecteurs locaux
cornéens
\item kératite bactérienne, parasitaire, mycosique : plage blanchâtre. Prélèver
sur l'abcès. \emph{ttt} collyre ATB (si important : "collyre fortifiés"). \danger
évolution : endophtalmie, perforation cornéenne, taie cornéenne cicatricielle
\item kératite sur sd sec
\item kératite d'exposition (paralysie faciale) : \emph{ttt} protecteurs cornéens en
prévention, tarsorraphie (=suture)
\end{itemize}
\item Uvéite antérieure : cercle périkératique, transparence cornée OK, myosis,
(synéchies iridocristalliniennes ou iridocornéennes), Tyndall, précipités
rétro-cornéens. FO systématique ! 
\begin{itemize}
\item Causes : inconnue, spondylarthrite ankylosante (diag = \{sacro-iléite, rachis,
Ag HLA B27), uvéite herpétique, arthrite juvénile idiopathique, sarcoïdose,
Behçet, lupus erythémateux disséminé.
\item \emph{Ttt} collyre mydriatique, collyres corticoïdes
\end{itemize}
\item Glaucome aigu par fermeture de l'angle : rare, pronostic sévère.
\begin{itemize}
\item prédisposition anatomique, pendant une mydriase
\item humeur aqueuse ne peut plus passer dans la chambre antérieure, s'accumule
dans champre postérieure et bloque le trabeculum.
\item signes fonctionnels .: douleurs profondes++ irradiant dans trijumeau,
souvent nausées, vomissement, BAV
\item examen : douleurs intenses, \oe{}il rouge, transparence cornée \dec diffuse,
semi-mydriase aréflexique, angle iridocornéen fermé, hypertonie oculaire++
\item cécité en qq jours sans ttt \danger
\item \emph{ttt} = urgence \skull : antalgique, inhibiteurs de l'anhydrase
carbonique \footnote{\dec production d'humeur aqueuse.}, solutés hyperosmolaires\footnote{Déshydratation du vitré}, collyre hypotonisants,
collyres myotiques
\item post-crise : iridotomie périphérique sur \textbf{les deux yeux} (laser ou chir)
\end{itemize}
\item Glaucome néovasculaire : VEGF crée néovaisseux qui empêche la résorption de
l'humeur aqueuse
\begin{itemize}
\item néovaisseaux sur l'iri
\item \emph{Ttt} hypotonisants locaux et généraux, photocoagulation ou injection anti-VEGF
\end{itemize}
\item Endophtalmie post-opératoire : douleur intense, \oe{}dème palpébral, hyalite
\end{itemize}
\section{7 Altération de la fonction visuelle}
\label{sec:orgc359fde}
\subsection{Examen}
\label{sec:orgda34b73}
Interrogatoire : BAV objective, altération CV  (myodésopsies, phosphènes,
métamorphopsies, éclipse viusuelle (qq secondes) ou cécité monoculaire
transitoire (qq min-heures), aura visuelle), installation, unilatéral ?, douleur
? ATCD, ttt, traumatisme ?

Examen ophtalmo : AV (avec correction), RPM, segment antériur, tonus oculaire,
\{cristallin, vitré, rétine, vaisseux, nerf optique\}

\subsection{BAV progressive \footnote{Cf chapitre suivant pour BAV brutale}}
\label{sec:org024b5db}
Si améliorié par correction optique, trouble de réfraction. Sinon 

\subsubsection{Transparence anormale}
\label{sec:org4d8220c}
Cataracte : BAV bilatérale, photophobie, myopie d'indice, diplopie
monoculaire. Perte de transparence du cristallin (opalescent). Étiologie : âge
surtout. \emph{Ttt} : chir

Autres : cornées, vitré (hyalite des uvéite)

\subsubsection{Atteinte nerf optique}
\label{sec:org0ca3142}
Glaucome chronique à angle ouvert (GCAO) : longtemps asymptomatique. Diag =
\{\inc tonus oculaire, altération CV, excavation glaucomateuse de la
papille\}. \emph{Ttt} : collyre hypotonisants, trabéculoplastie (laser/chir)

Autres : neuropathie toxiques (alcool-tabac, médic), héréditaires, compressive

\subsubsection{Atteinte de la rétine/macula}
\label{sec:org8b92f89}
Dystrophies rétiniennes héréditaires :
\begin{itemize}
\item maculoopathies héréditaires : maladie de Stargardt = débute vers 7-12 ans,
1/10e en fin d'évolution (!), en "\oe{}il de b\oe{}uf"
\item rétinopathies pigmentaires : héméralopie, rétrécissement progressif du CV
(débute dans l'enfance), aspect réticulé "en ostéoblastes"
\end{itemize}
"Interface vitréomaculaire" = séparation vitré-région maculaire
\begin{itemize}
\item membranes épi-/pré-maculaire (par tissu fibreux). "reflet" cellophane". OCT
maculaire. Chir possible
\item trous maculaires : OCT
\end{itemize}
Dégénérescence maculaire liée à l'âge 

\OE{}dème maculaire : en "pétales de fleur" si important. Causes : 
\begin{itemize}
\item rétinopathie diabétique : \emph{ttt} : injection IV anti-VEGF ou corticoïdes
\item occlusion veine centrale de la rétine : \emph{ttt} idem
\item chir cataracte
\item uvéite postérieures : \emph{ttt} : cause ou corticoïdes retard
\end{itemize}

Maculopathies toxiques aux antipaludéens de synthèse : potentiellement cécité
irréversible. Pas avant 5 ans ? Commence par périfovéolopathie \thus arrêt
immédiat du ttt \danger

\subsection{Altération du CV}
\label{sec:org1645999}
= altération vision périphérique

\subsubsection{Affections rétiniennes}
\label{sec:org0286aec}
Scotomes (para)centraux , déficits périphériques

\subsubsection{Atteinte des voies optiques}
\label{sec:org86bb657}
Atteinte nerf optique : cécité unilatérale (trauma, tumeur) 
\begin{itemize}
\item scotome central unilatéral ou caecocentral uni-/bi-latéral
\item déficicit fasciculaire possible
\item déficit altitudinal si neuropathie optique ischémique antérieure (NOIA)
\item étiologies :
\begin{itemize}
\item SEP (névrite optique rétrobulbaire)
\item NOIA
\item toxiques et métabolique : bilatéral, progressive. \{alcool-tabac, médicament,
professionnelle, métabolique\}, tumoral (tumeurs intraorbitaires, étage
antérieur du crâne)\}
\end{itemize}
\end{itemize}

Lésion du chiasma optique : hémianopsie/quadranopsie bitemporale (décussation
!). Étiologies : adénome hypophyse surtout

Lésions rétrochiasmatique : hémianopsie latérale homonyme. Si atteinte des
radiaton optiques, quadranopsie latérale homonymes. Étiologies vasc, tumoral,
trauma

Cécité corticale : bilatérale, brutale. Examen ophtalmo OK, RPM OK,
désorientation, hallucination visuelles, anosognosie
\section{8 Anomalies de la vision d'apparition brutale}
\label{sec:org28f83fc}
\subsection{Diagnostic}
\label{sec:org7c5dcb6}
Interrogatoire : BAV ? altération CV ? myodésopsies,  phosphènes,
métamorphopsies ? rapidité, latéralité, type de douleurs, ATCD, ttt, trauma ?

Examen ophtalmo ( 2 yeux) : AV, RPM, segment antérieur, tonus oculaire, FO

\subsection{Étiologie}
\label{sec:orgf4871b5}
BAV, \oe{}il rouge et douloureux 
\begin{itemize}
\item kératite aigüe : douleur superficielles importantes, photophobie,
blépharospasme. \dec transparence cornée, cercle périkératique, ulcération(s)
cornéenne(s)
\item glaucome aigü par fermeture de l'angle : douleurs profondes, intense,
irradiant dans trijumeaux. \inc\inc tonus oculaire ("bille de bois" à la
palpation)
\item uvéites 
\begin{itemize}
\item antérieure aigüe : BAV, douleurs, cercle périkératique, myosis par synéchies
iridocristalliniennes. Tyndall, chercher uvéite postérieure
\item postérieure : toxoplasmose oculaire le plus souvent. Myodésopises, BAV. FO =
foyer blanchâtre puis cicatrice. \emph{Ttt} antiparasitaire si AV menacée
\end{itemize}
\item autres  : glaucome néovasculaire, endophtalmie (contexte post-op)
\end{itemize}

BAV, \oe{}il blanc indolore
\begin{itemize}
\item FO non visible 
\begin{itemize}
\item hémorragie intravitréenne : précédée d'une "pluie de cendre", BAV
variable. Écho. B pour éliminer un décollement de la rétine. Causes :
rétinopathie diabétique proliférante, occlusions ischémique de la veine
centrale de la rétine, déchirure rétinienne, sd de Terson\footnote{Hémorragie itnravitréen uni-/bi-latérale et hémorragie méningée par
rupture d'anévrisme intracrânien}, plaie
perforante
\item uvéite intermédiaire (dans le vitré) : cellules inflammatoires
\end{itemize}
\item FO visible anormal
\begin{itemize}
\item occlusion de l'artère centrale de la rétine : BAV brutale, mydriase
aréflexique et RPM direct aboli, \dec diffuse du calibre artériel, \oe{}dème
blanc rétinien ischémique de la rétine (macula rouge cerise)
\item occlusion de la veine centrale de la rétine : BAV variable, \{\oe{}dème
papillaire, hémorragie rétiniennes disséminées, nodules cotonneux,
tortuosités veineuse\}. Préciser si ischémique
\item DMLA : BAV et métamorphopsies brutales, décolleme exsudatif de la rétine
maculaire \textpm{} hémorragies, exsudats profonds
\item décollement de la rétine rhegmatogène : après une déhiscence, le vitré va
sous la rétine. 
\begin{itemize}
\item causese : idiopathique (âgé), myopie (forte), chir cataracte
\item évolution spontanée = cécité
\item \emph{ttt} chir (semi-urgence \danger)
\item clinique : myodésopsies, phosphènes, amputation CV périphérique, BAV
\item diag par FO : rétine en relief, mobile, avec des plis
\item toujours examiner \oe{}il controlatéral (ttt préventif par photocoagulation)
\danger
\end{itemize}
\item neuropathie optique ischémique antérieure : BAV unilatérale brutale, \dec
RPM direct, \oe{}dème papillaire, déficit fasciculaire pour CV. Cause : surtout
artériosclérose mais penser à Horton (urgence \skull)
\end{itemize}
\end{itemize}
FO visible normal
\begin{itemize}
\item névrite optique rétrobulbaire : BAV unilatérale progresse en qq jours (!),
douleur \inc mouvement oculaires, RPM direct \dec, FO normal, scotome
(caeco)-central
\item atteinte des voies (rétro)chiasmatiques : tumorale (si progressive), vasc (si brutale)
\end{itemize}

Anomalies transitoire 
\begin{itemize}
\item cécité monoculaire transitoire (qq minute) = amarause. FO pour embole
rétinien. Urgence \skull. Cherhe athérome carotidien, cardiopathie embolinogène
\item insuf vertébrobasilaire, éclipses visuelle, scotome scintillant
\end{itemize}
\section{9 Prélèvement de cornée}
\label{sec:orgfffa8ce}
Le médecine prélèveur \(\ne\) celui qui a fait le constat de mort

Sérologies à faire : HIV, HTLV, hépatite B et C, syphils
CI :
\begin{itemize}
\item locale : chir sur segment antérieure, uvéite, conjonctivite, tumeur,
rétinoblastome, mélanome choroïdien
\item infectieuses (Sida, rage, Creutzfeld-Jakob, héatite..),
\item neuro inexpliquée, démence
\end{itemize}

Prélèvement \textbf{in situ}

\section{10 Greffe de cornée}
\label{sec:orge523879}
Couches (depuis l'extérieur) = épithelium, couche de Bowman, stroma, membrane
  de Descemet, endothelium

Techniques
\begin{itemize}
\item kératoplastie transfixiante = toute les couches
\item kératoplastie lamellaire antérieure = épithelium, Bowman et stroma seulement
\item kératoplastie endothéliale : membrane de Descemet et endotheliale seulement.
\end{itemize}

Indications : trauma perforant de la cornée, brûlures chimiques, dégénérescene
cornéenne (kératocône), kératite (herpétique), dystrophie bulleuse chez âgé

Bon pronostic dans \(\frac{2}{3}\)

Complications rare (retard d'épithélialisation, \oe{}dème cornéen précoce, rejet
immuntaire, récivide de maladie causale, hypertonie oculaire, astigmatisme post-op
\section{11 Traumatismes oculaires}
\label{sec:orgb82fb2a}
\subsection{Globe fermé (contusions)}
\label{sec:orgca533bb}
Dangerosité inversement \(\propto\) taille agent.

Interrogatoire : douleurs, AV, heure dernier repas, lésions associées.

Examen :
\begin{itemize}
\item contusions du segment antérieur 
\begin{itemize}
\item cornée : \emph{ttt} lubrifiant cicatrisant
\item conjonctive : plaie ou hémorragie sous-conjonctivale. Toujours chercher plaie
sclérale, corps étranger
\item chambre antérieure : hyphéma. Résorption spontanée
\item iris : iridodyalise \footnote{Désinsertion de la base de l'iris}, rupture sphincter irien, mydriase post-trauma
\item cristallin : (sub)luxation, cataracte contusive (plusieurs mois après\ldots{})
\item hypertonie oculaire : si lésions de l'angle iridocornée, hyphéma, luxation
antérieure du cristallin
\end{itemize}
\item contusions du segment postérieur
\begin{itemize}
\item \oe{}dème rétinien du pôle supérieur : guérison spontanée
\item hémorragie intravitréenne : résorption spontanée. Écho B si décollement de
rétine (DR) suspecté
\item déchirure rétiniennes périph. Photocoagulation prophylactique possible
(contre DR)
\item rupture de la choroïde : BAV définite si macula
\end{itemize}
\end{itemize}

\subsection{Globe ouverts}
\label{sec:org746811d}
Rupture du globe : pronostic plus péjoratif si postérieur

Trauma perforant : 
\begin{itemize}
\item AVP, accidents domestique, bricolage, agression.
\item larges, mauvais pronostic ou petite plais (cornée ou sclère)
\item risque : méconnaître plaie, corps étranger. Si doute, scanner !
\end{itemize}

\subsection{Corps étrangers}
\label{sec:org4067ffe}
Diagnosic évident : 
\begin{itemize}
\item superficiel 
\begin{itemize}
\item circonstance, symptôme de conjonctivite ou kératite
\end{itemize}
superficielle. Corps étranger souvent visible
\begin{itemize}
\item bon pronostic, ttt lubrifiant et antiseptique/ATB local
\end{itemize}
\item intraoculaire : circonstance, porte d'entrée et trajet visible
\end{itemize}
Délicat si trauma non remarqué, pas de porte d'entrée, trajet et corps non visible

Examens : TDM si doute. Pas d'écho B si transfixiante. Pas d'IRM \danger

Complications (si intraoculaire) : endophtalmie (grave++), DR, cataracte traumatique

Complications tardives : ophtalmie sympathique \footnote{Uvéite auto-immune sévère}, sidérose, chalcose
\section{12 Brûlures oculaires}
\label{sec:orgcaa2dd3}
Accident industriels (graves), domestiques, aggression

Brûlures 
\begin{itemize}
\item thermique : peu grave (brûlure par cigarette). Cicatrisation rapide, sans séquelles
\item acides : gravité modérée-moyenne, grave si très concentré
\item basiques : grave !
\end{itemize}

Classification de Roper-Hall
\begin{enumerate}
\item bon pronostic
\item bon pronostic : opacité cornéenne mais détails iris, ischémie < 1/3 \diameter
\item pronostic réservé : désépithélialisation cornéenne totale, opacité cornéenne
, ischémie \(\in\) [1/3, 1/2] \diameter
\item pronostic péjoratif
\end{enumerate}

Ttt d'urgence par lavage (20-30min) par sérum phy (ou eau), puis collyre
corticoïde ASAP

Autres :
\begin{itemize}
\item brûlures des UV (ski, UV)
\item soudure à l'arc sans lunette
\item phototraumatisme (éclipse) : BAV définitive !
\end{itemize}

\section{13 Cataracte}
\label{sec:org48d3746}
Déf: opacification (partielle ou non) du cristallin. 

\subsection{Diagnostic}
\label{sec:orga771d44}
Découvert sur BAV (progressive, vision de loin), photophobie, (diplopie
monoculaire), jaunissement

Examen clinique :
\begin{itemize}
\item interrogatoire : âge, profession, ATCD (diabète, corticoïdes), myodésopsies,
métamorphopsies
\item AV (\oe{}il par \oe{}il, bionculaire)
\item lampe à fente : 
\begin{itemize}
\item cristallin (caracte \{nucléaire, sous-capsulaire postérieure, corticale, totale\})
\item éliminer autre patho (cornée, iris, vitré, rétine (DMLA, glaucome)
\item mesure tonus oculaire (hypertonie, glaucome)
\end{itemize}
\end{itemize}

Diagnostic clinique mais en complémentaire :
\begin{itemize}
\item écho en mode B si décollement de rétine/tumeur intraoculaire
\item pour le cristallin artificiel : kératométire, longeur axiale
\end{itemize}

\section{14 Glaucome chronique}
\label{sec:orgf1ea443}
= Glaucome primite à angle ouvert (GPAO) Neuropathie optique progressive altération
fonction/structure.
FR : âge, hypertonie oculaire (\(\ne\) cause !!), ATCD familaux directs, noirs
d'origine africaine, myopie

Physio : perte accélére des fibres optiques liée à l'âge

\subsection{Formes cliniques :}
\label{sec:orgbaa4b39}
GPAO à préssion élevée (> 21 mmHg) (70\% des pop. occidentales)
\begin{itemize}
\item anomalies de structures visibles cliniquement : papille = \{\dec surface,
hémorragies péripapillaires en flammèches, atrophie péripapillaire \(\beta\), \inc
excavation papillaire\}. Rapport de taille entre papilles des 2 yeux > 0.2
\thus suspect
\item autres anomalies de structures : OCT
\item anomalies de foncton : champ visuel mais AV touchée très tardivement
\end{itemize}

GPAO à pression normale (70\% pop. asiatique) : plutôt femmes, migraine, acrosyndromes

DD : 
\begin{itemize}
\item hypertonie oculaire : pression intraoculaire > 21mmHg, angle ouvert (gonio),
\(\emptyset\) neuropathie optique. Pas forcément de ttt.
\item glaucomes à angles ouvert secondaires
\item glaucomes par fermeture de l'angle
\item crise aigüe de fermeture de l'angle : douleur, urgence \skull
\item neuropathie optiques non glaucomateuses
\end{itemize}

\subsection{Traitement}
\label{sec:org076c3b9}
Dépistage : seulement ATCD familaux de GPAO, myopie, > 40 ans

Ttt : \dec pression intra oculaire

Médicaments :
\begin{itemize}
\item à vie
\item 1ere intention : collyre à base de prostaglandines \footnote{EI : rougeur, irrit. oculaire. Puis iris plus sombre, \inc pousse cils.}(\inc élimination humeur
aqueuse) ou collyre betabloquant
(\dec sécrétion humeur aqueuse)
\item association possible mais \(\le\) 3
\item rarement : acétazolamide par voie générale
\end{itemize}

Trabéculoplastie au laser = photocoagulation de l'angle. Effet modeste, non durable.

Chir : trabéculectomie. Complications (rare) = cataracte, hyoptonie précoce
avec décollement choroïdien, infection globe oculaire. Principal échec :
fermeture de la voie par fibrose sous-conjonctivale.

\section{15 Dégénérescence maculaire liée à l'âge}
\label{sec:org294fc8a}
Atteinte de la macula chez > 50 A. Débutante/intermédaire : drusent, altération
pigmentaire, AV normale ou peu \dec. Évoluée : atrophique ou exsudative

Prévalence : 18\% après 50 ans, 37\% à 85 ans.

FR : âge, pop européennes, polymorphisme facteur H du complément, tabac, régime
pauvre en anti-oxydant/riches en (acide gras saturés et cholestérol)

\subsection{Diagnostic}
\label{sec:org78c8036}
\begin{itemize}
\item Mesure AV (loin et près, recherche scotome central ou métamorphopsies (grille
\end{itemize}
d'Amsler)
\begin{itemize}
\item FO (drusen, altération pigmentaires, atrophie épithelium pigmentaire,
\end{itemize}
forme exsudative [cf /infra/].
\begin{itemize}
\item OCT : suivi, ou diagnostic si + FO
\end{itemize}

\subsection{Formes cliniques}
\label{sec:orgd7483c8}
\begin{itemize}
\item débutante : \emph{drusen} = résidus de phagocytose des photorécepteurs. Petites
lésions profondes jaunâtre. OCT
\item atrophique : disparition de l'épithelium pigmentaire. FO : atrophie de
l'épithelium pigmentaire et choroïde. BAV sévère avec scotome centrale
\item exsudative (néovaisseaux sous rétine) : \oe{}dème intrarétinien, hémorragies,
décollement maculaire exsudatif (BAV, métamorphopsies brutales). Sans ttt :
BAV sévère, scotome central définitif. BAV chez drusen = urgence \skull
\end{itemize}

\subsection{Ttt}
\label{sec:org474fb6d}
\begin{itemize}
\item débutante : vit E, C, zinc, lutéine, zéaxantine
\item atrophique : \(\emptyset\)
\item exsudative : injection d'anti-VEGF : ranibizumab,
aflibercet. bévacizumab. Stoppent néovaisseux, font régresser l'\oe{}dème. 40\%
ont amélioration visuelle à 2 ans.
\end{itemize}
Laser possible mais risque thrombose
\section{16 Occlusions artérielles rétiniennes}
\label{sec:orgfd77ec0}
Artères ciliaires postérieure alimente les couches profondes (épithelium
pigmentaire de la rétine, photorécepteurs). 
Artère centrale de la rétine = couche internes (cellules bipolaires,
ganglionnaires, fibres optiques).

Arrêt circulatoire \thus lésions définitives en 90min \danger

\subsection{Occlusion de l'artère centrale de la rétine}
\label{sec:org9488232}
Diagnostic : 
\begin{itemize}
\item BAV brutale (amaurose transitoires précédentes possibles)
\item Oeil blanc indolore, AV effondrée, mydriase aréflective (RPM direct aboli)
\item FO : dans les heures : oedème ischémique rétinien blanchâtre, tache "rouge
cerise de la macula")
\end{itemize}

Étiologie : 
\begin{itemize}
\item embolies : athérome carotidien++, cardiopathie embolinogène, ( fractures des os
lng ou emboles tumoraux)
\item thromboses : maladie de Horton (urgence \skull, à rechercher), (maladies de systèmes)
\item troubles coagulation : anomalie primitive, sd antiphospholipides, hyperhomocystéinémie
\end{itemize}

Évolution spontanée quasiment toujours défavorable

CAT : urgence \danger \skull (fonction visuelle et patho sous-jacente)
\begin{itemize}
\item bilan étio : athérome, carotidien, cardiopathie embolinogène, dissection
carotidienne (si jeune), horton
\item ttt décevant : hypotonisant (acétazolamide IV/per os), vasodilatateur,
anticoagulant si  bespoin fibrinolytique
\item bilan cardio
\item Aspirine dans tous les cas. Si jeune, bon état général, ttt max.
\item selon étio : antiagrégant plaquettaire (athérome carotidien), anti-vit K (emboles
cardiaques), endartériectomie
\end{itemize}

\subsection{Occlusion de branche l'artère centrale de la rétine}
\label{sec:org10487c4}
Amputation du CV. BAV possible. FO : \oe{}dème rétinie ischémique en secteur

Évolution : \inc AV en qq semaines, pronostic visuel bon mais amputation
perisiste.

Même étio, sausf Horton. Même ttt.

\subsection{Nodules cotonneux}
\label{sec:orgd74c8d4}
Occlusion d'artérioles rétiniennes précapillaire \thus nodules cotonneux =
petites lésions blanches superficielles d'aspect duveteux.

Étio : HTA, diabète, occlusions veineuses rétiniennes, sida, lupus érythémateux
disséminé, périartérite noueuses, embolies graisseuses, pancréatite aigüe, sd Purtscher
\section{17 Occlusions veineuses rétiniennes}
\label{sec:org925f290}
\subsection{Occlusion de la veine centrale de la rétine}
\label{sec:org41dd719}
SF : vision trouble brutalement, BAV variable, \oe{}il blanc indolore
FO : 
\begin{itemize}
\item diagnostic = \{\oe{}dème papillaire, veines rétitiennes tortueuses et dilatées,
hémorragies sur la surface rétitienne, nodules cotonneux\}
\item formes non ischémiques (freq) : AV > 2/10, hémorragies en flammèches, ischémie
peu étendue
\item formes ischémiques\footnote{Différence avec l'angiographie fluorescinique} : AV < 1/10, réflexe pupillaire direct diminué, hémorragies
plus volumineuses, en tache
\end{itemize}

Suivi par OCT

Étiologie inconnue 
\begin{itemize}
\item mais > 50 ans avec FR vasc \thus recherche \{tabac, HTA,
\end{itemize}
diabète, hypercholestérolémie, SAS\}, hypertonie oculaire++
\begin{itemize}
\item si < 50 ans, 0 FR ou OVCR bilatérale, chercher anomalie primitive de la
coagulation, sd antiphospholipides, hyperhomocystéinémie, contraception, hyperviscosité
\end{itemize}

Évolution :
\begin{itemize}
\item formes non ischémiques : normalisation AV en 3-6 mois. Sinon : conversion en
forme ischémique, peristance d'un \oe{}dème maculaire cystoïde (BAV permanente !)
\item formes ischémiques : 
\begin{itemize}
\item pas de récupération fonctionnelle.
\item Pire complication = néovascularisation irienne \thus progression rapide vers glaucome
\end{itemize}
néovasculaire (3 mois) \thus prévention par photocoagulation panrétinienne
\begin{itemize}
\item néovascularisation prérétiniennes ou précapillaire (hémorragie intravitréenne)
\end{itemize}
\end{itemize}

Ttt 
\begin{itemize}
\item formes non ischémique : injection d'anti-VEGF/triamcinolone si \oe{}dème
maculaire cystoïde avec BAV persistante. Surveillance tous les mois
\item formes ischémiques : photocoagulation panrétinienne (PPR) pour éviter glaucome
néovasculaire
\item glaucome néovasculaire : PPR en urgence après hypotonisant local \danger
\end{itemize}


\subsection{Occlusion de branche veineuse rétinienne}
\label{sec:org626a8e1}
Identique à OVCR mais territoire plus limité.
Signe du croisement (cf chap 23) \thus > 60 ans, FR d'artériosclérose

Clinique : BAV variable, FO identique OVCR

Évolution favorable en majorité. Défavorable si maculopathie ischémique,
\oe{}dème maculaire persistante, néovaisseaux prérétiniennes \thus hémorragie du
vitré mais \textbf{pas} de GNV

Ttt similaire. 
\begin{itemize}
\item Si \oe{}dème maculaire peristant : injection intravitréenne
\item photocoagulation maculaire en grille (si \(\ge\) 3 mois, \oe{}dème maculaire
perisistante, AV ɇ 5/10)
\item photocoagulation sectorielle
\end{itemize}
\end{document}
