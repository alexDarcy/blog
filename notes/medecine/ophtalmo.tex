% Created 2018-11-26 Mon 17:08
% Intended LaTeX compiler: lualatex
\documentclass[11pt]{article}
\usepackage[hidelinks]{hyperref}
\usepackage{booktabs}
\documentclass{article}
\usepackage[hidelinks]{hyperref}
\usepackage{longtable}
\usepackage{booktabs}
%\usepackage[draft]{graphicx}
\usepackage{graphicx}
\usepackage{fancyhdr}
% French
\usepackage[T1]{fontenc}
\usepackage[francais]{babel}
\usepackage{caption}
\usepackage[nointegrals]{wasysym} % Male-female symbol
% Smaller marign
\usepackage[margin=2.5cm]{geometry}
\usepackage{latexsym}
\usepackage{subcaption}
%-------------------------------------------------------------------------------
% For graphs
\usepackage{tikz}
\usepackage{tikzscale}
\usetikzlibrary{graphs}
\usetikzlibrary{graphdrawing}
\usetikzlibrary{arrows,positioning,decorations.pathreplacing}
\usetikzlibrary{calc}
\usegdlibrary{trees, layered}
\usetikzlibrary{quotes}
%-------------------------------------------------------------------------------
% No spacing in itemize
\usepackage{enumitem}
\setlist{nolistsep}
% tightlist from pandoc
\providecommand{\tightlist}{%
  \setlength{\itemsep}{0pt}\setlength{\parskip}{0pt}}
 % Danger symbol (need fourier package)
\newcommand*{\TakeFourierOrnament}[1]{{%
\fontencoding{U}\fontfamily{futs}\selectfont\char#1}}
\newcommand*{\danger}{\TakeFourierOrnament{66}}
% Skull (need Symbola font)
\usepackage{amsmath,fontspec,newunicodechar}
\newfontface{\skullfont}{Symbola}[Scale=MatchUppercase]
\NewDocumentCommand{\skull}{}{%
  \text{\skullfont\symbol{"1F571}}%
}
% Hospital sign
\usepackage{fontspec} % For fontawesome
\usepackage{fontawesome}
% Itemize in tabular
\newcommand{\tabitem}{~~\llap{\textbullet}~~}
% No numbering
\setcounter{secnumdepth}{0}
% Set header
\pagestyle{fancy}
\fancyhf{}
\fancyhead[L]{\leftmark}
\fancyhead[R]{\thepage}
%\renewcommand{\headrulewidth}{0.6pt}
% Custom header : no uper case
\renewcommand{\sectionmark}[1]{%
  \markboth{\textit{#1}}{}}

\usepackage[linesnumbered,ruled,vlined]{algorithm2e}
\author{Alexis Praga}
\date{\today}
\title{Ophtalmologie}
\hypersetup{
 pdfauthor={Alexis Praga},
 pdftitle={Ophtalmologie},
 pdfkeywords={},
 pdfsubject={},
 pdfcreator={Emacs 26.1 (Org mode 9.1.9)}, 
 pdflang={English}}
\begin{document}

\maketitle
\tableofcontents

\input{bacteries-header}
\section{1 Sémiologie oculaire}
\label{sec:org35e1b30}
\subsection{Globe}
\label{sec:org2819582}
Membranes (d'avant en arrière)
\begin{itemize}
\item externe = \{cornée (avant), conjonctive (sur sclère antérieure), sclère\}
\item intermédiaire (uvée) = \{iris, corps ciliaires, choroïde\}
\item interne (rétine) = \{épithelium pigmentaire, neurosensorielle (photorécepteurs,
fibres optiques\}\}. NB : phototransduction = pigment visuel
(épithelium) \(\rightarrow\) photorécepteurs (bâtonnets = \{vision périphérique,
nocturne\}, cône = \{détails, couleurs\} \(\in\) macula)
\end{itemize}
Contenu : 
\begin{itemize}
\item humeur aqueuse \(\in\) chambre antérieure, évacuée dans l'angle iridocornéen. P
normale si \(\le\) 22 mmHg.
\item cristallin : biconvexe, convergente, déformation par zonule \thus accomodation
\item corps vitré : 4/5e cavité oculaire \thus segment antérieur vs postérieur
\end{itemize}
Voies optiques : 
\begin{itemize}
\item nerf optique - chiasma - bandelettes optiques - corps genouillés externes - radiations optiques - cortex occipital
\item mydriase "sensorielle"\footnote{Plus de voie afférente  \thus RPM (réflexe photomoteur) direct et
consensuel aboli pour l'oeil atteint} vs "paralytique"\footnote{Plus de voie efférente \thus oeil atteint : RPM direct aboli. Oeil sain : RPM consensuel aboli}
\item voie efférente sympathique : si atteinte, sd Claude-Bernard-Horner (myosis, ptosis)
\end{itemize}
Annexes
\begin{itemize}
\item oculomoteur : nerf IV = \{oblique supérieur\}, VI = \{droit externe\} et III =
\{droit sup, droit inf, droit médial, oblique inf\}
\item protection : paupière (muscle orbiculaire, tarse), conjonctive (face interne
paupière, globe antérieur), film lacrymal
\end{itemize}

\subsection{Examen}
\label{sec:org95b0ad9}
Interrogatoire : 
\begin{itemize}
\item baisse d'acuité visuelle (BAV), fatigue visuelle, myodésopsise,
\end{itemize}
métamorphopsies, héméralopie, scotome/amputation champ visuel, 
\begin{itemize}
\item mode d'installation : urgente si brutal
\item douleur superficielles/profondes
\item diplopie mono/bi-noculaire
\item évolution : aggration rapide = grave
\end{itemize}
Acuité visuelle : échelle de Monoyer (à 5m), de Parinaud (33cm)

Segment antérieur : "lampe à fente"
\begin{itemize}
\item conjonctive : rougeur (diffuse/localisée/culs-de-sac inférieure/avec
sécrétions/cercle périkératique/paupière), chemosis\footnote{Oedème conjonctival}
\item cornée (transparence)
\item iris : myosis/mydriase
\item chambre antérieure : inflammation (Tyndall\footnote{Cellules inflammatoires, protéines dans l'humeur aqueuse}, précipités rétro-cornéens,
synéchies iridocristaliniennes), hypopion (pus), hyphéma (sang)
\end{itemize}
Pression intraoculaire par tonomètre à air pulsé. Hypertonie si \(\ge\) 22 mmHg

Gonioscopie : angle iridocornéen

Fond d'oeil
\begin{itemize}
\item direct, indirect (apprentissage++), biomicroscopie
\item aspect : pôle postérieur = \{papille, vaisseaux rétitiens, macula\}, rétine
périphérique (si besoin)
\item lésions : microanévrisme (points rouges), hémorragie (intravitréennes,
prérétiniennes, sous-rétiniennes, intra-rétiniennes\footnote{Punctiformes, flammèches, profondes}), nodules cotonneux,
exsudats profonds, oedème papillaire\footnote{Si BAV, vasculaire probable. Sinon HTIC}
\end{itemize}

Oculomotricité

\subsection{Complémentaires}
\label{sec:org0f56592}
Fonctions visuelles
\begin{itemize}
\item Champ visuel :
\begin{itemize}
\item sensibilité lumineuse \dec périphérie, papille = zone aveugle
\item périmétrie cinétique (Goldman : trace isoptères) ou statique (dépistage
\end{itemize}
glaucome)
\item Vision des couleurs : si dépistage anomalie congénitale (planches colorées) ou
\end{itemize}
affection acquise (Farnsworth = classer couleurs). Utile pour antipaludéens de
synthèse ou \{éthambutal, isoniazide\}
Angiographie du fond d'oeils : fluorescine ou vert d'indocyanine (DMLA++)

Électrophysio : électrorétinogramme (si lésion rétiniennes étendues), potentiels
évoqués visuels (cortex occipital) (SEP++), électro-oculogramme (épithelium pigmentaire)

Echographie : mode A (longueur globe oculaire) ou B (décollement rétine, corps
étranger intraoculaire, tumeur)

Tomographie en cohérence optique (OCT) : affection maculaire (trou, DMLA),
dépistage glaucome chronique

\section{2 Réfraction}
\label{sec:orgd15c0df}
Oeil \(\approx\) 60 dioptries (cornée = 42, cristallin = 20)
Emmétrope = oeil normal. Amétrope = anomalie de réfraction

Punctum remotum (PR) : point le plus éloigné visible \emph{sans} accomoder. Punctum
proximum (PP) = point le plus proche visible \emph{en} accomodant.

Acuité visuelle = \(\frac{1}{\alpha}\) où \(\alpha\) = pouvoir séparateur de
l'oeil. Mesurée par l'échelle de Monoyer (de loin) en 10eme (!) et de Parinaud
(de près)

\subsection{Accomodation}
\label{sec:orgc445e66}
Amplitude \dec jusque 0 (79 ans) \thus presbytie (BAV vision de près). Compensée
par verres sphériques convexes progressifs ou implant cristallinien

Sinon, \dec vision de près par : médicaments, paralysie III, maladie générale,
spasmes de l'accomodation

\subsection{Anomalies de la réfraction}
\label{sec:org9aa54a0}
Différencier maladie de l'oeil/voie optique et anomalie réfraction !

Examen : réfractomètres automatiques \thus réfraction, kératométrie\footnote{Courbure de la cornée}. \footnote{Chez l'enfant, cycloplégique}

\subsubsection{Myopie}
\label{sec:orgf0b509b}
Oeil trop convergent. \(\approx\) 20\% population occidentale. PR à distance finie,
PP plus proche.

Types : myopie d'indice (\inc indice de réfraction), de courbure (courbure
cornée \inc) ou axile (longueur axiale \inc)

Myopie
\begin{itemize}
\item faible : < 6 dioptrie
\item forte : > 6 dioptries ou longueur axiale \(\ge\) 26mm. Héréditaire, \(\in\) [1/10,
5/10] \emph{après} correction. Complications : glaucome chronique à angle ouvert,
cataracte, décollement de la rétine++
\end{itemize}

Correction : verres sphériques concaves.

Chirurgie par photoablation au laser ecited dimer\footnote{Abrasion épithelium cornéen ou volet dans cornée.}. Chirugie du cristallin
possible.

\subsubsection{Hypermétropie}
\label{sec:orgdad5fe6}
Fréquent (enfant++). Pas assez convergent. PR = virtuel à l'arrière. Correction
par verres sphériques convexes, lentilles ou chir.

\subsubsection{Astigmatisme}
\label{sec:org6ead89f}
Rayons de courbures différents pour les méridiens. Régulier si 2 méridiens
principaux \bot.

1 point à l'infini = 2 droite perpendiculaire (= focales) \thus myopique,
hypermétropiique ou mixtes

Correction par verres cylindriques convexes/concaves, lentilles ou chir

\section{3 Suivi d'un nourisson}
\label{sec:org746be0d}
Déficits mineurs (amétropie, strabisme) ou sévères (grave !) (milieux transparents,
  malforamtion, rétinopathie, atteintes neuro centrales)

20\% d'enfant < 6 ans avec anomalie visuelle. Si non traitée, amblyopie (BAV),
définitive > 6ans !

Développement :
\begin{itemize}
\item 1ere semaine : réflexe lumière, RPM
\item 2-4e semaine : reflexe de poursuite
\item 4-12e semaine : reflexe de poursuite
\item 3e mois : vision des formes
\item 4-5e mois : coordinatio oeil-tête-main
\item > 2 ans : AV mesurable
\end{itemize}
Examens obligatoires pour \{strabisme, nystagmus, anomalie organique, trouble
comportement visuel\}:
\begin{itemize}
\item dépistage anténal (écho)
\item 1ere semaine
\item 4eme mois
\item 9eme mois
\item 2 ans
\item 3-6 ans
\end{itemize}

Dépistage : leucocorie, glaucome congénital, malformations, infections
maternelles, maladies enfants secoués, rétinopathie des prématurés
\section{4 Strabisme de l'enfant}
\label{sec:org20b9c61}

Position de l'oeil anormale et altération vision binoculaire. Provient d'une
perturbation de la fusion.

Conséquences :
\begin{itemize}
\item Si aigu : diplopie possible. Si ancien : corrigé par cerveaux \thus vision
\end{itemize}
binoculaire non acquise si strabisme dans premiers mois de vie !. Amblyopie
possible
\begin{itemize}
\item Perturbation vision stéréoscopie (3D)
\end{itemize}

Souvent dans l'enfance. 4\% population. \textbf{Dépistage avant 2 ans}

\subsection{Dépistage}
\label{sec:orgd63d334}
Jamais normal, toujours symptôme

Interrogatoire :
\begin{itemize}
\item date d'apparition
\item horizontal : \emph{eso-} si convergent, \emph{exo-} si divergent. Vertical : \emph{hyper-},
\emph{hypo-}. Si divergent < 9 mois, examen neuroradio
\item intermittent ?
\item oeil dominé
\end{itemize}

Examen : 
\begin{itemize}
\item motilité : strabisme paralytique ?
\item segment antérieur et FO (fond d'oeil) : perte transparence, patho rétinienne ? Si nystagmus : électrorétinogramme, PEV, IRM
\item réfraction sous cycloplégique : amyotropie ? Hypermétropie fréuente
\item acuité visuelle : amblyopie ? (> 2/10 entre 2 yeux)
\item mesure de l'angle de déviation (si chir), vision binoculaire (pronostique)
\end{itemize}

\subsection{Traitement}
\label{sec:orgb6d5055}
Correction optique. Si amblyopie, occlusion de l'oeil dominant (jusque 6-8ans)

Chir si angle résident avec correction. Correction optique après opération
\section{5 Diplopie (binoculaire)}
\label{sec:orgf2ff00e}
Binoculaire : disparaît à l'occlusion d'un oeil\footnote{Monoculaire : cause cornéenne, irienne, cristalinienne \thus pas une urgence.}. Souvent une urgence
\danger

Noyaux des nefs oculomoteurs \(\in\) tronc cérébral - racine - troncs -
muscle. Voies supranucléaire (latéralité, verticalité), internucléraires.

\begin{table}[htbp]
\caption{Champ d'action (\danger \(\ne\) action)}
\centering
\begin{tabular}{ll}
\toprule
oblique inférieur & droit supérieur\\
\midrule
droit médial & droit latéral\\
\midrule
oblique supérieur & droit inférieur\\
\bottomrule
\end{tabular}
\end{table}

Mouvement bilatéraux : synergie des muscles

Vision binoculaire :
\begin{itemize}
\item loi de Hering : même influx nerveux pour muscles antagonistes. Loi de
Sherrington : muscle antagonistes se relachent quand muscles synergistes se
contractent.
\item si correspondance rétinienne anormale (oeux non \(\parallel\)) : diplopie
\end{itemize}

\subsection{Diagnostic}
\label{sec:org7f4de1c}
\begin{itemize}
\item Signes fonctionnels : dédoublement toujours même direction\footnote{\danger méconnu si ptosis/oedème palpébral}, disparaît à
\end{itemize}
l'occlusion
\begin{itemize}
\item Interrogatoire : terrain, circonstance, brutal/progressif, \{douleurs, vertiges,
\end{itemize}
céphalées, nausées\}, \{horizontale, verticale, oblique\}, moment journée
\begin{itemize}
\item Attitude vicieuse ? Chercher déviation en position primaire par reflets
cornéens
\end{itemize}

Examens :
\begin{itemize}
\item motilité
\item cover-test (oeil dévié puis se redresse)
\item au verre rouge (dissociation point rouge et blanc)
\item test de Lancaster (superposer flèche de couleurs différente) \thus diagnostic
paralysie oculomotrie
\item RPM, inégalité pupillaire
\end{itemize}

\subsection{Sémiologie}
\label{sec:org93b154b}
\begin{itemize}
\item Paralysie du III : totale (ptosis total, mydriase aréflective, 0 accomodation)
\end{itemize}
ou partielle
\begin{itemize}
\item Paralysie du IV : diplopie verticale oblique
\item Paralysie du VI : convergence oeil atteint, déficit abduction
\item Formes particulières :
\begin{itemize}
\item paralysie supranucléaire : sd Foville (latéralité), sd Parinaud (verticalité
et cv \thus pinéalome++)
\item paralysie intranucléaire : ophtalmoplégie intranucléaire (parallélisme OK mais
déficie adduction) \thus SEP
\item paralysie intraaxile : \{fonction et diplopie, diplopie et signe neuro
controlatéraux\}
\end{itemize}
\end{itemize}

\subsection{DD}
\label{sec:org78890ac}
Diplopie monoculaire, simulation, hystérie

\subsection{Étiologie}
\label{sec:org6d4476a}
\begin{itemize}
\item Traumatique : fracture du plancher de l'orbite (élévation globe douleureuse),
hémorragie méningé traumatique
\item Tumeurs : HTIC, de la base du crâne
\item Vasculaires : AVC, insuf vertébrobasilaire, \textbf{anévrime
intracranien} (Y penser si atteinte partielle, signes
pupillaires, sujet jeune, 0 FR vasc, céphalée \thus angioscanner urgence
\skull)A, fistule carotidocaverneuse
\item Avec exophtalmie : Basedow, tumeurs de l'orbite
\item douleureuse : penser anévrisme intracrânnien, dissection carotidienne, fistule
carotidocaverneuse = urgence \skull. Maladie de Horton. Sd Tolosa-Hunt
\item SEP : paralysie VI, ophtalmoplégie internucléaire
\item Myasthénie : diag = \{test Prostigmine, Ac anti récepteur acétylcholine, électromyographie\}
\end{itemize}
\end{document}
