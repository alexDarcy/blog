\documentclass[a5paper,12pt,norsk]{article}
\usepackage[utf8]{inputenc}
\usepackage{babel,fouriernc,parskip,booktabs,array}
\usepackage[margin=1mm,portrait]{geometry}
\usepackage{enumitem}

\usepackage{pgfpages}                                 % <— load the package
  \pgfpagesuselayout{2 on 1}[a4paper,border shrink=5mm, landscape] % <— set options


\setlist{nolistsep}
\renewcommand\labelitemi{--}

% Redefine section commands to use less space
\makeatletter
\renewcommand{\section}{\@startsection{section}{1}{0mm}%
                                {-1ex plus -.5ex minus -.2ex}%
                                {0.5ex plus .2ex}%x
                                {\normalfont\large\bfseries}}
\renewcommand{\subsection}{\@startsection{subsection}{2}{0mm}%
                                {-1explus -.5ex minus -.2ex}%
                                {0.5ex plus .2ex}%
                                {\normalfont\normalsize\bfseries}}
\renewcommand{\subsubsection}{\@startsection{subsubsection}{3}{0mm}%
                                {-1ex plus -.5ex minus -.2ex}%
                                {1ex plus .2ex}%
                                {\normalfont\small\bfseries}}
\makeatother


% Don't print section numbers
\setcounter{secnumdepth}{0}


\setlength{\parindent}{0pt}
\setlength{\parskip}{0pt plus 0.5ex}

\begin{document}
\thispagestyle{empty}

\begin{center}
     \Large{\textbf{Examen clinique}} \\
\end{center}

\subsection{ATCD}
\label{sec:org32b2aff}
chir, med, gynéco, familial, perso
\subsection{Ttt}
\label{sec:orge6567b5}
habituel, allergies (dernier repas)
\subsection{MDV}
 profession, main dominante, sport\\
 toxiques, alcool, tabac, viral\\
 conditions de vie, autonome
\subsection{HDM}
\label{sec:org13306b7}
 anamnèse: déclenché par ? favorisé par ?\\
 clinique : quantifier, signes\\
 complications, retentissement, fièvre, AEG
\subsection{Signes généraux}
 SF\\
 Température\\
 Hydratation
\subsection{Urgences}
 Choc ?\\
 Détresse respi aigüe ?\\
 Confusion, coma ? (sd méningé, glycémie)
\subsection{Vasculaire}
 SF: douleur thoracique\\
 Turgescence jug, oedème\\
 Auscultation
\subsection{Vasculaire artériel}
 SF: AIT, HTA, ischémie\\
 PA, FC, IPS\\
 TRC, pouls, auscult vasc.
\subsection{Veine}
 SF: douleur, membre chaud\\
 Troubles trophiques
\subsection{Pulmonaire}
 SF: dyspnée, expectoration, toux\\
 FR, régularité\\
 Thorax, cyanose, hippocratisme digital\\
 Palpation, auscultation
\subsection{Digestif}
 SF: douleur, transit, hémorragie, rectal\\
 Cicatrice, éventration, hernie, ictère, IHC\\
 Palpation (HMG, point Mc Burney, sd péritonéal)\\
 Percussion, auscultation\\
 TR
\subsection{Neuro}
SF: céphalée, douleur neuro, confusion/mémoire, 5 sens, équilibre\\
droitier ?\\
tremblements ?\\
testing musculaire, ROT\\
RCP\\
sensibilité (tact, thermique, doigt-nez)\\
nerf crâniens (II [AV, RPM], oculomotricité, trijumeau, VII [occlusion palpébrale, grimace], VII [Romberg, systagmus], IX et X [déglutation], XI [sterno-cléido-mastoïdien] XII [langue])\\
Signe méningé\\
Doigts-nez, marionnettes\\
Romberg, marche, ordres\\
Fonction sup : langage, articulation, mémoire
\subsection{Locomoteur}
SF: douleur, raideur, blocage, marche\\
Rachis (épineuses, paravértébraux)\\
Sacroiliaque\\
Articulation périph : inflammation, déformation, mobilité passive/actives\\
Genou :
\begin{itemize}
\item épanchement (choc rotulien), palpé rotulien, rabot, Zohlen (rotule bas, quadri contracté), ressaut rotulien
\item méniscal : palpation, grinding test
\item ligaments : LCA (Lachman, tiroir), LCP, latéraux
\end{itemize}
Épaule : conflit sous-acromial (Neer, Yocum), coiffe (Jobe = supra, Patte  infa, belly = sous-scap)\\
Poignet : Finkelsten (déviation ulnaire $\rightarrow$ tendinop. de De Quervain)

Pied : interligne Chopart, Lisfranc, éversion/inversion

Cheville : Ottawa (5eme métatarsien/naviculaire ; malléole +/6cm)

\subsection{Uro}
SF : douleur, brûlures, couleurs, érection\\
Palpation (abdo, flanc, globe, bourse)\\
Perscussion (fosse, globe)\\
TR/TV

\subsection{Gynéco}
SF: douleur, écoulement, aménorrhée, prurit, urinaire, mammaire\\
Abdo\\
Sein\\
Périnée (prolapses)\\
Spéculum, TV

\subsection{Hémato}
SF (asthénie, dyspnée)\\
Hémorragique : épistaxis, purpuras\\
Sd infectieux

\subsection{Endocrino}
Thyroïde\\
Diabéte : SF (angor, AIT, vision), pied : cutané, pouls, TRC, réflexe, monofilament
\subsection{ORL}
Audition\\
Équilibre\\
Larynx\\
Oropharynx\\
Fosses nasales\\

\subsection{Ophtalmo}
SF: BAV, phosphène, diplopie, rougeur, prurit\\
Paupière, glande, cornée, conjonctives, globe, pupille\\
AV

\end{document}
