\documentclass[12pt,landscape]{article}
\usepackage{multicol}
\usepackage{calc}
\usepackage{ifthen}
\usepackage[landscape]{geometry}
\usepackage{hyperref}

% To make this come out properly in landscape mode, do one of the following
% 1.
%  pdflatex latexsheet.tex
%
% 2.
%  latex latexsheet.tex
%  dvips -P pdf  -t landscape latexsheet.dvi
%  ps2pdf latexsheet.ps


% If you're reading this, be prepared for confusion.  Making this was
% a learning experience for me, and it shows.  Much of the placement
% was hacked in; if you make it better, let me know...


% 2008-04
% Changed page margin code to use the geometry package. Also added code for
% conditional page margins, depending on paper size. Thanks to Uwe Ziegenhagen
% for the suggestions.

% 2006-08
% Made changes based on suggestions from Gene Cooperman. <gene at ccs.neu.edu>


% To Do:
% \listoffigures \listoftables
% \setcounter{secnumdepth}{0}


% This sets page margins to .5 inch if using letter paper, and to 1cm
% if using A4 paper. (This probably isn't strictly necessary.)
% If using another size paper, use default 1cm margins.
\ifthenelse{\lengthtest { \paperwidth = 11in}}
        { \geometry{top=.5in,left=.5in,right=.5in,bottom=.5in} }
        {\ifthenelse{ \lengthtest{ \paperwidth = 297mm}}
                {\geometry{top=1cm,left=1cm,right=1cm,bottom=1cm} }
                {\geometry{top=1cm,left=1cm,right=1cm,bottom=1cm} }
        }

% Turn off header and footer
\pagestyle{empty}


% Redefine section commands to use less space
\makeatletter
\renewcommand{\section}{\@startsection{section}{1}{0mm}%
                                {-1ex plus -.5ex minus -.2ex}%
                                {0.5ex plus .2ex}%x
                                {\normalfont\large\bfseries}}
\renewcommand{\subsection}{\@startsection{subsection}{2}{0mm}%
                                {-1explus -.5ex minus -.2ex}%
                                {0.5ex plus .2ex}%
                                {\normalfont\normalsize\bfseries}}
\renewcommand{\subsubsection}{\@startsection{subsubsection}{3}{0mm}%
                                {-1ex plus -.5ex minus -.2ex}%
                                {1ex plus .2ex}%
                                {\normalfont\small\bfseries}}
\makeatother

% Define BibTeX command
\def\BibTeX{{\rm B\kern-.05em{\sc i\kern-.025em b}\kern-.08em
    T\kern-.1667em\lower.7ex\hbox{E}\kern-.125emX}}

% Don't print section numbers
\setcounter{secnumdepth}{0}


\setlength{\parindent}{0pt}
\setlength{\parskip}{0pt plus 0.5ex}


% -----------------------------------------------------------------------

\begin{document}

\raggedright
\footnotesize
\begin{multicols}{3}


% multicol parameters
% These lengths are set only within the two main columns
%\setlength{\columnseprule}{0.25pt}
\setlength{\premulticols}{1pt}
\setlength{\postmulticols}{1pt}
\setlength{\multicolsep}{1pt}
\setlength{\columnsep}{2pt}

\begin{center}
     \Large{\textbf{Examen clinique}} \\
\end{center}

\subsection{ATCD}
\label{sec:org32b2aff}
chir, med, gynéco, familial, perso
\subsection{Ttt}
\label{sec:orge6567b5}
habituel, allergies (dernier repas)
\subsection{MDV}
 profession, main dominante, sport\\
 toxiques, alcool, tabac, viral\\
 conditions de vie, autonome
\subsection{HDM}
\label{sec:org13306b7}
 anamnèse: déclenché par ? favorisé par ?\\
 clinique : quantifier, signes\\
 complications, retentissement, fièvre, AEG
\subsection{Signes généraux}
 SF\\
 Température\\
 Hydratation
\subsection{Urgences}
 Choc ?\\
 Détresse respi aigüe ?\\
 Confusion, coma ? (sd méningé, glycémie)
\subsection{Vasculaire}
 SF: douleur thoracique\\
 Turgescence jug, oedème\\
 Auscultation
\subsection{Vasculaire artériel}
 SF: AIT, HTA, ischémie\\
 PA, FC, IPS\\
 TRC, pouls, auscult vasc.
\subsection{Veine}
 SF: douleur, membre chaud\\
 Troubles trophiques
\subsection{Pulmonaire}
 SF: dyspnée, expectoration, toux\\
 FR, régularité\\
 Thorax, cyanose, hippocratisme digital\\
 Palpation, auscultation
\subsection{Digestif}
 SF: douleur, transite, hémorragie, rectal\\
 Cicatrice, éventration, hernie, ictère, IHC\\
 Palpation (HMG, point Mc Burney, sd péritonéal)\\
 Percusion, auscultation\\
 TR
\subsection{Neuro}
SF: céphalée, doeuler neuro, confusion/mémoire, 5 sens, équilibre\\
droitier ?\\
tremblements ?\\
testing musculaire, ROT\\
RCP\\
sensibilitaté (tact, thermique, doigt-nez)\\
nerf crâniers (II = AV, RPM; oculomotricité, trijumeau, VII=occlusion palpébrale, grimace, VII = Romberg, systagmus, IX et X = déglutation, XI: sterno-cléido-mastoïdien, XII: langue)\\
Signe ménigé\\
Doigts-nez, marionnettes\\
Romberg, marche, ordres\\
Fonction sup : langage, articulation, mémoire
\subsection{Locomoteur}
SF: douleur, raideur, blocaque, marche\\
Rachis (épineuses, paravértébraux)\\
Sacroiliqaue\\
Articulation périhp : inflammation, déformation, mobilité passive/actives\\
Genou :
\begin{itemize}
\item épanchement (choc rotulien), palpé rotulien, rabot, Zohlen (rotule bas, quadri contracté), ressaut rotulien
  \item méniscal : palpation, grinding test
    \item LCA (Lachman, tiroir), LCP,
\end{itemize}
Épaule : conflit sous-acromial (Neer, Yocum), coiffe (Jobe = supra, Patte  infa, belly = sous-scap)\\
Poignet : Finkelsten (déviation ulaine -> tendinop. de De Quervain)

Pied : interligne Chopart, Lisfranc, éversion/inversion

Cheville : Ottawa (5eme métatarsion/naviculaire ; malléole +/6cm)

\subsection{Uro}
SF : douleur, brûlures, couleurs, érection\\
Palpation (abdo, flanc, globe, bourse)\\
Perscussion (fosse, globe)\\
TR/TV

\subsection{Gynéco}
SF: douleur, écoulement, aménorrhée, prurit, urinaire, mammaire\\
Abdo\\
sein
Périnée (prolapses)
Spéculum, TV

\subsection{Hémato}
SF (asthénie, dyspnée)\\
Hémorragique : épistaxis, purpuras\\
Sd infectieux

\subsection{Endocrino}
Thyroïde\\
Diabéte : SF (angor, AIT, vision), pied : cutané, pouls, TRC, réflexe, monofilament
\subsection{ORL}
Audition\\
Équilibre\\
Larynx\\
Oropharynx\\
Fosses nasales\\

\subsection{Ophtalmo}
SF: BAV, phosphène, diplopie, rougeau, prurit\\
Inspectieus : paupière, glande, cornée, conjonctives, globe, pupille\\
AV
\end{multicols}
\end{document}
