% Created 2018-10-19 Fri 13:53
% Intended LaTeX compiler: lualatex
\documentclass[11pt]{article}
\usepackage[hidelinks]{hyperref}
\usepackage{booktabs}
\documentclass{article}
\usepackage[hidelinks]{hyperref}
\usepackage{longtable}
\usepackage{booktabs}
%\usepackage[draft]{graphicx}
\usepackage{graphicx}
\usepackage{fancyhdr}
% French
\usepackage[T1]{fontenc}
\usepackage[francais]{babel}
\usepackage{caption}
\usepackage[nointegrals]{wasysym} % Male-female symbol
% Smaller marign
\usepackage[margin=2.5cm]{geometry}
\usepackage{latexsym}
\usepackage{subcaption}
%-------------------------------------------------------------------------------
% For graphs
\usepackage{tikz}
\usepackage{tikzscale}
\usetikzlibrary{graphs}
\usetikzlibrary{graphdrawing}
\usetikzlibrary{arrows,positioning,decorations.pathreplacing}
\usetikzlibrary{calc}
\usegdlibrary{trees, layered}
\usetikzlibrary{quotes}
%-------------------------------------------------------------------------------
% No spacing in itemize
\usepackage{enumitem}
\setlist{nolistsep}
% tightlist from pandoc
\providecommand{\tightlist}{%
  \setlength{\itemsep}{0pt}\setlength{\parskip}{0pt}}
 % Danger symbol (need fourier package)
\newcommand*{\TakeFourierOrnament}[1]{{%
\fontencoding{U}\fontfamily{futs}\selectfont\char#1}}
\newcommand*{\danger}{\TakeFourierOrnament{66}}
% Skull (need Symbola font)
\usepackage{amsmath,fontspec,newunicodechar}
\newfontface{\skullfont}{Symbola}[Scale=MatchUppercase]
\NewDocumentCommand{\skull}{}{%
  \text{\skullfont\symbol{"1F571}}%
}
% Hospital sign
\usepackage{fontspec} % For fontawesome
\usepackage{fontawesome}
% Itemize in tabular
\newcommand{\tabitem}{~~\llap{\textbullet}~~}
% No numbering
\setcounter{secnumdepth}{0}
% Set header
\pagestyle{fancy}
\fancyhf{}
\fancyhead[L]{\leftmark}
\fancyhead[R]{\thepage}
%\renewcommand{\headrulewidth}{0.6pt}
% Custom header : no uper case
\renewcommand{\sectionmark}[1]{%
  \markboth{\textit{#1}}{}}

\author{Alexis Praga}
\date{\today}
\title{Endocrinologie}
\hypersetup{
 pdfauthor={Alexis Praga},
 pdftitle={Endocrinologie},
 pdfkeywords={},
 pdfsubject={},
 pdfcreator={Emacs 26.1 (Org mode 9.1.9)}, 
 pdflang={English}}
\begin{document}

\maketitle
\tableofcontents


\section{35 : Contraception}
\label{sec:orga441f9a}
\subsection{Contraception hormonale}
\label{sec:org6644de6}
Oestroprogestatifs
\begin{itemize}
\item Contient \{oestrogène, progestatif (gen 1,2 ou 3), autres progestatif\}
\item Administration orale, transdermique ou vaginale
\item Action : \{pas d'ovulation, endomètre peu à apte à la nidation, glaire
cervicale imperméable aux spermatozoïdes\}
\end{itemize}
Progestatifs seuls 
\begin{itemize}
\item Microprogestatifs : action sur glaire cervicale, endomètre
\item Macroprogestatifs : si CI oestroprogestatifs
\item Administration : orale, injection, implant, intra-utérin (stérilet)
\end{itemize}
\subsection{Pratique}
\label{sec:orgd4b812c}
Oestroprogestatifs : 
\begin{itemize}
\item à commencer le premier jour des règles pendant 21j puis 7 j
\end{itemize}
d'arrêt.
\begin{itemize}
\item au même moment
\item Si oubli < 12h, ASAP sinon contraception mécanique \(\ge 7\) jours
\end{itemize}
Microprogestatifs : toujours à la même heure. Si oubli < 3h, ASA, sinon
contraception mécanique \(\ge 7\) jours
Macroprogestatifs : ccommencer le 5eme jour du cycle

\subsection{Contre-indications}
\label{sec:orgd9d0f5a}
Oestroprogestatifs : absolues =
\begin{itemize}
\item thromboemboliques veineux/artériels, prédisposition thromboses
\item lupus évolutif, connectivites, porphyries
\item vasc, cardiaque, cérébrales, oculaires
\item valvulopathie, troubles rythmes thrombogènes
\item HTA non contrôlée
\item diabète et micro/macroangiopathie
\item tumeur hormono-dépendantes (sein, utérus\ldots{})
\item hépatiques sévères
\item hémorragies génitales non diagnostiquées
\item (tumeurs hypophysaires)
\end{itemize}
Macro/microprogestatifs : cancers \{sein, endomètre\}, insuf hépatique, accident
TEV récents

\subsection{Recommandation}
\label{sec:org153172e}
Sans CI, oestroprogestatif minidosé et progestatif 2eme génération monophasique
(Minidril)

\subsection{Efficacité}
\label{sec:orgdc4b1a5}
Indice de Pearl\footnote{\(\frac{N}{N_e/10}\times 100\) avec N = nb grossesses
accidentelles, \(N_e\) nombres de mois d'exposition} < 0.07\% pour oestroprogestatif
(< 2\% pour les microprogestatifs)

Attention : certains inducteurs enzymatiques réduisens l'efficacité (ou
millepertuis).

Ado : sous- ou mal utilisée

\subsection{Tolérance}
\label{sec:org0775b31}
Oestroprogestatifs :
\begin{itemize}
\item bien tolérée, pas de perte de poids
\item surveiller métabolisme
\item active coagulation mais \inc fibrinolyse. Légère augmentation du risque
d'accident TEV
\item vasc : faible \inc PA
\item cancer : ovaire = risque -50\%, idem pour l'endomètre, faible \inc pour sein
\end{itemize}
Microprogestatifs : troubles des règles (spotting, aménorrhées), grossesse
extra-utérine

Macroprogestatifs : hypoestrogénie, aménorrhées, spotting

\subsection{Surveillance}
\label{sec:orgf567e88}
Consulter si céphalée, déficit sensitivomoteur, (douleur ou oedème) MI, dyspnée,
douleur thoracique

Examen clinique : 
\begin{itemize}
\item préthérapeutique : gynéco, frottis cervico-vaginal dès 25 ans si
asymptomatique.
\item PA à +3mois puis tous 6 mois
\item hyperoestrogénie (tension mammaire), hypoestrogénie (sécheresse vaginale)
\end{itemize}
Biologie : cholestérol total, triglycérides, glycémie à jeun à +3mois. Si FR, le
faire avant (!) prescription

Gynéco : métroragies, spottings. NB frottis cervico-utérin dès 25 ans (+1 an pui
tous 3 ans) indépendamment contraception

\subsection{Femmes à risques}
\label{sec:orga0c2423}
Diabétique :
\begin{itemize}
\item non hormonale : si diabète 1 > 15 ans ou micro/macroangiopathie \thus locale
(nullipare, peu de rapports) ou intra-utérin (multipare, diabète équilibré)
\item hormonale :pas d'oestropregestatifs si \{tabac, non équilibré, HTA, surpoids,
diabète compliqué\} \thus progestatif
\end{itemize}
Dyslipidémie : oestroprogestatif si < 3g/L cholestérol total, triglycérides <
9g/L

Thrombose veineuse
\begin{itemize}
\item prédisposant : anomalies de l'hémostase (génétique, acquises), ATCD familiaux
\item dépistage : thrombose, multiples fausses couches, ATCD thrombose < 45 ans
\item acétate de chlormadinone, CI oestrogène
\end{itemize}

Autres :
\begin{itemize}
\item HTA : oestroprogestatifs si 0 FR
\item tabac = CI, migraine \(\land\) vascularite = spécialiste
\end{itemize}

\subsection{Urgence}
\label{sec:org78d9809}
\begin{itemize}
\item lévonorgesttrel ASAP < 72h
\item ulipristal acétate ASAP < 120h mais 3x plus cher
\end{itemize}

\section{37 : Stérilité du couple}
\label{sec:org5b65cd4}
Infertile : 0 grossesses après 1 an de rapports non protégés. Stérilité si
définitf.

Stérilité = partagée !!

\subsection{Interrogatoire}
\label{sec:orga205d3d}
\begin{itemize}
\item Couple
\item Femme : âge++ (détérioration après 35 ans), \{grossesses antérieure,
avortements\}, infections/curetages++, ATCD chir/infectieux, douleurs
pelviennes (rapports, règles), conditions de vie, radio/chimio
\item Homme : trouble libido/érection, ATCD cryptorchidie/trauma testiculaire, ATCD
chir pelvienne/scrotale, ATCD médicaux (oreillons++), tabac/anabolisants\ldots{}
\end{itemize}

\subsection{Examen clinique}
\label{sec:orgea9962d}
\begin{itemize}
\item \female : âge++, obésité/maigreur, tour taille et hanche, pilosité, PA,
galactorrhé provoqué, gynéco.
\begin{itemize}
\item Si anovulation (a-/oligo-ménorrhée) : hyperprolactinémie, hyperandrogénie,
troubles comportement alimentaire, bouffées chaleur
\end{itemize}
\item \male : IMC, pilosité, hypoandrisme, cicatrice chir, varicocèle,
gynémocomastie, gynoïde/enuchoïde
\begin{itemize}
\item volume testiculaire++, palpation cordospermatiques
\end{itemize}
\end{itemize}

\subsection{Examens complémentaires}
\label{sec:org15f7191}
Premiere intention, femme
\begin{itemize}
\item Hormonale++ : oestradiole, LH, FSH, prolactine plasmatique. Puis progestérone
\end{itemize}
plasmatique (si cycle réguliers)
\begin{itemize}
\item Écho ovarienne++
\item Hystérographie++
\end{itemize}
Première intention, homme :
\begin{itemize}
\item spermogramme++ (concentration, mobilité, morphologie. Attention aux
variabilités !
\item hormonale++ si oligo-/azoo-spermie : testostérone, LH, FSH puir SHBG
\end{itemize}
Test poist-coïtal (discuté)

\subsection{Étiologie}
\label{sec:org8c1b204}
Femme :
\begin{itemize}
\item anovulation : très fréquent ! Souvent aménorrhées ou irrégularités. Causes :
sd des ovaires polymicrokystiques, hyperprolactinémie, insuf ovarienne
primitive, déficit gonadotrope, psycho-nutritionnel
\item obstacle mécanique :
\begin{itemize}
\item anomalie du col utérin et insuf glaire cervilaie : post-conisation/curetage
\item obstacle, anomalie utérie : manoeuvres post-partum, polypes muqueux\ldots{} \thus
echographie
\item obstacle tubaire : cause majeure++. Souvent salpingite (Chlamydia++)
\end{itemize}
\item Endométriose : rarement si modéré. hypstérosalpingographie pui coelioscopie
\end{itemize}
Homme :
\begin{itemize}
\item azoospermie 
\begin{itemize}
\item sécrétoire : diagnostic = volume testiculaire < 10ml, concentration FSH
faible
\end{itemize}
\thus caryotype, analyse bras long Y, écho testiculaire (élimine
   cancer), déficit gonadotrope (rare)
\begin{itemize}
\item obstructive : volume et concentration ormale, volume séminal \dec \thus
examen clinique
\begin{itemize}
\item cause congénitale : agénésie bilat des canaux déférent++ (soit anomalie
biallélique gène CFTR, soit isolée)
\item acquis : infectieux  (gonocoque, tuberculose, Chlamydia) \thus échographie
\end{itemize}
\item exploration chir testiculaire et des voies excrétrices : si azoospermie
confirmée par plusieurs spermogrammes, bilan génétique
\end{itemize}
\item oligo-asthéno-térato-spermie : \dec nombre, mobilité, \inc hormes anormales
\thus caryotype, brang long Y. Traitement = assistance médicale procréation
\end{itemize}
\section{40 : Aménorrhée}
\label{sec:org44a9f45}
Déf: absence de cycle menstruel après 16 ans (primaire) ou interruption chez
femme réglée (secondaire). Physiologique : grossesse, lactation, ménopause

Tout arrêt > 1 mois \thus enquête étiologique \danger

Atteinte de l'axe hypothalamo-hypophysio-ovarien ou anomalie tractus utérin

\textbf{Pas de traitement oestrogénique sans enquête étiologique}

\subsection{Conduite}
\label{sec:org015ce65}
\subsubsection{Primaire}
\label{sec:orgb9d06d9}
Forte proba de cause génétique/chromosomique. Chercher carences nutritionnelle

\begin{itemize}
\item Si absence de dév. puberture : doser FSH, LH
\begin{itemize}
\item Si basses, tumeur hypothalamo-hypophysaire, dénutrition ou génétique : \{sd
de Kalmann (anosmie), mutation récepteur GnRH (rare), atteinte
gonadotrophines (exceptionnels), mutation LH\}
\item Si hautes : sd de Turner (caryotique 45, XO),
\end{itemize}
\item Examen gynéco, écho pelvienne
\begin{itemize}
\item Pas d'utérus : sd de Rokitanski, tissu testiculaire dans les canaux
inguinaux (CAIS)
\item ambiguité OGA : dysgénésie gonadique, hyperplasie congénitale surrénales,
anomalies sensibilité/biosynthèse androgènes
\end{itemize}
\end{itemize}
\subsubsection{Secondaire}
\label{sec:org95acfc5}
Souvent acquises. 

Interrogatoire : médic, maladie endoc/chronique,
gynoc/obstétriques, insuf ovarienne (bouffées de chaleur). Douleurs pelviennes
cycliques : cause utérine

Examen clinique : 
\begin{itemize}
\item poids et taille (carence nutritionnelle
\item hyperandrogénie : sd ovaires polykystiques, déficit 21-hydroxylase, (sd
Cushing)
\item carence oestrogénique : pas de glaire +2 semainesa après saignement \thus
anovulation
\item pas de signe d'appel : enquête nutritionnelle
\end{itemize}

Dosages hormonaux :
\begin{itemize}
\item hCG : grossesse
\item prolactine \inc : médciament ou tumeux hypothalamo-hypophysaire (ou adénome)
\item FSh \inc, estradiol \dec : insuf ovarienne primitive
\item estradiol, LH, FSH basse : hypothalamo-hypophysaire ou nutritionnelle
\item testostérone : tumeur surrénalienne/ovarienne
\item sinon SOPK\footnote{Sd des ovaires polykystiques}
\end{itemize}

\subsection{Causes}
\label{sec:org03f5c52}

\subsubsection{Déficit gonadotrope organique/fonctionnel}
\label{sec:orgdfca5ef}

\begin{enumerate}
\item Hypothalamus, prolactine normales
\label{sec:orgf868da4}
Hypothalamus n'arrive pas à libérer la GnRH au bon rythme. 

Atteintes organiques : tumeur/infiltration \thus IRM
\begin{itemize}
\item macroadénomes hypophysaires, craniopharyngiomes
\item chercher hyperprolactinémie, insuf antéhypophysaire associé
\end{itemize}

Atteintes fonctionnelles : apports nutritionnels insufissants par rapport à l'activité physique intense+++
\item Hypothalamo-hypophysaire dû à une hyperprolactinémie
\label{sec:org17bf81d}
Cause majeure

Médicaments ou tumeurs \thus pas de traitement dopaminergique sans imagerie \danger
\item Autres
\label{sec:orge33b748}
\begin{itemize}
\item Endocrinopathies : sd de Cushing, dysthyroïdes déficits 21-hydroxylase
\item Hypophysaire (rare) : auto-immune (majorité), sd de Sheehan (très rare, nécrose hypophysaire post-partum)
\end{itemize}
\end{enumerate}
\subsubsection{Anovulation non hypothalamique}
\label{sec:org01c77ee}
\begin{enumerate}
\item SOPK (majorité)
\label{sec:org6c39ebd}
Pas de pic de LH, ni de progestérone. Oestradiol mais non cycliques

Irrégularité menstruelles, puis aménorrhées avec acné, hirsutisme

Diagnostic :
\begin{itemize}
\item 2 parmi : \{hyperandrogénie clinique\footnote{Séborrhéee, acné, hirsutisme, \inc testostérone}, oligo-/a-novulation, hypertrophie
ovarienne (écho)\}
\item exclure bloc 21-hydroxylase, tumeur de l'ovaire, sd Cushing
\item exclure hyperprolactinémie
\end{itemize}

Diagnostic parfois difficile :
\begin{itemize}
\item sans hyperandrogénie \thus écho
\item \{atteinte partielle axe gonadotrope, macroprolactinémie\} peuvent y ressembler
\end{itemize}

Acné : cherche hyperandrogénie, régularité cycle menstruel \thus éliminer
hyperplasie congénitale des surrénales

2 causes :
\begin{itemize}
\item tumeur ovarienne ou résistance insuline
\begin{itemize}
\item virilisation si tumeur
\item imagerie si testostérone > 1.5ng/mL. Si normale, cherche hypothécose
(obésité morbide androïde, acanthosis nigricans, insulino-résistance)
\end{itemize}
\item pathologie surrénale :
\begin{itemize}
\item sd de Cushing si signes hypercortisolisme \thus cortisol libre urinaire et
freinage minute
\item tumeur surrénale \thus scanner des surrénales
\item déficit enzymatique en 21-hydroxylase (\danger formes tardives qui peuvent
mimer SOPK)
\end{itemize}
\end{itemize}
\end{enumerate}
\subsubsection{Insuf ovarienne primitive}
\label{sec:orgbfa7fb3}
\inc FSH

Causes :
\begin{itemize}
\item chir, chimio, radiothérapie
\item anomalie caryotype (sd Turner)
\item anomalie gènes \emph{FMR1} (sd X fragile)\footnote{Transmission mère-fils. Expansion instable des triplets CGG jusqu'à
l'absence de transcript de FMR1 (Fragile X Mental Retardation 1). \\
\danger Risque d'IOP pour la pré-mutation seulement \thus dépister chez
\female + IOP < 40 ans par PCR et Southern Blot}
\item auto-immune
\end{itemize}
\subsubsection{Anomalie utérine}
\label{sec:org1d3c640}
\begin{enumerate}
\item Congénitales :
\label{sec:orgd852c9e}
Si dév pubertaire normale et douleurs pelviennes cycliques, imperfortanio
hyménéale/malformation vaginale \thus examen gynéco.\\
Idem sans douleurs \thus agénésie utérus ?

Difficulté : différence agénésie mullérienne isolée (46,XX)- anomalies androgènes
(46,XY) \thus testostérone
\item Secondaires :
\label{sec:org1a1cf45}
Synéchies utérines (trauma de l'utérus), tuberculose utérine
\end{enumerate}
\section{47 : Puberté}
\label{sec:orgf5f038c}
\subsection{Normale}
\label{sec:org3012922}
\textasciitilde{}4 ans, acquisition de la taille définitive, fonction de
reproduction. Classification de Tanner (5 stades)

\female : seins \textasciitilde{}11 ans [8,13]ans, règles \textasciitilde{}13ans [10,15]

\male : \inc volume testiculaire \textasciitilde{}11.5 [9.5,14], \inc taille verge \textasciitilde{}12.5ans.

Accélération de la croissance : 5 -> 8cm (163cm) \female, 5-10cm (175cm)
\subsection{Retards}
\label{sec:orge0fd653}
\male :  volume testiculaire < 4mL ou longeur < 25mm > 14 ans 

\female : pas de seins à 13 ans, pas de règles à 15 ans

Centrale ou périphérique ?
\begin{itemize}
\item centrale : congénital (pas de cassure de croissance, micropénise,
cryptorchidie), acquis (tumeur ?), "fonctionnel" (maladie générale, trouble
comportement alimentaire), isolé
\item périphérique : sd de Turner chez \female, sd Klinefelter \male
\item retard simple (élimination)
\end{itemize}

Clinique : 
\begin{itemize}
\item parents, grossesse, courbe de croissance. Chercher trbles digestifs,
\end{itemize}
polyuro-polydispise, céphalée, anomalies champ visuel
\begin{itemize}
\item pathologie acquise, OGE, testicules, anosmie (Kallmann)
\end{itemize}

Âge osseux : 13 ans \male, 11 ans \female

Biologie : 
\begin{itemize}
\item stéroïdes sexuels, si FSH, LH basses \thus hypothalamo-hypophysaire
\item testostérone chez \male, oestradial/écho chez \female
\end{itemize}

IRM indispensable ! si déficit gonadotrope (tumeur)

Caryotype si FSH élevé. Toujours chez \female de taille < -2DS avec retard
pubertaire/gonadotrophine \inc

\subsubsection{Étiologies}
\label{sec:org72c91ef}
Hypogonadotropes
\begin{itemize}
\item congénitaux : isolés, sd de Kallman, autres déficits hypophysaires, sd
polymalformatifs
\item acquis : tumeurs hypophysaires, post-radiothérapie
\end{itemize}
Hypogonadotropes fonctionnels
\begin{itemize}
\item chroniques digestives/cardaques/respi
\item sport intense
\item endocrio
\end{itemize}
Hypergonadotropes
\begin{itemize}
\item congénitaux : sd Turner, sd Klinefelter, autres atteintes primitevs
\item acquis : castration, trauma, oreillons, chimio/radio
\end{itemize}

\subsubsection{Traitement}
\label{sec:org03b0f5a}
Cause si possible. Sinon doses \inc de testostérone (\male) ou oestrogènes puis
oestroprogestatif (\female)
\subsection{Précoces}
\label{sec:org695683a}
Avant 8ans \female ou 9.5 ans \male
\subsubsection{Centrales}
\label{sec:org4007756}
8x plus fréquent chez \female que \male. Chez \female, causes
idiopathiques. Chez \male, causes tumorales à 50$\backslash$%

Clinique : 
\begin{itemize}
\item dév prématuré harmoniaux (pas de règles chez \female)
\item crises de rires (harmatome hypothalamique), tâches cutanées (neurofibromatose
I ou sd McCune-Albright)
\end{itemize}
Biologie :
\begin{itemize}
\item testostérone élevée chez \male mais variabilité d'oestradiol chez \female
\end{itemize}

IRM hypothalamo-hypophysaire indispensable \danger (petite taille
définitive). Écho pelvienne pour \female

Traitement si risque de petite taille adulte : analogues GnRH jusque âge normal
de puberté
\subsubsection{Périphériques}
\label{sec:org289fb06}
Clinique : \inc vitesse de croissance, avance maturation osseusse

Stéroïdes élevées, LH et FSH bas. Écho pelvienne chez fille

Étiologie :
\begin{itemize}
\item tumeurs ovarienne (rares) : écho puis histologies
\item kystes folliculaires : bénins, régression spontanée possible
\item sd McCune-Albright : \{puberté précoce ovarienne, taches cutanées
"café-au-lait", dysplasie fibreuses os\}. \danger tableau pas toujours complet
!\\
\end{itemize}
Oestradiol élevé, gonadotrophines basses, écho = utérus stimulé, kystes
ovariens. Dominance \female
\begin{itemize}
\item médicaments
\item testotoxicose (rare, cellule de Leydig activé et LH basses), adénome leydigien
(très rare)
\item tumeurs à hCG (\male)
\end{itemize}
\subsubsection{Avances dissociées}
\label{sec:orgd2ec535}
\begin{itemize}
\item Isolé des seins : beaucoup de filles [3mois, 3 ans]
\item Métrorragies isolées : chercher vulvite, vulvovaginite, prolapsus urétrale,
corps étranger. Éliminter kyste ovarien, sd McCune-Albright par l'absence des
sein
\end{itemize}
\thus écho pelvienne
\begin{itemize}
\item Pilosité pubienne isolée : chercher forme d'hyperplasie congénitale des
surrénales (\inc 17-hydroxyprogestérone, stimulation ACTH), prémature pubarche
(élimination !)
\end{itemize}

\section{48 : Cryptorchidie}
\label{sec:orge5fb6b6}
\subsection{Enfant}
\label{sec:org635503f}
Localisation anormale et inaboutie du testicule. Très fréquente : 3\%
nouveaux-nés, 20\% préma. 2/3 descendent spontanément à 1 an de vie

Clinique : checher micropénis (< 2cm, hypospadias, autres)

Explorations : endocrinienne pour toute cryptorchidie \danger
\begin{itemize}
\item bilatérale : doser 17-hydroxyprogestérone chez \female virilisée pour élimire hyperplasie
congénitale des surrénales
\item testostérone, INSL3 (cellule de Leydig), AMH, inhibine B sérique (c. de
Sertoli), FSH, LH mesurée jusque 4-6mois\footnote{\danger Testostérone, FSH, LH interprétables [6mois, puberté]}
\item écho pour vérifier l'absece de dévirés mülleriens
\end{itemize}

Étiologie
\begin{itemize}
\item hypogonadisme hypogonadotrope congénital
\item anorchidie rare
\item si hypospade en plus, chercher dysgénésie testiculaire
\item sd polymalformatif
\end{itemize}

Suivre l'âge de l'apparition de la puberté !

Traitement : chir dès 2 ans, indispensable ! (risque de cancer)
\subsection{Adulte}
\label{sec:org84047be}
\begin{itemize}
\item Risque : hypogonadisme, infertilité, cancer testicule
\item Examen clinique : scrotume, gynécomastie, signes d'hypogonadisme
\item Complémentaire : \{FSH, LH, testostérone\}, hCG si tumeur à la palpation, écho
scrotale, spermogramme
\end{itemize}
\section{51 : Retard de croissance}
\label{sec:org3f5f038}
\danger Ne pas passer à côté de pathologies sévères

Phases : foetale (rapide, \{nutrition, insuline, IGF-2\}), précoce 0-3ans (rapide,
\{insuline, IGF, hormones thyroïdiennes\}), prépubertaire (plus lente, décroît,
\{génétique, GH/IGF, hormones thyroïdiennes\}), pubertaire (\{stéroïdes sexuels,
GH, nutrition\})

Retard statural = \{taille < -2DS, ralentissements croissance, croissance <<
parents\}

Prise de poids, obésite, ralentissement croissance \thus chercher
hypercorticisme, tumeux craniopharyngiome sur l'hypothalamus, hypopituitarisme

Examen :
\begin{itemize}
\item ATCD : taille, parents, néonatale, médicaux/chir, contexte social
\item morphotype, dév. pubertaire, tous les système, psychoaffectif
\end{itemize}

\subsection{Principales causes}
\label{sec:org34ef32f}
\begin{table}[htbp]
\caption{Retard poid}
\centering
\begin{tabular}{ll}
\toprule
Maladie coeliaque & IgA \{totales, transglutamase\}, fibro\\
Crohn & VS, écho anse grêle\\
Mucoviscidose & Test sueur\\
Anorexie mentale & Courbe de poids\\
Insuf rénale chroniques & Créat, iono, explo fonctionnelles\\
Anémie chroniques & NFS\\
Rachitisme hypophosphatémique & Bilan phosphocalcique\\
Patho mitochondriales & lactate/pyruvate, génétique, biopsie musc, fond d'oeil\\
Nanisme psychosocial & \\
\bottomrule
\end{tabular}
\end{table}

\begin{table}[htbp]
\caption{Retard statural}
\centering
\begin{tabular}{lll}
\toprule
Endocrino & Déficit GH (congénital, acquis [tumeur]) & IRM\\
 & Hypothyroïdie & T4L, TSH, Ac anti-TPO\\
 & Hypercorticisme (iatrogène) & Cortisol libre urinaire/à 23h, ACTH\\
 & Déficit hormones sex. & Testostérone, GnRH, IRM\\
\midrule
Constitutionelles & Sd Turner & Caryotype\\
 & Sd noonan & Gène PTPN11\\
\midrule
Autres & Osseuses (a-/hypo-chondroplasie) & Radio\\
 & RCIU & Taille naissance\\
 & Petite taille idiopathique & Élimination\\
\bottomrule
\end{tabular}
\end{table}

\begin{itemize}
\item Test de stimulation de l'hormone de croissance (\danger si doute, IRM)
\item Ralentissement sévère \thus bilan en urgence (craniopharyngiome, thyroïdite de
Hashimoot) \skull
\item\relax [0, 3] ans : digestives pédiatrique (coeliaque, mucoviscidose), [3,puberté] :
endoc constitutionnelle, à la puberté : déficit hormone, patho osseuse
\item Savoir différenceier retard pubertaire syimple d'un vrai  retard
\end{itemize}

\subsection{Exploration :}
\label{sec:orgab54bd0}
\begin{itemize}
\item Caryotype : fille taille < -2DS ou < -1.5DS sous taille parentale moyenne
\item NFS, VS, foie, rein
\item IgA totales, anti-transglutaminase
\item GF-1, T4L, TSH
\item Radio
\end{itemize}

\section{78 : Dopage}
\label{sec:orgc286ec4}
\subsection{Substances augmentant la testostérone}
\label{sec:org2661e9f}
\begin{itemize}
\item Stéroïdes anabolisant, testostérone : \inc masse musc, puissance
\item Risque : thrombotique, rupture musculo-tendineuse, touble personnalité, foie,
\end{itemize}
trouble libido, adénome/cancer de la prostate
\begin{itemize}
\item Femmes : masculinisation, hirsutisme, acné, aménorrhée, anovulation,
hypertrophie clitoridienne, libido exacerbée
\end{itemize}

\vspace*{0.5cm}
\begin{itemize}
\item Testostérone : test chromatographique + spectrométrie de masse (très sensible
et spécifique)
\item Dihydrotestostérone (DHT) : traitement gynécomastie
\item Anabolisants : \inc tissu cellulaire (muscle). ES : rétention hydrosodée, HTA,
IDM, hépatite
\item hCG : diminuer épitestostérone/testostérone après dopage (IM, SC). Testée dans
le sang ou urine.
\item Anti-oestrogène : stimule production testiculaire de stéroïdes
\end{itemize}

\subsection{Hormone de croissance (GH), IGF-1}
\label{sec:orgaf34381}
\begin{itemize}
\item GH \inc masse musculaire, modifie architecture sequelette, acromégalie \emph{mais}
pas d'effert sur volume d'activité physique. Détection difficile : approche indirecte (cascade biologique) et mesure des
\end{itemize}
forme circulante et comparaison à r-hGH
\begin{itemize}
\item IGF-1 mime certains effet GH
\end{itemize}

\subsection{Glucocorticoïdes, ACTH}
\label{sec:org9fc2c30}
\begin{itemize}
\item Glucocorticoïdes : antalgiques, psychostimulants, combativité. ES : HTA,
oedème, rupture ligament/tendon
\end{itemize}
\danger arrêt brutal = dangereux \skull

\section{120 : Ménopause et andropause}
\label{sec:org69081f0}
\label{sec:120}
\subsection{Ménopause}
\label{sec:orgf28f374}
Déf: plus de règle > 1 an \textpm{} sd climatérique, lié à une carence
oestrogénique. Vers 51 ans.

Pré-ménopause : irrégularités cycles puis dysovulation puis anovulation \textasciitilde{}5 ans
avants.

\subsubsection{Diagnostic}
\label{sec:orgaeac7b6}
Clinique seulement \danger : bouffées de chaleur, \female > 50 ans. Bio
seulement si hystérectomie \thus \dec oestradiol et \inc FSH

En pratique : progestatif seul 10j/mois x3 \thus pas de saignement à l'arrêt =
diagnostic

Aménorrhée < 40 ans = pathologique !

\subsubsection{Conséquences}
\label{sec:orgd98c1a7}
Court terme : bouffées de chaleur, trouble sommeil/humeur, \dec sécrétions
vaginailes
Moyen terme : douleurs ostéoarticulaires, \inc perte osseuse (selon ATCD d'insuf
ovarienne prématurée, fractures no traumatique, médicaments, calcium/vit D)

Long terme :: \inc risque CV. Incertitude sur SNC

\subsubsection{Traitement}
\label{sec:org65c9d0e}
Bénéfices
\begin{itemize}
\item court terme : qualité de vie à +5-10 ans
\item long terme :
\begin{itemize}
\item prévention ostéoporose
\item cardiovasculaire et neuro = incertain
\item cancer du côlon
\end{itemize}
\end{itemize}
Risques :
\begin{itemize}
\item \inc cancer du sein, accident veineux thromboemboliques (mais chiffres absolus
faibles)
\item \inc AVC ischémique, lithiase bilaires
\end{itemize}

\begin{enumerate}
\item Thérapeutique
\label{sec:org2de3738}
\begin{itemize}
\item oestrogène (17\(\beta\)-oestradiol)
\end{itemize}
oral/percutané/transdermique\footnote{Percutané, transdermique : limite \inc facteur de coagulation.} 25 jours/mois
\begin{itemize}
\item \textbf{et} progestatif (au moins les 12 derniers jours) per os/transdermique
\end{itemize}
\danger hémorragie de privation possible. Si pendant le traitement, faire écho
pelvienne, hystéroscopie

\item CI
\label{sec:orgbf936c1}
Cancer du sein, endomètre, ATCD thromboembolique artériel (ischémique,
cardiopathie embolinogène) ou veineux, hémorragie génitale sans diag, hépatique

\item Mise en route
\label{sec:org906c1d1}
\begin{itemize}
\item Interrogatoire : ATCD \{cancer, métabolique, vasculaire\}, carence oestrogénique
\item Examen physique : poids, PA, palpation seins, gynéco, frottis cervico-vaginal
\item Mammograhpie !
\item Cholestérol, triglycérides, glycémie
\end{itemize}

\item En pratique
\label{sec:orgaed7e88}
1ere intention si trouble fonctionnels importants. 2eme si risque
d'ostéoporose. Sinon au cas par cas.

\item Surveillance
\label{sec:org9395f6e}
3-6mois (surdosage = douleur, tension mammaire). Puis tous les 6-12 mois,
mammographie tous les 2 ans, frottis CV tous les 3 ans.

Traitement \(\ge\) 5 ans !!

\item Alternatives
\label{sec:orgb66548f}
\begin{itemize}
\item Modulateurs spécifiques du récepteur des oestrogènes : raloxifène
\item tibolone
\end{itemize}

NB : traitement local préserve tractus urogénital. Dépister FR CV. Promouvoir
exercice, calcium, vit D
\end{enumerate}

\subsection{Andropause}
\label{sec:orgb103a23}
Chez majorité des hommes mûrs/âgés en bonne santé non obèse, baisse de
testostérone inconstante (2\%).

\subsubsection{Démarche}
\label{sec:org4990e83}
\begin{itemize}
\item Interrogatoire : libido, érection, énergie vitale, mobilité/activité physique
\item Examen clinique : IMC, virilisation, gynécomastie, palper testicules
\item Mesure de testostérone totale :
\begin{itemize}
\item > 3.2ng/mL = normale \thus étiologies non endocrino
\item \(\in\) [2.3, 3.2] : dosier SHBG, calculer index de T libre, si bas cherche cause
\item < 2.3 ng/mL : chercher cause
\end{itemize}
\end{itemize}
\subsubsection{Étiologie}
\label{sec:org1bc9aa1}
Si FSH, LH élevée, insuf testiculaire primitive 
\begin{itemize}
\item lésionnelle : chimio, radiation, alcoolisme surtout. Autres : castration,
torsion, orchite ourlienne
\item cryptorchidie bilatérale
\item chromosomique : sd Klienfelter++
\item lié à sénescene, cause génétique (rare !)
\end{itemize}
Sinon hypogonadisme hypogonadotrope 
\begin{itemize}
\item tumeur région hypothalamo-hypophysaire : craniopharyngiome, adénome
hypophysaire++, autres
\item infiltratif : sarcoïdose, hémochromatose
\item chir, radiothérapie, traumau
\item hyperprolactinémie, carence nutritionnelle, Cushing, tumeur testiculaire
\end{itemize}

\section{122 : Troubles de l'érection}
\label{sec:orge92fc06}
Nécessite : vasculaire, musculaire lisse, retour veineux, signal  nerveux,
hormonal, psychisme = fonctionnels

Déf : incapacité persistante à obtenir/maintenir érection pour rapport sexuel satisfaisant

Âge = FR (déficit neurosensorile, \inc testostérone, comorbidités)

\subsection{Conduite  diagnostique}
\label{sec:org32a1ea2}
\subsubsection{Interrogatoire}
\label{sec:org7620b89}
\begin{itemize}
\item DD avec perte désire, trouble éjaculation, douleurs pendant, anomalies morphologiques
\item Caractérisation : primaire/secondaire, brutal/progressif,
permanent/situationnel, sévérité (délai trouble-consult, capacité résiduelle,
masturbation)
\item Pathologies, facteur :
\begin{itemize}
\item primaire : trouble psychogène perso, complexe identitaire, trouble
relationnel, conflit socioprof, anomalie génitale
\item secondaire : ATCD abdo-pelvien, diabète, FR CV, patho CV, neuro, trouble
miction, endocrinopathie, troubles sommeil, traitement, déficit
androgénique, sd dépressif, troubles addictifs
\end{itemize}
\end{itemize}
\subsubsection{Clinique}
\label{sec:org0dd78d6}
\begin{itemize}
\item Gynécomastie, hypoandrisme, petits testicules, anomalies du pénis (La Peyronie)
\item CV : HTA, pouls, souffle
\item neuro : sensibilités périnée, MI
\item endoc : anomalie CV
\end{itemize}
\subsubsection{Bio}
\label{sec:orgae0c66a}
Glycémie, lipidique (si > 1 an), \{NFS, iono, créat\}, foie (si > 5 ans), déficit
androgénique

Doser prolactine, hormones thyroïdiennes

\subsection{Bilan secondaire et approfondi}
\label{sec:org5eb131d}
Secondaire : sexo/psychologique, épreuve pharmacologique (prostaglandien,
inhibiteur de la phospohdiestérase 5)

\subsection{Étiologies}
\label{sec:org3d909d6}
Plus fréquentes :
\begin{itemize}
\item vasculaire : FR = HTA
\item endocrino++ : diabète
\item génito-pelvien : chir pelvienne
\item trauma médullaire
\item neuro dégénératif
\item iatrogène : antihypertenseur
\end{itemize}

\subsection{Aspects endocriniens}
\label{sec:orgd44d97b}
\subsubsection{Androgènes circulants}
\label{sec:orgb7d5b92}
Influe libido, intérêt sexuel. Pour l'érection . seulement spontanées !

Hypogonadisme (diag difficile) : 
\begin{itemize}
\item asthénie, gynécomastie, dépilation, perte force musculaire, adiposité androïde
\item doser testostérone totale \textpm{} SHBG, prolactine. FSH, LH pour l'origin (cf
Sec. \ref{sec:120})
\end{itemize}

\subsubsection{Hyperprolactinémie}
\label{sec:orgd7060bf}
Tumeur hypophysaire (IRM), champ visuel si tumeur
supra-sellaire, \{T4L, cortisol, IGF-1, testostérone\}
\thus agoniste dopaminergique

\subsubsection{Diabète}
\label{sec:orgd305226}
sucré = 1ere cause de trouble érectile (TE). TE fréquents chez diabètique. 

Facteurs : mal équilibré, complications, âge, ancienneté diabète

Physiopatho : neuropathie autonomie, microangiopathie \thus défaut relaxation
musculaire. Macroangiopathie \thus ischémie organes érectiles

\danger facteurs psychènes hyportants !

Diabète et TE \thus mesure testostérone systématique (hypogonadisme ?)

Clinique : 
\begin{itemize}
\item TE peut révéler diabète.
\item diabète et TE : cherche trouble endoc, vasc, neuro, médicament, dépression
\item TE = FR d'ischémie myocardite silencieuse \danger
\end{itemize}

Thérapeutique : inhibiteurs des phosphodiésterases de type 5 moins efficaces

\subsection{PEC}
\label{sec:org5a0ef1a}
Ttt étiologie : seulement trouble psychogène pur, chir possible, endocrino

\subsubsection{Trouble endocrinien}
\label{sec:org8a23a90}
\begin{itemize}
\item Si hypogonadisme confirmé par bio\footnote{Baisse libido, testostérone totale < 3 ng/mL} : androgène oraux/intramusc/transderm
\item CI : nodule prostatique palpable, PSA > 3ng/mL
\item Surveiller prostate, foie, hématocrite
\end{itemize}

\subsubsection{Pharmacologique}
\label{sec:orgadf82aa}
\begin{itemize}
\item FR, Hb glyquée < 7\%, psycho/sexologique
\item 1ere intention : inhibiteurs des phosphodiésterases type 5 = efficaces
\item Sinon apomorphine, yohimbine = peu efficace
\item "Pompe" = efficace mais résistance psycho
\item 2eme intention : droge vasoactive = efficace mais douleurs peniennes, priapisme
\item Prothèses péniennes = dernier recours, par chirurgien spécialisé
\end{itemize}

\section{124 : Ostéopathies}
\label{sec:orge3c71f8}
Ostéoporose : fragilité excessive du squelette (\dec minéraux osseux, modif
microarchitecture). T-score < -2.5 DS valeur moyenne par DXA

Dominance \female. Primaire ou secondaire :
\begin{itemize}
\item endocrino : hypogonadisme, sd Cushing, hyperthyroïdie, hyperparathyroïdie, diabète
\item digestives, générale, génétique, médicaments, autres
\end{itemize}

\subsection{Hypogonadisme}
\label{sec:org231faeb}
Carence oestrogénique \inc ostéoclastogénèse. Aggravé par précocité,
déminarilsation
\subsubsection{Anorexie mentale}
\label{sec:orga69f6bf}
Biochimie : \{isoenzyme des phosphastales alcalines, ostéocalcine\} (formation)
\dec, \{CTx, NTx\} (résorption) normaux

Aggravé par troubles nutritionnels. Hypercortisolisme hypothalamique réversible

Ostéoporose fréquente, risque de fractures \(\times 7\)

Traitement 
\begin{itemize}
\item multidisciplinaire
\item pilule oestroprégestative en pratique (limite perte osseuse)
\end{itemize}

\subsubsection{Activité physique intensive}
\label{sec:org48ca968}
Hypoestrogénie hypothalamique

Facteur : activité\footnote{Marathon, danse classique, demi-fond, triathlone, gymnastique, cyclisme}, troubles menstruels, apports alimentaires

Résorption généralisée (rachis++), \inc fractures de fatigue

Traitement : si aménorrhées, \dec activité ou oestroprogestatifs

\subsubsection{Pathologies hypophysaires}
\label{sec:orgffb6c7e}
Prolactinomes, adénome corticotrope influent remodelage osseux

Perte osseuse rapide (8\% par an), récupération variable.

Traitement : \female non ménopausée : oestrogénothérapie

\subsubsection{Iatrogènes}
\label{sec:orgf44e06d}
Agonistes GnRH (patho utérines), inhibiteurs aromatase (cancer sein)

Réversible à l'arrêt (moins bien si âgée)

Traitement : bisphosphonates, denosumab

\subsubsection{Dysgénésies gonadique}
\label{sec:org8ae99b8}
Sd Turner = plus fréquent (1/2500 à naissance)

\dec masses osseuse, continue à l'adolescence. \inc risue fracture chez
l'adulte.

Traitement : oestrogénisation (hypogonadisme) et hormone de crossance. Adulte :
oestroprogestatif

\subsection{Hyperthyroïdie, traitement par hormones thyroïdiennes}
\label{sec:org2e8fbe9}
Cause fréquente d'ostéoporose secondaire \thus dosage systématique TSH

Hormones thyroïdiennes \inc remodelage : résorption sur l'os cortical++ et trabuclaire

En pratique, rarement évolution jusque l'atteinte osseuse (ttt
rapide). Adapter posologie hormones thyroïdiennes au cancer thyroïdien.

Prévention :
\begin{itemize}
\item densitométrie
\item bisphosphonates sujet âgé ou risque extrémité supérieure du fémur
\item surveillance si ttt suppressif de fonction thyroïdienne
\end{itemize}

\subsection{Hypercortisolisme, corticothérapie}
\label{sec:orgde2cd31}

\dec ostéoblastes, \inc activité ostéoclaste. \dec absorption intestinale
calcium, \inc pertes urinaires calcium, hyperparathyroïdisme

Surtout trabuculaire (vertèbres, côtes, radius). Aggravé si prépubertaire, hypogonadisme
Fractures vertébrales fréquentes, surtout sd Cushing avec adénome
corticotrope/surrénalien

Traitement : 
\begin{itemize}
\item pré-corticothérapie : status osseux, FR
\item supplément vitaminocalcique
\item bisphosphonates, tériparatide si corticothérapie > 3 mois, prednison > 7.5mg/j
et T-score \(\ge -1.5\)
\end{itemize}

\subsection{Hyperparathyroïdie primitive}
\label{sec:orgfc55c93}
Fréquent, notamment chez femme ménopausée. Ostéoporose fréquente \thus dépistage
systématique par DXA

Production continue PTH : \inc résorption os cortical (tiers proximal radius,
fémur). NB : souvent aussi extr supérieure du fémur et vertèbres. 

Diminution limitée (10\%). Souvet favorable post-parathyroïdectomie.

Traitement : chir si T-score < -2:5. Sinon anti-ostéoclastiques\footnote{Oestrogènes, raloxifène, bisphosphonates}, calcimimétique\footnote{Cinacalcet}

\subsection{Chez l'homme}
\label{sec:orgd5854d2}
Pas de T-score reconnu. 

Fracture radius distal plus rares.

Ostéoporoses secondaires plus fréquentes chez l'homme : hypercorticisme,
hypogonadisme congénital/acquis/iatrogène, alcoolisme, hypercalciurie
idiopathiques, génétique
\section{207 :Sarcoidose}
\label{sec:orged81b45}
Atteinte hypothalamo-hypophysaire exceptionnelle. Conséquences : diabète
inspide central, insufisance gonadotrope

Radio : IRM centrée sur hypothalamo-hypophyse = référence (T1,T2 injecté) \thus
infiltration plancher 3eme venticule, infundibulum, tige hypophysaire épaissie
\textpm{} hypophyse augmente de volume

DD : tuberculose, histiocytose, lymphome, autres tumeurs de la région 

Si patient avec sarcoïdose : diagnostic = déficit endocrinien et imagerie\footnote{Faire bilan hormonal : natrémie, testostérone totale et libre/ostradiol,
FSH, LH, T4L, TSH, prolactine}

Sinon : atteinte rare\footnote{adénome hypophysaire 90\%, méningiome, craniopharyngiome, patho
inflammatoires infiltratives}

Radio et arguments sarcoïdose.

Traitement : sarcoïdose et déficits hormonaux

\section{215 : Hémochromatose}
\label{sec:org3ad9cb1}
Hémochromatose primitive : génétique, surcharge en fer. 5 pour 1 000 !

Physiopatho : 
\begin{itemize}
\item Absorption intestinale régule stockage de fer
\item Fer entre dans l'entérocyte (DMT1), puis stocké via ferritine ou relargé par ferroportine
\item Hepicidine \dec quand besoins fer \inc \sout{+}
\item Hémochromatose : hepcidine effondrée, DMT1 et ferroportine \inc
\end{itemize}

Génétique : gène HFE à 95\% et mutation C282Y/C282Y ou C282Y/H63D

\subsection{Clinique}
\label{sec:orgefabef9}
En pratique, suspicion aux "3 A" : asthénie, arthralgies, \inc ALAT

\subsubsection{Atteintes :}
\label{sec:org7aadc70}
\begin{itemize}
\item foie : \inc ALAT ou hépatomégalie. Cirrhose \(\approx\) 90\% décès
\item coeur : cardiopathie dilatée, troubles rythme
\item endocrino :
\begin{itemize}
\item diabète++ (accumulation pancréatique de fer) insulino-pénie/-résistance
\item hypogonadisme+ : impuissance \male, aménorrhée \female, \dec libdio,
ostéoporose
\item insuf thyriodienne exceptionnelle
\end{itemize}
\item articulaire : arthrite chronique ("poignée de main"), chrondocalcinose
\item cutané : mélanodermie (tardive)
\end{itemize}


\subsection{Diagnostic}
\label{sec:org2ebee13}
\begin{itemize}
\item Si CS-Tf\footnote{Coefficient de saturation de la transferrine} < 45\% : si ferritine \inc, cherche hépatosidérose dysmétabolique,
acéruléoplasminémie, mutation gène de la ferroportine 1
\item Sinon, CS-Tf > 45\% : 
\begin{itemize}
\item si C282Y/C282Y ou C282Y/H63D : diagnostic
\item sinon, si ferritine \inc, test génétique de 2eme intention, biopsie
hépatique
\end{itemize}
\end{itemize}

Examen complémentaires : glycémie, transaminases, écho abdo, ECG \textpm{} écho
cardiaque, radio articulation, bilan testostérone

Dépistage chez parents (1er degér) : bilan martial \textpm{} dépistage
génétique. \danger mutation \(\neq\) maladie

\subsection{Stades}
\label{sec:orgf293804}
\begin{enumerate}
\item Asymptomatique, CS-Tf, ferritinémie normaux
\item CS-Tf \inc
\item CS-Tf \inc et ferritine \inc
\item Idem et expression clinique affectant qualité de vie
\item Idem et expression clinique affectant pronostic vital
\end{enumerate}

\subsection{Traitement}
\label{sec:orgd6cd1ab}
À partir du stade 2

\subsubsection{Saignées}
\label{sec:org88fbfae}
Référence. Objectif : ferritine < 50\g g/L (hebdomadaire) puis entretien tous les
  2-4 mois. Ne pas dépasser 550mL !

CI : anémie sidéroblastique, thalassémie majeure, cardiopathies sévères

\subsubsection{Autres}
\label{sec:orge31d2cc}
\begin{itemize}
\item Érythraphérèse : coûteuse, plus difficile
\item Chelation du fer : 2eme intention (coût, effets indésirable)
\item diététique : pas d'alcool, éviter vitamine C mais \textbf{conserver} apports en fer !
\item Symptomatique
\end{itemize}

\subsection{Suivi}
\label{sec:org71c03b1}
Résultats en 3-6 mois sur été générale. 

Bilan ferrique (stade 0,1) ferritinémie, hémoglobine (stade 2 à 4)
\end{document}
