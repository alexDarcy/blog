% Created 2019-01-22 Tue 22:02
% Intended LaTeX compiler: lualatex
\documentclass[11pt]{article}
\usepackage[hidelinks]{hyperref}
\documentclass{article}
\usepackage[hidelinks]{hyperref}
\usepackage{longtable}
\usepackage{booktabs}
%\usepackage[draft]{graphicx}
\usepackage{graphicx}
\usepackage{fancyhdr}
% French
\usepackage[T1]{fontenc}
\usepackage[francais]{babel}
\usepackage{caption}
\usepackage[nointegrals]{wasysym} % Male-female symbol
% Smaller marign
\usepackage[margin=2.5cm]{geometry}
\usepackage{latexsym}
\usepackage{subcaption}
%-------------------------------------------------------------------------------
% For graphs
\usepackage{tikz}
\usepackage{tikzscale}
\usetikzlibrary{graphs}
\usetikzlibrary{graphdrawing}
\usetikzlibrary{arrows,positioning,decorations.pathreplacing}
\usetikzlibrary{calc}
\usegdlibrary{trees, layered}
\usetikzlibrary{quotes}
%-------------------------------------------------------------------------------
% No spacing in itemize
\usepackage{enumitem}
\setlist{nolistsep}
% tightlist from pandoc
\providecommand{\tightlist}{%
  \setlength{\itemsep}{0pt}\setlength{\parskip}{0pt}}
 % Danger symbol (need fourier package)
\newcommand*{\TakeFourierOrnament}[1]{{%
\fontencoding{U}\fontfamily{futs}\selectfont\char#1}}
\newcommand*{\danger}{\TakeFourierOrnament{66}}
% Skull (need Symbola font)
\usepackage{amsmath,fontspec,newunicodechar}
\newfontface{\skullfont}{Symbola}[Scale=MatchUppercase]
\NewDocumentCommand{\skull}{}{%
  \text{\skullfont\symbol{"1F571}}%
}
% Hospital sign
\usepackage{fontspec} % For fontawesome
\usepackage{fontawesome}
% Itemize in tabular
\newcommand{\tabitem}{~~\llap{\textbullet}~~}
% No numbering
\setcounter{secnumdepth}{0}
% In TOC, only section
\setcounter{tocdepth}{1}
% Set header
\pagestyle{fancy}
\fancyhf{}
\fancyhead[L]{\leftmark}
\fancyhead[R]{\thepage}
%\renewcommand{\headrulewidth}{0.6pt}
% Custom header : no uper case
\renewcommand{\sectionmark}[1]{%
  \markboth{\textit{#1}}{}}
% Footnote in section
\usepackage[stable]{footmisc}
% Chemical compound
\usepackage{chemformula}

% Negate \implies
\usepackage{centernot} 

%-------------------------------------------------------------------------------
% Custom commands
%-------------------------------------------------------------------------------
% Logical and, or
\def\land{$\wedge{}$}
\def\lor{$\vee{}$}
\def\dec{$\searrow{}$}
\def\inc{$\nearrow{}$}


\usepackage[linesnumbered,ruled,vlined]{algorithm2e}
\newglossaryentry{SHBG}{name=SHBG,description={Sex Hormone-Binding Globulin. Diminue avec des androgènes, augmente avec les oestrogènes}}
\newacronym{SOPK}{SOPK}{Syndrome des ovaires polymicrokystiques}
\newacronym{OGE}{OGE}{Organes génitaux externes}
\newacronym{CAIS}{CAIS}{Complete Androgen Insensitivity Syndrome}
\newacronym{IOP}{IOP}{Insuffisance ovarienne primitive}
\newglossaryentry{Leydigcell}{name={cellule de Leydig},description={Produit de la testostérone. Localisé près des tubules séminifères (testicules)}. Activé par LH}
\newglossaryentry{Sertolicell}{name={cellule de Sertoli},description={Participe à la production du sperme. Localisé dans un tubule séminifère. Activé par FSH}}
\newacronym{HVG}{HVG}{Hypertrophie ventriculaire gauche}
\newglossaryentry{NF1}{name=NF1, description={Neurofibromatose 1. Tâches café au lait, neurofibromes (cutanées, nodulaires [le long d'un trajet d'un nerf] ou plexiformes [K possible]), nodules de Lisch sur l'iris.}}
\newacronym{NEM2}{NEM2}{Néoplasie endocrinienne multiple 2}
\newglossaryentry{VHL}{name={von Hippel-Lindau}, description={Hémangioblastome du cervelet/moelle épinière, de la rétine, phéochromocytome}}
\newglossaryentry{PCC}{name={Phéochromocytomes}, description={Tumeur de la médullo-surrénale}}
\newacronym{PGG}{PGG}{Paragangliomes}
\newacronym{ADP}{ADP}{Adénopathie}
\newglossaryentry{TPO}{name={Thyroid peroxydase (TPO)},description={Enzyme de la thyroïde servant à générer la thyroxine (T4) et triiodothyroine (T3)}}
\newacronym{ATS}{ATS}{Antithyroïdiens de synthèse}
\newacronym{GH}{GH}{Hormone de croissance (Growth hormone)}
\newacronym{CLU}{CLU}{Cortisol libre urinaire}
\newacronym{IS}{IS}{Insuffisance surrénale}
\newglossaryentry{PTH}{name={Parathyroide Hormone (PTH)},description={Stimule la résorbtion osseuse (ostéoclastes) pour libérer plus de calcium}}
\newglossaryentry{sdMetabolique}{name={Syndrome métabolique},
description={IMC > 28 kg/$m^2$, HTA,
(HDL < 0.35g/L ou TG > 2g/L ou dyslipidémie traitée),
ATCD diabète familial/gestionnel, temporairement induit.
Autre définition (NCEP III) : (\diameter abdo > 100cm \male ou 88cm \female),
hyperglycémie (glycémie à jeun > 1g/L),
dyslipidémie (TG > 1.5g/L et (HDL < 0.4g/L \male ou 0.5g/L \female)),
HTA (> 130mmHg systole ou > 85mmHg diastole)}}
\author{Alexis Praga}
\date{\today}
\title{Endocrinologie}
\hypersetup{
 pdfauthor={Alexis Praga},
 pdftitle={Endocrinologie},
 pdfkeywords={},
 pdfsubject={},
 pdfcreator={Emacs 26.1 (Org mode 9.1.9)}, 
 pdflang={English}}
\begin{document}

\maketitle
\tableofcontents

% Printing bacteria with biocon package
\newbact{chlamydia}{genus=Chlamydia, epithet=trachomatis}
\newbact{botulisme}{genus=Clostridium, epithet=botulinum}
\newbact{burnetii}{genus=Coxiella, epithet=burnetii}
\newbact{charbon}{genus=Bacillus, epithet=anthracis}
%
\newbact{tranchees}{genus=Bartonella, epithet=quintana}
%
\newbact{recurrente}{genus=Borreila, epithet=recurrentis}
\newbact{ecoli}{genus=Escherichia, epithet=coli}
%
\newbact{faecalis}{genus=Enteroccocus, epithet=faecalis}
%
\newbact{gardnerella}{genus=Gardnerella, epithet=vaginalis}
\newbact{ducreyi}{genus=Haemophilus, epithet=ducreyi}
\newbact{influenzae}{genus=Haemophilus, epithet=influenzae}
\newbact{granulomatis}{genus=Klebsiella, epithet=granulomatis}
\newbact{aeruginosa}{genus=Pseudomonas, epithet=aeruginosa}
\newbact{tuberculose}{genus=Mycobacterium, epithet=tuberculosis}
\newbact{genitalium}{genus=Mycoplasma, epithet=genitalium}
\newbact{gonocoque}{genus=Neisseria, epithet=gonorrhoeae}
\newbact{typhus}{genus=Rickettsia, epithet=prowazekii}
\newbact{conorii}{genus=Rickettsia, epithet=conorii}
%
\newbact{dore}{genus=Staphylococcus, epithet=aureus}
\newbact{gallolyticus}{genus=Staphylococcus, epithet=gallolyticus}
\newbact{saprophyte}{genus=Staphylococcus, epithet=saprophyticus}
%
\newbact{pneumocoque}{genus=Streptococcus, epithet=pneumoniae}
\newbact{pyogenes}{genus=Streptococcus, epithet=pyogenes}
\newbact{toxoplasmose}{genus=Toxoplasma, epithet=gondii}
\newbact{syphilis}{genus=Treponema, epithet=pallidum}
\newbact{trichomonose}{genus=Trichomonas, epithet=vaginalis}
%-------------------------------------------------------------------------------
%% Parasits
%-------------------------------------------------------------------------------
\newbact{saginata}{genus=Taenia, epithet=saginata}
\newbact{solium}{genus=Taenia, epithet=solium}


\section{35 : Contraception}
\label{sec:org539cd44}
\subsection{Contraception hormonale}
\label{sec:org3581d98}
Oestroprogestatifs
\begin{itemize}
\item Contient \{oestrogène, progestatif (gen 1,2 ou 3), autres progestatif\}
\item Administration orale, transdermique ou vaginale
\item Action : \{pas d'ovulation, endomètre peu à apte à la nidation, glaire
cervicale imperméable aux spermatozoïdes\}
\end{itemize}
Progestatifs seuls 
\begin{itemize}
\item Microprogestatifs : action sur glaire cervicale, endomètre
\item Macroprogestatifs : si CI oestroprogestatifs
\item Administration : orale, injection, implant, intra-utérin (stérilet)
\end{itemize}
\subsection{Pratique}
\label{sec:orgddf1a45}
Oestroprogestatifs : le premier jour des règles pendant 21j puis 7 j
d'arrêt. \emph{Tjrs au même moment}. Si oubli < 12h, ASAP sinon contraception mécanique \(\ge 7\) jours

Microprogestatifs : toujours à la même heure. Si oubli < 3h, ASAP, sinon
contraception mécanique \(\ge 7\) jours

Macroprogestatifs : commencer le 5eme jour du cycle

\subsection{Contre-indications}
\label{sec:org62bf106}
Oestroprogestatifs : absolues =
\begin{itemize}
\item thromboemboliques veineux/artériels, prédisposition thromboses
\item lupus évolutif, connectivites, porphyries
\item vasc, cardiaque, cérébrales, oculaires
\item valvulopathie, troubles rythmes thrombogènes
\item HTA non contrôlée
\item diabète et micro/macroangiopathie
\item tumeur hormono-dépendantes (sein, utérus\ldots{})
\item hépatiques sévères
\item hémorragies génitales non diagnostiquées
\item (tumeurs hypophysaires)
\end{itemize}
Macro/microprogestatifs : cancers \{sein, endomètre\}, insuf hépatique, accident
TEV récents

\subsection{Recommandation}
\label{sec:orgb8fb7eb}
Sans CI, oestroprogestatif minidosé et progestatif 2eme génération monophasique
(Minidril)

\subsection{Efficacité}
\label{sec:org83842d2}
Indice de Pearl\footnote{\(\frac{N}{N_e/10}\times 100\) avec N = nb grossesses
accidentelles, \(N_e\) nombres de mois d'exposition} < 0.07\% pour oestroprogestatif
(< 2\% pour les microprogestatifs)

Attention : certains inducteurs enzymatiques réduisens l'efficacité (ou
millepertuis).

Ado : sous- ou mal utilisée

\subsection{Tolérance}
\label{sec:org73cd727}
Oestroprogestatifs :
\begin{itemize}
\item bien tolérée, pas de perte de poids
\item surveiller métabolisme
\item active coagulation mais \inc fibrinolyse. Légère augmentation du risque
d'accident TEV
\item vasc : faible \inc PA
\item cancer : ovaire = risque -50\%, idem pour l'endomètre, faible \inc pour sein
\end{itemize}
Microprogestatifs : troubles des règles (spotting, aménorrhées), grossesse
extra-utérine

Macroprogestatifs : hypoestrogénie, aménorrhées, spotting

\subsection{Surveillance}
\label{sec:orgb475377}
Consulter si céphalée, déficit sensitivomoteur, (douleur ou oedème) MI, dyspnée,
douleur thoracique

Examen clinique : 
\begin{itemize}
\item préthérapeutique : gynéco, frottis cervico-vaginal dès 25 ans si
asymptomatique.
\item PA à +3mois puis tous 6 mois
\item hyperoestrogénie (tension mammaire), hypoestrogénie (sécheresse vaginale)
\end{itemize}
Biologie : cholestérol total, triglycérides, glycémie à jeun à +3mois. Si FR, le
faire avant (!) prescription

Gynéco : métrorragies, spottings. 

Frottis cervico-utérin dès 25 ans (+1 an puis tous 3 ans) indépendamment contraception

\subsection{Femmes à risques}
\label{sec:org9d30589}
Diabétique :
\begin{itemize}
\item non hormonale : si diabète 1 > 15 ans ou micro/macroangiopathie \thus locale
(nullipare, peu de rapports) ou intra-utérin (multipare, diabète équilibré)
\item hormonale :pas d'oestropregestatifs si \{tabac, non équilibré, HTA, surpoids,
diabète compliqué\} \thus progestatif
\end{itemize}
Dyslipidémie : oestroprogestatif si < 3g/L cholestérol total, triglycérides <
2g/L

Thrombose veineuse
\begin{itemize}
\item prédisposant : anomalies de l'hémostase (génétique, acquises), ATCD familiaux
\item dépistage : thrombose, multiples fausses couches, ATCD thrombose < 45 ans
\item CI oestrogène, acétate de chlormadinone à la place
\end{itemize}

Autres :
\begin{itemize}
\item HTA : oestroprogestatifs si 0 FR
\item tabac = CI
\item si migraine et vascularite, voir spécialiste
\end{itemize}

\subsection{Contraception d'urgence}
\label{sec:org07f96a6}
\begin{itemize}
\item lévonorgestrel ASAP < 72h
\item ulipristal acétate ASAP < 120h mais 3x plus cher
\end{itemize}

\section{37 : Stérilité du couple}
\label{sec:orgec86955}
Infertile : 0 grossesse après 1 an de rapports non protégés. Stérilité si
définitif.

Stérilité = partagée !!

\subsection{Interrogatoire}
\label{sec:org29b9515}
\begin{itemize}
\item Couple
\item Femme : âge++ (détérioration après 35 ans), \{grossesses antérieure,
avortements\}, infections/curetages++, ATCD chir/infectieux, douleurs
pelviennes (rapports, règles), conditions de vie, radio/chimio
\item Homme : trouble libido/érection, ATCD cryptorchidie/trauma testiculaire, ATCD
chir pelvienne/scrotale, ATCD médicaux (orchite ourlienne++), tabac/anabolisants\ldots{}
\end{itemize}

\subsection{Examen clinique}
\label{sec:org99f06ad}
\begin{itemize}
\item \female : âge++, obésité/maigreur, tour taille et hanche, pilosité, PA,
galactorrhé provoqué, gynéco.
\begin{itemize}
\item Si anovulation (a/oligo-ménorrhée) : hyperprolactinémie, hyperandrogénie,
troubles comportement alimentaire, bouffées chaleur
\end{itemize}
\item \male : IMC, pilosité, hypoandrisme, cicatrice chir, varicocèle\footnote{Dilatation variqueuse des veines du cordon spermatique},
gynécomastie, gynoïde/enuchoïde
\begin{itemize}
\item volume testiculaire++, palpation cordospermatiques
\end{itemize}
\end{itemize}

\subsection{Examens complémentaires}
\label{sec:orgf6ad85f}
Premiere intention, femme
\begin{itemize}
\item Hormonale++ : oestradiol, LH, FSH, prolactine plasmatique. Puis progestérone plasmatique (si cycle réguliers)
\item Écho ovarienne++
\item Hystérographie++
\end{itemize}
Première intention, homme :
\begin{itemize}
\item spermogramme++ (concentration, mobilité, morphologie). Attention aux variabilités !
\item hormonale++ si oligo-/azoo-spermie : testostérone, LH, FSH puis \gls{SHBG}
\end{itemize}
Test poist-coïtal (discuté)

\subsection{Étiologie}
\label{sec:org65dbbb6}
Femme :
\begin{itemize}
\item \emph{anovulation} : très fréquent ! Souvent aménorrhées ou irrégularités. Causes :
\gls{SOPK}, hyperprolactinémie, insuf ovarienne primitive, déficit gonadotrope, psycho-nutritionnel
\item \emph{obstacle mécanique} :
\begin{itemize}
\item anomalie du col utérin et insuf glaire cervicale : post-conisation/curetage
\item obstacle, anomalie utérine : manoeuvres post-partum, polypes muqueux\ldots{} \thus
echographie
\item obstacle tubaire : cause majeure++. Souvent salpingite (Chlamydia++)
\end{itemize}
\item \emph{endométriose} : rarement en cause si modérée. hystérosalpingographie puis coelioscopie
\end{itemize}
Homme :
\begin{itemize}
\item \emph{azoospermie}
\begin{itemize}
\item \emph{sécrétoire} : diagnostic = volume testiculaire < 10ml, concentration FSH
faible
\end{itemize}
\thus caryotype, analyse bras long Y, écho testiculaire (élimine K), déficit gonadotrope (rare)
\begin{itemize}
\item \emph{obstructive} : volume et concentration ormale, volume séminal \dec \thus
examen clinique
\begin{itemize}
\item cause congénitale : agénésie bilat des canaux déférent++ (soit anomalie
biallélique gène CFTR, soit isolée)
\item acquis : infectieux  (gonocoque, tuberculose, Chlamydia) \thus échographie
\end{itemize}
\item exploration chir testiculaire et des voies excrétrices : si azoospermie
confirmée par plusieurs spermogrammes, bilan génétique
\end{itemize}
\item \emph{oligo-asthéno-térato-spermie} : \dec nombre et mobilité, \inc formes anormales
\thus caryotype, brang long Y. Traitement = assistance médicale procréation
\end{itemize}
\section{40 : Aménorrhée}
\label{sec:org8f66a65}
Déf: absence de cycle menstruel après 16 ans (primaire) ou interruption chez
femme réglée (secondaire). Physiologique : grossesse, lactation, ménopause

Tout arrêt > 1 mois \thus enquête étiologique \danger

Atteinte de l'axe hypothalamo-hypophysio-ovarien ou anomalie tractus utérin

\begin{tcolorbox}
Pas de traitement oestrogénique sans enquête étiologique
\end{tcolorbox}

\subsection{Conduite}
\label{sec:org9d522b0}
\subsubsection{Primaire}
\label{sec:org26a95f9}
Forte proba de cause génétique/chromosomique. Chercher carences nutritionnelle

\begin{itemize}
\item Si absence de dév. pubertaire : doser FSH, LH
\begin{itemize}
\item Si basses, tumeur hypothalamo-hypophysaire, dénutrition ou génétique : \{sd
de Kalmann (anosmie), mutation récepteur GnRH (rare), atteinte
gonadotrophines (exceptionnels), mutation LH\}
\item Si hautes : sd de Turner (caryotipe 45, X0),
\end{itemize}
\item Examen gynéco, écho pelvienne
\begin{itemize}
\item Pas d'utérus : sd de Rokitanski, tissu testiculaire dans les canaux
inguinaux (ex: \acrshort{CAIS})
\item ambiguité \acrshort{OGE} : dysgénésie gonadique, hyperplasie congénitale surrénales,
anomalies sensibilité/biosynthèse androgènes
\end{itemize}
\end{itemize}
\subsubsection{Secondaire}
\label{sec:orgd87fb01}
Souvent acquises. 

Interrogatoire : médic, maladie endoc/chronique,
gynoc/obstétriques, insuf ovarienne (bouffées de chaleur). Douleurs pelviennes
cycliques : cause utérine

Examen clinique : 
\begin{itemize}
\item poids et taille (carence nutritionnelle)
\item hyperandrogénie : \gls{SOPK}, déficit 21-hydroxylase, (sd Cushing)
\item carence oestrogénique : pas de glaire +2 semainesa après saignement \thus
anovulation
\item pas de signe d'appel : enquête nutritionnelle
\end{itemize}


Dosages hormonaux : cf Table \ref{tab:amenorrhe_second}

\begin{table}
\begin{tabular}{llllll}
\toprule
hCG & prolactine \inc & FSH \inc & estradiol& testostérone \inc & sinon\\
& & & LH, FSH àdec & & \\
\midrule
grossesse & médicaments & \acrshort{IOP} & tumeur H-H & tumeur surrénales & \gls{SOPK}\\
 & adénome à prolactine &  & nutrition & tumeur ovarienne sécrét. & \\
 & tumeur H-H &  &  &  & \\
\bottomrule
\end{tabular}
\caption{Évaluation hormonale d'une aménorrhée secondaire. H-H = hypothalamo-hypophysaire}
\label{tab:amenorrhe_second}
\end{table}

\subsection{Causes}
\label{sec:org7e48c5f}

\subsubsection{1. Déficit gonadotrope organique/fonctionnel}
\label{sec:org90903d3}

\paragraph{Prolactine normale \footnote{Hypothalamus n'arrive pas à libérer la GnRH au bon rythme.}}
\label{sec:org7fc803c}
\begin{itemize}
\item Atteintes organiques : tumeur/infiltration \thus IRM
\begin{itemize}
\item macroadénomes hypophysaires, craniopharyngiomes
\item chercher hyperprolactinémie, insuf antéhypophysaire associé
\end{itemize}
\item Atteintes fonctionnelles : apports nutritionnels insufisants par rapport à l'activité physique intense+++
\end{itemize}

\paragraph{Hyperprolactinémie}
\label{sec:orgd8599e5}
Atteinte hypothalamo-hypophysaire (majeure++)

Médicaments ou tumeurs \thus pas de traitement dopaminergique sans imagerie \danger

\paragraph{Autres}
\label{sec:org5efba9e}
\begin{itemize}
\item Endocrinopathies : sd de Cushing, dysthyroïdes déficits 21-hydroxylase
\item Hypophysaire (rare) : auto-immune (majorité), sd de Sheehan (très rare, nécrose hypophysaire post-partum)
\end{itemize}

\subsubsection{2. Anovulation non hypothalamique}
\label{sec:org180cde2}
\paragraph{SOPK (majorité)}
\label{sec:org85338b5}
Pas de pic de LH, ni de progestérone. Oestradiol mais non cycliques

Irrégularité menstruelles, puis aménorrhées avec acné, hirsutisme

Diagnostic :
\begin{itemize}
\item 2 parmi : \{hyperandrogénie clinique\footnote{Séborrhéee, acné, hirsutisme, \inc testostérone}, oligo-/a-novulation, hypertrophie
ovarienne (écho)\}
\item exclure bloc 21-hydroxylase, tumeur de l'ovaire, sd Cushing
\item exclure hyperprolactinémie
\end{itemize}

Diagnostic parfois difficile :
\begin{itemize}
\item sans hyperandrogénie \thus écho
\item \{atteinte partielle axe gonadotrope, macroprolactinémie\} peuvent y ressembler
\end{itemize}

Acné : cherche hyperandrogénie, régularité cycle menstruel \thus éliminer
hyperplasie congénitale des surrénales

2 causes :
\begin{itemize}
\item tumeur ovarienne ou résistance insuline
\begin{itemize}
\item virilisation si tumeur
\item imagerie si testostérone > 1.5ng/mL. Si normale, cherche hypothécose
(obésité morbide androïde, acanthosis nigricans, insulino-résistance)
\end{itemize}
\item pathologie surrénale :
\begin{itemize}
\item sd de Cushing si signes hypercortisolisme \thus cortisol libre urinaire et
freinage minute
\item tumeur surrénale \thus scanner des surrénales
\item déficit enzymatique en 21-hydroxylase (\danger formes tardives qui peuvent
mimer SOPK)
\end{itemize}
\end{itemize}

\subsubsection{3. Insuf ovarienne primitive}
\label{sec:org6aa8f88}
\inc FSH

Causes :
\begin{itemize}
\item chir, chimio, radiothérapie
\item anomalie caryotype (sd Turner)
\item anomalie gènes \emph{FMR1} (sd X fragile)\footnote{Transmission mère-fils. Expansion instable des triplets CGG jusqu'à
l'absence de transcript de FMR1 (Fragile X Mental Retardation 1). \\
\danger Risque d'IOP pour la pré-mutation seulement \thus dépister chez
\female + IOP < 40 ans par PCR et Southern Blot}
\item auto-immune
\end{itemize}

\subsubsection{4. Anomalie utérine}
\label{sec:org3256169}
\paragraph{Congénitales}
\label{sec:orgf95baaf}
Si dév pubertaire normal :
\begin{itemize}
\item et douleurs pelviennes cycliques :  imperforation hyménéale/malformation vaginale \thus examen gynéco.\\
\item ou sans douleurs \thus agénésie utérus ?
\end{itemize}

Difficulté : différence agénésie mullérienne isolée (46,XX)- anomalies androgènes
(46,XY) \thus testostérone

\paragraph{Secondaires}
\label{sec:orgfd85482}
Synéchies utérines (trauma de l'utérus), tuberculose utérine
\section{47 : Puberté}
\label{sec:orga842b34}
\subsection{Normale}
\label{sec:orgaf1f3b6}
\textasciitilde{}4 ans, acquisition de la taille définitive, fonction de
reproduction. Classification de Tanner (5 stades)

\begin{table}[htbp]
\caption{Puberté normale}
\centering
\begin{tabular}{llll}
\toprule
\female &  & \male & \\
\midrule
seins & 11 ans [8,13] & volume testiculaire & 11.5 ans [9.5,14]\\
règles & 13 ans [10,15] & \inc taille verge & 12.5 ans\\
croissance & 5 \(\rightarrow\) 8cm/an & croissance & 5 \(\rightarrow\) 10cm/an\\
taille & 163cm & taille & 175cm\\
\bottomrule
\end{tabular}
\end{table}

\subsection{Retards}
\label{sec:orga820eb2}
\begin{tcolorbox}
\male :  volume testiculaire < 4mL après 14 ans \footnotemark\\
\female : pas de seins à 13 ans, pas de règles à 15 ans
\end{tcolorbox}
\footnotetext{ou longueur < 25mm}

\begin{tcolorbox}
Hypogonadisme\footnotemark central ou périphérique ?
\begin{itemize}
\item FSH, LH \inc : pour compenser le manque des gonades (hypergonadotrope = primaire) 
\item FSH, LH N ou \dec : problème hypothalamo-hypophysaire\footnotemark (hypogonadotrope = secondaire)
\end{itemize}
\end{tcolorbox}
\footnotetext\{Chez \male{}, manque de testostérone\}
\footnotetext{Rappel : LH entraîne la production de testostérone}

\begin{itemize}
\item centrale : congénital (pas de cassure de croissance, micropénis,
cryptorchidie), acquis (tumeur ?), "fonctionnel" (maladie générale, trouble
comportement alimentaire), isolé
\item périphérique : sd de Turner chez \female, sd Klinefelter \male
\item retard simple (élimination)
\end{itemize}

Clinique : 
\begin{itemize}
\item parents, grossesse, courbe de croissance. Chercher trbles digestifs, polyuro-polydispsie, céphalée, anomalies champ visuel
\item pathologie acquise, OGE, testicules, anosmie (Kallmann)
\end{itemize}

Âge osseux : 13 ans \male, 11 ans \female

Biologie : stéroïdes sexuels
\begin{itemize}
\item FSH, LH basses \thus hypothalamo-hypophysaire
\item testostérone chez \male, oestradial/écho chez \female
\end{itemize}

IRM indispensable si déficit gonadotrope (tumeur) \danger

Caryotype si :
\begin{itemize}
\item FSH élevé
\item toujours chez \female{} de taille < -2DS avec retard pubertaire/gonadotrophine \inc
\end{itemize}

\subsubsection{Étiologies}
\label{sec:org984c09b}
Hypogonadotropes
\begin{itemize}
\item congénitaux : isolés, sd de Kallman, autres déficits hypophysaires, sd
polymalformatifs
\item acquis : tumeurs hypophysaires, post-radiothérapie
\end{itemize}
Hypogonadotropes fonctionnels
\begin{itemize}
\item maladies chroniques digestives/cardiaques/respi
\item sport intense
\item maladies endocriniennes
\end{itemize}
Hypergonadotropes
\begin{itemize}
\item congénitaux : sd Turner, sd Klinefelter, autres atteintes primitevs
\item acquis : castration, trauma, oreillons, chimio/radio
\end{itemize}

\subsubsection{Traitement}
\label{sec:org9fc62d5}
Cause si possible. Sinon doses \inc de testostérone (\male) ou oestrogènes puis
oestroprogestatif (\female)
\subsection{Précoces}
\label{sec:org42abb44}
Avant 8ans \female ou 9.5 ans \male
\subsubsection{Centrales}
\label{sec:org4fd68ef}
8x plus fréquent chez \female{} que \male{}. Chez \female{}, causes
idiopathiques. Chez \male{}, causes tumorales à 50\%

Clinique : 
\begin{itemize}
\item dév prématuré harmonieux (pas de règles chez \female)
\item crises de rires (harmatome hypothalamique), tâches cutanées (neurofibromatose
I ou sd McCune-Albright)
\end{itemize}
Biologie :
\begin{itemize}
\item testostérone élevée chez \male{} mais variabilité d'oestradiol chez \female{}
\end{itemize}

IRM hypothalamo-hypophysaire indispensable \danger (petite taille
définitive). Écho pelvienne pour \female{}

Traitement si risque de petite taille adulte : analogues GnRH jusque âge normal
de puberté
\subsubsection{Périphériques}
\label{sec:org445207b}
Clinique : \inc vitesse de croissance, avance maturation osseusse

Stéroïdes élevées, LH et FSH bas. Écho pelvienne chez fille

Étiologie :
\begin{itemize}
\item tumeurs ovarienne (rares) : écho puis histologies
\item kystes folliculaires : bénins, régression spontanée possible
\item sd McCune-Albright : 
\begin{itemize}
\item \{puberté précoce ovarienne, taches cutanées "café-au-lait", dysplasie fibreuses os\}. \danger tableau pas toujours complet !
\item oestradiol élevé, gonadotrophines basses, écho = utérus stimulé, kystes ovariens. Dominance \female
\end{itemize}
\item médicaments
\item testotoxicose (rare, cellule de Leydig activé et LH basses), adénome leydigien
(très rare)
\item tumeurs à hCG (\male)
\end{itemize}
\subsubsection{Avances dissociées}
\label{sec:orgeb93510}
\begin{itemize}
\item Isolé des seins : beaucoup de filles ( de 3 mois à 3 ans)
\item Métrorragies isolées : chercher vulvite, vulvovaginite, prolapsus urétrale,
corps étranger. Éliminter kyste ovarien, sd McCune-Albright par l'absence des
sein
\end{itemize}
\thus écho pelvienne
\begin{itemize}
\item Pilosité pubienne isolée : chercher forme d'hyperplasie congénitale des
surrénales (\inc 17-hydroxyprogestérone, stimulation ACTH), prémature pubarche
(élimination !)
\end{itemize}

\section{48 : Cryptorchidie}
\label{sec:orga7fd930}
\subsection{Enfant}
\label{sec:orgfef3a1b}
Localisation anormale et inaboutie du testicule. Très fréquente : 3\%
nouveaux-nés, 20\% préma. 2/3 descendent spontanément à 1 an de vie

Clinique : checher micropénis (< 2cm, hypospadias, autres)

Explorations : endocrinienne pour toute cryptorchidie \danger
\begin{itemize}
\item bilatérale : doser 17-hydroxyprogestérone chez \female{} virilisée pour éliminer hyperplasie
congénitale des surrénales
\item testostérone, \gls{Leydigcell} (INSL3), \gls{Sertolicell} (AMH, inhibine B sérique), FSH, LH mesurée jusque 4-6mois\footnote{\danger Testostérone, FSH, LH interprétables [6mois, puberté]}
\item si bilatéral, écho (vérifier l'absence de dérivés mülleriens)
\end{itemize}

Étiologie
\begin{itemize}
\item hypogonadisme hypogonadotrope congénital
\item anorchidie rare
\item si hypospade en plus, chercher dysgénésie testiculaire
\item sd polymalformatif
\end{itemize}

Suivre l'âge de l'apparition de la puberté !

Traitement : chir dès 2 ans, indispensable ! (risque de cancer)
\subsection{Adulte}
\label{sec:orgc11ba55}
\begin{itemize}
\item Risque : hypogonadisme, infertilité, cancer testicule
\item Examen clinique : scrotum, gynécomastie, signes d'hypogonadisme
\item Complémentaire : \{FSH, LH, testostérone\}, hCG si tumeur à la palpation, écho
scrotale, spermogramme
\end{itemize}
\section{51 : Retard de croissance}
\label{sec:org9884db2}
\danger Ne pas passer à côté de pathologies sévères

Phases : 
\begin{itemize}
\item foetale (rapide, \{nutrition, insuline, IGF-2\})
\item précoce 0-3ans (rapide, \{insuline, IGF, hormones thyroïdiennes\})
\item prépubertaire (plus lente, décroît, \{génétique, GH/IGF, hormones thyroïdiennes\})
\item pubertaire (\{stéroïdes sexuels, GH, nutrition\})
\end{itemize}

Retard statural = \{taille < -2DS, ralentissements croissance, croissance \(\le\) parents\}

Prise de poids, obésité, ralentissement croissance \thus chercher
hypercorticisme, tumeux craniopharyngiome sur l'hypothalamus, hypopituitarisme

Examen :
\begin{itemize}
\item ATCD : taille, parents, néonatale, médicaux/chir, contexte social
\item morphotype, dév. pubertaire, tous les système, psychoaffectif
\end{itemize}

\subsection{Principales causes}
\label{sec:org26d07d3}
\begin{itemize}
\item Si poids < poids idéal : cf table \ref{tab:org55b5dc0}
\item Si poids \(\ge\) poids idéal : cf table \ref{tab:orgc3f8b0d}. Précisions :
\begin{itemize}
\item Test de stimulation de l'hormone de croissance (\danger si doute, IRM)
\item Ralentissement sévère \thus bilan en urgence (craniopharyngiome, thyroïdite de
Hashimoto) \skull
\item\relax [0, 3] ans : digestives pédiatrique (coeliaque, mucoviscidose), [3,puberté] :
endoc constitutionnelle, à la puberté : déficit hormone, patho osseuse
\item Savoir différencier retard pubertaire simple d'un vrai retard
\end{itemize}
\end{itemize}

\begin{table}[htbp]
\caption{\label{tab:org55b5dc0}
Causes de retard pondéral}
\centering
\begin{tabular}{ll}
\toprule
Maladie coeliaque & IgA totales, IgA anti-transglutamase, fibro\\
Crohn & VS, écho anse grêle\\
Mucoviscidose & Test sueur\\
Anorexie mentale & Courbe de poids\\
Insuf rénale chroniques & Créat, iono, explo fonctionnelles\\
Anémie chroniques & NFS\\
Rachitisme hypophosphatémique & Bilan phosphocalcique\\
Patho mitochondriales & lactate/pyruvate, génétique, biopsie musc, fond d'oeil\\
Nanisme psychosocial & \\
\bottomrule
\end{tabular}
\end{table}

\begin{table}[htbp]
\caption{\label{tab:orgc3f8b0d}
Causes de retard statural}
\centering
\begin{tabular}{lll}
\toprule
Endocrino & Déficit GH (congénital, acquis [tumeur]) & IRM\\
 & Hypothyroïdie & T4L, TSH, Ac anti-TPO\\
 & Hypercorticisme (iatrogène) & Cortisol libre urinaire/à 23h, ACTH\\
 & Déficit hormones sex. & Testostérone, GnRH, IRM\\
\midrule
Constitutionelles & Sd Turner & Caryotype\\
 & Sd Noonan & Gène PTPN11\\
\midrule
Autres & Osseuses (a-/hypo-chondroplasie) & Radio\\
 & RCIU & Taille naissance\\
 & Petite taille idiopathique & Élimination\\
\bottomrule
\end{tabular}
\end{table}


\subsection{Exploration :}
\label{sec:org15cda80}
\begin{itemize}
\item Caryotype : fille taille < -2DS ou < -1.5DS sous taille parentale moyenne
\item NFS, VS, foie, rein
\item IgA totales, anti-transglutaminase
\item GF-1, T4L, TSH
\item Radio
\end{itemize}

\section{78 : Dopage}
\label{sec:org77e8421}
\subsection{Substances augmentant la testostérone}
\label{sec:orgafce8a1}
\begin{itemize}
\item Stéroïdes anabolisant, testostérone : \inc masse musc, puissance
\item Risque : thrombotique, rupture musculo-tendineuse, trouble personnalité, foie, trouble libido, adénome/cancer de la prostate
\item Femmes : masculinisation, hirsutisme, acné, aménorrhée, anovulation, hypertrophie clitoridienne, libido exacerbée
\end{itemize}

\vspace*{0.5cm}
\begin{itemize}
\item \emph{Testostérone} : test chromatographique + spectrométrie de masse (très sensible
et spécifique)
\item \emph{Dihydrotestostérone} (DHT) : traitement gynécomastie
\item \emph{Anabolisants} : \inc tissu cellulaire (muscle).
\end{itemize}
ES : rétention hydrosodée, HTA, IDM, hépatite
\begin{itemize}
\item \emph{hCG} : diminuer épitestostérone/testostérone après dopage (IM, SC). Testée dans
le sang ou urine.
\item \emph{Anti-oestrogène} : stimule production testiculaire de stéroïdes
\end{itemize}

\subsection{Hormone de croissance (GH), IGF-1}
\label{sec:orge7edf04}
\begin{itemize}
\item GH \inc masse musculaire, modifie architecture sequelette, acromégalie \emph{mais}
pas d'effet sur volume d'activité physique. Détection difficile : approche
indirecte (cascade biologique) et mesure des forme circulante et comparaison à r-hGH
\item IGF-1 mime certains effet GH
\end{itemize}

\subsection{Glucocorticoïdes, ACTH}
\label{sec:orgd2c082f}
\begin{itemize}
\item Glucocorticoïdes : antalgiques, psychostimulants, combativité. ES : HTA,
oedème, rupture ligament/tendon
\end{itemize}
\danger arrêt brutal = dangereux \skull

\section{120 : Ménopause et andropause}
\label{sec:org7e6e969}
\label{sec:120}
\subsection{Ménopause}
\label{sec:org0be1514}
Déf: plus de règle > 1 an \textpm{} sd climatérique, lié à une carence
oestrogénique. Vers 51 ans.

Pré-ménopause : irrégularités cycles puis dysovulation puis anovulation \textasciitilde{}5 ans
avants.

\subsubsection{Diagnostic}
\label{sec:orgc75c3f1}
Clinique seulement \danger : bouffées de chaleur, \female > 50 ans. Bio
seulement si hystérectomie \thus \dec oestradiol et \inc FSH

En pratique : progestatif seul 10j/mois x3 \thus pas de saignement à l'arrêt =
diagnostic

Aménorrhée < 40 ans = pathologique !

\subsubsection{Conséquences}
\label{sec:orgf8d5614}
Court terme : bouffées de chaleur, trouble sommeil/humeur, \dec sécrétions
vaginales

Moyen terme : douleurs ostéoarticulaires, \inc perte osseuse (selon ATCD d'insuf
ovarienne prématurée, fractures non traumatiques, médicaments, calcium/vit D)

Long terme : \inc risque CV. Incertitude sur SNC

\subsubsection{Traitement}
\label{sec:orgec81c6c}
Bénéfices
\begin{itemize}
\item court terme : qualité de vie à +5-10 ans
\item long terme :
\begin{itemize}
\item prévention ostéoporose
\item cardiovasculaire et neuro = incertain
\item cancer du côlon
\end{itemize}
\end{itemize}
Risques :
\begin{itemize}
\item \inc cancer du sein, accident veineux thromboemboliques (mais chiffres absolus
faibles)
\item \inc AVC ischémique, lithiase bilaires
\end{itemize}

\paragraph{Thérapeutique}
\label{sec:orgc5ffbac}
\begin{itemize}
\item oestrogène (17\(\beta\)-oestradiol) oral/percutané/transdermique\footnote{Percutané, transdermique : limite \inc facteur de coagulation.} 25 jours/mois
\item \textbf{et} progestatif (au moins les 12 derniers jours) per os/transdermique
\end{itemize}
\danger hémorragie de privation possible. Si pendant le traitement, faire écho
pelvienne, hystéroscopie

\paragraph{CI}
\label{sec:orgda06f84}
Cancer du sein, endomètre, ATCD thromboembolique artériel (ischémique,
cardiopathie embolinogène) ou veineux, hémorragie génitale sans diagnostic, hépatique

\paragraph{Mise en route}
\label{sec:orga6511dc}
\begin{itemize}
\item Interrogatoire : ATCD \{cancer, métabolique, vasculaire\}, carence oestrogénique
\item Examen physique : poids, PA, palpation seins, gynéco, frottis cervico-vaginal
\item Mammograhpie !
\item Cholestérol, triglycérides, glycémie
\end{itemize}

\paragraph{En pratique}
\label{sec:orgdd635e2}
1ere intention si trouble fonctionnels importants. 2eme si risque
d'ostéoporose. Sinon au cas par cas.

\paragraph{Surveillance}
\label{sec:org1b8c3ab}
3-6mois (surdosage = douleur, tension mammaire). Puis tous les 6-12 mois,
mammographie tous les 2 ans, frottis CV tous les 3 ans.

Traitement \(\ge\) 5 ans !!

\paragraph{Alternatives}
\label{sec:orgab4c595}
\begin{itemize}
\item Modulateurs spécifiques du récepteur des oestrogènes : raloxifène
\item tibolone
\item traitement local préserve tractus urogénital
\end{itemize}

NB : Dépister FR CV. Promouvoir exercice, calcium, vit D

\subsection{Andropause}
\label{sec:orgf2a226e}
Chez majorité des hommes mûrs/âgés en bonne santé non obèse, baisse de
testostérone inconstante (2\%).

\subsubsection{Démarche}
\label{sec:org2c825c8}
\begin{itemize}
\item Interrogatoire : libido, érection, énergie vitale, mobilité/activité physique
\item Examen clinique : IMC, virilisation, gynécomastie, palper testicules
\item Mesure de testostérone totale :
\begin{itemize}
\item > 3.2ng/mL = normale \thus étiologies non endocrino
\item \(\in\) [2.3, 3.2] : doser SHBG, calculer index de T libre, si bas, chercher cause
\item < 2.3 ng/mL : chercher cause
\end{itemize}
\end{itemize}
\subsubsection{Étiologie}
\label{sec:org3c66ac6}
Si FSH, LH élevée, \emph{insuf testiculaire primitive} 
\begin{itemize}
\item lésionnelle : chimio, radiation, alcoolisme surtout. Autres : castration,
torsion, orchite ourlienne
\item cryptorchidie bilatérale
\item chromosomique : sd Klinefelter++
\item lié à sénescene, cause génétique (rare !)
\end{itemize}
Sinon \emph{hypogonadisme hypogonadotrope}
\begin{itemize}
\item tumeur région hypothalamo-hypophysaire : craniopharyngiome, adénome
hypophysaire++, autres
\item infiltratif : sarcoïdose, hémochromatose
\item chir, radiothérapie, traumau
\item hyperprolactinémie, carence nutritionnelle, Cushing, tumeur testiculaire
\end{itemize}

\section{122 : Troubles de l'érection}
\label{sec:org7edd3d4}
Nécessite : réseau vasculaire, appareil musculaire lisse, retour veineux, signal  nerveux,
appareil hormonal et psychisme fonctionnels

Déf : incapacité persistante à obtenir/maintenir érection pour rapport sexuel satisfaisant

Âge = FR (car déficit neurosensoriel, \inc testostérone, comorbidités)

\subsection{Conduite  diagnostique}
\label{sec:org29a8cb9}
\subsubsection{Interrogatoire}
\label{sec:org72da416}
\begin{itemize}
\item DD avec perte désir, trouble éjaculation, douleurs pendant, anomalies morphologiques
\item Caractérisation : primaire/secondaire, brutal/progressif,
permanent/situationnel, sévérité (délai trouble-consult, capacité résiduelle,
masturbation)
\item Pathologies, facteur :
\begin{itemize}
\item primaire : trouble psychogène perso, complexe identitaire, trouble
relationnel, conflit socioprof, anomalie génitale
\item secondaire : ATCD abdo-pelvien, diabète, FR CV, patho CV, neuro, trouble
miction, endocrinopathie, troubles sommeil, traitement, déficit
androgénique, sd dépressif, troubles addictifs
\end{itemize}
\end{itemize}
\subsubsection{Clinique}
\label{sec:org5d73eaa}
\begin{itemize}
\item Gynécomastie, hypoandrisme, petits testicules, anomalies du pénis (La Peyronie)
\item CV : HTA, pouls, souffle
\item neuro : sensibilités périnée, MI
\item endoc : anomalie CV
\end{itemize}
\subsubsection{Bio}
\label{sec:org78afc0a}
Glycémie, lipidique (si > 1 an), \{NFS, iono, créat\}, foie (si > 5 ans), déficit
androgénique

Doser prolactine, hormones thyroïdiennes

\subsection{Bilan secondaire et approfondi}
\label{sec:org3f1b94a}
Secondaire : sexo/psychologique, épreuve pharmacologique (prostaglandine,
inhibiteur de la phospohdiestérase 5)

\subsection{Étiologies}
\label{sec:org479a3f8}
Plus fréquentes :
\begin{itemize}
\item vasculaire : FR = HTA
\item endocrino++ : diabète
\item génito-pelvien : chir pelvienne
\item trauma médullaire
\item neuro dégénératif
\item iatrogène : antihypertenseur
\end{itemize}

\subsection{Aspects endocriniens}
\label{sec:org26d4fa4}
\subsubsection{Androgènes circulants}
\label{sec:orgb5e4b67}
Influe libido, intérêt sexuel, érection (seulement spontanée!)

Hypogonadisme (diag difficile) : 
\begin{itemize}
\item asthénie, gynécomastie, dépilation, perte force musculaire, adiposité androïde
\item doser testostérone totale \textpm{} SHBG, prolactine. FSH, LH pour l'origin
\end{itemize}

\subsubsection{Hyperprolactinémie}
\label{sec:orga7a359e}
Tumeur hypophysaire (IRM), champ visuel si tumeur
supra-sellaire, \{T4L, cortisol, IGF-1, testostérone\}
\thus correction par agoniste dopaminergique

\subsubsection{Diabète}
\label{sec:org18f41c4}
sucré = 1ere cause de trouble érectile (TE). TE fréquents chez diabètique. 

Facteurs : mal équilibré, complications, âge, ancienneté diabète

Physiopatho : neuropathie autonomie, microangiopathie \thus défaut relaxation
musculaire. Macroangiopathie \thus ischémie organes érectiles

\danger facteurs psychogènes hyportants !

Diabète et TE \thus mesure testostérone systématique (hypogonadisme ?)

Clinique : 
\begin{itemize}
\item TE peut révéler diabète.
\item diabète et TE : cherche trouble endoc, vasc, neuro, médicament, dépression
\item TE = FR d'ischémie myocardite silencieuse \danger
\end{itemize}

\subsection{PEC}
\label{sec:orga08f20d}
Ttt étiologie seulement pour : trouble psychogène pur, chir possible, endocrino

\subsubsection{Trouble endocrinien}
\label{sec:orgb78a2a6}
\begin{itemize}
\item Si hypogonadisme confirmé par bio\footnote{Baisse libido, testostérone totale < 3 ng/mL} : androgène oraux/intramusc/transderm
\item CI : nodule prostatique palpable, PSA > 3ng/mL
\item Surveiller prostate, foie, hématocrite
\end{itemize}

\subsubsection{Pharmacologique}
\label{sec:orged89f3d}
\begin{itemize}
\item FR, Hb glyquée < 7\%, psycho/sexologique
\item 1ere intention 
\begin{itemize}
\item inhibiteurs des phosphodiésterases type 5\footnote{Efficace si stimulation sexuelle (même chez diabétique)}
\item Sinon apomorphine, yohimbine = peu efficace
\item "Pompe" = efficace mais résistance psycho
\end{itemize}
\item 2eme intention : drogue vasoactive = efficace mais douleurs peniennes, priapisme
\item Prothèses péniennes = dernier recours, par chirurgien spécialisé
\end{itemize}

\section{124 : Ostéopathies}
\label{sec:orgc997287}
Ostéoporose : fragilité excessive du squelette (\dec minéraux osseux, modif
microarchitecture). T-score < -2.5 DS valeur moyenne par DXA

Dominance \female. Primaire ou secondaire :
\begin{itemize}
\item endocrino : hypogonadisme, sd Cushing, hyperthyroïdie, hyperparathyroïdie, diabète
\item digestives, générale, génétique, médicaments, autres
\end{itemize}

\subsection{Hypogonadisme}
\label{sec:org170d3b2}
Carence oestrogénique \inc ostéoclastogénèse. Aggravé par la précocité,
déminéralisation
\subsubsection{Anorexie mentale}
\label{sec:org1ef689d}
Biochimie : marqueurs de formation \dec (isoenzyme des phosphastales alcalines,
ostéocalcine), marqueurs de résorption normaux (CTx, NTx) 

Aggravé par troubles nutritionnels. Hypercortisolisme hypothalamique réversible

Ostéoporose fréquente, risque de fractures \(\times 7\)

Traitement 
\begin{itemize}
\item multidisciplinaire
\item pilule oestroprégestative en pratique (limite perte osseuse)
\end{itemize}

\subsubsection{Activité physique intensive}
\label{sec:orgb28d2bb}
Hypoestrogénie hypothalamique

Facteur : activité\footnote{Marathon, danse classique, demi-fond, triathlon, gymnastique, cyclisme}, troubles menstruels, apports alimentaires

Résorption généralisée (rachis++), \inc fractures de fatigue

Traitement : si aménorrhées, \dec activité ou oestroprogestatifs

\subsubsection{Pathologies hypophysaires}
\label{sec:org07a22e6}
Prolactinomes, adénome corticotrope influent remodelage osseux

Perte osseuse rapide (8\% par an), récupération variable.

Traitement : \female{} non ménopausée : oestrogénothérapie

\subsubsection{Iatrogènes}
\label{sec:org6fb2346}
Agonistes GnRH (patho utérines), inhibiteurs aromatase (cancer sein)

Réversible à l'arrêt (moins bien si âgée)

Traitement : bisphosphonates, denosumab

\subsubsection{Dysgénésies gonadique}
\label{sec:orge0efb7b}
Sd Turner = plus fréquent (1/2500 à naissance)

\dec masses osseuse, continue à l'adolescence. \inc risque fracture chez
l'adulte.

Traitement : oestrogénisation (hypogonadisme) et hormone de croissance. Adulte :
oestroprogestatif

\subsection{Hyperthyroïdie, traitement par hormones thyroïdiennes}
\label{sec:orgcd803a8}
Cause fréquente d'ostéoporose secondaire \thus dosage systématique TSH

Hormones thyroïdiennes \inc remodelage : résorption sur l'os cortical++ et trabéculaire

En pratique, rarement évolution jusque l'atteinte osseuse (ttt
rapide). Adapter posologie hormones thyroïdiennes au cancer thyroïdien.

Prévention :
\begin{itemize}
\item densitométrie
\item bisphosphonates sujet âgé ou risque extrémité supérieure du fémur
\item surveillance si ttt suppressif de fonction thyroïdienne
\end{itemize}

\subsection{Hypercortisolisme, corticothérapie}
\label{sec:org289dbfe}

\dec ostéoblastes, \inc activité ostéoclaste. \dec absorption intestinale
calcium, \inc pertes urinaires calcium, hyperparathyroïdisme

Surtout trabéculaire (vertèbres, côtes, radius). Aggravé si prépubertaire, hypogonadisme

Fractures vertébrales fréquentes, surtout sd Cushing avec adénome corticotrope/surrénalien

Traitement : 
\begin{itemize}
\item pré-corticothérapie : status osseux, FR
\item supplément vitaminocalcique
\item bisphosphonates, tériparatide si corticothérapie > 3 mois, prednisone > 7.5mg/j
et T-score \(\ge -1.5\)
\end{itemize}

\subsection{Hyperparathyroïdie primitive}
\label{sec:org56e85c0}
Fréquent, notamment chez femme ménopausée. Ostéoporose fréquente \thus dépistage
systématique par DXA

Production continue PTH : \inc résorption os cortical (tiers proximal radius,
fémur)\footnote{Souvent aussi extr supérieure du fémur et vertèbres.}

Diminution limitée (10\%). Souvent favorable post-parathyroïdectomie.

Traitement : chir si T-score < -2:5. Sinon anti-ostéoclastiques\footnote{Oestrogènes, raloxifène, bisphosphonates}, calcimimétique\footnote{Cinacalcet}

\subsection{Chez l'homme}
\label{sec:org1edd6c2}
Pas de T-score reconnu. 

Fracture radius distal plus rares.

Ostéoporoses secondaires plus fréquentes chez l'homme : hypercorticisme,
hypogonadisme congénital/acquis/iatrogène, alcoolisme, hypercalciurie
idiopathiques, génétique
\section{207 : Sarcoidose}
\label{sec:org2f52bf6}
Atteinte hypothalamo-hypophysaire exceptionnelle. Conséquences : diabète
inspide central, insufisance gonadotrope

Radio : IRM centrée sur hypothalamo-hypophyse = référence (T1,T2 injecté) \thus
infiltration plancher 3eme venticule, infundibulum, tige hypophysaire épaissie
\textpm{} hypophyse augmente de volume

DD : tuberculose, histiocytose, lymphome, autres tumeurs de la région 

Si patient avec sarcoïdose connue : diagnostic = déficit endocrinien et imagerie\footnote{Faire bilan hormonal : natrémie, testostérone totale et libre/ostradiol,
FSH, LH, T4L, TSH, prolactine}

Sinon : atteinte rare\footnote{adénome hypophysaire 90\%, méningiome, craniopharyngiome, patho
inflammatoires infiltratives}, diag = radio et arguments sarcoïdose\footnote{Atteinte poumon évocatrice, \inc{} ACE, pas de tuberculose, granulome
non caséeux (histologie)}.

Traitement : sarcoïdose et déficits hormonaux

\section{215 : Hémochromatose}
\label{sec:orgd0cdf97}
Hémochromatose primitive : génétique, surcharge en fer. 5 pour 1 000 !

Physiopatho : 
\begin{itemize}
\item Absorption intestinale régule stockage de fer
\item Fer entre dans l'entérocyte (DMT1), puis stocké via ferritine ou relargé par ferroportine
\item Hepcidine \dec quand besoins fer \inc (!)
\item Hémochromatose : hepcidine effondrée, DMT1 et ferroportine \inc
\end{itemize}

Génétique : gène HFE à 95\% et mutation C282Y/C282Y ou C282Y/H63D

\subsection{Clinique}
\label{sec:org0600d82}
En pratique, suspicion aux "3 A" : asthénie, arthralgies, \inc ALAT

\subsubsection{Atteintes :}
\label{sec:org5f45c0c}
\begin{itemize}
\item foie : \inc ALAT ou hépatomégalie. Cirrhose \(\approx\) 90\% décès
\item coeur : cardiopathie dilatée, troubles rythme
\item endocrino :
\begin{itemize}
\item diabète++ (accumulation pancréatique de fer) insulino-pénie/-résistance
\item hypogonadisme+ : impuissance \male, aménorrhée \female, \dec libdio,
ostéoporose
\item insuf thyriodienne exceptionnelle
\end{itemize}
\item articulaire : arthrite chronique ("poignée de main"), chrondocalcinose
\item cutané : mélanodermie (tardive)
\end{itemize}


\subsection{Diagnostic}
\label{sec:org4482947}
\begin{itemize}
\item Si CS-Tf\footnote{Coefficient de saturation de la transferrine} < 45\% : si ferritine \inc, cherche hépatosidérose dysmétabolique,
acéruléoplasminémie, mutation gène de la ferroportine 1
\item Sinon, CS-Tf > 45\% : 
\begin{itemize}
\item si C282Y/C282Y ou C282Y/H63D : diagnostic
\item sinon, si ferritine \inc, test génétique de 2eme intention, biopsie
hépatique
\end{itemize}
\end{itemize}

Examen complémentaires : pancréas (glycémie),  foie (transaminases, écho abdo), ECG \textpm{} écho
cardiaque, radio articulation, bilan testostérone

Dépistage chez parents (1er degré) : bilan martial \textpm{} dépistage génétique. \danger mutation \(\neq\) maladie

\subsection{Stades}
\label{sec:org3a36a76}
\begin{enumerate}
\item Asymptomatique, CS-Tf, ferritinémie normaux
\item CS-Tf \inc
\item CS-Tf \inc et ferritine \inc
\item Idem et expression clinique affectant qualité de vie
\item Idem et expression clinique affectant pronostic vital
\end{enumerate}

\subsection{Traitement}
\label{sec:org28dfb59}
À partir du stade 2

\subsubsection{Saignées = référence}
\label{sec:org63b4091}
Objectif : ferritine < 50 g/L (hebdomadaire) puis entretien tous les
  2-4 mois. Ne pas dépasser 550mL !

CI : anémie sidéroblastique, thalassémie majeure, cardiopathies sévères

\subsubsection{Autres}
\label{sec:org818b6c2}
\begin{itemize}
\item Érythraphérèse : coûteuse, plus difficile
\item Chelation du fer : 2eme intention (coût, effets indésirable)
\item diététique : pas d'alcool, éviter vitamine C mais \textbf{conserver} apports en fer !
\item Symptomatique
\end{itemize}

\subsection{Suivi}
\label{sec:org0033037}
Résultats en 3-6 mois sur été générale. 

Bilan ferrique (stade 0,1) ferritinémie, hémoglobine (stade 2 à 4)

\section{219, 220 :  Facteurs de risque CV, dyslipidémies}
\label{sec:org5813199}
Rappel : FR = causalité et \{relation forte, \(\propto\) dose, indépendant des autres
FR, plusieurs étude, exposition précède maladie, plausible, réversible++\}

Risque absolu = un individu. Relatif = \(\frac{R_{\text{exposé}}}{R_{\text{non exposé}}}\)

Prévention : primaire (avant accident), secondaire (éviter nouvel), tertaire
(traiter séquelles\}

\subsection{FR}
\label{sec:orgfb1d3b0}
\begin{itemize}
\item Non modifiable : âgé (> 50A \male, > 60A \female), ATCD familiaux IDM/mort
subite < 55A \male, 65A \female)
\item Modifiable : tabagisme, LDL \inc, HDL < 0.40g/L, diabète, insuf rénale
chronique
\end{itemize}

Estimation : Étude Framingham, SCORE. En pratique : 
\begin{itemize}
\item interrogatoire  : ATCD familiaux CV, personnels, FR
\item examen : athérome asymptomatique (pouls périphérique, souffle vasculaire),
symptomatique (ECG)
\end{itemize}

\subsection{Tabac}
\label{sec:orge9eb099}
30\% adulte, 25\% femmes enceintes

\inc rapide du risque après sevrage (mécanisme prothrombotique du tabac). 

RR = 3 de maladie coronarienne, = 5 d'IDM/mort subite, = 2-7 d'AOMI, = 2 d'AVC.

\danger tabac - contraception oestroprogestative

\subsection{Hyperlipidémie}
\label{sec:org4d8d9c5}
\hyphenation{hyper-cho-lesté-ro-lé-mie}
\showhyphens{hyperholestérolémie}

\subsubsection{Étiologies}
\label{sec:orga9b2083}
Secondaire
\begin{itemize}
\item Bilan selon contexte : TSH, glycémie, créat, protéinurie, BU
\item Comorbidité : 
\begin{itemize}
\item hypocholestérolémie : hypothyroïdie, (cholestase, anorexie mentale)
\item mixtes : sd néphrotique, grossesse
\item hypertriglycéridémie : insuf rénale chronique, alcoolisme, (obésité, diabète
avec sd métabolique)
\end{itemize}
\item Iatrogène : ciclosporine, corticoïdes, oestrogènes oraux, rétinoïdes,
IFN-\(\alpha\), certains antétroviraux, neuroleptiques, diurétiques thiazidiques, betabloquant
\end{itemize}
Primaire
\begin{itemize}
\item Hypercholestérolémies familiales monogéniques
\begin{itemize}
\item mutation du gène du récepteur LDL++ : hétérozygote (xanthomes tendineux,
complications CV précoces) ou homozygote (rare, DC vers 20 ans)
\item mutation du gène de l'apoliporotéine B
\item mutation du gène PCSK9
\end{itemize}
\item Hypercholestérolémies polygéniques : fréquent, complications CV tardives
\item Hyperlipidémies combinées familiales : fréquent++, pas de xanthèmes,
complicastions CV suivant intensité
\item Dysbetalipoprotéinémie : xanthomes pathognomoniques
\item Hypertriglycéridémie familiale : rare, pas de xanthomes
\item Hyperchylomicronémie primitive : souvent hypertriglycéridémies majeurs
\end{itemize}

\subsection{HTA et risque CV}
\label{sec:org2ad68bf}
PAs \(\ge\) 140, PAd \(\ge\) 90mmHg (3 consultations à 3 mois)\footnote{-5mmHg si automesure. -10mmHg si MAPA sur 14h}   

40\% adulte. Proba \inc si sd métabolique. Aggravé par \acrshort{HVG}, glomérulopathie

Clinique, bio, ECG

\subsection{Diabète et risque CV}
\label{sec:orgd0689da}
Complications coronariennes ischémique : RR \female{} > \male.

AOMI : RR \texttimes{} 5, AVC RR \texttimes{} 2.

Diabète 2 : maladie coronarienne peut précéder diabète ! \thus dépistage

\subsection{PEC}
\label{sec:orgc88cc22}
Dépistage familial si pathologie métaboblique \inc risque vasc
\subsubsection{Sevrage tabac}
\label{sec:orgd00b878}
Évaluer dépendance nicotine
Poids + 5kg en moyenne

Substituts nicotinique (6sem-6mois) : n'aggrave pas maladie
coronarienne/troubles rythmes

Varéniclide, bupropion = dernière ligne (8 semaines)
\subsubsection{Activité physique}
\label{sec:org48b60c8}
\begin{itemize}
\item \dec insulino-résistance, \dec triglycéridémie, \inc HDL
\item \dec PA repos, \inc périmètre marche AOMI, \inc pronostic complications coronariennes ischémiques
\end{itemize}

3x45min à 75\% \(O_2\)

\subsubsection{Diététique}
\label{sec:org46ed7e5}
\begin{itemize}
\item Lipides : graisses < 35\%, \dec gras saturés, \inc mono-insaturés, \{poisson gras, noix, \(\Omega_{\text{3}}\)\}, \dec cholestérol
\item Autres : \inc fruits, légumes, \{noix, noisettes, amandes\}, sel < 6g/j, alcool
< 3 verres vin, \dec sucres simples, -20\% calories
\end{itemize}

Hypertriglycéridémie : 
\begin{itemize}
\item modérées : -20\% calories ++, \inc activité physique
\item majeur : arrêt alcool, régime hypo- (si obèse) ou iso-calorique avec < 30g
lipides (si obèse) ou 20g
\end{itemize}

\subsubsection{Hypolipémiants :}
\label{sec:org3a464aa}
Statines :
\begin{itemize}
\item \dec LDL, \dec TG \inc HDL
\item ES : myalgies, \inc CPK, \inc transaminases, \inc risque diabète 2
\item CI : HS, grossesse, allaitement
\end{itemize}
\subsubsection{Traitement hypercholestérolémies}
\label{sec:org5aa2e59}
En primaire si LDL reste élèvé à +6 mois traitement. En secondaire si complication ischémique

Objectifs :
\begin{itemize}
\item primaire : 
\begin{itemize}
\item LDL < 1.3g/L si risque CV faible (pop générale, diabète ou hypercholestérolémie familale)
\item LDL < 1g/L sinon
\end{itemize}
\item secondaire : systématique
\end{itemize}

Molécules
\begin{itemize}
\item hypercholestérolémies : statines
\item hypertriglycéridémies : diététique si TG > 2g/L, statines si TG < 4g/L et HDL
bas, fibrate sinon
\end{itemize}

Augmenter doses progressivtement puis suivi : 2-3 mois tant que objectifs non
atteints puis 1-2/an

\subsection{Antihypertenseurs}
\label{sec:org3508162}
Diurétiques, betabloquant, inhibiteurs calcique, IEC, ARA II\footnote{Antagonistes des récepteurs de l'angiotensine II}

Monoprise, monothérapie en 1ere intention

Objectif : [130, 139] et < 90mmHg. Visites mensuelles jusque l'objectif

Suivi : pas d'hypotension orthostatique, \{iono, créat, DFG\} +15j après
chaquement changement, suspension diurétiques et ARA si déshydratation

\subsection{Antiagrégants plaquettaire}
\label{sec:org0566af7}
Prévention secondaire : systématique. Clipodigrel-aspirine systématique à +1 mois
après stent, +1 ans après stent actif

\section{221 : HTA, causes endocriniennes}
\label{sec:org366db22}
Déf: \(\ge\) 140/90 mmHg.

Dépistage d'une HTA secondaire : doit être systématique mais économe\ldots{}

Enquête :
\begin{itemize}
\item initiale : ATCD familiaux HTA, souffle para-ombilical, rein/masse abdo à la
palpation, signe d'hypercortisolisme/acromégalie, bio \thus
protéinurie/hématurie, imagerie, hormonale (selon signes)
\item si résistance malgré 3 antihypertenseurs (dont 1 diurétique), chercher toutes
les cause d'HTA
\end{itemize}

HTA curables : 3-5\%. Cf les catégories ci-dessous !

\subsection{Hyperminéralocorticisme primaire (HAP)}
\label{sec:org6dbe58d}
Physiopatho : aldostérone, cortisol, désoxycorticostérone \thus rétention sodée
\thus HTA et inhibe sécrétion de rénine.

Penser à HAP si hypokaliémie (< 3.5mmol/L) ou HTA résistante

\begin{tcolorbox}
Si la surrénale produit plus d'aldostérone : régulation négative par la rénine (en théorie)
\end{tcolorbox}

\begin{algorithm}
  \caption{Explorations des HAP}
  Arrêt diurétiques\;
  Vérifier natriurèse+, kaliurèse > 20mmol/j\;
  \If{aldostérine/rénine \times 2}{
  Si aldo \inc et rénine \dec : HAP\;
  Si aldo \inc et rénine \inc : hyperaldo. secondaire\;
  Sinon aldo \dec et rénine \dec : autre minéralocorticisme\;
  }
\end{algorithm}

Tests dynamiques possibles : stimulation (recherche un défaut de), freination
(recherche une freination)

\subsubsection{Adénome de Conn}
\label{sec:orge29f34f}
Forme généralement curable
\begin{itemize}
\item imagerie : nodule unilatéral > 10mm au scanner.
\end{itemize}
\danger il faut prouver une sécrétion unilatérale d'aldostérone \thus
cathétérisme si scanner douteux/patient jeune/HTA résistante
\begin{itemize}
\item chir possible (mais tumeur bénigne, risque récidive)
\item si autre HAP : médicaments en continus
\end{itemize}

\subsubsection{Hyperminéralocorticismes familiaux}
\label{sec:org69b0526}
Lié à l'aldostérone, désoxycorticostérone, cortisol

\subsection{HTA endocrines iatrogènes}
\label{sec:orga1892d6}
Contraception oestroprogestative, corticostéroides, réglisse

\subsection{Phéochromocytomes, paragangliomes fonctionnels}
\label{sec:org5b891dc}
\gls{PCC} : médullosurrénale. \gls{PGG} fonctionnels : autres    ganglions sympathiques

PCC : spontanément mortel. Dépistage :
\begin{itemize}
\item HTA avec céphalées, sueurs, palpitations, HTA paroxystiques/diabète sans
surpoids
\item sd familial : \gls{NF1}, \gls{VHL}, \gls{NEM2}, sd phéochromocytomes-paragangliomes familiaux
\end{itemize}

Diagnostic : métanéphrines \inc.

Puis imagerie : 
\begin{itemize}
\item PCC : uniques, \textasciitilde{}5cm.
\item PGG siègent dans l'organe de Zuckerkandl, vessie, hiles rénaux, médiastin postérieur, péricarde, cou.
\end{itemize}

Puis médecine nucléaire

Toujours dépistage :
\begin{itemize}
\item clinique : taches "café-au-lait", neurofibromatomes, nodules de Lish (NF1),
hémangioblastomes (VHL)
\item génétique : NEM2, VHL
\end{itemize}

Toujours traitement chir mais surveillance long terme

\subsection{Sd de Cushing}
\label{sec:org8674552}
Correspond hypersécrétion de cortisol

Signes : acné, ecchymoses, faiblesse musc, hirsutisme, oedèmes, ostéoporose, PAd
> 105mmHg, vergetures pourpres

Étiologies :
\begin{itemize}
\item maladie de Cushing (adénome corticotrope) à 66\%
\item tumeur non hypophysaire (15\%) : adénome sécrétant surrénalien ou
corticosurrénalome
\end{itemize}

\subsubsection{Démarche}
\label{sec:org98f0c9e}
\begin{tcolorbox}
CRH (hypothalamus) stimule ACTH (hypophyse) qui stimule la production de
glucocorticoïdes (surrénale)
\end{tcolorbox}

\begin{itemize}
\item Diagnostic positif : cortisol plasmatique \footnote{Physiologique = minimal à minuit, donc mesure à minuit. Mesure salivaire possible.}, cortisolurie (sur 24h), test de freinage rapide\footnote{1mg de dexaméthasone à minuit. Action de rétrocontrôle négative du
cortisol donc on vérifie le cortisol plasmatique le lendemain à 8h.}

\item Diagnostic étiologique selon ACTH :
\begin{itemize}
\item ATCH diminuée \thus adénome, corticosurrénalome, hyperplasie bilatérale
\item ATCH normale ou \inc \thus test CRH. si positif : tumeur ectopique ou
maladie de Cushing
\end{itemize}
\end{itemize}

\subsection{Causes rare}
\label{sec:orgf11df1c}
Tumeurs à rénine, acromégalie
\section{238 : Hypoglycémie}
\label{sec:org67be7d1}
Diagnostic : neuroglucopénie et glycémie < 0.50g/L (0.60 chez diabétique) et correction symptômes
à normalisation (triade de Whipple)

Causes :
\begin{itemize}
\item sécrétion inappropriée d'insuline (hypoglycémiante)
\item (rare) : défaut de sécrétion d'hormones hyperglycémiantes (GH, glucagon,
catécholamine, cortisol), déficit néoglucogénèse, défaut substrat
\end{itemize}

\subsection{Symptômes}
\label{sec:org527ad85}
Neuroglucopénie : faim brutale, troubles concentration, troubles moteurs,
troubles sensitifs, troubles visuels, convulsions focales/généralisése,
confusion

Coma hypoglycémique : début brutal, agité (sueurs), irritation pyramidale, hypothermie

\begin{itemize}
\item souvent signes adrénergiques : anxiété, tremblements, nausées, sueurs,
pâleur, tachycardie
\end{itemize}

\subsection{Causes}
\label{sec:org0012513}
Diabétique : traité par insulines, hypoglycémiants oraux

\subsubsection{Insulinome}
\label{sec:orgc7d5067}
1ere cause tumorale (mais rare). Maligne dans 10\%, < 2cm (90\%)

Clinique : manif. adrénergiques surtout

Diagnostic : épreuve de jeûne, cf table \ref{tab:org03790d7}

\begin{table}[htbp]
\caption{\label{tab:org03790d7}
Diagnostic d'hypoglycémie (jeûne) avec DD}
\centering
\begin{tabular}{llll}
\toprule
 & Diagnostic positif & Insuline cachée & Sulfonylurée cachée\\
\midrule
Glycémie & basse & basse & basse\\
Signes & neuroglucopénie &  & \\
Insulinémie & normale mais inadaptée & dosable & dosable\\
Peptide C & bas & dosable & indosable\\
Sulfamides & 0 &  & \\
pro-insuline & élevée &  & indosable\\
\bottomrule
\end{tabular}
\end{table}

Scanner en coupe fine du pancréas et écho-endoscopie si médecin habitué

Traitement : chir

\begin{tcolorbox}
Hypoglycémie par sécrétion inaproppriée d'insuline : triade de Whipple, glycémie \le 0.45g/L\footnotemark avec
insulinémie \ge 3 mUL/L, peptide C \ge 0.6ng/mL
\end{tcolorbox}
\footnotetext{Spontanément/jeûne}

\section{239 : Goitre, nodules thyroïdiens, cancers thyroïdiens}
\label{sec:org91d435b}
Besoins en iode quotidiens (synthèse hormones thyroïdiennes) :  \(\approx\) 150 \(\mu\)g/jour (ado,
  adulte, \texttimes{} 2 chez enceinte)

Goitre = hypertrophie de la thyroïde :
\begin{itemize}
\item palpation > dernière phalange du pouce
\item écho : volume > 20 \(cm^3\) (18 femme adulte, 16 ado)
\end{itemize}

\subsection{Évaluation}
\label{sec:org23701f9}
Clinique : mobile déglutition/visible cou en extension/visible à
distance. Chercher : gene fonctionelle, signes de compression, signes de
dysfonction thyroïdienne, \acrshort{ADP}

Bio : 
\begin{itemize}
\item TSH++ : \inc, déficit production, si \dec, imprégnation excessive en hormones thyroïdiennes.
\item compléter par T4, et si TSH \inc : Ac anti-TPO, anti-Tg
\end{itemize}

Échographie

\subsection{Goitre simple}
\label{sec:orgf4ad1a7}
Hypertrophies normo-fonctionnelles non inflammatoires non cancéreuses

Facteurs : \female, tabac, déficience iodée

\subsubsection{Évolution}
\label{sec:org6e13add}
Constitution à l'adolescence (cliniquement latente) puis plurirondulaire : gêne
cervicale \thus TSH, écho, ponction, scintigraphie
\danger cherche caractère plongeant sur radio !

À ce stade, complications : hématocèle, strumite, hyperthyroïdie, compression
organes de voisinages, cancerisation (5\%)

\subsubsection{PEC}
\label{sec:org2393fd0}
\begin{itemize}
\item Ado : levothyroxine (1 à 1.5 \$\(\mu\)\$g/kg/j) jusque V normal. Vérifier TSH
\item Adulte/agé : si multinodulaire non malin, surveillance. Si symptomatique,
thyroïdectomie totale
\item Goitre ancien, négligé : iode 131
\end{itemize}
\inc iode, notamment grossesse

\subsubsection{Autres pathologies responsables}
\label{sec:org8363432}
\begin{itemize}
\item Maladie de Basedow
\item Thyroïdites : 
\begin{itemize}
\item Hashimoto = hypertrophique. Goitre très ferme, expose à l'hypothyroïdie. Ac Ant-TPO\inc\inc{}, écho : goitre diffus, hypoéchogène
\item autres thyroïdites
\end{itemize}
\item Troubles de l'hormonosynthèse
\end{itemize}

\subsection{Nodules thyroïdes}
\label{sec:org1382bef}
Déf : toute hypertrophie localisée de la gande thyroïde. Majorité = bénin (5\%
cancers, de très bon pronostic)

Prévalence \(\approx\) décennie du sujet. \texttimes{} 2 chez \female. \inc si grossesse,
déficience iode, irradiation cervicale

\subsubsection{Évaluation :}
\label{sec:org00a8d0e}
Si signe d'accompagnement :
\begin{itemize}
\item nodule douloureux brutal : hématocèle
\item nodule douloureux + fièvre : thyroïdite subaigüe
\item nodule compressif + ADP : cancer
\item nodule + hyperthyroïdie : nodule toxique
\item nodule + hypothyroïdie : thyroïdite lymphocytaire
\end{itemize}
Si isolé : 
\begin{itemize}
\item TSH \dec : nodule hyperfonctionnel ? \thus scintigraphie
\item TSH N : tumeur \thus écho, cytologie
\item TSH \inc : thyroïdite lymphocytaire ? \thus Ac anti-TPO
\end{itemize}

Pronostic plutôt suspect : 
\begin{itemize}
\item homme, enfant/âgé, ATCD irradiation cervicale, > 3cm, ovalaire, dur, irrégulier, > 20\% en un an
\item écho : hypoéchogène, contour irrégulier, microcalcifications, ADP
\end{itemize}

Bio : TSH surtout. 
\begin{itemize}
\item si nodule, calcitonine > 100pg/mL = argument solide pour cancer médullaire thyroïde.
\item calcitonine \(\in\) [20,50]pg/mL : idem ou hyperplasise des cellules C ou insuffisant rénal
\end{itemize}

Examens : 
\begin{itemize}
\item Échographie (classification TI-RAD de 1 à 6)
\item Cytologie si nodule suspect (classification Bethesda de 1 à 6)
\item Scinti si cytologie ininterprétable 2 fois ou indéterminée
\end{itemize}

\subsubsection{Thérapeutique}
\label{sec:orge3df9ce}
\begin{itemize}
\item Chir si suspect clinique/écho/cyto/calcitonine \inc\inc{} : thyroïdectomie si dystrophie controlatérale
\item Surveillance sinon
\item Hormonal si bénin dans familles avec goitres plurinodulaire, < 50 ans.
\end{itemize}

Kystes, hématocèles : anéchogène \thus ponction \textpm{} hormonothérapie , alcoolisation.

Grossesse : chir possible 2e trimestre ou après accouchement 

Nodule oculte : < 1cm. Risque de cancer 5\%, faible pouvoir agressif
\begin{itemize}
\item \danger si ADP, hérédité cancer médullaire thyroïde, fixation au TEP
\item ponction seulement si hypoéchogène et > 8mm
\end{itemize}
\subsection{Cancers thyroïdiens}
\label{sec:orga5007f2}
1.5\% cancers, 4eme chez la femme

Découverte : fortuite++, ADP cervicale, signes de compression, flushes/diarrhée,
localisation métastatique

Anatomie :
\begin{itemize}
\item carcinomes différenciés d'origine vésiculaire : papillaire (85\%, excellent
pronostic), vésiculaires (5\%), peu différenciés (2\%)
\item carcinomes anaplasiques (1\%)
\item carcinomes médullaires au dépens des cellules C
\item autres
\end{itemize}

Risque de rechute/décès : 
\begin{itemize}
\item taille tumeur, effraction capsule thyroïdienne, métastase (clasif TNM de I à IV)
\item mortalité \(\propto\) âge, dépend de l'histologie, exérèse
\end{itemize}

\subsubsection{Thérapeutique}
\label{sec:orgc3cee28}
\begin{itemize}
\item Plan cancer
\item Chir en 1ere intention (anatomopatho pendant = certitude) : thyroïdectomie
totale. Curage ganglionnaire si besoin (systémique si carcinome médullaire,
si enfant/ado). \\
Complications : hémorragie postopératoire \skull,  hypoparathyroïdie (calcium + vit D), paralysie transitoire/définitive nerfs récurrents
\end{itemize}
\vspace*{10pt}

\emph{Cancers différenciés d'origine vésiculaire}
\begin{itemize}
\item iode 131 : seulement post-thyroïdectomie totale (haut risque). Nécéssite
stimulation par L-T4 ou injection TSH. Puis hospit après en chambre 2-5 j
et contraception 6-12 mois. \\
ES : \{nausées, oedèmes\}, \{agueusie, sialadénite\}. \\
Scinti  obligatoire à +2-8j : fixation extracervicale à  distance = métastases
\item hormonal : L-T4 si haut risque ou échec traitement initial. Puis mesurer TSH à
+6sem-2mois (pas avant !)
\item surveillance : 80\% des récidives à 5 ans \thus écho cervicale, rhTSH,
Tg\footnote{Thyroglobuline} à 6-12mois : cytoponction puis imagerie si Tg > seuil. Sinon \dec LT4
\item traitement récidives : chir si cervicale. Plus compliqué si métastates
(iode131 si fixant sinon ttt local ou molécules ciblées). Maintenir LT4
\end{itemize}

\emph{Cancers anaplasiques}\\
Tuméfaction cervicale rapidement progressive, dure, adhérente, sujet âgé \thus radio-chimio. Pronostic très péjoratif

\emph{Cancers médullaires}
\begin{itemize}
\item TTT : chir \textpm{} curage ganglionnaire
\item Surveillance : calcitonine > 150\(\mu\)g/L \thus bilan de localisation.
\item Temps doublement : 6 mois = pronostic très mauvais.
\item Traitement métastases = local.
\end{itemize}

Étude génétique dans tous les cas : positif \thus chercher phéochromocytome,
hyperparathyroïdie + enquêtes apparentés
\section{240 : Hyperthyroïdie}
\label{sec:orgcf5659f}
\begin{tcolorbox}
Examen en 1ere intention : TSHus (puis T4L !)
\end{tcolorbox}

Déf : hyperfonctionnement de la glande thyroïdienne. Sd de thyrotoxicose =
conséquence sur les tissus.

Prévalence élevée, 7\texttimes{} femme

Physiopatho :
\begin{itemize}
\item TSH, \gls{TPO} et Tg peuvent être des auto-antigènes
\item thyroïde produit surtout thyroxine (T4\footnote{[T4] n'est à l'équilibre que +5 semaines après modification de T4}), convertie en T3 par foie, muscle
squelette.
\item effet : 
\begin{itemize}
\item \inc production chaleur, \inc production énergie, \inc consommation \(O_2\)
\item \inc débit cardiaque, système nerveux, \inc ostéclasie, \inc lipolyse, \inc
glycémie, rétrocontrole négatif hypophysaire
\end{itemize}
\end{itemize}

\subsection{Sd de thyrotoxicose}
\label{sec:org957a38e}
Clinique (par fréquence \dec) :
\begin{itemize}
\item CV : tachycardie (régulière, repos, \inc effort), \inc intensité bruits
coeurs, \inc PAs
\item neuropsy : nervosité, tremblement fin régulier des extrémités, fatigue
générale, troubles sommeil
\item thermophobie, hypersudation,
\item amaigrissement rapide, important, avec appétit conservé
\item autre : polydipsie, amyotrophie, \inc frequence selles, rétraction paupière
supérieure (gynécomastie, troubles règle)
\end{itemize}

Examen complémentaire : TSH effondrée. T4 ou T3 libre pour l'importance

Complications : 
\begin{itemize}
\item cardiaque (surtout personnes fragiles) : troubles rythme supraV (FA), insuf
cardiaque (droite, avec débit N ou \inc), aggravation insuf coronaire
\item crise aigüe thyrotoxique (exceptionnelle)
\item musculaire (âgé)
\item ostéoporose (\female ménopausée) : rachis
\end{itemize}

\subsection{Étiologies (fréquence \dec)}
\label{sec:org7653659}
\subsubsection{Auto-immunes}
\label{sec:org25523d7}
\emph{Maladie de Basedow}\\
1\% population. Auto-immune, sur terrai génétique. Poussées puis rémissions

Clinique : 
\begin{itemize}
\item goitre diffus homogène, élastique, souffle
\item oculaire (spécifique, inconstant) : rétraction et asynérgie palpébrale,
inflammation, exophtalmie, oedème paupières, inflammation conjonctive,
limitation mouvement regard
\thus examen ophtalmo ! (acuité visuel, cornée, papille, oculomotricité, tonus
intraoculaire)\\
Mauvais pronostic : exophtalmie importante, paralysie complète, neuropathie
optique, hypertonie oculaire avec souffrance papillaire
\item dermopathie (exceptionnelle) placard rouge, surélevé, induré, face ant jambes
\end{itemize}

Diagnostic : manif oculaire suffit. sinon : écho (hypoéchogène, vascularisé),
(scinti), Ac anti-récepteur TSH

\emph{Autres auto-immune}\\
\begin{itemize}
\item Thyroïdite post-partum (5\%) : hyperthyroïdie transitoire puis hypothyroïdie. Ac
anti-TPO mais pas Ac anti-récepteur TSH
\item Thyroïdite d'Hashimoto : goitre irrégulier, très ferme. Écho :
hypoéchogène. Ac anti-TPO mais pas anti-récepteur TSH
\end{itemize}

\subsubsection{Nodules thyroïdiens hypersécrétans}
\label{sec:org523121b}
Âge plus avancé, sd de thyrotoxicose pur (pas de manif oculaire) 
\begin{itemize}
\item Goitre multinodulaire toxique : à la clinique, puis écho. Scinti : "en damier"
\item Adénome toxique : palpation nodule unique, écho : tissulaire/partiellement
kystique. Scinti nécessaire : reste du parenchyme "froid"
\end{itemize}

\subsubsection{Iatrogènes}
\label{sec:org0fe40a2}
\begin{itemize}
\item Iode : produits contraste, amiodarone. 2 formes : fonctionnelle ou lésionnelle
(lyse des cellules)
\end{itemize}
\danger sous amiodarone : T4L \inc mais T3L, TSH N 
\begin{itemize}
\item Hormones thyroïdiennes : pour maigrir. Diag : scinti (pas de fixation), Tg
effondrée
\item Interféron (fréq++)
\end{itemize}

\subsubsection{Thyroïdite subaigüe de De Quervain}
\label{sec:org29b804a}
Affection banale virale. Diagnostic clinique (goitre dur et douleureux). Hyper-
puis hypo-thyroïdie. Echo = hypoéchogène

\subsubsection{Thyrotoxicose gestionnelle transitoire}
\label{sec:org15147f4}
Fréquent (2\% grossesse). 1er trimestre : nervosité, tachycardie, pas de prise de
poids

DD : Basedow (pas Ac anti-récepteur TSH)

\subsubsection{Rares}
\label{sec:org239154f}
Mutations activatrices du récepteur TSH, métastase massives sécrétantes (K
thyroïdiens vésiculaire différencié), tumeurs placentaires/testiculaires, \{sd
résistance hormones thyroïdiennes, adénome hypophysaire\}

\subsection{Forme clinique}
\label{sec:org8bdc929}
\begin{itemize}
\item Enfant : généralement Basedow (néonatale/acquise) : avance staturale et
osseuses, hyperactivité \textpm{} signes oculaires
\item Femme enceinte : passage d'Ac \thus hyper- ou hypo-thyroïde. Passage
d'antithyroïdiens de synthèse \thus goitre, hypothyroïdie possible. Contraception !
\item Âgé : évolution discrète (AEG, fonte musculaire, cachexie, insuf
cardiaque). Penser thyrotoxicose si troubles rythme/insuf cardiaque
\end{itemize}

\subsection{Traitement}
\label{sec:org07c73c0}
\skull{} Urgence : crise aigüe thyrotoxicose, cardiothyréose chez âgé/cardiqaue,
orbitopathie maligne, cachexie vieillard, Basedow chez \female{} enceinte

Repos, sédatifs, bêtabloquant, contraception

\gls{ATS} :
\begin{itemize}
\item -mazole (30-60mg/j), -thiouracile (300-600mg/j) : bloque TPO
\item ES : allergies cut, \inc enzymes hépatiques, neutropénie, agranulocytose++
(\skull !!)
\item surveillance : T4 libre jusque N puis T4L et TSH. NFS 10jours pendant 2 mois (agranulocytose)
\end{itemize}

Chir : thyroidectomie totale sauf si adénome toxique (lobectomie)

Radio-iode : simple, sans risque génétique/cancérisation secondaire (\danger{} orbitopathie\ldots{}). CI : femme enceinte.

\subsubsection{Résultats}
\label{sec:org9e5efda}
\begin{itemize}
\item Basedow : thyroïdectomie \thus hypothyroïdie définitive. Radio-iode \thus
hypothyroïdie 50\%, risque aggravation orbitopathie. Donc ttt médical (1-2
ans) puis chir/iode si récidive
\item Adénome/goitre multinodulaire toxique : chir, iode
\item Induite par l'iode : arrêt si possible
\item Thyroïdite subaigüe : anti-inflammatoire (AINS/corticoïde)
\end{itemize}

\subsubsection{Formes particulières}
\label{sec:org2c10ecf}
\begin{itemize}
\item Cardiothyréose : propanolol et anticoag. Si insuf cardiaque : tonicardiaque,
diurétiques, vasodilatateurs, betabloquant, anticoag. Pour thyrotoxicose : ATS
puis chir/iode 131
\item Crise aigüe thyrotoxique : soins intensifs, réa, ATS, propanolol, corticoïdes,
iode131 après 24h ATS
\item Orbitopathie : pas d'effet ATS, iode peut aggraver !! Si simple, collyre. Si maligne : cf spécialiste
\item Femme enceinte : si transitoire, repos. Si Basedow : repos si mineur. Si forme
importante : ATS faible dose. Si formes grave, chir (2eme trimestre) possible)
\end{itemize}
\thus surveillance avant et après accouchement  

\section{241 : Hypothyroïdie}
\label{sec:org3612558}
\begin{tcolorbox}
Rappel : TRH (hypothalamus) stimule la production de TSH (hypophyse) qui stimule la thyroïde
\end{tcolorbox}

\begin{itemize}
\item Atteinte de la glande thyroïde  : \inc TSH et 
\begin{itemize}
\item soit T4L N : hypothyroïdie frustre
\item soit T4L \dec : hypothyroïdie patente
\end{itemize}
\item Ou hypothalamo-hypophysaire : T4L \dec et 
\begin{itemize}
\item soit TSH \dec ou N : hypophysaire
\item soit TSH légèrement /inc : hypothalamus
\end{itemize}
\end{itemize}

\subsection{Sémiologie}
\label{sec:orga272108}
Général :
\begin{itemize}
\item sd d'hypométabolisme\footnote{Asthénie, somnolence, hypothermie, frilosité, constipation, bradycardie,
prise poids modeste}
\item peau pâle/jaune, sèche, squameuse, dépilée; cheveux secs cassants
\item myxoedeme cutanéomuqueux : faciès "lunaire", voix rauque, hypoacousie,
macroglossie
\item neuromusc : crampes, myalgies
\item endocrinien : (galactorrhée), troubles règles, troubles libido
\end{itemize}
Cliniques (rare, diag fait avant) :
\begin{itemize}
\item CV : bradycardie sinusale, \dec contractilité, (insuf cardiaques, troubles
rythme V), épanchement péricardique, favorise athérome coronarien
\item neuromusc, neuropsy : dépressif, sd confusionnel, démence, myopathie prox,
apnée sommeil
\item coma myxoedemateux : si hypothyroïdie primaire profonde et
aggression. Convulsion, EEG non spécifique. Hyponatrémie. Pronostic sévère
\end{itemize}

Palpation : glande ferme hétérogène, pseudonodulaire

Grossesse : 
\begin{itemize}
\item complication mère : HTA, prééclampsie, fausse couche, hémorragie post-partum
\item complications foetus : troubles developpement neuro-intellectuel, hypotrophie
\item 1er trimestre : TSH \dec, T4L limite sup. Puis TSH normale, T4L basses
(physiologique !)
\end{itemize}

Anomalies bio :
\begin{itemize}
\item hémato : anémie normocytaire normochrome (si macrocytose, penser anémie de
Biermer) troubles de coagulation,hémostase
\item hypercholestérolémie, \inc CPK, hyponatrémie dilution
\end{itemize}

\subsection{Étiologies}
\label{sec:org34c65fc}
\subsubsection{Hypothyroïdie primaire}
\label{sec:org97a7e33}
Auto-immunes :
\begin{itemize}
\item Thyroïdite d'Hashimoto : 
\begin{itemize}
\item goitre ferme, irégulier, Ac anti-TPO.
\item infiltration lymphocytaire du parenchyme thyroïdien. Facteurs environnementaux,
terrain génétique.
\item penser à lymphome si \inc rapide du goitre
\item écho thyroïdiennes : hypoéchogène, hétérogène, vasc hétérogène (scint
inutile)
\end{itemize}
\item Thyroïdite atrophique : pas de goitre, Ac anti-thyroidiens moins
élevés. Souvent une évolution d'Hashimoto, > 50 ans.
\item Thyroïdite du post-partum : idem, petit goitre. Normalement résolutif dans
l'année. 5\% des grossesses
\end{itemize}
Non auto-immune :
\begin{itemize}
\item thyroïdite subaigüe de De Quervain : inflammation du parenchyme. Phase de
thyrotoxicose puis hypothyroïdie
\item thyroïdite sans Ac
\item thyroïdite iatrogène : interferon++, amiodarone, ATS, iode131, radiothérapie
cervicale, lithium, ttt anti-tyrosine kinase (cancéro)
\end{itemize}
Autres : carences iodées (endémie++), hypothyroïdie congénitale (dépistage à
naissance + 72h\footnote{Clinique discrète : ictère prolongé, constipation, hypotonie, pleurs
rauques, difficulté succin, fontanelles larges, hypothermie})

\subsubsection{Démarche diagnostique}
\label{sec:org3d87a54}
TSH puis (T4L (profondeur) et Ac anti-TPO, échographie pour étiologie)
\subsubsection{Insuffisance thyréotrope}
\label{sec:org83434d0}
\begin{itemize}
\item compression région hypothalamo-hypophysaire (HH) par tumeur (adénome hypophysaire
souvent)
\item séquelle post-chir, post-radio des tumeurs de la région HH
\item séquelles méningite, trauma crânien, hémorragie méningée
\item génétiques (rare)
\end{itemize}

IRM systématique !

\subsection{Traitement}
\label{sec:org1c731ba}
Lévothyroxine (T4) 
\begin{itemize}
\item hypothyroïdie patente : L-T4 50 à 150 \(\mu\)g/j. Si coronarien : \inc progressivement
de 12.5 à 25\(\mu\)g/j. \danger Surveillance ! (ECG hebdo si grave, hospit si coronarien
récent, sinon patient doit consulter si douleurs thoraciques)
\item hypothyroïdie frustre : 3 cas
\begin{itemize}
\item TSH > 10mUI/L ou Ac anti-TPO : ttt
\item TSH < 10mUI/L et pas d'Ac anti-TPO : surveillance
\item si grossesse : dès TSH \(\ge\) 3mUI/L
\item à discuter sinon
\end{itemize}
\end{itemize}

Suivi
\begin{itemize}
\item hypothyroïdie primaire : objectif : TSH \(\in [0.5, 2.5]\) mUI/L (\(\approx\) 10mUI/L pour âgé, et < 2.5mUI/L pour femme eceinte)
\item insuf thyréotrope : suivi sur T4L seulement
\end{itemize}

Situations particulières:
\begin{itemize}
\item grossesse : \inc posologie dès diagnostic grossesse
\item \inc si interférence avec l'absorption intestinale\{sulfate de fer, carbonate de calcium, hydroxyde
d'alimunie, cholestyramine\}, la clairance \{phénobabrital, carbamazépinex, rifampicine,
phénytoïne, sertraline, chlooriqune\}, oestrogenes
\item néonatale : L-T4 à vie
\end{itemize}

\subsection{Dépistage ?}
\label{sec:orgc91734e}
\begin{itemize}
\item Adulte : si risque : signes clinique, goitre, hypercholestérolémie, ATCD
thyroïdiens, auto-immunité thyroïdienne, irradiation cervicale, \{amiodarone,
lithium, interféron, cytokines\}
\item Femme enceinte : si signes, contexte thyroïdien (perso/familial), auto-immunité
\end{itemize}

\section{242 : Adénome hypophysaire}
\label{sec:org2f9cadb}
\subsection{Sd tumoral}
\label{sec:org117b991}
Clinique : 
\begin{itemize}
\item céphalées : rétro-orbitaire, localisése
\item trouble visuels : "voile", par compression des voies optiques. Fond d'oeil,
acuité visuelle OK. Quadra-/hémi-anopsie temporale (sup si quadra)
\item apoplexie hypophysaire (rare) brutal : céphalées violentes, sd méningé sd
 confusionnel, troubles visuel.\\
Imagerie en urgence \danger : adénome en nécrose/hémorragie \skull
\end{itemize}

IRM = examen de référence :
\begin{itemize}
\item microadénome : arrondi, homogène, hypo-T1, hypointense  après injection,
signes indirects
\item macroadénome :  > 10mm, iso-T1 et hyperintension après injection. Regarder
expansion vers chiasma optique et sinus (sphénoïdal, cavernux)
\item DD : craniopharyngiome intra-sellaire (hétérogène, hyperT2), méningiome
intra-sellaire (dure-mère spiculée)
\end{itemize}

\subsection{Sd d'hypersécrétion}
\label{sec:org6b7dd71}
\subsubsection{Hyperprolactinémie\footnote{Prolactine est sécrétée par l'hypophyse}}
\label{sec:org54f32ec}
Fréquente \thus cherche adénome hypophysaire (même si majorité = médicament)
Signes :
\begin{itemize}
\item \female : galactorrhée (pas forcément causée par prolactine ! mais chercher
quand même), troubles du cycle menstruel (doser !)
\item \male : galactorrhée, gynécomastie (rare), troubles sexuels. Hypogonadisme
\thus doser prolactine
\item 2 sexes : risque d'ostéoporose
\end{itemize}

\begin{tcolorbox}
Diagnostique d'hyperprolactinémie
\begin{enumerate}
\item  Vérifier hyperprolactinémie (kits, agrégats)
\item  Écarter grossesse, médicaments, hypothyroïdie périphérique, insuf rénale
\item  IRM : 
\begin{itemize}
  \item microadénome ?
  \item sinon différentier macroadénome à prolactine VS autre tumeur associée à 
hyperprolactinémie : régression sous agoniste dopaminergique si macrodénome
\end{itemize}
\end{enumerate}
\end{tcolorbox}

\subsubsection{Acromégalie (excès de \gls{GH})}
\label{sec:org35b98f5}
Clinique :
\begin{itemize}
\item Sd dysmorphique : extrémités élargies, visage (nez élargi, front bombé, lèvres épaisses, tendance prognathisme\}
\item Signes fonctionnels : sueurs, céphalées, paresthésies des mains, douleurs articulaires, asthénie fréquente, HTA (50\%)
\end{itemize}

Complications :
\begin{itemize}
\item CV : hypertrophie myocardite (écho), débit \inc. Puis insuf cardiaque
congestive
\item arthropathie périphérique : grosses articulations, radio (interligne \inc,
ostéophite), rachis (spondylose d'Erdheim)
\item diabète/intolérance glucose, SAS, goitre, polypes du colons
\end{itemize}

\begin{tcolorbox}
Diagnostic d'acromégalie : absence de freinage de la GH pendant hyperglycémie provoquée
  orale (GH > 0.4ng/mL), \inc IGF-1\footnotemark
\end{tcolorbox}
\footnotetext{La GH stimule la production d'IGF-1.}

Puis bilan tumoral, retentissement fonctionnel hypophysaire, retentissement
acromégalie

\subsubsection{Sd de Cushing (excès de glucocorticoïdes)}
\label{sec:orgd0cdee3}
Excès de glucocorticoïdes : causes iatrogènes ou adénomes hypophysaires
corticotropes

Clinique :
\begin{itemize}
\item anomalies acquises. Amyotrophie ceinture et abdomen, lenteur cicatrisation,
peu amincie (mains), ecchymoses au moindre choc, vergétures cutanée (flancs,
racine des membres, mammaire, péri-ombilic), visage érythrosique, congestif,
varicosité
\item moins spécifique : +10kg, graisse facio-tronculaire (visage arrondi), bosse de
bison, hyperandrogénie, OMI
\item autres : ostéoporose (asymptomatique), spanioménorrhée, \dec libido, HTA,
troubles psy
\end{itemize}
Bio : intolérance glucide

\begin{tcolorbox}
Diagnostic d'hypercorticisme :\\
  Si (\acrshort{CLU} \inc, réponse anormale au freinage minute\footnotemark, cortisol minuit > 72ng/mL) :
\begin{enumerate}
  \item Confirmation par freinage standard\footnotemark
  \item Dosage ACTH : 
  \begin{itemize}
    \item Si bas, scanner surrénales pour adénome\footnotemark
    \item Sinon IRM hypophysaire, freinage fort, test CRH, 
    test métopirone \thus sécrétion ectopique (normal) ou maladie de Cushing
    \end{itemize}
\end{enumerate}
\end{tcolorbox}

\footnotetext{Freinage minute = cortisolémie matin après 1mg dexaméthasone à 23h  (rétrocontrole négatif en théorie des glucocorticoïdes sur cortisol)}
\footnotetext{Freinage standard/faible : urines sur 48h puis idem après 0.5mg  dexméthasone. On vérifie également le rétrocontrôle négatif}
\footnotetext{ACTH stimule la production de glucocorticoïdes (via surrénales). Un excès de cortisol est censé diminuer la production d'ACTH. Si   oui, atteinte surrénale !}

DD : stress intense, dépression sévère, psychose, alcoolisme (CLU modéré,
  freinage minute anormal) \thus épreuve du temps 

\subsection{Insuffisance antéhypophysaire}
\label{sec:org49dce73}
Clinique : face pâle, "veillot", dépigmentation aréole mammaire et OGE,
dépilation complète aisselles pubis

Signes des déficits hypophysaires :
\begin{itemize}
\item gonadotrope : \male = \{\dec libido, pilosité visage \dec, petits
testicules mou, infertile\}, \female = \{aménorrhée, dyspareunie\}, ostéoporose,
(retard pubertaire)
\item corticotrope : asthénie, hypotension, amaigrissement, pas de déficit
aldostérone ! (hyponatrémie de dilution). Risque de collapsus CV
\item hypothyroïdie modérée
\item somatotrope : adulte = \{\dec masse et force musc, adiposité abdo\}, enfant =
retard croissance, accidents hypoglycémiques
\end{itemize}

\subsubsection{Bilan hypophysaire}
\label{sec:org411850d}


\begin{itemize}
\item Déficit corticotrope : test Métopirone, cortisol < 200ng/mL si hypoglycémie (<
0.49g/L), autres test (cortisolémie, synacthone, CRH)
\item Déficit thyréotrope  : \dec{} T4L sans augmentation de TSH
\item Déficit gonadotrope : 
\begin{itemize}
\item \female{} préménopause : aménorrhée, oestradiol \dec, gonadotrophines
normales
\item \female{} postménopause  : gonadotrophines basses\footnote{Ou dans les valeurs des femmes jeunes}
\item \male{} : troubles sexuels \dec testostérone
\end{itemize}
\item Déficit somatotrope : déficit GH enfant++ (retard croissance, pas de réponse
à stimulation GH) ou adulte (faire au moins 2 tests stimulation)
\item Prolactine normale/élevée
\end{itemize}

IRM si déficit hypophysaire
\section{243 : Insuffisance surrénale}
\label{sec:org15901ed}
\subsection{Insuffisance surrénale lente}
\label{sec:org2c8ea27}
Rare mais grave \skull

Physiopatho :
\begin{itemize}
\item cortisol \footnote{Stimulé par ACTH, rétrocontrole nég. sur ACTH}) : hyperglycémiant, stimule = \{catabolisme protidique, lipogenèse, SNC, tonus vasculaire\}, inhibe
= \{hormone antidiurétique\}, anti-inflammatoire et antipyrétique,
minéralocorticoïde\\
Minimum 0-2h, maximum 7-9h
\item aldostérone : réabsportion Na+ et Cl-, excrète K+
\item androgènes surrénalien (stimulé par ACTH)
\end{itemize}

\begin{table}[htbp]
\caption{Insuffisance surrénale primaire (maladie d'Addison)/secondaire : clinique}
\centering
\begin{tabular}{ll}
\toprule
Primaire (surrénale) & Secondaire (hypophysaire)\\
\midrule
Fatigue, dépression, anorexie, nausées & \\
\dec poids, hypotension, hypotension orthostatique & \\
Hyperpigmentation & Pâleur\\
HyperK, hypoNa (manque sel) & HypoNa (dilution)\\
\bottomrule
\end{tabular}
\end{table}

\subsubsection{Diagnostic}
\label{sec:orgf9cc5df}
\danger ne pas attendre résultats pour commencer traitment \skull

\begin{table}[htbp]
\caption{Insuffisance surrénale : diagnostic}
\centering
\begin{tabular}{lll}
\toprule
 & Primaire & Secondaire\\
\midrule
cortisolémie à 8h & basse & basse\\
ACTH & haute & basse\\
aldostérone & basse & N\\
rénine & haute & N\\
Synacthène & réponse insuffisante du cortisol & réponse insuffisante\\
\bottomrule
\end{tabular}
\end{table}

Positif 
\begin{itemize}
\item cortisolémie : à compléter avec tests dynamiques si valeurs "intermédiaires"
\item primaire ou secondaire ? ACTH \inc{} si primairei, rénine \inc si primaire
\item test Synacthène (+ Métopirone ou hypoglycémie insulinique si doute)
\end{itemize}

NB : femme enceinte = \{\inc seuil, faisceau d'args\}, enfant : répéter dosages
voire ttt probabiliste

\subsubsection{Étiologies de l' \acrshort{IS} primaire}
\label{sec:org2cec53b}
\begin{itemize}
\item Auto-immune (80\% adulte, 20\% enfant) : 
\begin{itemize}
\item \emph{polyendocrinopathie de type 1} (mutation facteur de transcription AIRE)
\item ou \emph{type 2} (IS + 1 parmi \{thyroïdite d'Hashimoto+++, Basedow, diabète 1\}
\item \thus autoAc anti-21-hydroxylase, scanner (surrénales atrophiques)
\end{itemize}
\item \emph{tuberculose bilatérale surrénale} (10\%) : transplanté ou ID avec TCD tuberculose
\thus scanner surrénales (\inc puis atrophie et calcification). Bilan des
localisation tuberculose
\item \emph{VIH} (stade avancé) : iatrogène, infection opportuniste (CMV++), atteinte de
l'hypophyse (lymphome, CMV), corticoïde anti-inflammatoire et ritonavir
\end{itemize}
\danger dénutrition \thus spécialiste
\begin{itemize}
\item autres : \emph{iatrogènes}\footnote{Surrénalectomie bilatére, anticortisolique de synthèse, nécrose hémorragique}, \emph{métastases bilatérales}\footnote{\danger éliminer phéochromocytomes avant biopsie surrénale}, lymphomes, maladies
infiltratives, causes vasculaires
\item enfant : génétiques surtout = bloc enzymatique++ (dépistage obligatoire), adrénoleucodystrophie
\end{itemize}

\subsubsection{Étiologies de l'IS secondaire}
\label{sec:orgc0a82c9}
\begin{itemize}
\item \emph{interruption corticothérapie prolongée} surtout (> 7mg prednisone)
\item autres\footnote{S'associe souvent à d'autres insuffisances de l'axe hypothalamo-hypophysaire} : tumeur région hypothalamo-hypophysaire, hypophysite (auto-immune),
granulomatose, trauma, chir hypophysaire, radiothérapie, sd de Sheehan
\end{itemize}

\subsubsection{Prise en charge}
\label{sec:org5f94948}
Ttt substitutif :
\begin{itemize}
\item glucocorticoïdes (hydrocortisone) 15-25mg/j
\item minéralocorticoïde (fludrocortisone) 50-150\(\mu\)g/j si IS primaire
\end{itemize}

TTT cause

Éducation du patient : 
\begin{itemize}
\item avoir carte, hydrocortisone (comprimés et injection), régime normosodé
\item pas de laxatifs, diurétiques, millepertuis, réglisse, jus de pamplemousse
\item ttt à vie
\item hydrocortisone en SC si > 2 vomissement/diarrhées en < 1/2 journée
\item adapter à chaleur, exercice, voyage
\end{itemize}

Surveillance clinique :
\begin{itemize}
\item fatigue, poids, PA
\item surdosage en hydrocortisone (gonflement/rougeur visage, \inc poids, HTA, os,
métabolisme, CV)
\item surdosage en fludrocortisone
\item cortisolémie et ACTH inutile !!
\end{itemize}

\subsection{Insuffisance surrénale aigüe}
\label{sec:org93c2f0f}
\subsubsection{Y penser}
\label{sec:orge500a5e}
\begin{itemize}
\item déshydratation extracellullaire avec pli cutané, hypotension
\item confusion, crises convulsives
\item troubles digestifs
\item douleurs (musc, céphalées)
\item fièvre
\end{itemize}
Biologie : 
\begin{itemize}
\item hémoconcentration, insuf rénale fonctionnelle++
\item hypoNa, hyperK++
\item hypoglycémie, acidose métabolique, anémie, hyperlymphocytose,
hyperéosinophilie, natriurèse conservée
\end{itemize}

\subsubsection{Confirmation}
\label{sec:org8d0fa74}
Si diagnostic \textbf{non} posé : dosage cortisol (\dec\dec), ATCH (\inc si primitive, N ou \dec
si secondaire). 
Ne pas attendre les résultats \skull

\subsubsection{Causes}
\label{sec:orgb4ed3a0}
\begin{itemize}
\item Insuf surrénale chronique décompensée++
\item D'emblée si bloc enzymatique surrénalien (21-hydroxylase) complet (néonatale)
ou hémorragie bilat surrénale ou apoplexie hypophysaire
\item Décompensation par n'importe quelle patho intercurrente
\end{itemize}

\subsubsection{PEC}
\label{sec:org20b6be1}
\danger Urgence extrème \skull
\begin{itemize}
\item Domicile : 100mg hydrocortisone (IV, IM, SC) puis transport
\item Hopital : réa puis 
\begin{itemize}
\item perfusion NaCL (et G30\% si hypoglycémie)
\item ttt facteur déclenchant
\item surveiller : PA, FC, FR, oxymétrie de pouls, diurèse, T, glycémie, CS, ECG
si hyperK
\end{itemize}
\end{itemize}

Ttt préventif : patient doit \inc ses doses, médecin traitant au courant

\subsection{Arrêt d'une corticothérapie}
\label{sec:org93113a2}
Expose au rebond de la maladie causale, insuf surrénale secondaire
(corticotrope), sd de sevrage

À risque : (ttt \(\ge\) 3 semaines par \(\ge\) 20mg prednisone) ou (corticoïdes et inhib enzymatique du
cytochromie P450 (ritonavir)) ou sd Cushing iatrogène
\section{244 : Gynécomastie}
\label{sec:org3e8e77e}
Hyperplasie tissue glandulaire mammaire, fréquente. Dû à oestrogène \inc{} et testostérone \dec{}. Regarder aussi TeBG,
SHBG
\subsection{Démarche}
\label{sec:org79c5d17}
\begin{itemize}
\item Clinique : palpation = ferme/rugueux, mobile arrondi, centré par le mamelon (rien si adipomastie)
\item Mammographie si doute : opacité nodulaire/triangulaire (rien si adipomastie). Élimine cancer du sein (rare)
\item Physiologique ? 
\begin{itemize}
\item 2/3 des nouveaux-nés
\item pubertaire : de 13 jusque 20 ans, rétrocède . Palper testicule pour atrophie testiculaire/tumeur
\item fréquente > 65 ans. Palpation testiculaire
\end{itemize}
\end{itemize}
\subsection{Étiologie}
\label{sec:org43c4c4b}
\begin{tcolorbox}
Causes fréquentes : médic, idiopathique, cirrhose, insuf testiculaire/gonadotrope, (tumoral)
\end{tcolorbox}
Évidente :
\begin{itemize}
\item insuf rénale chronique, cirrhose, médicaments (surtout spironolactone,
antiandrogène, kétoconazole, neuroleptiques, ATB antirétroviraux, antiulcéreux)
\end{itemize}
Sinon exploration hormonale : T4L, TSH, hCG, testostérone totale, LH, FSH,
prolactine, oestradiol

Causes endocriniennes :
\begin{itemize}
\item hyperthyroïdie
\item insuffisance testiculaire/hypogonadisme périphérique (8\%) : sd de Klinefelter
le plus fréquent
\item hypogonadisme d'origine hypothalamique/hypophysaire: testostérone basse, LH,
FSH normales/abaissées \thus imagerie hypophysaire, dosage
prolactine. Hyperprolactinémie ou tumorale
\item tumeur sécrétant oestrogène : oestradiol \inc, testostérone \dec \thus tumeur
testiculaire (ou surrénalienne rarement) \thus echo testiculaire ou scanner
abdo
\item tumeur sécrétant hCG : \inc hCG \thus écho testiculaire, scanner
cérébrales. Dans les bronches ou le foie parfois. Chimio.
\item Résistance androgènes (exceptionnelle) : testostérone \inc, LH \inc
\item idiopathique (25\%)
\end{itemize}

\subsection{Traitement}
\label{sec:orgd7f2568}
Traiter la cause. Sinon
\begin{itemize}
\item Pubertaire : ne rien faire
\item Idiopathique : androgènes non aromatisables 3 mois. Si inefficace, chir
plastique possible
\end{itemize}
\section{245 : Diabète}
\label{sec:org8399064}
  \begin{tcolorbox}
Définition : glycémie à jeun \ge 1.26g/L (2 reprises) ou aléatoire \ge 2g/L \footnotemark
\end{tcolorbox}
\footnotetext{Normale à jeûn < 1.10g/L}

Caractéristiques diabète 1 (le diabète 2 s'y oppose) : 
\begin{itemize}
\item ATCD familiaux rares
\item < 25 ans
\item début rapide explosif
\item symptomatologie bruyante
\item poids normal ou \dec
\item hyperglycémie majeure > 23g/L
\item souvent cétose
\item pas de complications dégénératives
\item mortalité par insuf rénale (CV pour diabète 2)
\end{itemize}

\subsection{Diabète 1}
\label{sec:org7a4c0f5}
Prévalence : 1/200 000 (10\% des diabétiques). Peut survenir à tout âge. \inc
incidence. Sex-ratio = 1

\subsubsection{Physiopathologie}
\label{sec:org047339e}
Carence en insuline par destruction cellules beta du pancréas. Soit auto-immun
(fréquent), soit idiopathique

Prédisposition génétique (Ag HLA, VNTR, CTLA-4, PTP-N22), facteurs
environnementaux.

Processus auto-immuns : au moins un Ac parmi les suivants dans 97\% : Ac
anti-\{ICA, GAD, IA2, insuline, ZnT8\}

Diabète \(\in\) sd polyendocrinien auto-immun : 10\% d'autres maladies auto-immunes =
thyroïdopathies (Basedow, thyroïdite), Addison, atrophie gastrique de Biermer,
maladie coeliaque, vitiligo 
\begin{itemize}
\item doser Ac anti-TPO ou TSH (thyroïdite), anti-surrénale (Addison), anti-transglutaminase \textpm{} anti-endomysium (coeliaque), anti-paroi gastrique, anti-facteur intrinsèque (Biermer)
\item si positif : surveillance annuelle
\end{itemize}

\subsubsection{Clinique}
\label{sec:org19eb2ad}
\begin{tcolorbox}
Diabète 1 : \{polyuro-polydipsie, amaigrissement, polyphagie\}, glycémie $> 2g/L$ \pm cétonurie
\end{tcolorbox}
Habituel : début rapide/explosif, sd cardinal = polyuro-polydispsie,
amaigrissement, polyphagie, troubles visuels transitoires, examen pauvre (fonte
musc, signes d'acidose\footnote{Dyspnée de Kussmaul, haleine acétonique}, glycémie veineuse, cétonurie (acidocétose
inaugurale)

Formes
\begin{itemize}
\item Diabète 1 lent (LADA\footnote{Latent Autoimmune Diabetes in the Adult}) : début tardif, progessif, Ac positif. Insulinothérapie en 2-10 ans
\item Révélé par acidocétose : fréquente chez enfants, ne devrait plus être vues
\item Non insulinodépendantes : 3 auto-Ac positifs \thus 100\% d'avoir diabète 1 dans
5 ans. Rémission de quelques mois possibles.
\item cétosique du sujet noir d'origine africaine : mécanisme
auto-immun. Décompensation cétosique. Auto-Ac spécifique du diabète 1
nésgatifs.
\end{itemize}

Affirmer type 1 :
\begin{itemize}
\item clinique : triade maigreur/amaigrissement, cétose, < 35 ans
\item sinon auto-Ac
\item sinon :
\begin{itemize}
\item hérédité dominante : MODY, mutation SUR1/KIR6-2 (si diabète néonatal)
\item symptômes inhabituels : sd de Wolfram (atrophie optique, surdité, diabète
insipide < 20 ans), mitochondropathie (surdité, dystrophie maculaire,
cardiomyopathie transmission par la mère)
\item secondaire : cancer pancréas (amaigrissement), pancréatite chronique,
mucoviscidose, hémochromatose, médicaments
\end{itemize}
\end{itemize}

\subsubsection{Évolution}
\label{sec:orgf2909b7}
Schéma théorique : phase préclinique (destruction cellules beta), clinique (85\% détruites), clinique séquellaire

Diabète instable : 
\begin{itemize}
\item itérations de cétoacidoses ou hypoglycémies sévères, psycho.
\item DD : gastroparésie, déficit systèmes contra-insuliniques, Ac anti-insuline
\end{itemize}

\subsubsection{PEC}
\label{sec:org808d1af}
Patient : contrôle glycémie, injection d'insuline, prévenir complications
métaboliques et vasculaires, adapter ttt, contrôle de l'alimentation \thus
éducation thérapeutiques

Objectifs : HbA1c < 7\% (enfants : entre 7.5 et 8.5, complication/sujet âgé : 8\%)

Autosurveillance : urinaire rare, glycémique 4/jour \footnote{De temps en temps 3h matin, postprandial}, (glucose en SC si
objectifs thérapeutiques non remplis) :

Surveillance :
\begin{itemize}
\item HbA1c : en pourcentage Hb totale, aucun sens si modif durée de vie moyenne des globules rouges\footnote{Hémoglobinopathie, anémie hémolytique, urémie, EPO, saignées}
\item diabétologue/pédiatre endocrinologue 3/an
\item \{lipides, créat, microalbuminurie\}
\item ophtalmo, cardiologie 1/an (sympto/âgé,compliqué), dentiste 1/an
\end{itemize}

Traitement insulinique (Table \ref{tab:orgae693a1} ): palliatif à vie

\begin{table}[htbp]
\caption{\label{tab:orgae693a1}
Traitement insulinique du diabète 1}
\centering
\begin{tabular}{llll}
\toprule
Type & Injection & Durée & Utilisation\\
\midrule
insuline humaine recombinante & IV, IM, SC & 7-8 & Prandiale, hyperglycémie\\
analogue rapide de l'insuline & IV, IM, SC & 4-6h & Pompe\\
forme lente & SC & 9-16h & \\
analogue lents &  & 16-40h & \\
\bottomrule
\end{tabular}
\end{table}

Injection :
\begin{itemize}
\item Résorption SC \inc si injection dans muscle, chaleur, vasodilatition, selon zone et dose,
\item Stylo à insuline (pompe si échec)
\item Schéma : analogue lent (1-2/j) et analogue rapide (3-4)
\item dose "pour vivre", "pour manger", "pour traiter", activité physique \thus
éducation nutritionnelle
\end{itemize}

Effets secondaires : hypoglycémie, prise de poids légère, allergie rarissime,
lipoatrophie insulinique (immuno), lipohypertrophie si piqûres au même endroit

Non insulinique : accompagnement, alimentation variée sans interdits, exercice physique

\subsubsection{Cas particuliers}
\label{sec:org30604e5}
Enfant/ado : 
\begin{itemize}
\item cétoacidoses fréquentes (risque d'oedème cérébral \inc si réa).
\item difficile à équilibrer et accepter chez l'ado
\item pompe chez très petit enfant
\end{itemize}

Femme :
\begin{itemize}
\item dépister diabète 1 pendant grossesse
\item contraception à discuter : pilule oestrogestative possible si \female{} jeune,
sans complication, non fumeuses, diabète bien équilibré
\item grossesse : pronostic quasi normal si équilibre dès conception
jusqu'accouchement et non compliqué. Analogue de l'insuline. \\
\danger{} HbA1c \dec , \inc besoin fin grossesse et \dec post-partum, aggravation rétinopathie
et néphropatie
\item CI absolue : insuf coronaire instable
\item HbA1c < 6.5\%, glycémie jeun < 0.9g/L
\end{itemize}

Ménopause : hormonothérapie seulement si médicalement indiqué

Jeûne : 
\begin{itemize}
\item si intolérance gastrique : pas d'arrêt insuline, collations liquide/hospit. Vérifier cétonurie
\item pour examen à jeun : laisser agir analogue lent, surveiller glycémie
\item prolongé/stress : soluté glucosé et insuline en IV
\end{itemize}

\subsection{Diabète 2}
\label{sec:org048a072}
90\% de diabète. Prévalence 4\%. À risque : obèse, anomalie métabolisme
glucidique, ATCD familiaux diabète 2, ethnie noire/hispanique.
Age adulte.

\subsubsection{Physiopatho}
\label{sec:orgffd5cd7}
Insuline n'arrive pas à avoir une réponse amximale : défaut de captation
musculaire du glucose, \inc production hépatique du glucose, lipolyse \inc acides gras
libres circulants \thus (aggrave \dec insulinosécrétion et utilisation du glucose)

Insuf de sécrétion d'insuline, s'aggrave avec l'âge et la durée

Facteurs génétique. Hyperglycémie aggrave insulinosécrétion,
insulino-résistance. \dec sécrétion adiopkines

\subsubsection{Cliniques}
\label{sec:orged85310}
Signes secondaires hyperglycémie, souvent inaperçus. Décompression sévères \thus
polyurie, polydispie, amaigrissement, prurit vulvaire/balanite, infections
récidivantes

Dépistage : si
\begin{itemize}
\item glycémie veineuse à jeun pour : signes cliniques de diabètes, > 45 ans (tous les 3 ans), \(\ge\) 1 FR
\item non caucasien/migrant, \glslink{sdMetabolique}{sd métabolique}
\end{itemize}

DD : diabète 1 lent (LADA) (minceur, 0 ATCD, IA2 et GAD positifs), génétique
(MODY2, mitochondrial), secondaire (pancréatopathie, hémochromatose,
mucoviscidose, médicaments)

Évolution : insulinopénie -> insulinoréquerant. Pronostic selon complications.

\subsubsection{Traitement}
\label{sec:orgce73fc3}
Objectifs : normalisation HbA1c (< 6.5\%), améliorer glycémie, insulinosensibilité

Moyens  : activité physique, régime (hypocalorique si surpoids) sans sucres
rapides, traitement oraux, analogues GLP-1, insuline

Ttt oral :
\begin{itemize}
\item biguanide (Metformine)+++ : 
\begin{itemize}
\item ES = digestif.
\item CI = insuf rénale (?), hépatique, respiratoire
\end{itemize}
\item autres sulfamides, glinides, inhibiteurs DPP-4, inhibiteurs \(\alpha\)-glucosidase
\end{itemize}

Surveillance glycémique :
\begin{itemize}
\item HbA1c essentielle : < 7\% pour plupart des patients (8\% si comorbidité grave,
âgé fragile, ATCD complication macrovasc, insuf rénale chronique sévère, 9\% si
agé dépendante)
\item autosurveillance glycémique : pas systématique si ttt oral (1-3 cycles/j),
nécessaire si insuline
\end{itemize}

Hygiéno-diététique
\begin{itemize}
\item Activité physique 
\begin{itemize}
\item avantages : \dec incidence diabète 2, \inc insulinorésistance, \inc TA
effort, \inc masse maigre, \dec masse grasse
\item intensité modérée  \(\ge\) 30min et intense (> 60\% \(VO_{2max}\)) de 20min
\item 30min/jour, 3-5/semaine
\item CI : insuf coronarienne, rétinopathie proliférante non stabilisée
\item surveiller si risque hypoglycémie (reprise, intensité/durée
inhabituelle). \danger{} pieds !
\end{itemize}
\item Alimentation : 
\begin{itemize}
\item hypocalorique si surpoids, équilibrée\footnote{50\% glucides, 30\% lipides, 20\% protides. Légumes et féculents pour \inc
absorption des glucides}, sans sucres rapides.
\item objectif : -5 à 10\% du poids
\end{itemize}
\end{itemize}

Traitement médicamenteux :
\begin{itemize}
\item oral : metformine (sinon sulfamide, puis inhibiteurs DPP-4 ou inhibiteurs
\(\alpha\)-glucosidase)
\item insulinothérapie 
\begin{itemize}
\item quand : insulinorequérance (amaigrissement, asthénie, amyotrophie), observance thérapeutique, HbA1c > objectfis, CI oraux,
\end{itemize}
affections intercurrentes
\begin{itemize}
\item combiné = insuline intermédiaire/analogue lent + hypoglycémiant oraux si
insulinorequérance partielles. 0.2 U/kg/j à adapter
\item exclusive : autosurveillance glycémique quotidienne, même gestion que
diabète 1.
\item CI au renouvellement permis poids lourds
\end{itemize}
\end{itemize}

\subsection{Complications}
\label{sec:org8da55c1}
Souffrance vasculaire : micro- (rein, oeil, nef) et macro-angiopathie (\inc
athérosclérose). AOMI x6-10

\subsubsection{Physiopatho}
\label{sec:org4ed095d}
Excès de glucose entre dans les cellules endothéliales, musculaire lisses, péricytes.
Glycolyse : voies mineures \inc, systèmes de protection de la mitochondrie
débordés \\
\thus stress oxydant dans la cellule.

Autres causes d'agression: inflammation, activation rénine-angiotensine, voies profibrosantes,
induites par l'hypoxies. Systèmes de protection moins efficaces : antioxydants,
anti-inflammatoire, cellules progénitrices vasculaires, angio-, artério-genèse

Conséquences : épaississement des membranes basales, troubles perméabilité
vasculaire, spécifiques = \{profil vasculaire (rétine), fibrose (rein)\}

\subsubsection{Rétinopathie diabétique}
\label{sec:org3379bdb}
\paragraph{Déf}
\label{sec:org647d8fb}
diabète = risque d'une rétinopathie. Complications dont on peut éviter la
cécité ! \thus examen ophtalmo et surveillance annuelle, contrôler glycémie et
HTA, laser (si (pré-)proliférante), laser ou injection anti-VEGF (maculopathie
oedémateuse)

\paragraph{Épidémio}
\label{sec:orgb4de828}
90\% de rétinopathie après 30 ans de diabète, dont 30\% menaçant pronostic
visuel. Diabète = 1ere cause de cécité acquise en France chez < 50 ans. FR =
durée et intensité de l'hyperglycémie

\paragraph{Physiopatho}
\label{sec:orge6dc81d}
anomalie vasculaire (microanévrisme, trouble perméabilité capillaire) \thus soit
oedème (dangereux si maculaire), soit (ischémie puis angiogenèse\footnote{Risques : saignement entre rétine et vitré, traction sur rétine,
hypertonie oculaire} puis rétinopathie proliférante)

\paragraph{Examens}
\label{sec:orgd8bb7cb}
cécité possible du jour au lendemain \skull \thus dépistage tous les 2 ans
(annuel si diabète/TA mal contrôlé, \{avant, trimestre, post-partum\} pour femme
enceinte)

Surveillance par examen ophtalmo, photo au rétinographe(dépistage seulement), tomographie de
cohérence optique

\paragraph{Gravité}
\label{sec:orgfcae8e9}
\begin{itemize}
\item au fond d'oeil : 
\begin{itemize}
\item microanévrisme < exsudats < nodules blancs cotonneux, irrégularités \diameter{} veines, capillaires dilatés (préproliférante) < néovaisseaux (proliférante) < décollement rétine, hémorragies, glaucome néovasculaire
\item maculopathie : exsudat, oedème maculaire, ischémie
\end{itemize}
\item \dec acuité visuelle car hémorragie intravitréenne, décollementd rétine,
glaucome néovasculaire, maculopathie diabétique
\item complémentaire : angiographie fluorescéine, tomographie cohérence optique
(oedème maculaire), écho en mode B
\item risque d'évolution rapide : ado, puberté, grossesse, intensification de
l'insuline, chir cataracte, \inc TA, dégradation fonction rénale
\end{itemize}

\paragraph{Traitement}
\label{sec:org2d7b9a5}
équilibre glycémie et TA++, laser\footnote{\dec de 50\% risque de cécité, \dec néovascularisation dans 90\%. Danger
si oedème maculaire, traite seulement l'exsudat} (+ injection intraoculaire
d'inhibiteurs VEGF)

Autres complications : cataracte (+ freq), glaucome néovasculaire (redoutable),
paralysie oculomotrice (régresse spontanément qq mois)

\subsubsection{Néphropathie}
\label{sec:org471519b}
Diabète = 1ere cause d'insuf rénale terminale. Diabète 2 = \(\frac{3}{4}\)
diabétiques dyalisé. Risque CV x10 (D1), x3 (D2). Décès insuf rénale terminale
30\% (D1), 5\% (D2)

Physiopatho : \inc pression intra-glomérulaire \thus dilatation des
glomérules. Filtration améliorée puis décroît (sclérose) avec \inc
albumine\footnote{Microalbuminurie, macro quand détectable à la BU}. Et toxicité directe glucose.

\paragraph{Dépistage}
\label{sec:org9516b30}
1/an par BU (protéinurie, hématurie, infection urinaire),
rapport albuminurie/créatinurie (excrétion urinaire) et vérif à 6 mois

\paragraph{Diagnostic}
\label{sec:org8c0528e}
Signes cliniques tardifs (HTA si protéinurie, oedème si protéinurie et insuf
rénale)

\danger{} faux positifs pour microalbuminurie : orthostatisme prolongé, activité
physique intense, \(\Delta\) PA, tabac, fièvre, \inc insuf cardiaque, hyperglycémie,
infection urinaire

Signes associés : 
\begin{itemize}
\item rétinopathie (surtout D1). FR CV : microalbuminurie, DFG \dec.
\item Penser sténose des artères rénale (surtout D2) si HTA résistante ou \dec\dec
rein.
\item Hyperkaliémie (\inc si IEC, sartans) \thus surveiller
\end{itemize}

Diagnostic histologique :
\begin{itemize}
\item pas besoin si \{rétinopathie (hyperglycémie prolongé), excrétion urinaire d'albumine \inc répétée et croissante\}
\item biopsie seulement si 0 rétinopathie, < 10 ans après diagnostic, aggravation rapide, hématurie HTA sévère, signes extra-rénaux
\item diabète 1 : hypertrophie mésangiale, glomérulaire < épaississement membrane basale, dépôts
mésangiaux < hyalinose artériolaire < glomérulosclérose nodulaire
\item diabète 2 : 1/3 typique, 1/3 vasculaire, 1/3 non néphropatie diabétique
\end{itemize}

Classification en 5 stades : 
\begin{enumerate}
\setcounter{enumi}{3}
\item Néphropathie incipiens : microalbuminurie\footnote{30-300mg/24h ou 20-200mg/L}
\item Néphropathie : PA élevée, DFG \dec de 10mL/min/an, nodule de sclérose,
hyalinose artériolaire
\item Insuffisance rénale
\end{enumerate}

\paragraph{Traitement}
\label{sec:org7ffae2e}
Prévention 
\begin{itemize}
\item primaire (diabète, FR HTA)
\item si microalbuminurie : HbA1c < 7\%, PA < 140/85, IEC ou sartans, FR, régime
hypoprotidique, sel < 6g/j
\item si macroalbuminurie : contrôle tension++ < 140/85mmHg. (IEC ou sartan) et
diurétique thiazidique. Protéinurie < 0.5g
\item si insuf rénale : 
\begin{itemize}
\item si DFG < 30mL/min/1.73m\(^{\text{2}}\), HbA1c < 8\% et seuls autorisé : insuline, répaglinide, inhib \(\alpha\)-glucosidase,
\item surveiller glycémie (HbA1c souvent pertubrée si IR chronique)
\item PAs < 130mmHg
\item traiter anomalies phosphocalciques, anémie arégénérative, préparer suppléance rénale
\end{itemize}
\end{itemize}

Éviter AINS. Si nécessaire, pas d'IEC/sartan. Limiter produits contrastes iodés

Néphrologue si : doute diagnostique, DFG < 45mL/min/1.73m\(^{\text{2}}\), protéinurie brutale.

IEC, sartans : se méfier d'une sténose des artères rénales : doser kaliémie, créatininémie

Infections urinaires : \(\times 3\) dont 90\% asymptomatique (basses)
\begin{itemize}
\item Risque = contamination du haut appareil urinaire (pyélonéphrite, nécrose papillaire,
\end{itemize}
pyélonéphrite emphysémateuse), aggravation néphropathie glomérulaire.
\begin{itemize}
\item Ttt : oui si symptomatique. Sinon pas de consensus
\end{itemize}

\subsubsection{Neuropathie}
\label{sec:org9f56605}
\begin{itemize}
\item Autonome : tardive
\item Périphérique : 50\% des diabètes à 20 ans. FR : grande taille, tabac, âge,
AOMI, carences nutritionnelles/vitaminiques, alcool, insuf rénale
\end{itemize}

Atteinte métabolique et vasculaire.

Dépistage sur examen clinique et interrogatoire (+ examens complémentaires si
autonome)

Si débutante, souvent silencieuse ! Examen des pieds !

\paragraph{Sensorimotrice}
\label{sec:org581be5f}
Fibres les plus longues en premier ("chaussettes" puis "gants")\footnote{Exceptionnellement, douleurs abdominales}. Examen
clinique pour perte de sensibilité, interrogatoire pour douleur

\begin{itemize}
\item Polynévrite symétrique distale : fréquente (40\% diabétiques après 25 ans) avec
\begin{itemize}
\item hypoesthésie pression/tact/thermique/proprioceptique ignorée
\item parfois paresthésies distales, douleurs "arc électrique"
\item ROT achilléen aboli (puis rotulien)
\item voûte plantaire se creuse (tardivement)
\item complication : neuroarthropathie (pied "cubique" de Charcot)
\item rares : mononeuropathies avec signes moteurs déficitaires, douleurs à
exacerbation nocturne
\end{itemize}
\item Polynévrite asymétrique proximale : beaucoup plus rare. L2, L3
(L4). Fatigabilité et amyotrophie douloureuse proximale
\item Polyradiculopathie thoracique : rare, douleurs abdo aux niveaux T4-12
\item Mononévrite : 5-10\%, asymétrique. Souvent nerfs crâniens. Si MS
: compressive. Si MI : sensitif
\item Multinévrite : DD = vascularite
\end{itemize}

\paragraph{Autonome}
\label{sec:org8f04f4d}
Diabète ancien, mal équilibré \thus nerfs vagues, système sympathique
lésés. Régression rare

\begin{itemize}
\item CV : \{tachycardie sinusale quasi-permanente, (bradycardie permanente),
allongement QT\}
\item Vasomotrice : hypotension orthostatique \emph{sans} accélération du pouls, troubles
microcirculation périphérique\footnote{Hyperémie, rougeur, oedème}
\item Troubles sudation : sécheresse cutanée MI, parfois hypersudation partie
supérieure. Prurit possible
\item Digestive gastro-intestinale : \{parésie tractus digestif, dysphagie, gastroparésie (fréq), diarrhée banale/motrice capricieuse (diag d'élimination !\footnote{maladie coeliaque, pullulation microbienne, endocrine}, constipation, incontinence fécale (rare)\}
\item Vésicale : pas de perception de la vessie pleine, hypoactavité détrusor et
favorisé par polyurie de l'hyperglycémie. Résidu post-miction \thus
incontinence, rétention, infection urinaire \thus clinique, écho
(prostate, vessie)
\item Dysfonction érectile : psychogène, sd de Leriche\footnote{Thrombus bloquant l'aorte abdominale avant bifurcation} (rare). À rechercher
\emph{systématiquement}. \\
DD : examen génital, testostérone, prolactinémie. Traitement efficace presque toujours.
\end{itemize}


Examen 
\begin{itemize}
\item clinique : interrogatoire (hypotension OS, diarrhée\ldots{}), inspection pieds, ROT abolis au niveau des troubles sensitifs, monofilament, sensibilité épicritique, thermoalgique, vibratoire\footnote{Grosses fibres\label{org8c6a3c8}}, proprioceptiques\textsuperscript{\ref{org8c6a3c8}}
\item ECG annuel, EMG si atypique
\item chercher dénervation cardiaque parasympathique : variation de la FC entre
inspiration et expiration\footnote{Sensible mais pas interprétable > 60 ans ou patho bronchorespiratoire}, rapport RR long/court pendant épreuve de
Valsalva, variation FC de couché à debout
\end{itemize}

DD : neuropathies métaboliques (insuf rénale, amylose, hypothyroïdie), toxiques
(alcool, tabac, iatrogène), paranéoplasiques, carentielles, inflammatoire,
infectieuse (Lyme, lèpre), autre (Charcot-Marie-Tooth, péri-artérite noueuse)

Traitement : 
\begin{itemize}
\item préventif = glycémie. FR : alcool, tabac, insuf rénale, carence
vitamines B, médicaments.
\item Si installées, stabiliser et éviter les complications
(mal perforant plantaire++)
\item Antalgiques, hydratation peau
\end{itemize}

\subsubsection{Macroangiopathie}
\label{sec:org9d9ab3a}
\diameter > 200 \(\mu\)m. Plus fréquente et sévère. Artères visibles sur radio.

Prévention CV = \textbf{problème majeur} des diabétiques 2 : \(\frac{3}{4}\) DC d'une cause
CV. 

Risque CV \texttimes{}2-3 (\texttimes{}3-4 chez \female). Risque coronarien \texttimes{}2-4, AV ischémique
\texttimes{}2, AOMI \texttimes{}5-10 !

\paragraph{Dépistage}
\label{sec:orgf6b7869}
Risque > 1\% = élevé\footnote{Calcul par les études UKPDS ou SCORE (mais \texttimes{}2-4 pour ce dernier)}

\emph{FR} :
\begin{itemize}
\item CV : > 50 ans \male (> 60 \female), diabète > 10 ans, ATCD IDM/mort subite (< 55
ans \male, < 65 ans \female) ATCD AVC constitué < 45 ans, tabac, HTA
permanente, HDLc < 0.4g/L, microalbuminurie > 30mg/24h
\item autres : obésité abdominale (> 102cm \male, > 88cm \female) ou IMC >
30k/m\(^{\text{2}}\), sédentarité, > 3 verres vin/j (2 si \female), pyschosociaux
\end{itemize}

Montrer atteinte artérielle : 
\begin{itemize}
\item coronaropathie : ECG repos annuel, scinti avec épreuve d'effort ou coronarographie
\item carotides ? auscultation \thus écho si AIT possible
\item AOMI ? pieds, pouls, claudication, IPS cheville/bras < 0.7 ? Écho-doppler
\end{itemize}

\paragraph{Diagnostic}
\label{sec:org7cbb8fc}
Ischémie myocardique silencieuse fréquente ! Y penser si troubles digestifs,
asthénie à l'effort, troubles du rythme cardiaque, déséquilibre inexpliqué du
diabète, \dec PA \thus dépistage systématique si risque

Risque extrême : diabète et microangiopathie sévère (glomérulopathie et
protéinurie > 1g/L), atteinte vasculaire

AOMI : fréquemment avec neuropathie. 1/3 proximale (HTA), 1/3 distale sous
genou (glycémie, tabac), 1/3 proximale et distale. Pouls pédieux = bon pronostic

\paragraph{Traitement}
\label{sec:org3392675}
Revascularisation : stents par défaut (risque de resténose) et chir si atteinte
3 coronaires.

\begin{itemize}
\item Glycémie : Objectif 6.5\% si jeune et prévention primaire, 7\% si âgé ou plus à
risque. Metformine systématique
\item Activité physique systématique
\item Contrôle lipidique : LDL < 1.3g/L (1.0 si risque CV élevée ou
néphropatie). Statines (simvastatine, pravastatine, atorvastatine) ou fibrates\footnote{Pas en association !}
\item PAs \(\in\) [130, 139] et PAd < 90mmHg. Hygiéno-diététique et antihypertenseurs si
échec
\item Prévention thrombose si \(\ge\) 1 FR : aspirine 75-150mg
\item Poids : IMC < 25kg/m\(^{\text{2}}\), tour taille < 94cm \male, 80cm
\female. Hygiéno-diététique
\item Arrêt tabac : substituts nicotiniques (bupropion sinon). Anticiper polyphagie
réactionnelle et modification transitoire sensibilité insuline !
\end{itemize}

\subsubsection{Pied diabétique}
\label{sec:orgdea3ba8}
1 patient sur 10 à risque d'1 amputation d'orteils. Éviter les plaies pour prévenir l'amputation
\skull

Mal perforant plantaire, plaie ischémie d'orteil/membre

À risque : diabète et pouls faible, neuropathie et trouble statique pied,
troubles sensibilité \{algique, vibratoire, thermique, profonde\}, ulcération
pieds

\paragraph{Mal perforant plantaire (MPP)}
\label{sec:org15f1807}
Neuropathies entraîne hypoesthésie et déformations
ostéoarticulaires \thus durillons puis fissure et infection \thus
dermo-hypodermite.

Révélé par : pus, fonte purulente localisé tisseux adipeux

\paragraph{Autres}
\label{sec:org6bd46ac}
\begin{itemize}
\item Ischémie/nécrose : si oblitération/sténose artères de moyen-petit calibre. Peau
froide, fine, dépilée, livedo. \danger nécrose peut arriver en qq heures \thus
revasculariser en urgence \skull
\item Combinaison nécrose et MPP
\item Dermo-hypodermite nécrosante : très rare, urgence vitale \skull. \{Teinte gris,
hémodynamique altérée, plaie odeur fétide\} \thus débrider en urgence, ATB. \\
Cas particulier : gangrène gazeuse à \bact{perfringens} (crépitations à la
palpation, clartés parties moelles\} \thus urgence vitale \danger
\end{itemize}

\paragraph{CAT}
\label{sec:org46137df}
\begin{itemize}
\item Dater l'apparition, neuropathie ou artériopathie. Plaie : localisation, couleur,
signes de diffusion, \{fièvre, frisson, teint gris\}, douleur.
\item Contact : orthopédiste si drain d'une infection purulente. Chir vasculaire si
doute sur l'ischémie. Réanimateur si tableau sévère.
\item Radio pieds bilatérale (ostéite ?), si infection : NFS, iono, CRP
\item Surveillance si état clinique
\item TTT : 
\begin{itemize}
\item décharge (chaussure, arrêt de travail), excision kératose à domicile : peut
suffire
\item anticoagulation si alitement, antalgique,
\item réhydratation, équilibrer glycémie si besoin
\item anti-escarre si ischémie, vaccin anti-tétanos !
\item si infection : parage et drainage, ATB (cocci G+ si récent, sinon bacille G-)
\item revascularisation si plaie artériopathique
\end{itemize}
\end{itemize}

Ostéite : grave mais pas une urgence. Basé sur radio. Ttt . résection
chirurgicale ou ATB 6-12semaine et sans l'appui

\subsubsection{Autres}
\label{sec:org83ea325}
Peau : 
\begin{itemize}
\item nécrobiose lipoïdique (rare, ttt mal codifié)
\item dermopathie diabétique (fréquente, cicatrices brunâtre, régresse spontanément)
\item bullose diabétique (cicatrise spontanément)
\item lipodystrophie (hypertrophie le plus souvent, dû à des injections trop souvent
au même endroit)
\item acanthosis nigricans (placards cutanés brunâtres, cou/aisselles, plis inguinaux)
\item vitiligo (taches achromiques)
\item xanthomatose éruptives (nodules rouge/jeun si grande hypertriglycéridémies)
\end{itemize}
Infections : bactériennes plus nombreuses, fonction polynucléaires altérée
\begin{itemize}
\item otite nécrosante : écoulement auriculaire, douleur intense et insomniante,
inflammation conduit auditif externe avec granulome/nécrose du plancher du
conduit \thus \danger urgence, ORL spécialisé
\item mucormycose : rhino-cérébro-orbitale, destruction osseuse, nécrose muqueuse
paroi des sinus. Fièvre, obstruction et écoulement nasal, oedème jugal et
palpébral \thus urgence \skull
\end{itemize}
Foie : hépatologue dès anomalie transaminases ou \(\gamma\)-GT

Articulations : 
\begin{itemize}
\item capsulite rétractile : douleur diffuse des épaules, limitations
mouvements actifs et passifs, frequence \texttimes{}4. Ttt : antalgiques, corticoïdes locaux,
physiothérapie
\item Maladie de Dupuytren : sclérose réractile de l'aponévrose palmaire moyenne
\item Chéiroarthropathie : raideur des doigts, peau épaissie et cireuse, signe de la
prière
\item Arthrose : fréquente diabétique 2
\end{itemize}

Dents : maladie parodontale 
\begin{itemize}
\item car \{plus de plaque dentaire, \inc production
toxine, matrice extracellullaire altérée, vasculairisation gencive aussi\}
\item contrôle diabète, brossage, fil dentaire, soins dentaires prévention
\item signes : dents branlantes, saignements gencive brossage/mastication
\thus dentiste tous 6 mois
\end{itemize}

\textbf{Suivi diabète} : 
\begin{itemize}
\item complications oculaires, rénales, neuro, cv
\item fond d'oeil annuel, ECG repos annuel, bilan cardio approfondi si risque CV
\inc, écho-doppler MI (si > 40 ans, diabète > 20 ans) tous 5 ans
\item bio : HbA1c 4/an, glycémie veineuse, lipides 1/an, microalbuminurie 1/an,
créatininémie jeun, clairance créat 1/an, TSH
\end{itemize}

\subsubsection{Complications métaboliques}
\label{sec:org3c6d81a}
Coma cétoacidosique
\begin{itemize}
\item acétonurie, glycosurie, glycémie 2.5g/L, pH veineux < 7.25, bicarbonate <
15mEg/L
\item cause : déficit insuline absolu/relatif, inconnue
\item évolution : cétose puis cétoacidose (Kussmaul, stupeur, déshydratation mixte)
\item gravité : âgé, ph < 7, kaliémie 4-6 mmol/L, coma profond, TA instable, pas de
diurèse après 3h, vomissements incoercibles
\item DD : urgence abdo, coma hyperosmolaire
\item Régression sous ttt en 24-48h. Complication iatrogène : oedème cérébral,
surcharge hydrosodée
\item ttt :
\begin{itemize}
\item éducation : si cétose, maintenir injections, supplément insuline rapide,
acétonurie si glycémie > 2.5g/L
\item curatif : insuline rapide IV, recharge volumique, apport potassimu, glucose
si besoin, facteur déclenchant
\end{itemize}
\end{itemize}

Coma hyperosmolaire :
\begin{itemize}
\item glycémie > 6g/L, osmolalité > 350mmol/kg, natrémie corrigée > 155mmol/L, pas
de cétose ni d'acidose
\item FR : > 80 ans, infection aigüe, diurétique, pas d'accès aux boissons, corticothérapie
\item ttt : réhydratation prudente, lente, insuline IV, surveillance, héparine
préventive, ttt causal, soins yeux/bouche/aérosols/aspiration bronchique
\end{itemize}

Hypoglycémie : inévitable 
\begin{itemize}
\item mais pas mortelle, pas séquelle au cerveaux, ne déclenche pas d'accident vasculaire/cardiaque.
\item peur chez diabiétique, déstabilise diabète, attention si âgé ou alcoolique
\item favorisé par : hypoglycémie mineure répétées ignorées, neuropathie végétative
\item cause : repas insuffisants, effort physique, erreur d'injection
\end{itemize}
\section{249 : Amaigrissement}
\label{sec:org28dd9c8}
Fréquent

\subsection{Interrogatoire}
\label{sec:org533ee74}
\begin{itemize}
\item Histoire pondérale, conditions, de vie, psychologique, activité physique excessive et apports alimentaires insuffisants
\item Anorexie, \{troubles digestifs, palpitations, sd polyuro-polydipsie\}, troubles
libido/érection, amnénorrhée (anorexie mentale ou hypothalamique
fonctionnelle), médicaments (nausée, anorexie), dépression masquée++
\end{itemize}

\subsection{Examens :}
\label{sec:org8de6a36}
Clinique : poids, taille, IMC, pli cutané, fonte musculaire, carences vitamines,
pâleur cutanéomuqueuse

Complémentaires :
\begin{itemize}
\item bio : NFS (anémie), VS/CRP (inflammatoire), iono (hyponatrémie \thus insuf
surrénale), BU (glycosurie), calcémie, \{transaminase, \(\gamma\)-GT\}(foie), TSH
(hyperthyroïdie), \{B12, folates, TP, albuminémie\}, graisses fécales ?
(pancréatite chronique calcifiante), dénutrition\footnote{(Pré-)Albumine, IGF-1, ferritine sérique}
\item Radio thoracique (tuberculose), écho abdo (abcès/tumeur), fibro (obstacle),
DEXA (composition corporelles)
\end{itemize}

\subsection{Étiologie}
\label{sec:org90bf74b}
\begin{itemize}
\item Poids stables, apports nutritionnels normaux, examens normaux :  maigreur
constitutionnelle
\item Si perte de poids confirmée, éliminer anorexie mentale, maladies digestives,
iatrogène, cancer extradigestif, maladies infectieuses, neuro, grande
défaillance cardiaque/rénale/respi/hépatique, alcool
\item Sinon, causes endocrines :
\begin{itemize}
\item diabète 1 ou 2 : glycémie, HbA1C
\item hyperthyroïdie : TSH \dec\dec, hormones thyroïdiennes \inc
\item hypercalcémie : si \gls{PTH} inadaptée, hyperparathyroïdie primaire
\item insuf surrénalienne : cortisol, ACTH plasmatique
\item panhypopituitarisme\footnote{Insuffisance antéhypophysaire complète} : cortisol \dec
\item phéochromocytomes : (nor)métanéphrines dans urines 24h, imagerie surrénales
\end{itemize}
\end{itemize}
\section{251 : Obésité}
\label{sec:org0159ff1}
\subsection{Adulte}
\label{sec:org2e4818a}
Surpoids = IMC \(\in\) [25, 29.9]kg/m\(^{\text{2}}\). Obésité :
\begin{itemize}
\item grade 1 : IMC \(\in\) [30, 34.9]kg/m\(^{\text{2}}\).
\item grade 2 : IMC \(\in\) [35, 39.9]kg/m\(^{\text{2}}\).
\item grade 3 : IMC \(\ge\) 40kg/m\(^{\text{2}}\).
\end{itemize}

Limites : sous-estimé chez asiatiques. Seulement pour [18,65] ans

Phases : prise de poids, constituée, perte, rechutes

Localisation : viscéral (scanner, IRM), sous-cutanée, ectopique (muscle, foie)

Épidémio : +27.5\% 1980-2013 (monde). France : de plus en plus jeune, \inc chez >
65 ans

Étiologie :
\begin{itemize}
\item génétique: envisager si précoce (naissance +24 mois), troubles du
comportement alimentaire
\item obésités communes liées à des facteurs environementaux (majorité) : surtout
déséquilibre apport caloriques- dépense
\begin{itemize}
\item antipsychotiques, glucocorticoïdes, antidépresseurs,
antiépileptiques, antidiabétiques
\item arrêt du tabac, privation de sommeil (?), hypothalamique (rare)
\end{itemize}
\end{itemize}

Complications : \inc RR mortalité, métabolique, CV, respi, ostéoarticulaire,
digestive, rénale, gynéco, cutanée, néoplasiques, psychosociale

\subsubsection{Clinique}
\label{sec:org0c9a6c7}
\begin{itemize}
\item Interrogatoire : 
\begin{itemize}
\item ATCD familiaux d'obésité, poids naissance, âge surpoids,
poids max et min, circonstances déclenchantes, tentatives antérieures, phases
\item Comportement alimentaire (carnet), évaluation dépense énergétique, pyscho-comportementale
\item Complications (SAS)
\end{itemize}
\item Examen : poids, taille, PA, tour de taille\footnote{Obésité abdominale : > 88cm \female, > 102cm \male}, obésité secondaire
\item Complémentaires : glycémie à jeune, lipides, hépatique, uricémie, ECG repos
\end{itemize}

\subsubsection{Traitement}
\label{sec:org0b89533}
\begin{itemize}
\item Diététique, activité physique (\(\forall\) IMC)
\item Psychologique
\item Médicaments (IMC \(\ge\) 30 ou (\(\ge\) 27 et comorbidités)) : orlistat
\item Chir bariatrique :  \{anneau gastrique ajustable, sleeve gastrectomie\},
\{court-circuit gastrique, dérivation biliopancréatique\} : < 65 ans. Prise en
charge 6 mois avant et post-op à vie (carences vitaminiques)\footnote{CI : troubles cognitifs sévères, troubles sévères non stabilisés du
comportement alimentaire, dépendances à l'alcool / substances psychoactives, pas
de PEC médicale, pronostic vital mis en jeu, CI à l'anesthésie générale,
incapacité à faire un suivi médical prolongé}
\end{itemize}
\subsection{Enfant/ado}
\label{sec:org5316694}
\danger{} évolutivité. Surpoids : IMC > 25. Obésité 
\begin{itemize}
\item grade 1 : > 30kg/m\(^{\text{2}}\)
\item grade 2 : > 35kg/m\(^{\text{2}}\)
\item grade 3: > 40kg/m\(^{\text{2}}\)
\end{itemize}

Épidémio : stabilisation mais obésités sévères \texttimes{}4

\subsubsection{Étiologies}
\label{sec:org3358ba5}
\begin{itemize}
\item génétiques : mutation sur récepteur de la mélanocortine type 4 = 2.5-5\%
\item communes (majorité) : facteurs environnementaux et prédisposition génétique
\begin{itemize}
\item repond d'adiposité à 6 ans. Risque d'obésité \(\propto\) précocité du rebond
\item tour de taille/taille > 0.62 = forte valeur prédictive
\item FR : surpoids parent, poids excessif/tabac pendant grossesse, anomalie de
croissance foetale, \inc\inc poids à naissance + 2ans, difficulté
socio-éoc, manque d'activité physique, troubles sommeil, psychopatho
\end{itemize}
\item secondaires (rare) : ralentissement de la vitesse de croissance naturelle
\end{itemize}

\subsubsection{Complications}
\label{sec:org4c0aa19}
\begin{itemize}
\item HTA : > 97e percentile + 10mmHg
\item Insulinorésistance avec glycémie normale fréquente
\item \inc TG et \dec HDL
\item Stéatose hépatique non alcoolique
\item Rachialgies, gonalgies, troubles statique vertébrales. Penser à l'épiphysiolyse
de la tête fémorale : garçons [10,15] ans avec douleur mécanique de hanche
\thus radio de profil
\item Psychologique
\end{itemize}

\subsubsection{Clinique}
\label{sec:org181761d}
Interrogatoire : 
\begin{itemize}
\item ATCD familaux,
\item personnels : poids, taille naissance, âge d'appartition, changements environnementaux, tentatives antérieures, troubles des règles
\item comportement alimentaire (difficile)
\end{itemize}
Examen clinique :
\begin{itemize}
\item poids, taille, PA, tour de taille, pli-cutané (masse grasse < 20\% après 5 ans), courbes de
croissance (ralentissement = pathologique !), dermato (acanthosis nigricans =
insulinorésistance, vergétures= hypercorticisme, intertrigo, mycose)
\end{itemize}
Pas d'examens complémentaires !

\subsubsection{Traitement}
\label{sec:org6f29b11}
Prévention surtout. Modifier style de vie (efficacité faible). Chir possible
avec équipes spécialisées
\section{252 : Diabète gestationnel}
\label{sec:orgd87a74f}
Physio chez femme enceinte selon moitié:
\begin{itemize}
\item non diabétique : (\inc insulinéme, insulinosensibilité) puis (insulinorésistance
\thus hyperinsulinisme ou diabète gestationnel)
\item à risque de diabète : (hypoglycémie, cétose) puis (insulinosécrétion
postprandiale insuffisante)
\end{itemize}

\subsection{PEC du diabète pré-gestationnel}
\label{sec:org02b0895}
Grossesse à risque mais fécondité normale (sauf si sd ovaires polykystiques).

\danger{} Normalisation glycémie préconception \(\rightarrow\) accouchement
\begin{itemize}
\item HbA1c \(\le\) 6.5\%
\item glycémie à jeun \(\in\) [0.6, 0.9]g/L
\item glycémie repas + 1h < 1.40g/L et +2h 1.20g/L
\end{itemize}

\subsubsection{Risque foetus}
\label{sec:orge7056f9}
\begin{itemize}
\item Fausses couches spontanées \texttimes{}2, \(\propto\) hémoglobine glycquée
\item Malformation congénitales \texttimes{}2, constituée pendant 8 premières semaines :
cardiaque, neuro, rénale \thus \inc fausses couches spontanées, mortalité
foeatale/néonatale, malfomations
\item 2e trimestre : macrosomie, hypoxie tissulaire, retard maturation pulmonaire,
hypertrophie cardiaque septale
\item 3e trimestre : mort foetale
\item Accouchement : \inc prématurés, césariennes. Danger : trauma foetal,
hypoglycémie sévère, hypocalcémie, hyperbilirubinémie/polyglobulie, détresse
respi transitoire, maladie des membranes hyalines
\item Long terme : surpoids/obésité et diabète 2
\end{itemize}
\subsubsection{Risque mère}
\label{sec:org3b4a2f0}
\begin{itemize}
\item HTA (30\%) : si > 20 SAc, risque de toxémie gradivique. \texttimes{}5 si
diabète 1. Risque vital \skull
\item Rétinopathie : ttt préalable si rétinopathie proliférative. CI : rétinopathie
proliférative floride non traitée
\item Néphropathie : 
\begin{itemize}
\item FR = \{HTA, déséquilibre glycémique, rétinopathie évoluée dès
départ, diabète ancien, insuf rénale, hydramnios, correction trop rapide d'une
hyperglycémie chronique\}.
\item Insuf rénale \thus hypotrophie foetale, prééclampsie. Si IR préexistante : 50\%
mortalité foeatale \textbf{in utero}
\item dépistage : créat plasmatique, microalbuminurie, protéinurie
\item IEC contre-indiqués
\end{itemize}
\item Coronaropathie : exceptionnelle mais gravissime. Dépister si diabète ancien et
complications microvasculaire (ECG, effort)
\item Infection urinaire \inc, risque pyélonéphrite, décompensation diabétique
\item Diabète 1 : \inc risque dysfonction thyroïdiennes
\end{itemize}

\subsubsection{PEC}
\label{sec:orgfe3e3eb}
\begin{itemize}
\item Avant grossesse : glycémie \(\in\) [0.7, 1.20] préprandial, \(\in\) [1, 1.4]
postprandial et HbA1c < 7\%
\begin{itemize}
\item diabète 1 : \inc insuline
\item diabète 2 : insuline si régime ne suffit pas/arrêt ttt oral
\end{itemize}
\item Pendant    
\begin{itemize}
\item équilibre glycémique++ (6 glycémies capillaires/jour)
\begin{itemize}
\item \danger variations physiologiques : insuline \dec puis \inc puis \dec\dec
\item cétonémie/cétonurie si glycémie > 2g/L
\end{itemize}
\item \(\ge\) 1600kcal/j 2eme et 3eme tri
\item surveiller poids, PA, créat plasmatique, microalbuminurie, protéinurie, FO,
BU, protéinurie
\item surveillance obstétricale : dater++ (12-14SA), malformations (22-24), placenta et liquide
amniotique (32-34SA), cardiomyopathie hypertrophique (32-34SA), bien-être
foetal
\item pas de bêtamimétique si prématuré
\end{itemize}
\end{itemize}
\subsubsection{(Post)partum}
\label{sec:org6bd13fb}
Accouchement programmé souvent, facilité si rétinopathie sévère, insuline
  SC/IV et glucosé avec surveillance horaire

Puis : insuline selon besoin pré-grossesse (D1) ou arrêt (D2)

\subsection{Diabète gestationnel}
\label{sec:org29b17e7}
Si lié à la grossesse, apparait en 2eme partie. Risque : pré-éclampsie,
césarienne (\(\propto\) hyperglycémie matenrelle). FR : surpoids 

Même complications liées à l'hyperinsulinisme que pré-gestationnel

\subsubsection{Dépistage}
\label{sec:orgf2eb18b}
Si FR seulement : 
\begin{itemize}
\item \(\ge\) 35ans
\item IMC \(\ge\) 25kg/m\(^{\text{2}}\)
\item ATCD : diabète gestationnel, macrosomie, diabète chez parents 1er degré
\end{itemize}
Diagnostic :
\begin{itemize}
\item début de grossesse si glycémie jeun \(\ge\) 0.92g/L \thus PEC immédiate
\item sinon à 24-28SA et (glycémie jeun < 0.92g/L ou non faite) : hyperglycémie
provoquée oralement
\end{itemize}

\subsubsection{Traitement}
\label{sec:orgd152022}
\begin{itemize}
\item Diététique (30-35kcal/kg [25 si surpoids]), activité physique, antidiabétique
CI \skull, insuline si régime ne suffit pas après 8 jours
\item Surveillance : glycémie (6/jour puis 4/jour), cétonurie (si glycémie > 2g/L),
HTA
\item Objectif : glycémie jeun < 0.95g/L et postprandiale +2h < 1.20g/L
\end{itemize}

Post-partum : arrêt insuline et surveillance glycémie (diabète antérieur
?). Vérifier glycorégulation à 3 mois. Risque de récidive si grossesse
\section{253 : Nutrition chez le sportif}
\label{sec:org49b1f6c}
\subsection{Examen d'aptitude}
\label{sec:org159f684}
Dépister les pathologies induisant un risque vital/fonctionnel grave : mort
subite (1-4/100 000 après 35 ans)
Obligation légale si compétition (licencié ou non)\footnote{Médecin qualifé pour : alpinisme, armes à feu, mécaniques, aériens, sous-marins, de combat
avec HS}

Examen :
\begin{itemize}
\item ATCD sportif, médicaux familiaux (CV, hypercholestérolémie familiale),
conduites à risque, alimentaire, ttt, toxiques
\item Clinique : 
\begin{itemize}
\item poids, taille, IMC, (courbe de croissance)
\item maturation pubertaire
\item ostéoarticulaire, cardiorespiratoire, test dynamique sous-maximal
(Ruffier-Dickson)
\end{itemize}
\item Complémentaire : ECG repos\footnote{Pour 1er certificat puis tous les 3 ans puis tous les 5
ans jusque 35 ans}, CV
\end{itemize}

\subsection{Bénéfices/inconvénients}
\label{sec:org1d10f92}
Adulte :
\begin{itemize}
\item Bénéfices :
\begin{itemize}
\item maintien santé : \dec mortalité prématurée, \inc qualité de vie, \inc
autonomie (âgé), régule poids
\item prévention : cancers (colon, sein), CV, métabolique, ostéoporose \female
\item ttt : anxiété, cardiomyopathie ischémique, BPCO, obésité, diabète 2, neuro,
rhumatismales, dégénératives
\end{itemize}
\item Surveillance : dépistage d'insuf coronarienne > 40 ans, \danger nutrition et
hydratation si > 3h/semaine
\item Recommandation : 150min/semaine (modéré) ou 75min/semaine (soutenu). Idéal : x2
\end{itemize}

Enfant : 
\begin{itemize}
\item Bénéfices :
\begin{itemize}
\item dev psychosocial : \dec stress, anxiété, \inc intégration sociale, \inc
confiance en soi
\item dev psychomoteur : concentration, coordination, équilibre
\item \inc masse maigre, \inc densité osseuse
\item prévention : sd métabolique, surpoids, CV
\end{itemize}
\item Surveillance : nutrition (éviter retards de croissance/pubertaire), attitude
alimentaires restrictives
\item Recommandation : 60min/jour (modéré-soutenu) et renforcement musculaire,
osseux 3x/semaine
\end{itemize}

\subsection{Besoins nutritionnels}
\label{sec:org4799504}

\begin{center}
\begin{tabular}{llll}
\toprule
Intensité & durée & Energie & Limitation\\
\midrule
Très intense & secondes & ATP, P-Cr & \\
Intense & minutes & Glycogène musculaire & Lactate\\
Faible-élevée & prolongée & glycogène musculaire/lipides & VO\(_{\text{2}}\) max\\
\bottomrule
\end{tabular}
\end{center}

Macronutriments :
\begin{itemize}
\item Glucides : détermine l'épuisement si endurance \thus index glycémique faible à
distance, IG élevé juste avant. Pendant : maintenir glycémie. Après :
reconstituer les stocks de glycogène
\item Lipides à limiter si intensité élevé/compétition
\item Protides : endurance 1.2-1.4g/kg/j, force : 1.3-1.5g/kg/j si maintien masse, sinon jusque 2.5g-kg/j
\end{itemize}

Hydrosodé : avant = 500ml en 2h (prévention). Pendant : NaCl si \(\ge\) 1h selon
intensité (jusque 1.5L/h). Après : 150\% perte pondérale.

Minéraux, vitamines:
\begin{itemize}
\item attention situation à risque : déficit en fer, contrainte de poids,
alimentation glucidiques mais faible densité nutritionnelle, exclusion de
groupes d'aliments
\item endurance : vit B énergétiques\footnote{Thiamine, riboflavine, niacine, B6} , vit. "antioxydantes"\footnote{Vit C, E, \(\beta\)\{xcarotène}
\item force : \inc vit B6, \inc "antioxydantes"
\end{itemize}

\subsubsection{Enfant}
\label{sec:orgda1e961}
Apport insuffisants \thus retard croissance staturo-pondéral ou pubertaire,
\dec masse musculaire, déminéralisation osseuse, déficit immunitaire.

Surveiller calcium, vit D, fer.

\section{265 : Hypocalcémie, dyskaliémie, hyponatrémie}
\label{sec:org95ee1f6}

\subsection{Hypocalcémie (hypoCa)}
\label{sec:orgc6de158}
Éliminer fausses hypoCa dues à l'hypoalbuminémie\footnote{Une partie du calcium est lié à l'albumine}.
Calcémie = équilibre absorption intenstinale, résorption osseuse, excrétion
rénale. Régulé par PTH, calcitriol

Clinique : 
\begin{itemize}
\item hyperexcitabilité neuromusc : paresthésie main, pieds, péribuccales
(spontanées/effort), signe de Trousseau ("main d'accoucheur"), signe de
Chvosteck (peu spécifique), crises de tétanie (paresthésie, fasciculation
pouvant entraîner arrêt respi)
\item chronique : sd de Fahr\footnote{Cataracte sous-capsulaire, calcification des noyaux gris centraux} \thus signes extrapyramidaux, crises comitiales
\item \inc QTc \thus troubles du rythmes
\item dans l'enfance : musc, neuro, cardiaques
\end{itemize}

\subsubsection{Principales causes}
\label{sec:org819e265}
\begin{itemize}
\item Hypoparathyroïdes : anamnèse et \{hypoCa, PTH \dec, phosphatémie
normale/haute\}.
\begin{itemize}
\item post-chir++ : parathyroïdectomie totale
\item congénitale : sd Di George++\footnote{Hypoplasie des parathyroïdes et du thymus, dysmorphie faciale, anomalies cardiaques}
\end{itemize}
\item Pseudoparathyroïdies : génétiques : résistance à la PTH \thus PTH
\inc. Chondrodysplasie possible
\item Anomalie vitamine D
\begin{itemize}
\item carence vit D = 1ere cause hypoCa chez nourrisson \thus rachitisme
carentiel. Chez l'adulte, seulement si déficit prolongé et profond
\item malabsorption digestive, insuf rénale chronique, cirrhose
\end{itemize}
\end{itemize}

\subsubsection{TTt}
\label{sec:org851b0ed}
\begin{itemize}
\item Aigüe = urgence \skull \thus calcium IV lente (2-3x10ml). Suspension des ttt qui \inc
QTc, réduction digoxine
\item Chronique : vit D (ou dérivés actifs) et calcium per os
\end{itemize}

\subsection{Hyper-/hypo-kaliémie,}
\label{sec:org7560470}
Retentissement cardiaque \thus vital \skull

\subsubsection{HyperK}
\label{sec:orgcf364f1}
Principales causes
\begin{itemize}
\item Acidose (sort K+ de la cellule) et insulinopénie (réduit entrée K+) : ttt par
insuline à risque d'hypoK \danger \thus apport K+ dès normokaliémie
\item Hypoaldostéronisme
\begin{itemize}
\item insuf surrénale périphérique
\item secondaire : chez > 65 ans, diabétiques. Risque = aggravation si IEC ou ARA II
\end{itemize}
\item Pseudo-hypoaldostéronisme : résistance à l'aldostérone (génétique)
\end{itemize}

\subsubsection{HypoK}
\label{sec:org02883a1}
\begin{itemize}
\item Dénutrition sévère : anorexique, post-chir bariatrique sans suivi
\item Insulinothérapie : si cétoacidose et troubles digestifs majeurs \thus
insulinothérapie seulement après normokaliémie, sinon arrêt cardiocirculatoire \skull
\item \inc activité \(\beta\)adrénergique
\item Paralysie périodique famililiale : exceptionnelle, paralysie brutale
transitoire des 4 membres
\item Hyperaldostéronisme ou hypercorticisme : y penser si HTA (non constante) et hypoK avec
kaliurèse \inc
\item Polyurie : hyperglycémie \inc
\item Hypomagnésémie : si Mg \dec, malabsorption, pertes digestives causées par
IPP. Sinon : pertes urinaires acquises/génétique
\item Bloc 11-\(\beta\)hydroxystéroïde déshydrogénase : tableau similaire à
hyperaldostéronisme primaire mais avec aldostérone \dec. Si HTA et hypoK,
vérifier réglisse et pastis (glycyrrhizine)
\end{itemize}

\subsection{Hyponatrémie endocrinienne}
\label{sec:org477c763}

HypoNa = anomalie électrolytique la plus commune chez hospitalisés

Osmolarité (mosm/L) : 2\texttimes{}([Na+] + [K+]) + glycémie + urée

\begin{figure}[htpb]
  \centering
  \resizebox{0.9\linewidth}{!}{
    \tikz \graph [
  % Labels at the middle 
      edge quotes mid,
  % Needed for multi-lines
      nodes={align=center},
      sibling distance=3cm,
      level distance=1.5cm,
      edges={nodes={fill=white}}, 
    layered layout]
    {
      Osmolalité -> {
        Augmentée -> "Hyperglycémie";
        Normale -> "HyperTG\\Hyperprotidémie";
        Diminuée -> Volémie -> {
	  "Augmentée\\(hyperhydrat. extracell)" -> "Insuf cardiaque\\Cirrhose\\Sd néphrotique"
	  -> "Sérum salé\\isotonique";
	  "Normal\\(hyperhydrat intracell)" -> "Hypothyroïdie\\Insuf corticotrope\\SIADH"
          -> "Sérum salé\\hypertonique";
	  "Diminué\\(déshydrat extracell)" -> "Perte digestives\\rénales, cérébrales\\Insuf corticosurrénales aigüe"
          -> "Restriction hydrosodée";
	};
      };
    };
  }
  \caption{Démarche diagnostique et ttt devant une hyponatrémie}
\end{figure}

Physiopatho : hormone anti-diurétique (ADH) : répond au stimulus osmotique,
volémique et stress etc. Action vasoconstrictive, corticotrope (stress),
antidiurétique

\subsubsection{SIADH}
\label{sec:org2bbb436}
PA et FC normale, pas de pli cutané, (déshydratation extra-cellulaire) ni
d'oedème (hyperhydratation extra-cellulaire)

DD : cf figure. Si hyponatrémie hypoosmolaire normovolémique :
\begin{itemize}
\item insuf corticotrope : cortisolémie et ACTH
\item insuf surrénale aigüe
\item hyporthyroïdie proto-thyroïdienne : TSH \inc
\item hypopituitarisme antérieure : cortisolémie, TSH, T4L
\end{itemize}

Étioliogies :
\begin{itemize}
\item iatrogènes : neuroleptiques, antidépresseurs, chimio, carbamazépine,
desmopressine
\item quasi toutes affections neuro, notament intervention trans-sphénoïdale
(adénome corticotrope)
\item pulmonaires
\item tumeurs malignes : cancer bronchique à petites cellules++
\item rares : mutation récepteur V2 ADH, marathonien, VIH
\item Intoxication aigüe à l'eau
\end{itemize}

\subsubsection{Traitement}
\label{sec:org23f3fb0}
Urgence si < 115mmol/L ou \{délire, coma, convulsion\} \skull \thus sérum salé
hypertonique jusque Natrémie = 120mmol/L (puis restriction hydrique).

\danger{} < 12mmol/24h sinon tableau d'AVC (myélinolysie centropontine) !! \skull

Thérapeutique :
\begin{itemize}
\item restriction hydrique : mal tolérée
\item déméclocycline : induit diabète inspidie néphrogénique
\item aquarétique (tolvaptan)
\end{itemize}

Indications :
\begin{itemize}
\item symptômes cliniques sévères/récent : sérum salé hypertonique
\item symptômes plus modérés : sérum et tolvaptan
\item sinon restriction hydrique et tolvaptan (ou déméclocycline)
\end{itemize}
\section{266 : Hypercalcémie}
\label{sec:org4abc6d2}
Diagnostic = double dosage calcémie. Étiologie selon parathormone (PTH)

Physio : calcémie régulée par PTH et calcitriol
\begin{itemize}
\item PTH: \inc absorption intestinale du calcium et phosphore, \inc résorption
osseuse, \dec réabsorption phoshpore et \inc absorption calcium (rein)
\item PTH régulée par récepteur sensible au calcium (CaSR)
\end{itemize}

Bio : calcémie totale = \{calcium ionisé, calcium lié = \{lié à l'albumine,
complexé aux anions\}\}. Calcium ionisé \(\approx\) 50\% calcium total\footnote{Sauf si acidose, hyperprotidémie, \inc phosphore/sulfate sériques}

Clinique : asthénie, \{polyuro-polydipsie, lithiases rénales\}, \{anorexie,
constipation, nausées\}, \{apathie, somnolence, confusion, psychose, coma\}, \{HTA,
\dec QT\}

\danger hyperglycémie maligne = urgence \skull{} avec déshydratation, \{confusion,
coma, insuf rénale\} et risques de troubles du rythme cardiaque, bradycardie avec asystolie

\subsection{Étiologies}
\label{sec:org42dac98}
\subsubsection{Hypercalcémie PTH dépendante (PTH N ou \inc)}
\label{sec:org38d8ccb}
\begin{itemize}
\item \textbf{Hyperparathyroïdie\footnote{La PTH est produite par la parathyroïde\ldots{}} primaire} (55\%) : lésion parathyroïde. 
\begin{itemize}
\item Signes cliniques précédents avec rénaux, osseux (clinique et radio\footnote{Ostéite fibrokystique de von Recklinhausen, exceptionnelle.}) \thus créatinine
plasmatique, scénal rénal non injecté
\item Surtout densité osseuse (tier distal du radius)
\item Bio : hypercalcémie et PTH non adaptée (N ou \inc). 
\begin{itemize}
\item \danger corriger déficit vitamine D avant doser calcémie.
\item \danger DD : sd hypercalcémie-hypocalciurie familiale,
hyperparathormonémie avec ttt au lithium
\item calcémie et phosphorémie n'ont de sens qu'avec une fonction rénale normale
\item calciurie : si augmentée, enlève les DD précédents
\end{itemize}
\item imagerie : bio primaire mais sert si indication opératoire seulement (écho, scinti)
\item étiologie :
\begin{itemize}
\item majorité : sporadique, isolé
\item NEM1\footnote{Néoplasie endocirinienne multiple de type 1} (1\%) : hyperparathyroïdie primaire = 95\%. Recherche tumeurs
endocrines pancréas et duodenum, adénomes hypophysaires
\item NEM2 : cancer médullaire de thyroïde puis phéochromocytome bilat et
hyperparathyroïdie primaire avec atteinte multiglandulaire
\item \danger hyperparathyroïdie primaire chez jeune = suspicion transimission
génétiques
\item hyperparathyroïdie \emph{secondaire} : adaptation à hypocalcémie (chercher chez
insuf rénaux chronqiue)
\item hyperparathyroïdie \emph{tertaire} : insuf rénaux chronique
\end{itemize}
\end{itemize}
\item \textbf{hypercalcémie-hypocalciurie familiale bénigne} : hypercalcémie,
hypophosphorémie, (hypermagnésémie), calciurie \dec\dec{}, PTH inadaptées
(N ou \inc)
\item Lithium
\end{itemize}

\subsubsection{hypercalcémie PTH-indépendante}
\label{sec:orgc480281}
\begin{itemize}
\item \textbf{Hypercalcémie des affections malignes} (30\%) : PTH \dec\dec{}. 
\begin{itemize}
\item Tumeurs : poumon, sein, rein, tractus digestif
\item Production tumorale de PTHrp (mime PTH)
\end{itemize}
\item Autres : 
\begin{itemize}
\item granulomatose : hyperphosphorémie, PTH \dec
\item iatrogènes : vitamine D (hypercalcémie, hyperphosphorémie, PTH passe),
vitamine A (asthénie sévère, douleurs musc et osseuse, alopécie des
sourcils, chéilite fissuraire), diurétiques thiazidique, buveurs de laits
(plutôt fortes doses d'antiacide ou carbonate de calcium)
\end{itemize}
\end{itemize}

\subsection{Traitement}
\label{sec:org26fb781}
Hypoparathyroïdie primitive : guérison par ablation des adénome(s) par chir
conventionnelle ou mini-invasive (faire imagerie avant !)

Sinon, traitement palliatif : bisphosphonates (inhibe résorption osseuse),
calcimimétiques (\dec PTH), 

\danger Hypercalcémie = urgence \skull{} : 
\begin{itemize}
\item sérum phy
\item bisphosphonate en perf lente ou corticothérapide IV (myélome/hémopathie) ou dialyse (maligne)
\end{itemize}
\section{303 : Tumeurs de l'ovaire (hormono-sécrétante)}
\label{sec:org78fc714}
\subsection{Sécrétant des oestrogènes}
\label{sec:orgf786305}
Tumeurs de la granulosa : 
\begin{itemize}
\item malignes, les plus fréquentes des tumeurs des cordons sexuels et du stroma.
\item plutôt femmes [30,50] ans
\item jeune fille : pseudo-puberté précoce. Femme :
aménorrhées/ménométrorragie. Ménopausée : saignement vaginal dû à hyperplasie
endométriale\footnote{Tumeurs souvent > 10cm, kystique, multiloculaire, unlatérale}
\item ttt : ovariectomie unilatérale mais récidives 10-33\%
\end{itemize}

Thécomes : 
\begin{itemize}
\item très rare, surtout péri-/post-ménopause
\item Tumeurs solides, bénignes \thus exérèse = guérison
\end{itemize}
Sd Peutz-Jeghers (très très rare)

\subsection{Sécrétant des androgènes}
\label{sec:orgbbd5609}
Tumeurs à cellules de Sertoli-Leydig 
\begin{itemize}
\item sécrète testostérone
\item rare. Y penser si hirsutisme récent avec signes de virilisation
\item DD : corticosurrénalome (faire scanner surrénales), sd Cushing (faire freinage
minute), block 21-hydroxylase (doser 17-hydroxyprogestérone)
\item femme 30-40ans
\item détecté à l'écho ovarienne vaginale ou IRM pelvienne
\item si < 5 cm, bon pronostic \thus ttt conservateur chez femme jeune
\end{itemize}

Tumeurs à cellules de Leydig
\begin{itemize}
\item cristaux de Reinke (caractéristique)
\item typiquement : virilisantes chez ménopausée
\item petite taille, bénigne \thus ovariectomie bilatérale
\end{itemize}

Tumeurs germinales sécrétantes
\begin{itemize}
\item tumeur ovarienne sécrétant de l'hCG : chez femme jeune, aménorrhée, douleurs
abdo/métrorragie. Tttt : conservateur si jeune, chimio si étendu
\item gonadoblastome : chez sd de Turner avec mosaïque et chromosome Y (risque
7-20\%) \thus gonadectomie préventive
\item autres : sécrétant hCG, T4, sérotonine
\end{itemize}

\section{305 Tumeurs du pancréas (endocrine)}
\label{sec:org9762786}
Rare, concerne pancréas et duodénome. Diagnostic histologique, compléteté par
immunohistochimie

Pronostic péjoratif : > 2 cm, invasion vasculaire, dissémination métastase

\begin{table}[htbp]
\caption{Caractéristiques des tumeurs endocrines duodéno-pancréatiques}
\centering
\begin{tabular}{ll}
\toprule
Sécrétion & Clinique\\
\midrule
Insuline & Hypoglycémie organiques\\
Gastrine & Ulcère oestro-gastro-duodénaux, diarrhées\\
ACTH & Cushing\\
Glucagon & Diabète, érythème migrateur, diarrhée, amaigrissement, thromboses\\
VIP & Diarrhée hydroélectrolytique profuse, hypokaliémie\\
GHRH & Acromégalie\\
\bottomrule
\end{tabular}
\end{table}

Imagerie : scanner spiralé TAP \textpm{} IRM abdo

Formes familiales : NEM1, neurofibromatose 1, von Hippel-Lindau

\section{310 : Tumeurs du testicule (aspects endocriniens)}
\label{sec:orge1d449a}

Prévalence : 9/100 000, ado/adulte jeune

\subsection{Tumeurs stromales}
\label{sec:orgf829d90}
Cellules de Leydig. Unilatérales, bénignes
\begin{itemize}
\item Garçon < 9 ans : pseudo-puberté précoce \thus testostérone plasmatique, écho
testiculaire
\item Adulte : féminisation, infertilité \thus oestradiol \inc, testostérone N ou \dec
\end{itemize}

Cellules de Sertoli : rares (enfant) ou exceptionnelles
(adulte). Féminisation/pseudo-puberté précoce à 50\%. Testostérone/oestradial
\inc, LH et FSH \dec, inhibine B \inc.\footnote{Peut appartenir à : complexe Carney, sd Petuz-Jeghers}

\subsection{Autres}
\label{sec:org8847683}
\begin{itemize}
\item Tumeurs germinales : fréquentes, écho testiculaire
\begin{itemize}
\item séminomateuses : fréquentes, pronostic bon
\item non séminomateuses : pronostic réservé
\end{itemize}
\item Inclusion surrénaliennes : par excès ACTSH. Marqueur : 17-hydroxyprogestérone
\end{itemize}

\subsection{PEC}
\label{sec:orgba232d4}
Glucocorticoïdes si inclusion surrénaliennes. Sinon chir 1ere intention. Chimio si métastases pulmonaires/ganglionnaires.


\printglossaries

\section{Annexes}
\label{sec:org9e029f8}
\subsection{Hormones}
\label{sec:orgae7ef15}


\begin{figure}[htpb]
  \centering
  \resizebox{!}{5cm}{
    \tikz \graph [decision]
    {
      ht/hypothalamus[organ] -> ["CRH"] hh/hypophyse[organ] -> ["ACTH"] cs/corticosurrénale[organ];
      cs -> cort/cortisol;
      cort ->[bend left=60, "-"] ht;
      cs -- ["+"] hh;
      hh --["+"] ht;
    };
  }
\resizebox{0.4\linewidth}{!}{
    \tikz \graph [decision]
    {
      ht/hypothalamus[organ] -> ["GnRH"] hh/hypophyse[organ] -> "FSH, LH" -> {
        testicules[organ] -> test/testosterone;
        ovaires[organ] -> est/estrogène;
      };
      test -> [bend left=70, "-"] ht;
      test -> [bend left=60, "-"] hh;
      est -> [bend right=70, "-"] ht;
    };
}
\resizebox{!}{5cm}{
    \tikz \graph [decision, layer distance=1.5cm]
    {
      ht/hypothalamus[organ] -> ["TRH"] hh/antéhypophyse[organ]
      ->["TSH"] th/thyroide[organ] -> "T4, T3";
      th -> [bend left=60, "-"] hh;
      th -> [bend left=70, "-"] ht;
    };
}
\resizebox{!}{5cm}{
    \tikz \graph [decision, layer distance=1.5cm]
    {
      ht/hypothalamus[organ] -> ["GHRH"] hh/hypophyse[organ] -> Foie[organ] -> "IGF-1" -> {
        os;
	muscle;
	graisse..;      
      }
    };
}
\end{figure}

\begin{table}[htbp]
\caption{Hormones produite par les surrénales (du moins au plus profond)}
\centering
\begin{tabular}{ll}
\toprule
Zone de la surrénale & Hormones\\
\midrule
corticale (glomerusa) & minéralocorticoïdes (aldostérone)\\
corticale (fasciculata) & glucocorticoïdes (cortisol)\\
corticale (reticularis) & androgènes\\
médullaire & épinephrine, norepinéphrine\\
\bottomrule
\end{tabular}
\end{table}

\subsection{Syndromes génétiques}
\label{sec:orga811095}
\begin{center}
\begin{tabular}{llll}
\toprule
 & Klinefelter & Turner & Kallmann\\
\midrule
Sexe & \male & \female & \male{}, \female\\
Frequence & 1/500 & 1/2000 \female & 1/30 000 \male{}, 1/250 000 \female{}\\
Caryotype & 46 XX,Y & genes manquants & \\
 &  & sur bras court d'un chr. X & \\
Caractéristiques & Hypogonadisme, infertilité & Petite taille & Anosmie\\
 & gynécomastie & Aménorrhée & \male : Micropénis, cryptorchidie\\
 & trouble apprentissage, comm. & Pas de seins ? & Pas de dev. sexuel secondaire\\
 &  &  & \\
\bottomrule
\end{tabular}
\end{center}


\printglossaries
\end{document}