% Created 2018-10-16 Tue 20:04
% Intended LaTeX compiler: lualatex
\documentclass[11pt]{article}
\usepackage[hidelinks]{hyperref}
\usepackage{booktabs}
\documentclass{article}
\usepackage[hidelinks]{hyperref}
\usepackage{longtable}
\usepackage{booktabs}
%\usepackage[draft]{graphicx}
\usepackage{graphicx}
\usepackage{fancyhdr}
% French
\usepackage[T1]{fontenc}
\usepackage[francais]{babel}
\usepackage{caption}
\usepackage[nointegrals]{wasysym} % Male-female symbol
% Smaller marign
\usepackage[margin=2.5cm]{geometry}
\usepackage{latexsym}
\usepackage{subcaption}
%-------------------------------------------------------------------------------
% For graphs
\usepackage{tikz}
\usepackage{tikzscale}
\usetikzlibrary{graphs}
\usetikzlibrary{graphdrawing}
\usetikzlibrary{arrows,positioning,decorations.pathreplacing}
\usetikzlibrary{calc}
\usegdlibrary{trees, layered}
\usetikzlibrary{quotes}
%-------------------------------------------------------------------------------
% No spacing in itemize
\usepackage{enumitem}
\setlist{nolistsep}
% tightlist from pandoc
\providecommand{\tightlist}{%
  \setlength{\itemsep}{0pt}\setlength{\parskip}{0pt}}
 % Danger symbol (need fourier package)
\newcommand*{\TakeFourierOrnament}[1]{{%
\fontencoding{U}\fontfamily{futs}\selectfont\char#1}}
\newcommand*{\danger}{\TakeFourierOrnament{66}}
% Skull (need Symbola font)
\usepackage{amsmath,fontspec,newunicodechar}
\newfontface{\skullfont}{Symbola}[Scale=MatchUppercase]
\NewDocumentCommand{\skull}{}{%
  \text{\skullfont\symbol{"1F571}}%
}
% Hospital sign
\usepackage{fontspec} % For fontawesome
\usepackage{fontawesome}
% Itemize in tabular
\newcommand{\tabitem}{~~\llap{\textbullet}~~}
% No numbering
\setcounter{secnumdepth}{0}
% Set header
\pagestyle{fancy}
\fancyhf{}
\fancyhead[L]{\leftmark}
\fancyhead[R]{\thepage}
%\renewcommand{\headrulewidth}{0.6pt}
% Custom header : no uper case
\renewcommand{\sectionmark}[1]{%
  \markboth{\textit{#1}}{}}

\author{Alexis Praga}
\date{\today}
\title{Fiches d'endocrinologie}
\hypersetup{
 pdfauthor={Alexis Praga},
 pdftitle={Fiches d'endocrinologie},
 pdfkeywords={},
 pdfsubject={},
 pdfcreator={Emacs 26.1 (Org mode 9.1.9)}, 
 pdflang={English}}
\begin{document}

\maketitle
\tableofcontents


\section{35 : Contraception}
\label{sec:orgacfdc67}
\subsection{Contraception hormonale}
\label{sec:org771a9e1}
Oestroprogestatifs
\begin{itemize}
\item Contient \{oestrogène, progestatif (gen 1,2 ou 3), autres progestatif\}
\item Administration orale, transdermique ou vaginale
\item Action : \{pas d'ovulation, endomètre peu à apte à la nidation, glaire
cervicale imperméable aux spermatozoïdes\}
\end{itemize}
Progestatifs seuls 
\begin{itemize}
\item Microprogestatifs : action sur glaire cervicale, endomètre
\item Macroprogestatifs : si CI oestroprogestatifs
\item Administration : orale, injection, implant, intra-utérin (stérilet)
\end{itemize}
\subsection{Pratique}
\label{sec:org984b5d4}
Oestroprogestatifs : 
\begin{itemize}
\item à commencer le premier jour des règles pendant 21j puis 7 j
\end{itemize}
d'arrêt.
\begin{itemize}
\item au même moment
\item Si oubli < 12h, ASAP sinon contraception mécanique \(\ge 7\) jours
\end{itemize}
Microprogestatifs : toujours à la même heure. Si oubli < 3h, ASA, sinon
contraception mécanique \(\ge 7\) jours
Macroprogestatifs : ccommencer le 5eme jour du cycle

\subsection{Contre-indications}
\label{sec:org1024290}
Oestroprogestatifs : absolues =
\begin{itemize}
\item thromboemboliques veineux/artériels, prédisposition thromboses
\item lupus évolutif, connectivites, porphyries
\item vasc, cardiaque, cérébrales, oculaires
\item valvulopathie, troubles rythmes thrombogènes
\item HTA non contrôlée
\item diabète et micro/macroangiopathie
\item tumeur hormono-dépendantes (sein, utérus\ldots{})
\item hépatiques sévères
\item hémorragies génitales non diagnostiquées
\item (tumeurs hypophysaires)
\end{itemize}
Macro/microprogestatifs : cancers \{sein, endomètre\}, insuf hépatique, accident
TEV récents

\subsection{Recommandation}
\label{sec:org9de7633}
Sans CI, oestroprogestatif minidosé et progestatif 2eme génération monophasique
(Minidril)

\subsection{Efficacité}
\label{sec:orgf008781}
Indice de Pearl\footnote{\(\frac{N}{N_e/10}\times 100\) avec N = nb grossesses
accidentelles, \(N_e\) nombres de mois d'exposition} < 0.07\% pour oestroprogestatif
(< 2\% pour les microprogestatifs)

Attention : certains inducteurs enzymatiques réduisens l'efficacité (ou
millepertuis).

Ado : sous- ou mal utilisée

\subsection{Tolérance}
\label{sec:org7e20528}
Oestroprogestatifs :
\begin{itemize}
\item bien tolérée, pas de perte de poids
\item surveiller métabolisme
\item active coagulation mais \inc fibrinolyse. Légère augmentation du risque
d'accident TEV
\item vasc : faible \inc PA
\item cancer : ovaire = risque -50\%, idem pour l'endomètre, faible \inc pour sein
\end{itemize}
Microprogestatifs : troubles des règles (spotting, aménorrhées), grossesse
extra-utérine

Macroprogestatifs : hypoestrogénie, aménorrhées, spotting

\subsection{Surveillance}
\label{sec:orgaff5342}
Consulter si céphalée, déficit sensitivomoteur, (douleur ou oedème) MI, dyspnée,
douleur thoracique

Examen clinique : 
\begin{itemize}
\item préthérapeutique : gynéco, frottis cervico-vaginal dès 25 ans si
asymptomatique.
\item PA à +3mois puis tous 6 mois
\item hyperoestrogénie (tension mammaire), hypoestrogénie (sécheresse vaginale)
\end{itemize}
Biologie : cholestérol total, triglycérides, glycémie à jeun à +3mois. Si FR, le
faire avant (!) prescription

Gynéco : métroragies, spottings. NB frottis cervico-utérin dès 25 ans (+1 an pui
tous 3 ans) indépendamment contraception

\subsection{Femmes à risques}
\label{sec:orgc9fb2bb}
Diabétique :
\begin{itemize}
\item non hormonale : si diabète 1 > 15 ans ou micro/macroangiopathie \thus locale
(nullipare, peu de rapports) ou intra-utérin (multipare, diabète équilibré)
\item hormonale :pas d'oestropregestatifs si \{tabac, non équilibré, HTA, surpoids,
diabète compliqué\} \thus progestatif
\end{itemize}
Dyslipidémie : oestroprogestatif si < 3g/L cholestérol total, triglycérides <
9g/L

Thrombose veineuse
\begin{itemize}
\item prédisposant : anomalies de l'hémostase (génétique, acquises), ATCD familiaux
\item dépistage : thrombose, multiples fausses couches, ATCD thrombose < 45 ans
\item acétate de chlormadinone, CI oestrogène
\end{itemize}

Autres :
\begin{itemize}
\item HTA : oestroprogestatifs si 0 FR
\item tabac = CI, migraine \(\land\) vascularite = spécialiste
\end{itemize}

\subsection{Urgence}
\label{sec:org180de1b}
\begin{itemize}
\item lévonorgesttrel ASAP < 72h
\item ulipristal acétate ASAP < 120h mais 3x plus cher
\end{itemize}

\section{37 : Stérilité du couple}
\label{sec:orga4fefcd}
Infertile : 0 grossesses après 1 an de rapports non protégés. Stérilité si
définitf.

Stérilité = partagée !!

\subsection{Interrogatoire}
\label{sec:orgb175b36}
\begin{itemize}
\item Couple
\item Femme : âge++ (détérioration après 35 ans), \{grossesses antérieure,
avortements\}, infections/curetages++, ATCD chir/infectieux, douleurs
pelviennes (rapports, règles), conditions de vie, radio/chimio
\item Homme : trouble libido/érection, ATCD cryptorchidie/trauma testiculaire, ATCD
chir pelvienne/scrotale, ATCD médicaux (oreillons++), tabac/anabolisants\ldots{}
\end{itemize}

\subsection{Examen clinique}
\label{sec:orgce7b352}
\begin{itemize}
\item \female : âge++, obésité/maigreur, tour taille et hanche, pilosité, PA,
galactorrhé provoqué, gynéco.
\begin{itemize}
\item Si anovulation (a-/oligo-ménorrhée) : hyperprolactinémie, hyperandrogénie,
troubles comportement alimentaire, bouffées chaleur
\end{itemize}
\item \male : IMC, pilosité, hypoandrisme, cicatrice chir, varicocèle,
gynémocomastie, gynoïde/enuchoïde
\begin{itemize}
\item volume testiculaire++, palpation cordospermatiques
\end{itemize}
\end{itemize}

\subsection{Examens complémentaires}
\label{sec:orge0f67be}
Premiere intention, femme
\begin{itemize}
\item Hormonale++ : oestradiole, LH, FSH, prolactine plasmatique. Puis progestérone
\end{itemize}
plasmatique (si cycle réguliers)
\begin{itemize}
\item Écho ovarienne++
\item Hystérographie++
\end{itemize}
Première intention, homme :
\begin{itemize}
\item spermogramme++ (concentration, mobilité, morphologie. Attention aux
variabilités !
\item hormonale++ si oligo-/azoo-spermie : testostérone, LH, FSH puir SHBG
\end{itemize}
Test poist-coïtal (discuté)

\subsection{Étiologie}
\label{sec:orgaf26bb1}
Femme :
\begin{itemize}
\item anovulation : très fréquent ! Souvent aménorrhées ou irrégularités. Causes :
sd des ovaires polymicrokystiques, hyperprolactinémie, insuf ovarienne
primitive, déficit gonadotrope, psycho-nutritionnel
\item obstacle mécanique :
\begin{itemize}
\item anomalie du col utérin et insuf glaire cervilaie : post-conisation/curetage
\item obstacle, anomalie utérie : manoeuvres post-partum, polypes muqueux\ldots{} \thus
echographie
\item obstacle tubaire : cause majeure++. Souvent salpingite (Chlamydia++)
\end{itemize}
\item Endométriose : rarement si modéré. hypstérosalpingographie pui coelioscopie
\end{itemize}
Homme :
\begin{itemize}
\item azoospermie 
\begin{itemize}
\item sécrétoire : diagnostic = volume testiculaire < 10ml, concentration FSH
faible
\end{itemize}
\thus caryotype, analyse bras long Y, écho testiculaire (élimine
   cancer), déficit gonadotrope (rare)
\begin{itemize}
\item obstructive : volume et concentration ormale, volume séminal \dec \thus
examen clinique
\begin{itemize}
\item cause congénitale : agénésie bilat des canaux déférent++ (soit anomalie
biallélique gène CFTR, soit isolée)
\item acquis : infectieux  (gonocoque, tuberculose, Chlamydia) \thus échographie
\end{itemize}
\item exploration chir testiculaire et des voies excrétrices : si azoospermie
confirmée par plusieurs spermogrammes, bilan génétique
\end{itemize}
\item oligo-asthéno-térato-spermie : \dec nombre, mobilité, \inc hormes anormales
\thus caryotype, brang long Y. Traitement = assistance médicale procréation
\end{itemize}
\section{40 : Aménorrhée}
\label{sec:org0d45a9a}
Déf: absence de cycle menstruel après 16 ans (primaire) ou interruption chez
femme réglée (secondaire). Physiologique : grossesse, lactation, ménopause

Tout arrêt > 1 mois \thus enquête étiologique \danger

Atteinte de l'axe hypothalamo-hypophysio-ovarien ou anomalie tractus utérin

\textbf{Pas de traitement oestrogénique sans enquête étiologique}

\subsection{Conduite}
\label{sec:orgda3dcf3}
\subsubsection{Primaire}
\label{sec:org31576d9}
Forte proba de cause génétique/chromosomique. Chercher carences nutritionnelle

\begin{itemize}
\item Si absence de dév. puberture : doser FSH, LH
\begin{itemize}
\item Si basses, tumeur hypothalamo-hypophysaire, dénutrition ou génétique : \{sd
de Kalmann (anosmie), mutation récepteur GnRH (rare), atteinte
gonadotrophines (exceptionnels), mutation LH\}
\item Si hautes : sd de Turner (caryotique 45, XO),
\end{itemize}
\item Examen gynéco, écho pelvienne
\begin{itemize}
\item Pas d'utérus : sd de Rokitanski, tissu testiculaire dans les canaux
inguinaux (CAIS)
\item ambiguité OGA : dysgénésie gonadique, hyperplasie congénitale surrénales,
anomalies sensibilité/biosynthèse androgènes
\end{itemize}
\end{itemize}
\subsubsection{Secondaire}
\label{sec:orgcf38a19}
Souvent acquises. 

Interrogatoire : médic, maladie endoc/chronique,
gynoc/obstétriques, insuf ovarienne (bouffées de chaleur). Douleurs pelviennes
cycliques : cause utérine

Examen clinique : 
\begin{itemize}
\item poids et taille (carence nutritionnelle
\item hyperandrogénie : sd ovaires polykystiques, déficit 21-hydroxylase, (sd
Cushing)
\item carence oestrogénique : pas de glaire +2 semainesa après saignement \thus
anovulation
\item pas de signe d'appel : enquête nutritionnelle
\end{itemize}

Dosages hormonaux :
\begin{itemize}
\item hCG : grossesse
\item prolactine \inc : médciament ou tumeux hypothalamo-hypophysaire (ou adénome)
\item FSh \inc, estradiol \dec : insuf ovarienne primitive
\item estradiol, LH, FSH basse : hypothalamo-hypophysaire ou nutritionnelle
\item testostérone : tumeur surrénalienne/ovarienne
\item sinon SOPK\footnote{Sd des ovaires polykystiques}
\end{itemize}

\subsection{Causes}
\label{sec:org2225097}

\subsubsection{Déficit gonadotrope organique/fonctionnel}
\label{sec:org8329539}

\begin{enumerate}
\item Hypothalamus, prolactine normales
\label{sec:orgd24db65}
Hypothalamus n'arrive pas à libérer la GnRH au bon rythme. 

Atteintes organiques : tumeur/infiltration \thus IRM
\begin{itemize}
\item macroadénomes hypophysaires, craniopharyngiomes
\item chercher hyperprolactinémie, insuf antéhypophysaire associé
\end{itemize}

Atteintes fonctionnelles : apports nutritionnels insufissants par rapport à l'activité physique intense+++
\item Hypothalamo-hypophysaire dû à une hyperprolactinémie
\label{sec:orgd3170cc}
Cause majeure

Médicaments ou tumeurs \thus pas de traitement dopaminergique sans imagerie \danger
\item Autres
\label{sec:orgd94db3e}
\begin{itemize}
\item Endocrinopathies : sd de Cushing, dysthyroïdes déficits 21-hydroxylase
\item Hypophysaire (rare) : auto-immune (majorité), sd de Sheehan (très rare, nécrose hypophysaire post-partum)
\end{itemize}
\end{enumerate}
\subsubsection{Anovulation non hypothalamique}
\label{sec:orgd80b9f8}
\begin{enumerate}
\item SOPK (majorité)
\label{sec:org082886e}
Pas de pic de LH, ni de progestérone. Oestradiol mais non cycliques

Irrégularité menstruelles, puis aménorrhées avec acné, hirsutisme

Diagnostic :
\begin{itemize}
\item 2 parmi : \{hyperandrogénie clinique\footnote{Séborrhéee, acné, hirsutisme, \inc testostérone}, oligo-/a-novulation, hypertrophie
ovarienne (écho)\}
\item exclure bloc 21-hydroxylase, tumeur de l'ovaire, sd Cushing
\item exclure hyperprolactinémie
\end{itemize}

Diagnostic parfois difficile :
\begin{itemize}
\item sans hyperandrogénie \thus écho
\item \{atteinte partielle axe gonadotrope, macroprolactinémie\} peuvent y ressembler
\end{itemize}

Acné : cherche hyperandrogénie, régularité cycle menstruel \thus éliminer
hyperplasie congénitale des surrénales

2 causes :
\begin{itemize}
\item tumeur ovarienne ou résistance insuline
\begin{itemize}
\item virilisation si tumeur
\item imagerie si testostérone > 1.5ng/mL. Si normale, cherche hypothécose
(obésité morbide androïde, acanthosis nigricans, insulino-résistance)
\end{itemize}
\item pathologie surrénale :
\begin{itemize}
\item sd de Cushing si signes hypercortisolisme \thus cortisol libre urinaier et
freinage minute
\item tumeur surrénale \thus scanner des surrénales
\item déficit enzymatique en 21-hydroxylase (\danger formes tardives qui peuvent
mimer SOPK)
\end{itemize}
\end{itemize}
\end{enumerate}
\subsubsection{Insuf ovarienne primitive}
\label{sec:orge2c9756}
\inc FSH

Causes :
\begin{itemize}
\item chir, chimio, radiothérapie
\item anomalie caryotype (sd Turner)
\item anomalie gènes \emph{FMR1} (sd X fragile)\footnote{Transmission mère-fils. Expansion instable des triplets CGG jusqu'à
l'absence de transcript de FMR1 (Fragile X Mental Retardation 1). \\
\danger Risque d'IOP pour la pré-mutation seulement \thus dépister chez
\female + IOP < 40 ans par PCR et Southern Blot}
\item auto-immune
\end{itemize}
\subsubsection{Anomalie utérine}
\label{sec:orgb789ff0}
\begin{enumerate}
\item Congénitales :
\label{sec:org55cdc8c}
Si dév pubertaire normale et douleurs pelviennes cycliques, imperfortanio
hyménéale/malformation vaginale \thus examen gynéco.\\
Idem sans douleurs \thus agénésie utérus ?

Difficulté : différence agénésie mullérienne isolée (46,XX)- anomalies androgènes
(46,XY) \thus testostérone
\item Secondaires :
\label{sec:org1ada624}
Synéchies utérines (trauma de l'utérus), tuberculose utérine
\end{enumerate}
\section{47 : Puberté}
\label{sec:org3f09c26}
\subsection{Normale}
\label{sec:org451296b}
\textasciitilde{}4 ans, acquisition de la taille définitive, fonction de
reproduction. Classification de Tanner (5 stades)

\female : seins \textasciitilde{}11 ans [8,13]ans, règles \textasciitilde{}13ans [10,15]

\male : \inc volume testiculaire \textasciitilde{}11.5 [9.5,14], \inc taille verge \textasciitilde{}12.5ans.

Accélération de la croissance : 5 -> 8cm (163cm) \female, 5-10cm (175cm)
\subsection{Retards}
\label{sec:orge0e2567}
\male :  volume testiculaire < 4mL ou longeur < 25mm > 14 ans 

\female : pas de seins à 13 ans, pas de règles à 15 ans

Centrale ou périphérique ?
\begin{itemize}
\item centrale : congénital (pas de cassure de croissance, micropénise,
cryptorchidie), acquis (tumeur ?), "fonctionnel" (maladie générale, trouble
comportement alimentaire), isolé
\item périphérique : sd de Turner chez \female, sd Klinefelter \male
\item retard simple (élimination)
\end{itemize}

Clinique : 
\begin{itemize}
\item parents, grossesse, courbe de croissance. Chercher trbles digestifs,
\end{itemize}
polyuro-polydispise, céphalée, anomalies champ visuel
\begin{itemize}
\item pathologie acquise, OGE, testicules, anosmie (Kallmann)
\end{itemize}

Âge osseux : 13 ans \male, 11 ans \female

Biologie : 
\begin{itemize}
\item stéroïdes sexuels, si FSH, LH basses \thus hypothalamo-hypophysaire
\item testostérone chez \male, oestradial/écho chez \female
\end{itemize}

IRM indispensable ! si déficit gonadotrope (tumeur)

Caryotype si FSH élevé. Toujours chez \female de taille < -2DS avec retard
pubertaire/gonadotrophine \inc

\subsubsection{Étiologies}
\label{sec:org318bd40}
Hypogonadotropes
\begin{itemize}
\item congénitaux : isolés, sd de Kallman, autres déficits hypophysaires, sd
polymalformatifs
\item acquis : tumeurs hypophysaires, post-radiothérapie
\end{itemize}
Hypogonadotropes fonctionnels
\begin{itemize}
\item chroniques digestives/cardaques/respi
\item sport intense
\item endocrio
\end{itemize}
Hypergonadotropes
\begin{itemize}
\item congénitaux : sd Turner, sd Klinefelter, autres atteintes primitevs
\item acquis : castration, trauma, oreillons, chimio/radio
\end{itemize}

\subsubsection{Traitement}
\label{sec:org2d34291}
Cause si possible. Sinon doses \inc de testostérone (\male) ou oestrogènes puis
oestroprogestatif (\female)
\subsection{Précoces}
\label{sec:orge8e6e4f}
Avant 8ans \female ou 9.5 ans \male
\subsubsection{Centrales}
\label{sec:org200f29f}
8x plus fréquent chez \female que \male. Chez \female, causes
idiopathiques. Chez \male, causes tumorales à 50$\backslash$%

Clinique : 
\begin{itemize}
\item dév prématuré harmoniaux (pas de règles chez \female)
\item crises de rires (harmatome hypothalamique), tâches cutanées (neurofibromatose
I ou sd McCune-Albright)
\end{itemize}
Biologie :
\begin{itemize}
\item testostérone élevée chez \male mais variabilité d'oestradiol chez \female
\end{itemize}

IRM hypothalamo-hypophysaire indispensable \danger (petite taille
définitive). Écho pelvienne pour \female

Traitement si risque de petite taille adulte : analogues GnRH jusque âge normal
de puberté
\subsubsection{Périphériques}
\label{sec:orge267cee}
Clinique : \inc vitesse de croissance, avance maturation osseusse

Stéroïdes élevées, LH et FSH bas. Écho pelvienne chez fille

Étiologie :
\begin{itemize}
\item tumeurs ovarienne (rares) : écho puis histologies
\item kystes folliculaires : bénins, régression spontanée possible
\item sd McCune-Albright : \{puberté précoce ovarienne, taches cutanées
"café-au-lait", dysplasie fibreuses os\}. \danger tableau pas toujours complet
!\\
\end{itemize}
Oestradiol élevé, gonadotrophines basses, écho = utérus stimulé, kystes
ovariens. Dominance \female
\begin{itemize}
\item médicaments
\item testotoxicose (rare, cellule de Leydig activé et LH basses), adénome leydigien
(très rare)
\item tumeurs à hCG (\male)
\end{itemize}
\subsubsection{Avances dissociées}
\label{sec:orge538a1e}
\begin{itemize}
\item Isolé des seins : beaucoup de filles [3mois, 3 ans]
\item Métrorragies isolées : chercher vulvite, vulvovaginite, prolapsus urétrale,
corps étranger. Éliminter kyste ovarien, sd McCune-Albright par l'absence des
sein
\end{itemize}
\thus écho pelvienne
\begin{itemize}
\item Pilosité pubienne isolée : chercher forme d'hyperplasie congénitale des
surrénales (\inc 17-hydroxyprogestérone, stimulation ACTH), prémature pubarche
(élimination !)
\end{itemize}
\end{document}
