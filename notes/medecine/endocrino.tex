% Created 2018-10-11 Thu 15:11
% Intended LaTeX compiler: lualatex
\documentclass[11pt]{article}
\usepackage[hidelinks]{hyperref}
\usepackage{booktabs}
\documentclass{article}
\usepackage[hidelinks]{hyperref}
\usepackage{longtable}
\usepackage{booktabs}
%\usepackage[draft]{graphicx}
\usepackage{graphicx}
\usepackage{fancyhdr}
% French
\usepackage[T1]{fontenc}
\usepackage[francais]{babel}
\usepackage{caption}
\usepackage[nointegrals]{wasysym} % Male-female symbol
% Smaller marign
\usepackage[margin=2.5cm]{geometry}
\usepackage{latexsym}
\usepackage{subcaption}
%-------------------------------------------------------------------------------
% For graphs
\usepackage{tikz}
\usepackage{tikzscale}
\usetikzlibrary{graphs}
\usetikzlibrary{graphdrawing}
\usetikzlibrary{arrows,positioning,decorations.pathreplacing}
\usetikzlibrary{calc}
\usegdlibrary{trees, layered}
\usetikzlibrary{quotes}
%-------------------------------------------------------------------------------
% No spacing in itemize
\usepackage{enumitem}
\setlist{nolistsep}
% tightlist from pandoc
\providecommand{\tightlist}{%
  \setlength{\itemsep}{0pt}\setlength{\parskip}{0pt}}
 % Danger symbol (need fourier package)
\newcommand*{\TakeFourierOrnament}[1]{{%
\fontencoding{U}\fontfamily{futs}\selectfont\char#1}}
\newcommand*{\danger}{\TakeFourierOrnament{66}}
% Skull (need Symbola font)
\usepackage{amsmath,fontspec,newunicodechar}
\newfontface{\skullfont}{Symbola}[Scale=MatchUppercase]
\NewDocumentCommand{\skull}{}{%
  \text{\skullfont\symbol{"1F571}}%
}
% Hospital sign
\usepackage{fontspec} % For fontawesome
\usepackage{fontawesome}
% Itemize in tabular
\newcommand{\tabitem}{~~\llap{\textbullet}~~}
% No numbering
\setcounter{secnumdepth}{0}
% Set header
\pagestyle{fancy}
\fancyhf{}
\fancyhead[L]{\leftmark}
\fancyhead[R]{\thepage}
%\renewcommand{\headrulewidth}{0.6pt}
% Custom header : no uper case
\renewcommand{\sectionmark}[1]{%
  \markboth{\textit{#1}}{}}

\author{Alexis Praga}
\date{\today}
\title{Fiches d'endocrinologie}
\hypersetup{
 pdfauthor={Alexis Praga},
 pdftitle={Fiches d'endocrinologie},
 pdfkeywords={},
 pdfsubject={},
 pdfcreator={Emacs 26.1 (Org mode 9.1.9)}, 
 pdflang={English}}
\begin{document}

\maketitle
\tableofcontents


\section{35 : Contraception}
\label{sec:orge04aa15}
\subsection{Contraception hormonale}
\label{sec:org431f4e9}
Oestroprogestatifs
\begin{itemize}
\item Contient \{oestrogène, progestatif (gen 1,2 ou 3), autres progestatif\}
\item Administration orale, transdermique ou vaginale
\item Action : \{pas d'ovulation, endomètre peu à apte à la nidation, glaire
cervicale imperméable aux spermatozoïdes\}
\end{itemize}
Progestatifs seuls 
\begin{itemize}
\item Microprogestatifs : action sur glaire cervicale, endomètre
\item Macroprogestatifs : si CI oestroprogestatifs
\item Administration : orale, injection, implant, intra-utérin (stérilet)
\end{itemize}
\subsection{Pratique}
\label{sec:org7a63d02}
Oestroprogestatifs : 
\begin{itemize}
\item à commencer le premier jour des règles pendant 21j puis 7 j
\end{itemize}
d'arrêt.
\begin{itemize}
\item au même moment
\item Si oubli < 12h, ASAP sinon contraception mécanique \(\ge 7\) jours
\end{itemize}
Microprogestatifs : toujours à la même heure. Si oubli < 3h, ASA, sinon
contraception mécanique \(\ge 7\) jours
Macroprogestatifs : ccommencer le 5eme jour du cycle

\subsection{Contre-indications}
\label{sec:orge226fbf}
Oestroprogestatifs : absolues =
\begin{itemize}
\item thromboemboliques veineux/artériels, prédisposition thromboses
\item lupus évolutif, connectivites, porphyries
\item vasc, cardiaque, cérébrales, oculaires
\item valvulopathie, troubles rythmes thrombogènes
\item HTA non contrôlée
\item diabète et micro/macroangiopathie
\item tumeur hormono-dépendantes (sein, utérus\ldots{})
\item hépatiques sévères
\item hémorragies génitales non diagnostiquées
\item (tumeurs hypophysaires)
\end{itemize}
Macro/microprogestatifs : cancers \{sein, endomètre\}, insuf hépatique, accident
TEV récents

\subsection{Recommandation}
\label{sec:org4b7d153}
Sans CI, oestroprogestatif minidosé et progestatif 2eme génération monophasique
(Minidril)

\subsection{Efficacité}
\label{sec:org88d4488}
Indice de Pearl\footnote{\(\frac{N}{N_e/10}\times 100\) avec N = nb grossesses
accidentelles, \(N_e\) nombres de mois d'exposition} < 0.07\% pour oestroprogestatif
(< 2\% pour les microprogestatifs)

Attention : certains inducteurs enzymatiques réduisens l'efficacité (ou
millepertuis).

Ado : sous- ou mal utilisée

\subsection{Tolérance}
\label{sec:orge967db8}
Oestroprogestatifs :
\begin{itemize}
\item bien tolérée, pas de perte de poids
\item surveiller métabolisme
\item active coagulation mais \inc fibrinolyse. Légère augmentation du risque
d'accident TEV
\item vasc : faible \inc PA
\item cancer : ovaire = risque -50\%, idem pour l'endomètre, faible \inc pour sein
\end{itemize}
Microprogestatifs : troubles des règles (spotting, aménorrhées), grossesse
extra-utérine

Macroprogestatifs : hypoestrogénie, aménorrhées, spotting

\subsection{Surveillance}
\label{sec:org1d32f89}
Consulter si céphalée, déficit sensitivomoteur, (douleur ou oedème) MI, dyspnée,
douleur thoracique

Examen clinique : 
\begin{itemize}
\item préthérapeutique : gynéco, frottis cervico-vaginal dès 25 ans si
asymptomatique.
\item PA à +3mois puis tous 6 mois
\item hyperoestrogénie (tension mammaire), hypoestrogénie (sécheresse vaginale)
\end{itemize}
Biologie : cholestérol total, triglycérides, glycémie à jeun à +3mois. Si FR, le
faire avant (!) prescription

Gynéco : métroragies, spottings. NB frottis cervico-utérin dès 25 ans (+1 an pui
tous 3 ans) indépendamment contraception

\subsection{Femmes à risques}
\label{sec:orgb1e1f2d}
Diabétique :
\begin{itemize}
\item non hormonale : si diabète 1 > 15 ans ou micro/macroangiopathie \thus locale
(nullipare, peu de rapports) ou intra-utérin (multipare, diabète équilibré)
\item hormonale :pas d'oestropregestatifs si \{tabac, non équilibré, HTA, surpoids,
diabète compliqué\} \thus progestatif
\end{itemize}
Dyslipidémie : oestroprogestatif si < 3g/L cholestérol total, triglycérides <
9g/L

Thrombose veineuse
\begin{itemize}
\item prédisposant : anomalies de l'hémostase (génétique, acquises), ATCD familiaux
\item dépistage : thrombose, multiples fausses couches, ATCD thrombose < 45 ans
\item acétate de chlormadinone, CI oestrogène
\end{itemize}

Autres :
\begin{itemize}
\item HTA : oestroprogestatifs si 0 FR
\item tabac = CI, migraine \(\land\) vascularite = spécialiste
\end{itemize}

\subsection{Urgence}
\label{sec:orgb19246e}
\begin{itemize}
\item lévonorgesttrel ASAP < 72h
\item ulipristal acétate ASAP < 120h mais 3x plus cher
\end{itemize}
\end{document}
