% Created 2018-11-14 Wed 20:44
% Intended LaTeX compiler: lualatex
\documentclass[11pt]{article}
\usepackage[hidelinks]{hyperref}
\usepackage{booktabs}
\documentclass{article}
\usepackage[hidelinks]{hyperref}
\usepackage{longtable}
\usepackage{booktabs}
%\usepackage[draft]{graphicx}
\usepackage{graphicx}
\usepackage{fancyhdr}
% French
\usepackage[T1]{fontenc}
\usepackage[francais]{babel}
\usepackage{caption}
\usepackage[nointegrals]{wasysym} % Male-female symbol
% Smaller marign
\usepackage[margin=2.5cm]{geometry}
\usepackage{latexsym}
\usepackage{subcaption}
%-------------------------------------------------------------------------------
% For graphs
\usepackage{tikz}
\usepackage{tikzscale}
\usetikzlibrary{graphs}
\usetikzlibrary{graphdrawing}
\usetikzlibrary{arrows,positioning,decorations.pathreplacing}
\usetikzlibrary{calc}
\usegdlibrary{trees, layered}
\usetikzlibrary{quotes}
%-------------------------------------------------------------------------------
% No spacing in itemize
\usepackage{enumitem}
\setlist{nolistsep}
% tightlist from pandoc
\providecommand{\tightlist}{%
  \setlength{\itemsep}{0pt}\setlength{\parskip}{0pt}}
 % Danger symbol (need fourier package)
\newcommand*{\TakeFourierOrnament}[1]{{%
\fontencoding{U}\fontfamily{futs}\selectfont\char#1}}
\newcommand*{\danger}{\TakeFourierOrnament{66}}
% Skull (need Symbola font)
\usepackage{amsmath,fontspec,newunicodechar}
\newfontface{\skullfont}{Symbola}[Scale=MatchUppercase]
\NewDocumentCommand{\skull}{}{%
  \text{\skullfont\symbol{"1F571}}%
}
% Hospital sign
\usepackage{fontspec} % For fontawesome
\usepackage{fontawesome}
% Itemize in tabular
\newcommand{\tabitem}{~~\llap{\textbullet}~~}
% No numbering
\setcounter{secnumdepth}{0}
% Set header
\pagestyle{fancy}
\fancyhf{}
\fancyhead[L]{\leftmark}
\fancyhead[R]{\thepage}
%\renewcommand{\headrulewidth}{0.6pt}
% Custom header : no uper case
\renewcommand{\sectionmark}[1]{%
  \markboth{\textit{#1}}{}}

\usepackage[linesnumbered,ruled,vlined]{algorithm2e}
\author{Alexis Praga}
\date{\today}
\title{Douleur}
\hypersetup{
 pdfauthor={Alexis Praga},
 pdftitle={Douleur},
 pdfkeywords={},
 pdfsubject={},
 pdfcreator={Emacs 26.1 (Org mode 9.1.9)}, 
 pdflang={English}}
\begin{document}

\maketitle
\tableofcontents

\input{bacteries-header}

\section{132 : Antalgiques}
\label{sec:org4c64834}
\subsection{Palier 1 et AINS}
\label{sec:org6e32bba}
\subsubsection{Paracétamol}
\label{sec:org0ff80ff}
1ere intention, surtout douleur chronique

Prescription : 3-4g/24h (adulte) avec délai 4-6h. Délai 1h, dure 4-6h. Enfant : 60mg/kg/24g
espacée de 6h 

Seul pour douleurs faibles/modérées, avec opioïdes sinon

Tolérance :
\begin{itemize}
\item si surdosage : 
\begin{itemize}
\item hépatotoxique \thus urgence \skull \thus N-acétylcysténine
\item insuf rénale aigüe et nécrose tubulaire
\item thrombocytopénie
\end{itemize}
\item sinon bonne tolérance. Surveiller HTA, hypersensibilité
\end{itemize}

\subsubsection{AINS}
\label{sec:orgf4b757c}
COX-1 : protège muqueuse gastro-duodénale et pro-agrégant. COX-2 : inflammation,
effet anti-agrégant

Antalgique à faible dose. Pour douleurs \{ostéo-articulaire, trauma, post-op,
néplasiques\}, coliques néphrétiques, dysménorrhées essentielles, migraines.

OS, injectable (limiter à 48h !) ou pommade/gel

Effets indésirables :
\begin{itemize}
\item digestif \footnote{NB : coxibs \dec probabilité de développer un ulcère simple/compliqué
mais retard circatrisation d'un ulcère gastrique \ldots{}}: 
\begin{itemize}
\item mineur : nausées, vomissements, gastralgies, douleurs abdo
\item grave : ulcère, perforation digestive, hémorragie
\item autres : ulcère oesophagien\footnote{Donc toujours prendre avec de l'eau, debout et sans être à jeun !}, aggravation diverticulose,
anorectites/brûlures anales
\item FR : \{ > 65 ans, ATCD ulcère/hémorragie dig, infection \bact{helicobacter},
maladie générale sévère\}, \{+ AINS, + corticoïdes, +anticoag/agrégants,
+aspirine\}
\thus FR \(\ge\) 3 ou aspirine : éviter !
\end{itemize}
\item rein :
\begin{itemize}
\item \danger risque hypovolémie, néphropathie, âgé, association (diurétique, IEC,
ARAII)
\item insuf rénale fonctionnel (créat !), rétention hydrosodée, hyperkaliémie,
néphropathie intestitielle, (nécrose papillaire)
\end{itemize}
\item cutanés/muqueus fréquents : bénin, urticaire (sd Lyelle/Stevens-johnson)
\item allergiques (rhinite, conjonctivite, oedème de Quincke, asthme) ou respiratoire
\item si anémie, chercher saignement digestif
\item hépatite (souvent silencieuse)
\item sd confusionnel (âgé)
\item CV : rétention hydrosodée, \inc risque thrombotique artériel
\end{itemize}

Éviter interactions : AINSE aspirine, \{anticoag, ticlopidine\}, \{diurétiques,
IEC\}, lithium, corticoïdes, méthotrexate

\fbox{Surtout douleurs aigües}

\subsection{Palier II}
\label{sec:orgc31422c}
60mg codéine = 50mg tramadol = 10mg morphine

\fbox{Douleurs modéres/intenses d'emblée ou ne répondant pas au palier I}
Douleurs aigües (courte) ou chronique (courte/long)

Codéine : 
\begin{itemize}
\item agoniste opioïde naturel \thus métabolite * morphine
\item seul (sirop, dihydrocodéine) ou avec paracétamol.
\item 1-2 comprimés toutes 6-8h
\end{itemize}

Tramadol : 
\begin{itemize}
\item seul, libération prolongée (LP) sur 12h \footnote{1 prise/j, 24h si 2 prises/j} ou immédiate (LI)\footnote{Toutes 4-6h}
\item max 400mg/j.
\item IV lente seulement en hôpital
\end{itemize}

Poudre d'opium avec paracétmal : 1-2 gélules toutes 4h (max 10/j)

Contre-indications : insuf respiratoire, asthme grave, insuf hépatocellulaire
sévère, enfants < 12 ans [codéine] ou 3 ans [tramadol], allaitement, épilepsie
non contrôlé [tramadol], +(ant)agoniste morphinique, +IMAO\footnote{Inhibiteurs de la monoamine oxydase} [tramadol].

Effets indésirables :
\begin{itemize}
\item ceux des opioïdes = sédation, vertige, \{constipation, nausées, vomissements\},
\{bronchospasme (dépression respiratoire)\}, rétetion d'urine
\item sécheresse buccale, douleurs abdo, troubles visuels, convulsion (si facteurs)
\end{itemize}

Éviter codéine si enceinte. \dec posologie si âgé
\end{document}
