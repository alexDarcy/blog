% Created 2019-01-15 Tue 21:13
% Intended LaTeX compiler: lualatex
\documentclass[11pt]{article}
\usepackage[hidelinks]{hyperref}
\usepackage{booktabs}
\documentclass{article}
\usepackage[hidelinks]{hyperref}
\usepackage{longtable}
\usepackage{booktabs}
%\usepackage[draft]{graphicx}
\usepackage{graphicx}
\usepackage{fancyhdr}
% French
\usepackage[T1]{fontenc}
\usepackage[francais]{babel}
\usepackage{caption}
\usepackage[nointegrals]{wasysym} % Male-female symbol
% Smaller marign
\usepackage[margin=2.5cm]{geometry}
\usepackage{latexsym}
\usepackage{subcaption}
%-------------------------------------------------------------------------------
% For graphs
\usepackage{tikz}
\usepackage{tikzscale}
\usetikzlibrary{graphs}
\usetikzlibrary{graphdrawing}
\usetikzlibrary{arrows,positioning,decorations.pathreplacing}
\usetikzlibrary{calc}
\usegdlibrary{trees, layered}
\usetikzlibrary{quotes}
%-------------------------------------------------------------------------------
% No spacing in itemize
\usepackage{enumitem}
\setlist{nolistsep}
% tightlist from pandoc
\providecommand{\tightlist}{%
  \setlength{\itemsep}{0pt}\setlength{\parskip}{0pt}}
 % Danger symbol (need fourier package)
\newcommand*{\TakeFourierOrnament}[1]{{%
\fontencoding{U}\fontfamily{futs}\selectfont\char#1}}
\newcommand*{\danger}{\TakeFourierOrnament{66}}
% Skull (need Symbola font)
\usepackage{amsmath,fontspec,newunicodechar}
\newfontface{\skullfont}{Symbola}[Scale=MatchUppercase]
\NewDocumentCommand{\skull}{}{%
  \text{\skullfont\symbol{"1F571}}%
}
% Hospital sign
\usepackage{fontspec} % For fontawesome
\usepackage{fontawesome}
% Itemize in tabular
\newcommand{\tabitem}{~~\llap{\textbullet}~~}
% No numbering
\setcounter{secnumdepth}{0}
% In TOC, only section
\setcounter{tocdepth}{1}
% Set header
\pagestyle{fancy}
\fancyhf{}
\fancyhead[L]{\leftmark}
\fancyhead[R]{\thepage}
%\renewcommand{\headrulewidth}{0.6pt}
% Custom header : no uper case
\renewcommand{\sectionmark}[1]{%
  \markboth{\textit{#1}}{}}
% Footnote in section
\usepackage[stable]{footmisc}
% Chemical compound
\usepackage{chemformula}

% Negate \implies
\usepackage{centernot} 

%-------------------------------------------------------------------------------
% Custom commands
%-------------------------------------------------------------------------------
% Logical and, or
\def\land{$\wedge{}$}
\def\lor{$\vee{}$}
\def\dec{$\searrow{}$}
\def\inc{$\nearrow{}$}


\usepackage[linesnumbered,ruled,vlined]{algorithm2e}
\usepackage[nonumberlist]{glossaries}
\makeglossaries
\newacronym{EVS}{EVS}{Echelle verbale simple}
\newacronym{EVA}{EVA}{Echelle visuelle analogiqu}
\newacronym{EN}{EN}{Echelle numérique}
\newacronym{ECPA}{ECPA}{Echelle comportementale d'évaluation de la douleur chez la personne âgée}
\newacronym{HAD}{HAD}{Hospital Anxiety and Depression Scale}
\newacronym{QDSA}{QDSA}{Questionnaire Douleur Saint-Antonne}
\newacronym{QCD}{QCD}{Questionnaire Concis de la Douleur}
\author{Alexis Praga}
\date{\today}
\title{Handicap et douleur (minifiches)}
\hypersetup{
 pdfauthor={Alexis Praga},
 pdftitle={Handicap et douleur (minifiches)},
 pdfkeywords={},
 pdfsubject={},
 pdfcreator={Emacs 26.1 (Org mode 9.1.9)}, 
 pdflang={English}}
\begin{document}

\maketitle
\tableofcontents

% Printing bacteria with biocon package
\newbact{chlamydia}{genus=Chlamydia, epithet=trachomatis}
\newbact{botulisme}{genus=Clostridium, epithet=botulinum}
\newbact{burnetii}{genus=Coxiella, epithet=burnetii}
\newbact{charbon}{genus=Bacillus, epithet=anthracis}
%
\newbact{tranchees}{genus=Bartonella, epithet=quintana}
%
\newbact{recurrente}{genus=Borreila, epithet=recurrentis}
\newbact{ecoli}{genus=Escherichia, epithet=coli}
%
\newbact{faecalis}{genus=Enteroccocus, epithet=faecalis}
%
\newbact{gardnerella}{genus=Gardnerella, epithet=vaginalis}
\newbact{ducreyi}{genus=Haemophilus, epithet=ducreyi}
\newbact{influenzae}{genus=Haemophilus, epithet=influenzae}
\newbact{granulomatis}{genus=Klebsiella, epithet=granulomatis}
\newbact{aeruginosa}{genus=Pseudomonas, epithet=aeruginosa}
\newbact{tuberculose}{genus=Mycobacterium, epithet=tuberculosis}
\newbact{genitalium}{genus=Mycoplasma, epithet=genitalium}
\newbact{gonocoque}{genus=Neisseria, epithet=gonorrhoeae}
\newbact{typhus}{genus=Rickettsia, epithet=prowazekii}
\newbact{conorii}{genus=Rickettsia, epithet=conorii}
%
\newbact{dore}{genus=Staphylococcus, epithet=aureus}
\newbact{gallolyticus}{genus=Staphylococcus, epithet=gallolyticus}
\newbact{saprophyte}{genus=Staphylococcus, epithet=saprophyticus}
%
\newbact{pneumocoque}{genus=Streptococcus, epithet=pneumoniae}
\newbact{pyogenes}{genus=Streptococcus, epithet=pyogenes}
\newbact{toxoplasmose}{genus=Toxoplasma, epithet=gondii}
\newbact{syphilis}{genus=Treponema, epithet=pallidum}
\newbact{trichomonose}{genus=Trichomonas, epithet=vaginalis}
%-------------------------------------------------------------------------------
%% Parasits
%-------------------------------------------------------------------------------
\newbact{saginata}{genus=Taenia, epithet=saginata}
\newbact{solium}{genus=Taenia, epithet=solium}


\section{Douleur}
\label{sec:orgc3e2804}
\subsection{Divers}
\label{sec:org377567d}
\begin{itemize}
\item Douleurs par excès de nociception : anti-inflammatoires, opioïdes en pratique
\item Douleurs chez personne âgée : interaction douleur, dépression et démence
\item Douleur chez l'enfant : anesthésique locaux topique (EMLA), mélange
oxygène-protoxyde d'azote (MEOPA)
\item Santé mentale : anxiété pour douleur aigüe, dépression pour douleur chronique
\item anesthésie : hypnotique (CS) + morphinique (analgésie) \textpm{} curare (relâchement
musc). Médullaire ou périphérique
\item douleurs neuro : antidépresseurs tricycliques et mixet et antiépileptiques =
efficaces sur douleurs continues et paroxystiques
\end{itemize}

\subsection{Autres médicaments}
\label{sec:org475373b}
\begin{itemize}
\item anesthésiques locaux : 
\begin{itemize}
\item EMLA : douleurs provoquées par les soins)
\item techniques de bloc : prévenir douleurs post-op
\end{itemize}
\item capsaïcine : 
\begin{itemize}
\item faible concentration : douleurs musc, arthrose, neuro (immédiat)
\item forte concentration : 3 mois sur douleurs neuro (neurotoxique !)
\end{itemize}
\item corticoïdes : anti-inflammatoire, et (+ anesthésiques locaux) sur
lésion/irritation nerf
\item kétamine : anesthésie seulement
\end{itemize}

\subsection{Évaluation de la douleur chez l'adulte}
\label{sec:org452c6a5}
\begin{center}
\begin{tabular}{ll}
\toprule
Quantitative & \gls{EVS}\\
 & \gls{EVA}\\
 & \gls{EN}\\
Qualitative & DN4\\
Retentissement & \gls{HAD}\\
 & \gls{QDSA}\\
 & \gls{QCD}\\
Âgé non communicant & Doloplus (chronique)\\
 & Algoplus (aigüe)\\
\bottomrule
\end{tabular}
\end{center}
\subsection{132 : Antalgiques}
\label{sec:org4a2e159}
\subsubsection{Palier 1 et AINS}
\label{sec:orgf617733}
\paragraph{Paracétamol}
\label{sec:orgc24b9b9}
1ere intention, surtout douleur chronique

Prescription : 3-4g/24h (adulte) avec délai 4-6h. Délai 1h, dure 4-6h. Enfant : 60mg/kg/24g
espacée de 6h 

Seul pour douleurs faibles/modérées, avec opioïdes sinon

Tolérance :
\begin{itemize}
\item si surdosage : 
\begin{itemize}
\item hépatotoxique \thus urgence \skull \thus N-acétylcysténine
\item insuf rénale aigüe et nécrose tubulaire
\item thrombocytopénie
\end{itemize}
\item sinon bonne tolérance. Surveiller HTA, hypersensibilité
\end{itemize}

\paragraph{AINS}
\label{sec:orgce1456a}
COX-1 : protège muqueuse gastro-duodénale et pro-agrégant. COX-2 : inflammation,
effet anti-agrégant

Antalgique à faible dose. Pour douleurs \{ostéo-articulaire, trauma, post-op,
néplasiques\}, coliques néphrétiques, dysménorrhées essentielles, migraines.

OS, injectable (limiter à 48h !) ou pommade/gel

Effets indésirables :
\begin{itemize}
\item digestif \footnote{NB : coxibs \dec probabilité de développer un ulcère simple/compliqué
mais retard circatrisation d'un ulcère gastrique \ldots{}}: 
\begin{itemize}
\item mineur : nausées, vomissements, gastralgies, douleurs abdo
\item grave : ulcère, perforation digestive, hémorragie
\item autres : ulcère oesophagien\footnote{Donc toujours prendre avec de l'eau, debout et sans être à jeun !}, aggravation diverticulose,
anorectites/brûlures anales
\item FR : \{ > 65 ans, ATCD ulcère/hémorragie dig, infection \bact{helicobacter},
maladie générale sévère\}, \{+ AINS, + corticoïdes, +anticoag/agrégants,
+aspirine\}
\thus FR \(\ge\) 3 ou aspirine : éviter !
\end{itemize}
\item rein :
\begin{itemize}
\item \danger risque hypovolémie, néphropathie, âgé, association (diurétique, IEC,
ARAII)
\item insuf rénale fonctionnel (créat !), rétention hydrosodée, hyperkaliémie,
néphropathie intestitielle, (nécrose papillaire)
\end{itemize}
\item cutanés/muqueus fréquents : bénin, urticaire (sd Lyelle/Stevens-johnson)
\item allergiques (rhinite, conjonctivite, oedème de Quincke, asthme) ou respiratoire
\item si anémie, chercher saignement digestif
\item hépatite (souvent silencieuse)
\item sd confusionnel (âgé)
\item CV : rétention hydrosodée, \inc risque thrombotique artériel
\end{itemize}

Éviter interactions : AINSE aspirine, \{anticoag, ticlopidine\}, \{diurétiques,
IEC\}, lithium, corticoïdes, méthotrexate

\fbox{Surtout douleurs aigües}

\subsubsection{Palier II}
\label{sec:org164bdeb}
60mg codéine = 50mg tramadol = 10mg morphine

\fbox{Douleurs modéres/intenses d'emblée ou ne répondant pas au palier I}

Douleurs aigües (courte) ou chronique (courte/long)

Codéine : 
\begin{itemize}
\item agoniste opioïde naturel \thus métabolite * morphine
\item seul (sirop, dihydrocodéine) ou avec paracétamol.
\item 1-2 comprimés toutes 6-8h
\end{itemize}

Tramadol : 
\begin{itemize}
\item seul, libération prolongée (LP) sur 12h \footnote{1 prise/j, 24h si 2 prises/j} ou immédiate (LI)\footnote{Toutes 4-6h}
\item max 400mg/j.
\item IV lente seulement en hôpital
\end{itemize}

Poudre d'opium avec paracétamol : 1-2 gélules toutes 4h (max 10/j)

Contre-indications : insuf respiratoire, asthme grave, insuf hépatocellulaire
sévère, enfants < 12 ans [codéine] ou 3 ans [tramadol], allaitement, épilepsie
non contrôlé [tramadol], +(ant)agoniste morphinique, +IMAO\footnote{Inhibiteurs de la monoamine oxydase} [tramadol].

Effets indésirables :
\begin{itemize}
\item ceux des opioïdes = sédation, vertige, \{constipation, nausées, vomissements\},
\{bronchospasme (dépression respiratoire)\}, rétention d'urine
\item sécheresse buccale, douleurs abdo, troubles visuels, convulsion (si facteurs)
\end{itemize}

Éviter codéine si enceinte. \dec posologie si âgé

\subsubsection{Palier III}
\label{sec:org98d3e49}
AMM : 
\begin{itemize}
\item douleurs \emph{non} cancéreuse = morphine, oxycodone, fentanyl (transdermique)
\item douleurs cancéreuse = idem et fentanyyl transmuqueus, hydromorphone
\end{itemize}

\paragraph{Formes}
\label{sec:orga308a79}
\begin{center}
\begin{tabular}{lllll}
\toprule
Catégorie & Molécule & Note & forme & Action\\
\midrule
Agonistes purs & Morphine & référence. & Chlorhydrate & \\
 &  &  & Sulfate & LP (12-24h) ou LI (4h)\\
 & Hydromorphone & K (2eme intention) &  & 12h (délai 2h)\\
 & Oxycodone) & morphine x2 &  & LP (12h) ou LI (4h)\\
 & Fentanyl & morphine x50-150 & Transdermique & 72h (délai 12-18h)\\
 &  &  & Transmuqueuse & 1-2h (délai 10min)\\
 & Méthadone & dpdance opiacés &  & \\
 &  &  &  & \\
Agonistes partiels & buprénorphine & morphine x30. &  & \\
 &  & Effet plafond &  & \\
Agoniste-antagonistes & nalbuphine & pédiatrie++ &  & \\
Antagonistes & naloxone & Effet plafond &  & \\
\bottomrule
\end{tabular}
\end{center}

\paragraph{CI}
\label{sec:org28aca21}
Insuf respi décompensée, insuf hépatocellulaire sévère, insuf rénale sévère,
épilepsie non contrôlée, trauma crânier et HTIC, intox alcoolique aigüe, +IMAO,
associer agonistes avec a. partiel ou a-antagonistes

\paragraph{EI}
\label{sec:org0938f11}
\begin{itemize}
\item Constipation (fréquent !):
\begin{itemize}
\item préventif = laxatif oral systématique, hygiéno-diététique, oxycodone et
naloxone
\item curatif : \inc laxatif, fécalome ?, ttt rectal.\footnote{Si 0 selles : bithérapie laxative, lavement rectal, antagoniste
morphinique périph}
\end{itemize}
\item Nausées, vomisseements(fréquent !):
\begin{itemize}
\item préventif : anti-émétique
\item curatif : neuroleptique action centrale, corticoïdes, sétrons, droperidol
\end{itemize}
\item Somnolence : \dec dose ou rotation
\item Dépression respi (FR < 10min) : réa et naloxone
\item Trouble s confusionnels, cognitifs : \dec doses, rotation
\item Dysurie, rétention :\footnote{Y enser si HTA, douleurs abdo, agitation inhabituelle} \dec doses, sondage, chercher médicaments favorisants
\item Prurit : antihsistaminique, rotation
\end{itemize}

Dépendance : pyschologique (recherche compulsvie), physique (sd sevrage)

\paragraph{Surdosage}
\label{sec:orgbae8870}
Somnolence, respi irrégulière, FR < 10/min
Échelle de sédation (0 à 3)\footnote{0 = éveillé, 3 = très somnolent, éveillable par stimulation tactile.}, de qualité de respiration (R0 à R3)\footnote{R0 = normale, R3 = pauses/apnée}

Réat et injection narcan (naloxone) : 0.4mg par dose de 0.04mg/2min jusque R1/R0

\paragraph{Indications :}
\label{sec:orgebf73d8}
\begin{tcolorbox}
Privilégier formale orale et LP
\end{tcolorbox}

\begin{itemize}
\item Douleurs aigüe : très intenses ou (modéré/sévère ne répondant pas au palier
\end{itemize}
II). Oral (parentéral si urgence)
\begin{itemize}
\item Douleurs chroniques cancéreuses ou (non cancéreuses après échec étiologique,
palier 1, II et techniques). Oral et LP.
\end{itemize}

\begin{tcolorbox}
 Équivalence :
\begin{itemize}
\item morphine : 1 oral = 1/2 SC = 1/3 IV 
\item 1 morphine Iv = 1 oxycodone IV/SC
\item oxycodone : 1 oral = 1/2 IV/SC
\end{itemize}
\end{tcolorbox}

Prescription : ordonnance sécurisée, \(\le\) 28j

\subsection{Antiépileptiques}
\label{sec:org97d7952}
Douleur neuropathique :
\begin{enumerate}
\item Gabapentinoïdes
\item \textpm{} tricycliques/opioïdes (2eme intention)
\end{enumerate}

\subsection{Antidépresseur}
\label{sec:orgf21ce39}
Classes :
\begin{itemize}
\item tricyclique : amitriptyline
\item inhibiteurs de la recapture de la noradrénaline et sérotonie : venlafaxine,
duloxétine, milnacipran
\end{itemize}

Douleurs chroniques : 
\begin{itemize}
\item neuropathiques, lombalgie, céphalée : amitriptyline
\item fibromyalgie : duloxétine, milnacipran
\end{itemize}

\printglossaries

\subsection{Grands syndromes}
\label{sec:org99fc91d}
Pneumo 
\begin{itemize}
\item Dyspnée : morphinique, benzodiazepine \textpm{} O\(_{\text{2}}\) (en phase terminale ?)
\item Encombrements bronchiques : demi-assis, soins de bouches, éviter aspirations,
anticholinergiques
\end{itemize}

Digestifs :
\begin{itemize}
\item nausées, vomissements : alimentation, soins de bouche, neuroleptiques,
corticoïdes, agonistes récepteurs 5-HT3 (si chimio) \textpm{} benzodiazepine
\item occlusion intestinale : origine mécanique ou fonctionnelle
(motricité). Fréquent si carcinose péritonéale
\begin{itemize}
\item à jeûn + réhydratation IV/SC
\item antiémétique (neuroleptique), antisécrétoires anti-cholinergique,
antisécrétoire analogue somatostatine, IPP, corticothérapie, antalgique \textpm{}
SNG
\end{itemize}
\item constipation : hygiéno-diététique, laxatif osmotique/stimulation (3e jour :
suppositoire de glycérine)
\end{itemize}

Neuropsy :
\begin{itemize}
\item confusion : neuroleptique (éviter benzodiazepine !)
\item anxiété : benzodiazepine, hydroxine. Si échec neuroleptique
\end{itemize}
\end{document}