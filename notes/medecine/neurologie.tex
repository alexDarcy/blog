\section{UE5 - Item 132 : Thérapeutiques antalgiques médicamenteuses et non
médicamenteuses}%
\label{sec:ue5_item_132}

\subsection{Médicaments}%

\paragraph{Analgésiques}%

Palier 1 :
\begin{itemize}
  \item Paracétamol : < 4g / jour. Toxicité hépatique
  \item Aspirine : < 500mg-1g toutes les 6-8h. Toxicité gastro-intestinale. Risque allergique, effet antiagrégant
  \item Néfopam (Acupan\copyright) < 20-120mg/j. CI : ATCD convulsion, adénome
  \item prostatique, glaucome angle fermé
\end{itemize}

Palier 2

\begin{itemize}
\item Codéine : posologie suivant paracétamol. ES: opioïdes
\item Tramadol : alternative codéine. Associer au paracétamol. < 400mg/j. ES : opioïdes
\item Lamaline\copyright : < 10 comprimés/jour
\end{itemize}

Palier 3
\begin{itemize}
  \item Anta-agonistes morphiniques : buprénorphine
  \item Agonistes morphiniques : morphine, oxycodone, hydromorphone, fentanyl
\end{itemize}

Antidote : naloxone (urgence seulement)

\paragraph{Douleurs neurogènes}%

Antiépileptiques : prégabaline (Lyrica\copyright), carbamazépine

Antidépresseur : 1ère intention : tricycliques (Laroxyl\copyright). Puis : inihibiteur de la recapture de la sérotonine


\subsection{Non médicamenteux}%
\label{sub:non_medicamenteux}

Neurochirugie, neurostimulation cutanée, suivi psy, thérapies physiques, ttt non
conventionnels, médico-social.

