% Created 2019-02-08 Fri 16:51
% Intended LaTeX compiler: lualatex
\documentclass[11pt]{article}
\usepackage{graphicx}
\usepackage{fontawesome}
\usepackage{grffile}
\usepackage{longtable}
\usepackage{wrapfig}
\usepackage{rotating}
\usepackage[normalem]{ulem}
\usepackage{amsmath}
\usepackage{textcomp}
\usepackage{amssymb}
\usepackage{capt-of}
\usepackage{hyperref}
\documentclass{article}
\usepackage[hidelinks]{hyperref}
\usepackage{longtable}
\usepackage{booktabs}
%\usepackage[draft]{graphicx}
\usepackage{graphicx}
\usepackage{fancyhdr}
% French
\usepackage[T1]{fontenc}
\usepackage[francais]{babel}
\usepackage{caption}
\usepackage[nointegrals]{wasysym} % Male-female symbol
% Smaller marign
\usepackage[margin=2.5cm]{geometry}
\usepackage{latexsym}
\usepackage{subcaption}
%-------------------------------------------------------------------------------
% For graphs
\usepackage{tikz}
\usepackage{tikzscale}
\usetikzlibrary{graphs}
\usetikzlibrary{graphdrawing}
\usetikzlibrary{arrows,positioning,decorations.pathreplacing}
\usetikzlibrary{calc}
\usegdlibrary{trees, layered}
\usetikzlibrary{quotes}
%-------------------------------------------------------------------------------
% No spacing in itemize
\usepackage{enumitem}
\setlist{nolistsep}
% tightlist from pandoc
\providecommand{\tightlist}{%
  \setlength{\itemsep}{0pt}\setlength{\parskip}{0pt}}
 % Danger symbol (need fourier package)
\newcommand*{\TakeFourierOrnament}[1]{{%
\fontencoding{U}\fontfamily{futs}\selectfont\char#1}}
\newcommand*{\danger}{\TakeFourierOrnament{66}}
% Skull (need Symbola font)
\usepackage{amsmath,fontspec,newunicodechar}
\newfontface{\skullfont}{Symbola}[Scale=MatchUppercase]
\NewDocumentCommand{\skull}{}{%
  \text{\skullfont\symbol{"1F571}}%
}
% Hospital sign
\usepackage{fontspec} % For fontawesome
\usepackage{fontawesome}
% Itemize in tabular
\newcommand{\tabitem}{~~\llap{\textbullet}~~}
% No numbering
\setcounter{secnumdepth}{0}
% In TOC, only section
\setcounter{tocdepth}{1}
% Set header
\pagestyle{fancy}
\fancyhf{}
\fancyhead[L]{\leftmark}
\fancyhead[R]{\thepage}
%\renewcommand{\headrulewidth}{0.6pt}
% Custom header : no uper case
\renewcommand{\sectionmark}[1]{%
  \markboth{\textit{#1}}{}}
% Footnote in section
\usepackage[stable]{footmisc}
% Chemical compound
\usepackage{chemformula}

% Negate \implies
\usepackage{centernot} 

%-------------------------------------------------------------------------------
% Custom commands
%-------------------------------------------------------------------------------
% Logical and, or
\def\land{$\wedge{}$}
\def\lor{$\vee{}$}
\def\dec{$\searrow{}$}
\def\inc{$\nearrow{}$}


\newacronym{CPRE}{CPRE}{Cholangio-pancréatographie rétrograde endoscopique}
\usepackage{adjustbox}
\newacronym{DT}{DT}{Delirium tremens}
\newacronym{ID}{ID}{Immunodéprimé}
\newacronym{HSH}{HSH}{Hommes ayant des relations sexuelles avec des hommes}
\newacronym{DO}{DO}{Déclaration obligatoire}
\newacronym{CHC}{CHC}{Carcinome hépato-cellulaire}
\newacronym{OGD}{OGD}{Oestro-gastro-duodénale}
\newglossaryentry{trophozoïtes}{name={Trophozoïtes},description={Formes végétatives mobiles}}
\newacronym{CST}{CST}{Coefficient de saturation de la transferrine}
\newglossaryentry{sdmetabolique}{name={Syndrome métabolique},
description={$\diameter \ge 94$ cm $\male{}$, 80 cm $\female{}$,
TG $\ge$ 1.7mmol/L, HDL < 1 mmol/L $\male{}$ ou 1.3mmol/L $\female{}$,
PAs $\ge 130$ mmHg ou PAd $\ge 85$ mmHg, glycémie jeun $\ge$ 1 g/L
}}
\newacronym{RGO}{RGO}{Reflux gastro-oesophagien}
\newacronym{IS}{IS}{Immunosuppresseurs}
\newacronym{HMG}{HMG}{Hépatomégalie}
\newacronym{SMG}{SMG}{Splénomégalie}
\newacronym{VBP}{VBP}{Voie biliaire principale}
\newacronym{BGT}{BGT}{bilirubine glucoronide-transférase}
\author{Alexis Praga}
\date{\today}
\title{HGE}
\hypersetup{
 pdfauthor={Alexis Praga},
 pdftitle={HGE},
 pdfkeywords={},
 pdfsubject={},
 pdfcreator={Emacs 26.1 (Org mode 9.1.9)}, 
 pdflang={English}}
\begin{document}

\maketitle
\tableofcontents

\section{74 : Addiction à l'alcool}
\label{sec:org9f62e29}
\subsection{Définition}
\label{sec:org4ddba54}
\begin{itemize}
\item Non-usage
\item Usage simple
\item Mésusage : 
\begin{itemize}
\item risque
\item nocif
\item dépendance : 3 parmi \{désir puissant, difficulté à contrvôler l'utilisation,
sd de sevrage, tolérance, abandon d'autres plaisirs/intérêt, consommation
malgré conséquences nocives\}
\end{itemize}
\end{itemize}
\subsection{Étiologie}
\label{sec:orgac16204}
Génétique (40-60\%), ATCD paternel, personnalité antisociale, facteurs
socioculturels.

Anatomie : noyau accumbens.
\subsection{Épidémiologie}
\label{sec:org4badc67}
Consommation : \dec{} vin, \inc{} boissons peu/très alcoolisées

Mésusage : 49 000 DC/an. Alcool = 3eme cause de DC. 2 700 DC/an sur route, 20\%
AT.

Souvent + conduite addictive (tabac, benzodiazépine).

50\% DC prématurés = pas de dépendance !
\subsection{Dépistage}
\label{sec:org1f0def7}
20-30\% adultes en CS ont un problème avec l'alcool
1 verre = 10g d'alcool pur\footnote{Contenu d'alcool = \(V_{alcool} \times \rho_{alcool}\) = degré d'alcool
\(\times V_{total} \times 0.8\). Volume en L, degré en pourcentage.}

Questionnaires : AUDIT-C, FACE
\subsection{Biologie \footnote{Biologie normale \(\ne\) pas de mésusage !}}
\label{sec:org7563e8c}
\begin{itemize}
\item alcoolémie
\item \(\gamma\)-GT : peu sensible, peu spécifique. Utile si (transaminases et
phosphatases alcalines normales) ou (transaminases < 5N avec \inc{} ASAT
\item VGM
\item transferrine désialylée
\end{itemize}
\subsection{Clinique}
\label{sec:org6cb05d4}
Intoxications alcoolique aigüe : 
\begin{itemize}
\item diagnostic = alcoolémie, régression de qq heures
\item désinhibition et euphorisant puis dépression sur SNC
\item pathologique : excitomotrice, délirante, dépressive/hypomaniaque, amnéie
lacunaire, convulsivante
\item risque = coma alcoolique avec pronostic vital engagé
\end{itemize}
Coma alcoolique : alcoolémie > 3 g/L
\begin{itemize}
\item neuro : coma calme, hypotonique, mydriase bilatérale symétrique
\item dépression repsiration
\item hypotension artérielle, bradycardie, collapsus CV
\item hypothermie
\item \danger cherche : hémorragie cérébroméningée, aVC, hémato sous/extra-durale,
épilepsie, Gayez-Wernicke, encéphalopathie hépatique, \gls{DT}, infection
cérébroméningée si fièvre
\end{itemize}
Dépendance : sevrage
\begin{itemize}
\item \(\frac{1}{3}\) n'ont pas de symptômes de sevrage
\item anxiété, sueurs, tremblemens, vomissement
\item \gls{DT}, épilespise chez 5\%
\end{itemize}
Accidents du sevrage
\begin{itemize}
\item convulsions : dans 48h (mais retard possible). Sans ttt, peut évoluer vers
DT. Éliminer autres causes !
\item DT : début brutal/petits signes de sevrage/crise comitiale. Sans ttt : sd
confusionnel. Sans PEC, DC 
\end{itemize}

\subsection{Complications somatiques}
\label{sec:orge5b0ba8}
\begin{tcolorbox}
5 arguments pour rapporter une pathologie à l'alcool : consommation à risque,
 complication connue, tableau évocateur, pas d'autre cause, s'améliore avec le sevrage/réduction
\end{tcolorbox}
\begin{table}[htbp]
% \caption{Complications somatiques}
\centering
\begin{tabular}{ll}
\toprule
Cancer & VAS, oesophage, hépatocellulaire, colorectal, sein\\
Digestif & Maladie alcoolique du foie, pancréatite aigüe/(chronique\\
 & calcifiante), oesophagite, gastrite, diarrhée montrice\\
SN & Troubles cognitifs, démence, encéphalopathies carentielles, métaboliques,\\
 & atrophie cérébelleuse, épilepsie, polyneuropathies SM, neuropathie optique,\\
 & traum, hémorragies cérébrales/méningées\\
CV & HTA, troubles rythme, cardiomyopathies non obstructives\\
Rhumato & Nécrose tête fémorale, ostéoporose, ostéomalacie\\
Trauma & Fracture\\
Hémato & Macrocytose, anémie, thrombopénie, leucopénie\\
Métabo & Hypoglyclémie, hyperTG, dénutrition, hyperuircémie\\
Dermato & Aggrav. psoriasis, rhinophyma\\
Psy & Dépression, trouble anxieux\\
Foetus & Dysmorphie, retard mental\\
Dysfonctions sexuelles & \\
\bottomrule
\end{tabular}
\end{table}

\subsection{PEC}
\label{sec:org9dd2327}
Attitude empathique

Intervention brève (5-20min)

Intoxication alcoolique aigüe : 
\begin{itemize}
\item éliminer hypoglycémie, autre intoxication,hématome IC, hémorragies méningées \thus EC et complémentaires si trouble
\end{itemize}
conscience. 
\begin{itemize}
\item \danger Vitamine B1 avant perfusion glucosée !
\end{itemize}

Usage nocif, dépendance :
\begin{itemize}
\item nocif/dépendance sans comorbidités sévères : réduction conosmmation
\item nocif et complications nombreuses/sévères : addictologie
\item dépendance sévère et comorbidité : arrêt
\item motivation++
\item surveiller ASAT, ALAT, \(\gamma\)-GT, TP, NFS
\end{itemize}

Sevrage : 
\begin{itemize}
\item hospit si : ATCD complication sevrage, dépendance sévère ou avec
benzodiazépines, échec de l'ambulatoire, environnement social, terrain
vulnérable
\item arrêt, hydratation orale, vitamine B1 \textpm{} correction troubles
hydroélectriques, \textpm{} benzodiazépnie (diazépam)
\end{itemize}

Complications de sevrage :
\begin{itemize}
\item DT
\begin{itemize}
\item hospit, réhydratation IV, correction trouble hydroélectrique
\item vitamine B1 forte dose avant sérum glucosé
\item diazépam jusque sédation
\end{itemize}
\item Crises convulsive : cf sevrage
\end{itemize}

Médicaments efficaces seulement si suivi psychosocial !
\begin{itemize}
\item \dec{} consommation : nalméfène (CI : opiacés) (ou baclofène)
\item maintien abstinence : acamprosate ou naltrexone (ou disulfirame, baclofène)
\end{itemize}

Autre : PEC comorbidités psy, addictive (tabac, cannabais++), non psy, sociale
\section{163 : Hépatites virales}
\label{sec:org170d341}
Définition : processus inflammatoire du foie. Chronique si > 6 mois.

Causes :
\begin{itemize}
\item \inc{} forte des transaminases : 
\begin{itemize}
\item \textbf{médicaments, virale} = \{VHA, VHB, VHC, VHE et Herpes (EBV, HSV, CMV, VZV)\}
\item migration lithiasique, hépatite auto-immune, maladie de Wilson
\end{itemize}
\item \inc{} modérée des transaminases : 
\begin{itemize}
\item \textbf{alcool, sd métabolique, VHB, VHC}
\item VHE si \gls{ID}, médicaments, hépatite auto-imune, hémochromatose, maladie de
Wilson, déficit \(\alpha\)-1-antitrypsine
\item DD : maladie coeliaque, myopathie, effort violent, hémolys
\end{itemize}
\end{itemize}

Diagnostic: 
\begin{itemize}
\item souvent asymptomatiques, ictère peut manquer (symptômes + 7 à 10j)
\item chercher signes de gravité : 
\begin{itemize}
\item TP < 50\% = hépatite sévère \thus centre spécialisé
\item encéphalopathie \thus hospit en urgence 
\end{itemize}
\item dans tous les cas : CI médicaments hépatotoxiques, neurosédatifs, éviter
contamination (VHA, VHE)
\end{itemize}





\begin{table}
\caption{Hépatites virales, résumé}
\label{tab:my-table}
\centering
\adjustbox{max width=\linewidth}{
\begin{tabular}{llllll}
\toprule
 & VHA & VHB & VHC & VHD & VHE\\
\midrule
 & ARN simple brin & ADN double brin & ARN simple brin & ARN simple brin & ARN\\
 & Sans enveloppe & Enveloppe &  & Enveloppe & monocaténaire\\
Famille & \emph{Hepatovirus} & \emph{Hepadnavirus} & \emph{Flavirus} &  & \\
Transmission & Féco-orale, \acs{HSH} & périnatale, sexuelle & parentéral & sang, sexuelle & \\
 &  & sang, contact &  &  & \\
Diagnostic & IgM anti-VHA & Ag HBs & Ac antiVHC & IgM/IgG anti-delta & IgM antiVHE\\
 &  &  & \textpm{} ARN & ARN & \\
 & \acs{DO} & DO (aigüe) &  &  & \\
\bottomrule
\end{tabular}
}
\end{table}

\subsection{VHA}
\label{sec:org2d6b308}
Incubation 2-6 semaines

Pas d'infection chronique !
Vaccin efficace.

Fréquent. Asymptotique et bénigne le plus souvent

\subsection{VHB}
\label{sec:org1948ad3}
2 milliards porteurs d'une infection (résolue ou non). 

Hépatite aigüe B :
\begin{itemize}
\item incubation 6 sem. - 4 mois
\item Morbidité/mortalité : risque de cirrhose (20\%) ou \gls{CHC}
\item \danger recherche VIH, IST, coinfection VHD.
\item chronique dans 5-10\%
\end{itemize}

Hépatite chronique B : Ag HBs positif > 6mois. 
\begin{itemize}
\item asymptomatique jusqu'à cirrhose, CHC
\item 4 phases :
\begin{enumerate}
\item tolérance immunitaire (réplication virale, hépatite peu active) : AgHBe
\item clairance immunitaire (réplication virale, hépatite) : Ag HBe
\item non réplicative (réplication virale faible, pas d'activité hépatite) : Ac
anti-HBe
\item hépatite B résolue : Ac anti-HBs
\end{enumerate}
\item 40\% de mutation
\item \gls{IS} = risque de réactivation 
\item CHC possible sans cirrhose \danger
\end{itemize}

\subsubsection{Traitement (chronique)}
\label{sec:orge24f5a4}
ALD, PEC 100\%

Chercher cirrhose : hypertension portable (endoscopie \gls{OGD}), prévention
hémorragies digestives, dépistage CHC (écho abdo tous 6 mois)

2 traitements :
\begin{itemize}
\item à vie : analogues nucléosidiques/nucléotidiques (entécavir, ténofovir) =
antiviral
\item 1 an : antiviral + immunomodulateur (plus rare)
\end{itemize}

Indications :
\begin{itemize}
\item ADN VHB > 2 000 UI/mL et transaminases > N ou fibrose hépatique modérée ou
activité modérée\footnote{Avec le score Metavir : activité \(\ge\) A2 et fibrose \(\ge\) F2 (entre 0 et 3,
entre 0 et 4 respectivement)}
\item ADN > 20 000 UI/mL et transaminases > 2N
\item cirrhose et ADN VHB
\item ACTD familiale de cirrhose/CHC ou atteinte extrahépatique
\item ISN
\end{itemize}
Pas indiqué en phase de "tolérance immunitaire" ou "non réplicative"

Élasticité hépatique (Fibroscan) :
\begin{itemize}
\item < 7kPa : cirrhose peu probable
\item > 17kPa : cirrhose très probable
\end{itemize}

Vaccination :
\begin{itemize}
\item obligatoire nourrison (hexavalent) : sinon rattrapage jusque 15 ans (2 inj. à
6 mois d'écart)
\item dépistage chez femme enceinte avant 10 SA : si positive, sérovaccination à la
naissance
\item professionnels de santé
\item sujet exposé
\item IgG anti-HBs seulement si contage accidentel chez non vacciné
\end{itemize}

\subsection{VHC}
\label{sec:orga8b89ef}
Pas de vaccin !

Chronique dans 75-85\%, cirrhose dans 10-20\% à 20 ans. CHC 1-4\% par an. Manifestations extrahépatiques

Aigüe : 
\begin{itemize}
\item incubation très variable (7-8 sem), souvent asymptomatique
\item ARN à +1 semaine
\item Ac anti-VHC à +12semaines
\end{itemize}

Chronique : (sérologie positive, ARN détectable) > 6 mois. 
\begin{itemize}
\item \danger ponction biopside non recommandée \footnote{Sans comorbidités}\thus Fibrotest, Fibromètre Fibroscan
\end{itemize}

\subsubsection{Traitement (chronique)}
\label{sec:org440df55}
Cherche cirrhose (cf VHB).

Guérison virologique.

Pas d'interféron, ni ribavirine.

Régression des lésions hépatiques si pas de cirrhose. \danger continuer
dépistage CHC

\subsection{VHD}
\label{sec:orga46f4c0}
Dépend du VHB !
\danger toujours cherche VHD chez VHB

Co/sur-infection. Risque : cirrhose, CHC

Ttt (chronique) : interféron pégylé, peu efficace

\subsection{VHE}
\label{sec:orgcc61a72}
Incubation 3-8semaines, 3-4 jours pré-ictériques.
Guérison sans séquelles en 1 mois souvent. Formes graves : hépatite fulminante, décompensation

Chez l'ID, chercher ARN pour diagnostic

\section{168 : Parasitoses digestives}
\label{sec:org39a7575}

\subsection{Téniasis : cf Tab \ref{tab:org24964bd}}
\label{sec:org8bf9a99}
\begin{table}[htbp]
\caption{\label{tab:org24964bd}
Caractéristiques des \emph{Taenia}}
\centering
\begin{tabular}{lllll}
\toprule
 & \emph{saginata} & \emph{solium} & \emph{hymenolepsis} & \emph{Diphyllobothrium}\\
 &  &  & \emph{nana} & \emph{latum}\\
\midrule
Caract & Anneaux, 4-10m, & \danger{} cysticercose\tablefootnote{Larves dans tissus SC, muscle, oeil, cerveau (HTIC, épilepsie !), moelle épinière \danger} & Petit & 10-15m, 10 ans\\
 & intestin grêle &  &  & \\
Contam. & Boeuf mal cuit & Porc mal cuit &  & Poisson mal cuit\\
Clinique & Latent &  &  & \\
Diagnostic & Anneaux selles/vêtement & Sérologie & \OE{}ufs dans selles & \OE{}ufs dans selles\\
 & EPS & \textpm{} biopsie-exérèse &  & \\
Ttt & Niclosamide & Albendazole & Niclosamidel & Niclosamide\\
 & praziquantel & praziquantel & praziquantel & praziquantel\\
 &  & (+cortico si cérébral) &  & \\
Prévention & Cuisson/congélateur\tablefootnote{> qq semaines} &  & hygiène des main & Cuisson\\
\bottomrule
\end{tabular}
\end{table}



\subsection{Ascaridiose\footnote{\emph{Ascaris lumbricoides}}}
\label{sec:org287638b}

Pays tropicaux hygène insuffisante

Ingestion \oe{}ufs

Phase de migration : signes allergiques, sd de Löffler\footnote{Fièvre, toux, dyspnée, infiltration radio fugace, hyperéosinophilie}. Phase d'état :
troubles digestifs

Complications : appendicite, angiocholite, pancréatite, occlusion intestinale

Diag: \oe{}ufs dans selles (+2 mois)

Ttt : flubendazole, albendazole

Prév: lavage mains, fruits, crudités, péril fécal

\subsection{Oxyurose \footnote{\emph{Enterobius vermicularis}}}
\label{sec:orgee706ce}

Fréquent (enfants âge scolaire)

Clinique : prurit anal/asymptomatique

Diag : vers blancs mobiles dans selles/Scotch test

Ttt : flubendazole, albendazole, pyrantel. Traiter toute la communaute !

\subsection{Giardiose\footnote{\emph{Giardia intestinalis}}}
\label{sec:orge5fa6ee}

Ingestion eau, aliments contaminés, féco-oral

Clinique : 
\begin{itemize}
\item asymptomatique
\item atypique : \{brutal, selles nbs liquide, douleurs épigastriques\}
\item subaigu/chronique (mois/année)
\item malabsorption et dénutrition si déficit immunitaire
\end{itemize}

Diag : kystes/\gls{trophozoïtes} à l'EPS. \texttimes{} 3 si besoin. Biopsie duodénale
si chronique

Ttt : métronidazole (contrôle + 1mois)

Prév : hygiène eau, boisson, aliments

\subsection{Am\oe{}bose\footnote{\emph{Entamoeba histolytica}}}
\label{sec:org8c4004d}

Une des parasitoses les	plus fréquentes. Kyste ou trophozoïtes.

Cont: féco-orale\footnote{Ou sexuelles oro-anales}

Dissémination par voie sanguine depuis tube digestif : atteinte intestinale
(tous âges) ou foie (\male{} 20-50 ans), poumons, cerveau

Forme intestinale 
\begin{itemize}
\item (sub)Aigue, diarrhée non hémorragiques, douleurs abdo, sans fièvre ni AEG
\item Autres formes : dysentérique aigüe, fébrile, colite aigüe grave amibienne
= urgence 
\end{itemize}

Forme hépatique : rare, collection purulente, après am\oe{}bose intestinale
\begin{itemize}
\item qq jours, fièvre élevée, frissons, douleur hypochondre droit/scapula droit
\item foie volumineux, douleureux palpation
\item hyperleucocytose à PNN, légère \inc{} transaminases, phosphatases,
bilirubinémie
\item DD : abcès du foie à pyogènes, tumeur maligne nécrose
\item complications rares, graves
\end{itemize}

\subsubsection{Diagnostic}
\label{sec:org1f0e171}
Intestinale
\begin{itemize}
\item EPS : seulement kyste, et confusion \emph{Entamoeba histolycita} et \emph{dispar} (non
patho)
\item endoscopie : "boutons de chemise", sérologies (sur biopsie)
\end{itemize}
Hépatique
\begin{itemize}
\item sérologie et clinique et écho typique \(\approx\) certittude
\item répéter si séro négative. Si négative et sans argument : ponction avec écho
\end{itemize}

\subsubsection{Traitement}
\label{sec:org88e1b52}
Intestinale : curatif/probabiliste: métronidazole 10 j puis am\oe{}bicide de
contact 10j (tiliquinol)

Hépatique : imidazolé et am\oe{}bicide de contact. Ponction seulement si abcès superficiels au bord
de la rupture

\subsection{Hydatidose}
\label{sec:orgad6a46b}
Parasite adlute (anneaux), hydatide (larve).

Cycle : intestin grêle du chien -> \oe{}ufs -> mammifères herbivores : hydatide
dans foie/poumon -> ingestion de protoscolex par le chien. Homme = accident

Clinique : asymptomatique, lent (années)
\begin{itemize}
\item hépatique : 
\begin{itemize}
\item fréquent, 10-15ans. HMG isolée, indolore, bien toléré.  pas
de ponction !
\item complications : compression voie biliaires ou veineux porte/cave,
fissuration/ruputre (réaction allergiques, hydatidose secondaire),
infection après fissure
\item IRM hépatique
\end{itemize}
\item autres : radio thorax suspectes, même complications. N'importe quel
endroit. Os = mauvais pronostic
\end{itemize}

Diag: sérologie (90\% hépatique, 70\% pulmonaire). Pas de ponction !!

Ttt : chir (albendazole seul sinon)

Prév : chien, mesures sanitaires
\section{197 : Transplantation d'organes}
\label{sec:org036554a}
Indication num. 1 de greffe hépatique : CHC

Complications :
\begin{itemize}
\item précoces = vasculaire, bilaire et sepsis
\item tardives : récidive maladie initiale, cancers de novo, CV
\end{itemize}

Survie : 62\% à 10 ans

Score "foie" pour défaillance chronique

Consentement présumé, gratuité du don, anonymat donneur-receveur
\section{215 : Pathologies du fer}
\label{sec:org0a499f7}

Absoprtion par les entérocytes du duodénum et jéjunum proximal. Puis stocké par
ferritine ou exporté par ferroportine dans le plasma. Il y est alors transporté
par transferrine.

Régulation : hepcidine influence l'absorption.

Réserve de fer : ferritine (foie, moelle osseuse, rate)

\subsection{Exploration}
\label{sec:org1d493ed}
\begin{itemize}
\item Ferritine = 1ere intention : si \dec, carence en fer, si \inc, pas forcément
un excès\ldots{}\footnote{Lyse cellulaire, sd inflammatoire, alcool, sd métabolique, sd
paranéoplasique, hyperthyroïde}
\item \gls{CST} = transport du fer. Montre surcharge mais sans la quantifié. \dec si
sd inflammatoire, \inc insuf hépatocellulaire, sd néphrotique, sd
inflammatoire
\item récepteur soluble de transferrine = 2eme intention
\end{itemize}

\subsection{Anémie par carence martiale}
\label{sec:org6fac169}
\subsubsection{Mécanisme}
\label{sec:org25cf649}
\begin{itemize}
\item Saignement : occulte. Digestif ou gynéco si femme non ménopausée
\item Malabsorption : hypochlorhydrie franche\footnote{Acide chlorhydrique garde Fe sous forme soluble}, altération muqueuse
duodénojéjunale, court-circuit duodénojéjunal. Maladie c\oe{}liaque ++
\item Carence d'apportA : 1mg/j (\male) ou 2 mg (\female). \inc besoins ou régime
végétarien très strict
\end{itemize}
\subsubsection{Diagnostic}
\label{sec:org8ad765b}
Pâleur, asthénie. Chercher signes de gravité.
Diagnostic biologique : anémie hypochrome microcytaire 
\begin{itemize}
\item 1ere intention : ferritine \inc
\item différencier d'un sd inflammatoire et thalassémie (cf encadré)
\end{itemize}

\begin{tcolorbox}
Carence en fer = ferritine \dec, fer sérique \dec, CST \dec et transferrine \inc
\end{tcolorbox}

\subsubsection{Étiologie}
\label{sec:org601cad3}
Clinique : ATCD, médic, gynéco, régles, troubles dig. Chercher masse tumorale,
méléna

Examens :
\begin{itemize}
\item 1ere intention : gynéco, endoscopie \gls{OGD} et coloscopie :
\begin{itemize}
\item NB: > 5 Ans : coloscopie systématique (K colique) \danger
\item pas de lésion à l'endoscopie \thus biopsie duodénale \textpm{} gastrique
\item \textbf{\textbf{pas}} des causes : hernie hiatale sans érosion, polype colique non ulcéré
< 15mm
\end{itemize}
\item 2eme intention : vidéocapsule endoscopie (sauf sis sténose), entéroscopie (biopsie),
scanner spiralé (tumeurs du grêles), scintigraphie au pertechnétate-99m
(diverticule de Meckel)
\item Si exploration négatives : supplémentation Fe
\end{itemize}

\subsubsection{Traitement}
\label{sec:org08b1fe2}
Cause si possible.

Martial : voie orale 150 à 300mg/j hors repas (selles noires, interactions
possibles). IV si intolérance. Contrôle ! Traiter jusque normalisation ferritinémie




\subsection{Surcharge en fer}
\label{sec:orga7f05c3}
Diagnostic = 
\begin{itemize}
\item CST < 45\% \thus \textbf{pas} origine génétique. si Ferritine \inc, hépatosidérose
métabolique (\(\in\) sd métabolique)
\item CST > 45\%, répéter test. Si positif et pas d'autres causes, chercher mutation C2V2Y du gène HFE : si
oui, diagnostic positif
\item \textpm{} IRm hépatique
\end{itemize}
DD : surcharge d'apport, malade chronique du foie, porphyrie cutanée, hémopathie

\subsubsection{Hémochromatose}
\label{sec:orgafef2b7}
10 à 20\% des homozygotes ont des symptômes. 4 phases :
\begin{enumerate}
\item asymptomatique
\item surcharge Fe biologique : CST > 45\%, ferritine normale
\item symptômes clinique (30 ans)
\item lésions viscérales (40-60 ans)
\end{enumerate}

Clinique : 
\begin{itemize}
\item asthénie
\item mélanodermie, dépilation et cheveux fins et cassants
\item ostéoarticulaire : petite articulations de la main. RX : athropathie sous
chondrale
\item HMG : risque CHC élevé
\item diabète tardif, hypogonadisme hypogonadotrope, cardiqaque
\end{itemize}

Si stade \(\ge\) 2, chercher atteinte pancréas, foie, coeur, gonades, os

Diagnostic des lésions hépatiques : biopsie hépatique \textpm{} IRM hépatique

Diagnostic génétique : individuel (cf supra) ou familal (apparenté 1er degré)

Traitement : 
\begin{itemize}
\item soustractions sanguines : 7mL/kg/semaine \(\ge\) 550mL tous les 7-10j puis 1-3
mois. Contrôle : PA, pouls, NFS. \danger arrêt si Hg < 11g/dL
\item chélateurs du fer sinon
\end{itemize}
Suivi des complications cirrhose si besoin

Pronostic \(\approx\) population générale avant cirrhose.

\subsubsection{Hépatosidérose métabolique++}
\label{sec:org449e6d7}
\gls{sdmetabolique} et surcharge en fe

Ferritinémie \inc, CST normal \thus IRM hépatique.

surcharge modérées mais les atteintes peuvent évoluer. 

PEC : troubles lipidiques, HTA, diabète, activité physique, équilbre alim. Pas
forcément normalisation Fe !!


\printglossaries
\section{248 : Dénutrition}
\label{sec:org5733b2c}
Référence nutritionnelle pour la population (couvre besoin). Sinon : apport
satisfaisant (apport moyen d'une population)

Besoins
\begin{itemize}
\item Eau : 2L/j \female{}, 2.5L/j \male{} (\(\approx\) 1L dans aliments)
\item Energie : 2400-2500 kcal (ou 35kcal/kg)
\begin{itemize}
\item base = 1500kcal/j \male{} (formule de Harris et Benedict)
\item thermorégulation (< 5\%), alimentation (10\%)
\item musculaire (15-30\%), 150-200 kcal
\item 1g glucide = 1g protide = 4 kcal; 1g lipide = 9 kcal, 1g alcool = 7 kcal
\end{itemize}
\item Nutriments : 10-20\% protides, 35-40\% lipides, 40-55\% glucides avec 0.75 g/kg/j
de protides et acide linoléique et acide alphalinolénique
\item Minéraux, vitamines : calcium, fer, folates, vitamines A, B1, B12, C, E, D
\item Âgé : \inc ea, calcium, vitamine D, folate. 30 kcal/kg/j
\end{itemize}

\subsection{Evaluation}
\label{sec:org3b25a70}
Clinique : 
\begin{itemize}
\item interrogatoire (contexte patho, activité physique, fatigabilité, poids
antérieur, anorexie, modification apports
\item EC : 
\begin{itemize}
\item épreuve du tabouret, \oe{}dèmes délives, peau, ongles, cheveux, lèvre,
langue, faciès, masse musculaire
\item poids vs poinds antérieur et idéal. IMC \(\in [18.5, 25] kg/m^2\) si < 70 ans
et \([21, 25] kg/m^2\) après\footnote{\danger Obésité sarcopéniques}
\item épaisseur cutanées tricipitale, périmètre bras au milieu
\end{itemize}
\item Biologie : 
\begin{itemize}
\item créatinine urinaire sur 24h (\(\approx\) masse musculaire)
\item albuminémie (\(t_{1/2} = 20\) j) : \dec si malnutrition, fuite rénale/digestive\footnote{Ou hémodiluton, insuffisance hépatocellulaire, sd infectieux/inflammatoire}
\item transthyrétine (\(t_{1/2} = 2\) j !)
\end{itemize}
\end{itemize}
Evaluation : protéino-énergétique ou protéique, aigüe ou chronique, grave ?

\subsection{Dénutrition}
\label{sec:org78e4931}
Rechercher chez tout malade en début d'hospitalisation et toutes les semaines :
\begin{itemize}
\item IMC insuffisant ou perte de poids de 2\% (1 semaine) ou 5\% (1 mois) ou 10\% (6
\end{itemize}
mis)
\begin{itemize}
\item doser albuminémie
\end{itemize}

Mécanisme = énergétique (balance énergétique négative), protéique (bilan azoté
  négatif) ou mixte
Causes : 
\begin{itemize}
\item \dec apport alimentaire = énergétique (marasme). 
\begin{itemize}
\item Volontaire, dysphagie, trouble déglutition, troubles digestifs
post-prandiaux
\item Maldigestion, malabsorption intestinale
\end{itemize}
\item Hypermétabolisme : affection septiques, néoplasique, inflammatoire
grave/étendues
\item Perte protéique anormale : cutanées, urinaires\footnote{Perte de plasma et protéines \thus hypoalbuminémie, \oe{}dèmes rapides}
\end{itemize}

Conséquence : sévère et épuisement des réserves lipidiques \thus pronostiv vital
!
\begin{itemize}
\item muscle (sarcopénie) \thus \dec force musculaire, \inc fatigabilité, \dec
ventilation
\item réponse aux aggresion inadaptée
\end{itemize}

PEC : 
\begin{itemize}
\item supplémentation alimentaire : débit énergétique basal \texttimes{} facteur
\item nutrition entérale (sonde dans tube digestif supérieur) par voie
nasogastrique, gastrostomie, jéjunostomie. 
\begin{itemize}
\item Bien tolérée
\item Broncho-pneumopathie d'inhalaion \thus faible débit et positino demi-assise
(nuit++)
\item 30 kcal/kg/j et 1.25 g/kg/j protéines chez dénutri chroniquue
\end{itemize}
\item nutrition parentérale : cathéter dans veine cave supérieure
\begin{itemize}
\item complication infectieuses (5-20\% patients) cher, difficile à domicile
\end{itemize}
\end{itemize}
À envisager si patho digestive, alimentation orale impossible/insuffisante,
pré-opératoire si dénutri

Privilégier nutrition entérale

\section{268 : Douleurs abdominales/lombaires aigües}
\label{sec:orge84ddc5}

\subsubsection{Clinique}
\label{sec:org319899a}
Irradiation : pointe omoplate droite (hépatobiliaire), épigastrique
(bilio-pancréatique), OGE (urologique)

Installation : brutale (perforation, embolie, rupture), rapide (obstacle,
ischémie, torsion), progressive (inflamation, infection, obstruction)

Exacerbé par marche, inspiration et calmée par décubitus : inflammation
intra-abdominale), soulagement par l'alimentation (ulcère), antéflexion
(pancréas), vomissement (obstruction, occlusion intestinale)

Déclenché par : alcool (pancréatite, hépatique alcoolique), médicaments (AINS,
aspirine), voyage récent

Terrain :
\begin{itemize}
\item signes généraux/organe
\item \female{} penser grossesse extra-utérine, gynéco
\item médicaments : AINS, aspirine (ulcère, gastrite aigüe), anticoagulant
(hématomes des muscle [paroi abdo. antérieure, psoas, paroi tube digestif),
corticoïdes
\end{itemize}

\subsubsection{Examen}
\label{sec:orgedd9e65}
\begin{tcolorbox}
Toujours chercher : cicatrice abdo, hernie, défense/contracture, toucher pelviens
\end{tcolorbox}

\begin{itemize}
\item Fièvre, FC, PA, choc
\item Ictère, pâleur, cyanose, cicatrice abdo++, hernie ballonnement, 0 mouvement
respi
\item palpation++
\item percussion : matité déclive/globe vésical/tympanisme (occlusion ou
pneumopéritoine)
\item auscultation : silence, bruits hydroaériques, souffle abdo (anévrisme de
l'aorte)
\end{itemize}
NB : irritation péritonéale : douleur à la décompression, défense/contracture
abdo, douleur toucher pelvien


\subsubsection{Examens complémentaires}
\label{sec:org3584305}
\begin{tcolorbox}
Aux urgences, penser : BU, \beta-HCG, ECG (IDM, péricardite)
\end{tcolorbox}

\begin{itemize}
\item Biologie : NFS, CRP, iono, hémoc si fièvre, lipasémie, \{transaminases, \(\gamma\)-GT,
phosphatases alcalines, bilirubine totale\} si clinique, BU, \(\beta\)-HCG,
\{hypercalcémie, acidocétose diabétique, insuf. surrénale aigüe\}, TP-TCA et
Rh-RAI si hémorragie digestive
\item ECG : infarctus antérieur, péricardite
\item RX pulmonaire si cause pleurale, pulmonaire
\item Echo si suspicion bilio-pancréatique, gynéco, urinaire, infection sous-abdo
\item \emph{Scanner abdo} : 1ere intention si diverticulite sigmoïdienne, occlusion par
obstruction, sd péritonéal
\end{itemize}

\subsection{Principaux tableaux}
\label{sec:org01a622d}
Douleur biliaire ou colique hépatique
\begin{itemize}
\item épigastre/hypochrondre droit (irradie épaule/omoplate droite)
\item forte intensité, plusieurs heures
\item aggravée par inspiration
\item signe de Murphy
\item causes : \emph{complications lithiase bilaire}, K vésicule/voie biliaire
principale, parasite, hémobilie
\end{itemize}
Douleur gastrique ou duodénale
\begin{itemize}
\item épigastre, torsion sans irradiation
\item parfois très intense, 30min à plusieurs heures
\item calmée par aliments, anti-acides, pansements gastrique
\item EC normal ou douleur provoquée du creux épigastrique
\item causes : maladie ulcéreuse gastrique/duodénale, K gastrique, dyspepsie fonctionnelle
\end{itemize}
Douleur colique
\begin{itemize}
\item épigastre/en cadre/fosses illiaque,/hypogastre (irradie le long du cadre
colique)
\item qq minutes-qq heures
\item calmée par selles/gaz++, antispasmodique
\item examen : douleur en cadre (colique)
\item causes : troubles fonctionnels intestinaux, K côlon, colites
inflammatoires/infectieuses/ischémiques
\end{itemize}
Douleur pancréatique
\begin{itemize}
\item épigastre/sus-ombilical, crampe (irradie dos)
\item très intense, "coup de poignants", pluiseurs heures
\item déclenchée par repas gras, alcool
\item soulagée par antéflexion, aspirine
\item examen : douleur provoquée épigastrique/périombilicale
\item causes : pancréatite aigüe/chronique, K pancréas
\end{itemize}
Ischémie intestinale aigüe = urgence avec pronostic vital  \footnote{Diagnostic difficile\label{org2da56b0}}
\begin{itemize}
\item y penser si terrain vasculaire
\item \emph{douleur abdo} : inaugurale aigǜe ou angor mésentérique puis diffuse rapidement
à tout l'abdomen. Intensite \inc{} sans répit
\item et \emph{terrain à risque}
\item angioscanner multibarretes en urgence
\end{itemize}
Ischémie intestinale chronique \textsuperscript{\ref{org2da56b0}}
\begin{itemize}
\item artérite oblitérante, athéromateuse++, inflammoire/radique
\item angor mésentérique avec douleurs abdo chroniques diffuses postprandiales
précoces 1-3h
\item peur alimentaire
\item perte poids
\item âgé, terrain CV
\end{itemize}

\subsection{Grandes causes}
\label{sec:org9868973}
Douleur épigastriques : 
\begin{itemize}
\item ulcère (hyperalgique, perforation), pancréatite aigüe
\item biliaire (colique hépatique, migration lithiase, cholécystite)
\item autres : aorte \footnote{Dissection/anévrisme}, cardiaque \footnote{Péricardite, infarctus postéro-inf}, pulmonaire \footnote{Pneumopathie infeciteuse, pleurésie}, digestive \footnote{Gastrite, \oe{}sophagite, appendicite aigüe}
\end{itemize}

Douleur de l'hypochondre droite :
\begin{itemize}
\item hépatobilaire : colique hépatique, cholécystite, angiocholite, tumeur/abcès du
foie, affections héaptique
\item ulcère perforé, appendicite sous-hépatique, abcès sous-phrénique,
pulmonaire, urinaire
\end{itemize}

Douleurs de l'hypochondre gauche (rare)
\begin{itemize}
\item queue du pancréas, ulcère gastrique, gastrite aigǜe
\item sd de l'intestin irritable, diverticulite de l'angle colique gauche
\item splénique, pleuropulmonaire, urologique, abcès sous-phrénique
\end{itemize}

Douleurs de l'hypogastre :
\begin{itemize}
\item gynéco, urologique, colique
\item appendicite pelvienne, diverticule de Meckel, sd de l'intestin irritable
\end{itemize}

Douleurs de la fosse illiaque droite :
\begin{itemize}
\item chirurgicale : appendicite, diverticule de Meckel, diverticulite (côlon
droit/sigmoïde), hernie étranglée, grossesse extra-utérine, torsion
d'annexe/fibrome utérin, anévrisme artériel iliaque
\item médicale : sd de l'intestin irritable, adénolymphite mésentérique, torsion de
frange épiploïque, iléite, salpingite, kyste ovarien, cystite, colique
néphrétique, abcès/hématome du psoas/grand droit
\end{itemize}
Douleurs de la fosse iliaque gauche
\begin{itemize}
\item sd de l'intestin irritable, colite diverticulaire, colite, K côlon gauche
compliqué, fécalome
\item grossesse extra-utérine, torsion d'annexe/fibrome, salpingite, colique
néphrétique, pyélonéphrite, cystite
\item anévrisme artérial iliaque, abcès/hématome du psoas/grand droit
\end{itemize}
Douleurs lombaires : 
\begin{itemize}
\item urologique, appendicite rétrocaecale, abcès/hématome psoas, fissuration anévrisme de l'aorte, douleur rachis
\end{itemize}

Douleurs abdominales diffuses 
\begin{itemize}
\item péritonite, occlusion, ischémie/infarctus mésentérique, médicales (cf supra)
\end{itemize}


\subsection{Douleurs pièges}
\label{sec:orgc1b1e69}
\begin{itemize}
\item IDM : douleur épigastrique, FR coronaire \thus ECG
\item Insuffisance surrénale : contexte, douleurs intenses diffuse, abdo souple, TR
indolore. HyperK, hypoglycémie. Urgence 
\item Hypercalcémie : urgence 
\item Acidocétose diabétique : diabète connu/situation révélatrice. Sd
polyuropolydipsique/trouble neur/Kussmau et haleine cétosique \thus
hyperglycémie, cétonurie, acidose
\item Acidocétose alcoolique : alcool et jeûne. Acidose métabolique, \inc corps
cétoniques, glycémie \dec
\item Maladie périodique (fièvre méditerranéenne) : jeune, ACTD crises. Sd
inflammatoire bio \thus mutation gène de la marénostrine
\item TRAPS sd \thus mutation gène TNFRSF1A
\item Sd hyper-IgD \thus doser IgD
\item Périhépatite : femme, péritonique, vénérienne. Sd inflammatoire, écho
hépatobiliaire normale.
\item Porphyrie hépatique aigüe intermittente : femme jeune avec
infection/médic. Urine rouge porto \thus doser acide delta aminolévulinique et PBG
\item Oedème angioneurotique : \oe{}dème récidivant dans l'enfants \thus diag = \dec
inhibiteur C1-estérase, C4, C3 normal
\item Sevrage opiacés, ingestion teoxiques, amphétamines, dérivés ergo de seigle,
intoxication au plomb
\item Purpura rhumatoïde
\item Autre vascularite : y penser si purpura
\item Drépanocytose : stress, effort physique, soleil. Anémie, hyperleucocytose
\item Phéochromocytome : crise brutale avec douleurs montantes et se finit par envie
impérieuse d'uriner abondamment
\item Douleurs rachidiennes projetées : zona, sd de Cyriax (compression du nerf intercostal)
\end{itemize}

\section{268 : Reflux gastro-\oe{}sophagien}
\label{sec:orgb56dc4c}

\gls{RGO} = passage à travers le cardia d'une partie du contenu gastrique hors
effort. Pathologique = symptômes/lésions d'oesophagite

20-40\% des adultes ont un pyrosis

Mécanismes :
\begin{itemize}
\item défaillance du sphincter inférieur de l'\oe{}sophage
\item hyper pression abdo, stase gastrique
\item hernie hiatale = glissement ou roulement mais pas de lien avec RGO
\end{itemize}

Signes fonctionnels :
\begin{itemize}
\item pyrosis, régurgitations = quasi pathognomonique
\item extra-digestifs (pulmonaire, ORL, cardiaque)
\item RGO compliqué\footnote{Sévérité des symptômes indépendantes de l'intensité des lésions !} : ulcérations du bas \oe{}sophage (étendues,
confluentes/circonférentielles) \thus risque = hémorragi digestive, sténose
\oe{}sophagienne
\end{itemize}

Endobrachy\oe{}sophage : épithelium normal de l'\oe{}sophage remplacé par épithelium
métaplastique intestinal \thus surveillance régulière (risque d'ulcère,
dysplasie, adénocarcinome)

\subsubsection{CAT}
\label{sec:org47a2b34}
Interrogatoire : éliminer sd de rumination (pas de brûlure, nausée,
vomissement).

< 50 ans, pyrosis ou régurgitation, 0 dysphagie, 0 amaigrissement, 0 anémie :
pas d'examen complémentaire

Sinon : endoscopie OGD 
\begin{itemize}
\item 1ere intention > 50 ans ou atypique
\item affirme diagnostic si perte de substance épithéliale, peu profonde
\item sans anomalies et (pré-chirurge ou persistance ou extradigestive), faire
ph-métrie des 24h
\end{itemize}
Autres : impédancemétrie \oe{}sophagienne (reflux persistant), manométrie
\oe{}sophagienne (si opération)

\subsubsection{Traitement}
\label{sec:org6e7373f}
Anti-acides, inhibiteurs de la sécrétion gastrique (IPP), protection de la muqueuse
\oe{}sophagienne (alginates)

\dec{} poids, arrêt tabac et alcool, tête de lit à 45\(^{\circ}\), 3h entre diner
et coucher

\begin{itemize}
\item RGO sans \oe{}sophagite: anti-acides/alginates/anti-H2 si < 1/semaine. Sinon
IPP demi-dose puis long cours
\item RGO avec \oe{}sophagite : IPP demi dose (ou pleine si sévère) 4 semaines
(ou 8) puis long cours
\item RGO principalement extra-digestif : pas d'IPP
\item RGO résistants IPP : non acide, erreur diag, \inc dose. Sinon chir
\item Sténose peptique : IPP pleine dose
\item Endobrachy\oe{}sophage : IPP si symptomatique ou \oe{}sophagite
\end{itemize}

\subsubsection{Chirurgie}
\label{sec:org34208aa}
Fundoplicature complète (Nissen)

\section{269 : Ulcère gastrique et duodénal}
\label{sec:orgdafa4e6}

\subsection{Ulcère gastrique et duodénal}
\label{sec:org18bdddb}
Déf: perte de substance en profondeur (jusque musculeuse). Chronique si socle
scléro-inflammatoire
\begin{itemize}
\item ulcères gastriques = altération des mécanismes de défense
\item ulcères duodénaux = altération des mécanismes de défense ou hypersécrétion
\end{itemize}

\subsubsection{Types}
\label{sec:org2b73141}
\begin{itemize}
\item \bact{helicobacter} : 
\begin{itemize}
\item oro-orale/féco-orale, pendant l'enfance
\item pays en voie de développement. Incidence \dec dans pays développés.
\item Gastrite aigüe \thus chronique. Complications (rare) : ulcère G ou D,
(adénocarcinome gastrique, lymphome)
\end{itemize}
\item AINS (inhibition de COX-1), aspirine faible dose
\item Sd Zollinger-Ellison : sécrétion tumorale de gastrine (exceptionnel)
\item Autre (20\%) :  gastrotoxique, tabac, Chron, vascularite
\end{itemize}
Autre facteur : génétique possible mais pas de stress, ni facteurs psycho

Épidémio : 0.2\% en France. Sex ratio H/F = 2 (UD) ou 1 (UG). 10\% de mortalité
des complications. 1/3 liés à l'aspirine/AINS

\subsubsection{Diagnostic}
\label{sec:orgc93ebea}
Symptômes :
\begin{itemize}
\item typique : douleur épigastrique sans irradiation, crampe/faim. Calmée par
aliments/anti-acides. Rythmée par repas. Poussées de qq semaines avec périodes asymptomatiques
\item atypique (+ fréquent) : sous-costal/postérieur, hyperalgique/frustre, non
rythmé par l'alimentation
\item asymptomatique
\item complication inaugurale
\end{itemize}

EC normal (si pas de complications)

Endoscopie digestive haute\footnote{Anesthésie locale pharyngée/générale. Morbidité et mortalité faibles} :
\begin{itemize}
\item perte de substance profonde, branchâtre, ovalaire, à bords régulieurs
\item gastrique : surtout antre. Biopsies systématiques
\item duodénal : buble, pas de biopsie de l'ulcère
\end{itemize}

\begin{tcolorbox}
Quelque soit l'ulcère, biopsie de l'antre, angle et fundus pour \bact{helicobacter}
\end{tcolorbox}

Recherche de \bact{helicobacter} :
\begin{itemize}
\item biopsie gastrique = référence
\item autre : teste respiratoire à l'urée = contrôle d'éradication hors endoscopie
\end{itemize}


\subsubsection{DD}
\label{sec:orgb4ef521}
\begin{itemize}
\item avant endoscopie : adénocarcinome/lymphome gastrique, douleur
pancréatique/biliaire, insuffisance coronarien, péricardite, ischémie
mésentérique, douleur vertébrale projetée, dypepsie non ulcéreuse
\item endoscopie : adénocarcinome, lymphome, Crohn
\item ulcère de stresse : en réa avec \(\ge\) 1 défaillance viscérale
\end{itemize}

\subsubsection{Complication}
\label{sec:orgdaae5c4}
\begin{itemize}
\item hémorragie digestive (fréquente) : à bas bruit ou aigüe, mortalité 10\% \thus
endocscopie \textpm{} hémostase
\item perforation ulcéreuse : plutôt en péritoine libre : douleur épigastrique en "coup de
poignard", choc, contracture épig. puis généralisée, cul-de-sac de Douglas
douloureux au TR, pneumopéritoine au scanner. CI à l'endoscopie \danger
\item sténose ulcéreuse (exceptionnelle) : clapotage gastrique à jeun, ondes
péristaltiques \thus évacuer stase puis diagnostic sur endoscopie
\item gastrite aigüe - atropie / métaplasie - dysplasie - cancer invasif
\end{itemize}


\subsubsection{Traitement UGD non compliqué}
\label{sec:org52cb0b0}
Associé à H. pylori 
\begin{itemize}
\item éradication : 
\begin{itemize}
\item si sensibilité : IPP matin et soir et amoxicilline 10j et clarithromycine (ou
lévoflaxicine si résistance)
\item sinon métronidazole, tétracycline, sous-citrate de bismuth et oméprazole
\end{itemize}
\item IPP en curatif avant éradication \emph{et} (6 semaines en curatif si ulcère ou AINS
ou persistance des douleur ou ulcèse GD compliqué) \emph{et} 6 semaines en préventif
\item Surveillance à +4 esmaines par teste respiratoire (UD) ou biopsie gastrique
(UG). 
\begin{itemize}
\item Si réussite : pas d'IPP long cours
\item Si échec : quadrithérapie ou endoscopie et ABTgramme. Si échec : IPP long
cours
\end{itemize}
\item Chir exceptionnelle : ()éliminer sd Zollinger-Ellison) vagotomie \textpm{} hémostase
\end{itemize}

Induite par AINS/aspirine
\begin{itemize}
\item curatif : 4 semaines (UD) ou 8 (UG). Maintenir IPP si AINS/aspirine
continué. Contrôle endoscopique
\item préventif si FR = > 6) ans, ATCD UGD, AINS avec antiagrégant/corticoïdes/anticoagulants
\end{itemize}
Autre
\begin{itemize}
\item éliminer Zollinger-Ellison, Crohn, lymphome, K gastrique
\item UD = IPP 4 semaines \textpm{} IPP long cours
\item UG = IPP 4-8 semaines \textpm{} IPP voire chir si échec
\end{itemize}


\subsubsection{Traitement UGD compliqué :}
\label{sec:orgf1748ee}
\begin{itemize}
\item Hémorragique, perforcé : ch chap "Hémorragie digestive", "Péritonite aigüe"
\item Sténose ulcéreuse pylorobulbaire :
\begin{itemize}
\item évacuer stage, correction torubles hydroélectriques, IPP
\item si échec : dilatation sténose et endoscopie. Si échec : chir
\end{itemize}
\end{itemize}

\subsection{Gastrite}
\label{sec:orgb95904d}
Définition : atteinte inflammatoire

Pas de corrélation histologie - symptômes

\subsubsection{Chronique à \bact{helicobacter}}
\label{sec:org4ae93bc}
20-50\% population. Évolution selon :
\begin{itemize}
\item sujet hypersécréteur : gastrite antrale, risque UD
\item sujet hyposécréteur : pangastrite, risque UG, adénocarcinem gastrique
\item lymphome gastrique MALT
\end{itemize}
Diag = biopsise de l'antre et du corps. Ttt = éradication (cf supra)

\subsubsection{Chronique de mécanisme immunitaire}
\label{sec:org5de089c}
\begin{itemize}
\item Auto-immune : si atrophie fundique sévère : 
\begin{itemize}
\item carence facteur intrinsuqèe \thus
anémie macrocytaire arégénérative, glossite, neuro
\item carence martiale
\item risque adénocarcinome, tumeurs endocrine \thus surveillance 3 ans si < 70
ans et bon état général
\item corriger carence B12 et fer
\end{itemize}
\item Lymphocytaire : souvent asymptomatique
\item Granulomateuse : granulomes épithélioïdes
\item À éosinophiles
\item Avec maladie de Crohn (30\% des atteints)
\end{itemize}

\subsubsection{Gastrite aigüe}
\label{sec:org4332836}
\begin{itemize}
\item \bact{helicobacter}
\begin{itemize}
\item immédiatement après contaminale, souvent asymptomatique, lésions de l'antre.
\item Diag = biopsie : \bact{helicobacter}, PNN
\end{itemize}
\item Phlegmoneuse : exceptionnelle, ID
\item Virale : ID
\end{itemize}


\subsubsection{DD}
\label{sec:orga1b4f0f}
\begin{itemize}
\item Gastropathie induite par AINS
\item Gastropathie chimique : excès d'alcool, post-gastrectomie (reflux)
\item Gastropathie congestive : hypertension portale ou ectasies vasculaires
atnrales
\item Gastropathies hypertrophiques : échoendoscoie si muqueuse épaissie
\begin{itemize}
\item maladie de ménétrie (exceptionnelle, étiologie inconnue) : exsudative, sd
\oe{}démateux
\item sd de Zollinger-Ellison
\end{itemize}
\item Gastropathie radique (> 45 Gy) : multiples biopsies, chronicité possible
\end{itemize}

\section{270 : Dysphagie}
\label{sec:org719aea7}
Déf : sensation de gêne/obstacle à la progression du bol alimentaire
(déglutition). \(\ne\) odynophagie, striction cervicale (anxiété), anorexie, satiété
précoce

2 types : oropharyngé (cervicale) et \oe{}sophagienne (rétrosternal)

\subsection{Dysphagie \oe{}sophagienne}
\label{sec:org6a2e44b}
\begin{enumerate}
\item Interrogatoire : 
\begin{itemize}
\item anorexie, asthénie, \emph{amaigrissement}, \emph{localisation}, solides
ou liquide, brutal ou progressif, évolution, terrain (âge, \emph{tabac-alcool},
agents irritants, affection maligne), symptômes
\item score d'Eckart pour quantifier
\end{itemize}
\item Chercher lésion organique de l'\oe{}sophage :
\begin{itemize}
\item endoscopie OGD 1ere intention (toujours biopsie pour \oe{}sophagite à
éosinophiles !)
\item autres : scanner thoracique (lésions médiastin), échoendoscopie (paro
\oe{}sophage), transit baryté de l'\oe{}sophage (K de l'\oe{}sophage,
diverticule, achalasie)
\end{itemize}
\item Si endoscopie normale, chercher trouble moteur : manométrie \oe{}sophagienne
\end{enumerate}
(haute résolution)

\subsection{Dysphagies lésionnelles}
\label{sec:org8e61c67}
Prédomine sur les solides, s'aggrave, retentit état général. Sténoe organique en
majorité.

Principales lésions organiques :
\begin{itemize}
\item sténoses tumorales
\begin{itemize}
\item carcinome épidermoïde (terrain alcool-tabac)
\item adénocarcinome \oe{}sophagien sur endobrachy\oe{}sophage
\item tumeurs bénignes de la paroi
\item compression extrinsèque (tumeur, ADP médiastinale, carcinome bronchopulmonaire)
\end{itemize}
\item sténoses non tumorales
\begin{itemize}
\item peptique
\item caustique
\item post-radique
\item après \oe{}sophagectomie
\item après résection endoscopique étendue
\item pendant \oe{}sophagite
\item anneaux de Schatzki (aspect de diaphragme)
\item sd de Plummer-Vinson/Kelly-Paterson
\item compression extrinsèque (ADP, anomalie artérielle)
\end{itemize}
\item \OE{}sophagite sans sténoe : médicaments, à éosinophiles\footnote{Endoscopie peut être normale !}, infectieuses
(ID)
\item Diverticule de Zenker
\end{itemize}


\subsection{Dysphagies non lésionnelles}
\label{sec:org8f6011b}
Liquides et solides, fluctue dans le temps. Endoscopie souvent normales

\begin{itemize}
\item Achalasie : trouble moteur d'origine inconnue avec relaxation du sphincter
inférieur de l'\oe{}sophage et sans contractions péristaltiques
\begin{itemize}
\item surtout les liquides, capricieuse, régurgitations. Endoscopie : \oe{}sophage
dilaté (stade évolué !), méga-\oe{}sophage, cardia en "bec d'oiseau"
\item Anomalies manométriques \oe{}sophagienne : type I (défault relaxation de la
joncion), type II (pressurisation \oe{}sophage), type III (contractions
\oe{}sophagiennes prématurées)
\item DD : toujours endoscopie (néoplasie !) : tumeur infiltrante du cardia
\end{itemize}
\item Autres troubles moteurs : 
\begin{itemize}
\item douleurs thoraciques peudo-angineuses : spasmes \oe{}sophagiens, \oe{}sophage
hypercontractile
\item RGO et clairance oesophage \dec : péristaltisme inefficace ou absent
\end{itemize}
\end{itemize}

\section{271 : Vomissements}
\label{sec:org322168c}
Nausées et vomissement : souvent activation des SN sympathiques et
parasympathiques

DD : régurgitations (remontée sans effort), rumination/mérycisme (remontée
volontaire)

Physio: 
\begin{itemize}
\item centre = substance réticulée du tronc cérébral
\item stimulation : plancher 4eme ventricule, cortex cérébral, vestibulaire, nerfs
vagues et sympathiques
\item efférences motrice
\end{itemize}

Complications :
\begin{itemize}
\item troubles hydroélectriques : déshydratation puis IR, hypochlorémie, alcalose
métabolique, hypoK
\item sd de Mallory-Weiss\footnote{Déchirurge longitudinale du cardia} : vomissement puis hématémèse
\item rupture de la paroi de l'\oe{}sophage : exceptionnel, urgence . Dyspnée,
emphysème SC, odynophagie \thus diag par TDM thoracique
\item inhalation bronchique avec pneumopathie
\item \oe{}sophagite
\item hémorragie sous-conjonctivale, fracture de côte, dénutrition, interrution ttt
oraux
\item encéphalopathie Gayet Wernicke : urgence, femmes enceintes
\end{itemize}

Aigus si \(\ge\) 7 jours

\subsection{Diagnostic}
\label{sec:org4993d75}
Sémiologie :
\begin{itemize}
\item matin, à jeun, liquide un peu glaireux, haut-le-c\oe{}ur : alcool, médic,
tabac, grossesse
\item matin, en jets, sans nausée ni haut-le-c\oe{}ur : HTIC (rare)
\item postprandial, répétés, aliments nauséabonds partiellement digérés :
obstruction chronique GD organique/fonctionnelle
\item fécaloïdes : obstruction basse, fistule gastrocolique (exceptionnel)
\item perprandiaux/juste après repas : psychogène (par élimination !)
\item fin de journée : sd obstructif, avec nausées, balonnements, satiété, crampes
\end{itemize}

Médicaments : antimitotiques, dérivés théophylline, digitaliques

\begin{tcolorbox}
Toujours penser à : grossesse, surdosage, intolérance médicaments/toxique, métabolique, HTIC
\end{tcolorbox}

Chercher complications, dénutrition, toxiques

Si déshydratation, perte de poids, AEG, vomissement ou personnes à risque :
\begin{itemize}
\item iono sanguin, NFS, urée, créat, iono urinaire
\item albuminémie, préalbuminuméie
\end{itemize}
Si vomissements chronique : endoscopie EOG. Éventuellement : scinti de vidange
gastrique, scanner (opacification digestive haute), IRM cérébrale

\subsubsection{Vomissements aigüs}
\label{sec:org453d311}
Causes :
\begin{itemize}
\item abdominopelviennes :
\begin{itemize}
\item medicale : \emph{gastroentérite virale, toxi-infection alimentaire}, hépatite
aigüe, sténose du pylore, colique hépatique/néphrétique
\item chir : douleur biliaire, pancréatite aigüe, infarctus mésentérique, torsion
kyste de l'ovaire, grossesse extra-utérine
\end{itemize}
\item médic
\item neuro : vestibulaire, migraine, trauma cérébral, méningite, HTIC, hémorragie
méningée
\item métabolique : acidocétose diabétique, IR aigüe\footnote{Piège !\label{org450d505}}, hyperCa, hypoglycémie/malaise
vagal, insuf. surrénale aigüe, hypoNa, hyperthyroïdie
\item autre : \emph{grossesse}, postop, mal des transports, glaucome aigue\textsuperscript{\ref{org450d505}}, IDM
inférieur\textsuperscript{\ref{org450d505}}, radiothérapie, psychogène
\end{itemize}

\subsubsection{Vomissements chronique}
\label{sec:orgc1dae74}
\begin{itemize}
\item Digestif supérieur : 
\begin{itemize}
\item mécanique : ulcère, K gastrique/duodénal, K pancréas, compression par
pseudo-kyste pancréatique
\item fonctionnelle : gastroparésie, chir gastrique, vagotomie
\end{itemize}
\item Intestion, côlon :
\begin{itemize}
\item obstruction mécanique tumorale
\item sténose mécanique non tumorale
\item fonctionnelle
\end{itemize}
\item SNC : HTIC, épilepsie
\item Psychogène
\item Autres : cataméniaux\footnote{Pendant menstruations}, sd des vomissements cyclique\footnote{Sans organicité ni toxiques}, sd d'hyperemesis aux cannabinoïdes
\end{itemize}
Grossesse
\begin{itemize}
\item 1er trimestre : fréquents, physiologique si pas d'AEG. Hyperemesis gravidarum
(forme grave) : anomalies disparaissent à l'arrêt
\item 3eme trimestre : non lié ou \{stéatose aigüe gravidique, prééclampsie\} =
urgences 
\end{itemize}
Chimio : aigü (< 24h), retardés, anticiéps

\subsection{Traitement}
\label{sec:org1611ee8}
Hospitalisation urgrente : urgence médicale/chir/obstétricale, trouble
hydroélectriques, trouble conscience, réhydratation orale impossible, ttt oral
indispensible impossible, décompensation, complication

Symptomatique :
\begin{itemize}
\item réhydratation per os/IV
\item sonde gastrique si risque d'inhalation
\item surveillance (déshydratation, FC, PA, diurèse, iono sanguin)
\item médic: 
\begin{itemize}
\item métoclopramide (neuroleptique, \(\in\) benzamides) : CI : dyskinésie des
neuroleptiques, phéochromocytome, alcool, lévodopa
\item dompéridone (neuroleptique, \(\in\) butyrophénones)
\item métopimazine (\(\in\) phénothiazines)
\end{itemize}
\item chimio : anti-5-HT3, aprépitant, corticoïdes, métoclopramide, alizapride
\end{itemize}

\section{273 : Hépatomégalie, masse abdo}
\label{sec:orgb23be14}

\subsection{\gls{HMG}}
\label{sec:orgde77426}
Projection ligne médio-claviculaire > 12cm\footnote{Entre limite supérieure (percussion) et bord inférieur (palpation)}. Écho abdominable si besoin.

DD : tumeur du rein/angle colique droit/estomac/pancreas => mobile avec
respiration échographie

\begin{table}[htbp]
\caption{Causes d'HMG}
\centering
\begin{tabular}{ll}
\toprule
Augmentation de volume & Causes\\
\midrule
diffuse, homogène & hépatite, cirrhose\\
 & stéatose, stéatohépatique \hfill [foie hyperéchogène, alcool/sd métabolique]\\
 & cholestase prolongée \hfill [ictère cholestatique/voie biliaires dilatées]\\
 & foie cardiaque \hfill [veines hépatiques dilatées]\\
 & sd Budd-Chiari, surcharge en fer, abcès du foie, autres\\
sectorielle, homogène & cirrhose, autres\\
hétérogène & cirrhose, tumeurs bénignes (kyste biliaire simple/hydatique)\\
 & polykystose hépatique, abcès du foie\\
 & tumeurs malignes (métastases foie, CHC, autres)\\
\bottomrule
\end{tabular}
\end{table}

\subsubsection{Moyens diagnostiques :}
\label{sec:org2844cfa}
\paragraph{Clinique}
\label{sec:orge92f6e0}
ATCD, symptômes, FR maladie du foie. Et :
\begin{itemize}
\item foie douloureux, sd inflammaatoire
\item angiomes stellaires, hypertension portale\footnote{Maladie chronique du foie}
\item reflux hépatojugulaire, expansion systolique du foie
\end{itemize}
NB foie cirrhotique = dur

\paragraph{Écho}
\label{sec:orgf9a7293}
sans attendre ! : diffuse/sectorielle, homogène/hétérogène, cirrhose,
stéatose, insuf cardiqaue droite, dilatation voies biliaires intrahépatiques

\paragraph{Autres :}
\label{sec:orga548179}
\begin{itemize}
\item hémogrammes, \{transaminases, phosphatases alcalines, \(\gamma\)-GT, TP,
bilirubinémie, électrophorèse des protéines plasmatique\footnote{Et si besoin : hépatetites virales/auto-immunes, surcharge en fere,
amibiases (abcès), hydatidose (kyste)}
\item écho Doppler/contraste, TDM, IRM, écho cardiaque si besoin
\item ponction-biopsie hépatique si cause introuvable \footnote{Pas pas voie transpariétale/transcapulaire si troubles de l'hémostase
non corrigés/dilatation diffues des voies biliaires intra-hépatiques/ascite}
\end{itemize}

\subsubsection{Démarche : HMG hétérogène}
\label{sec:orgfe293df}
\begin{enumerate}
\item Maladie chronique du foie ?
\item Si oui, CHC ? (centre spécialisé)
\item Sinon : 
\begin{itemize}
\item lésion kystique : si liquidienne sans paroi ni cloison \thus kyste biliaire
symple. Sinon centre spécialisé
\item tumeur solide :
\begin{itemize}
\item rehaussement périphérie - centre : hémangiome bénin
\item rehaussement périphérie (temps artériel) : abcès/tumeur nécrosée
\item rehaussement : métastase/adénome hépatocellulaire
\item rehaussement (temps artériel) :  CHCH ou autre \thus milieu spé
\end{itemize}
\end{itemize}
\end{enumerate}

\subsection{Masse abdominale}
\label{sec:org86a8b16}
Interrogatoire : découverte, date et évolution, SF, ATCD med et chir, ttt
(anticoag)

Examen : éliminer éventration, hernies, distension abdo. Localisation, taille,
forme, contours, consistante, mobile, percussion, auscultation (souffe).
Cherche métastase

Imagerie : écho abdo en 1ere intention. \emph{TDM = examen clé} \footnote{Éliminer grossesse !}

\subsection{Hypothèses}
\label{sec:org31e5f9f}
Épigastre :
\begin{itemize}
\item tumeur gastrique : masse pierreuse, AEG, signes digestifs hauts \thus diag =
endoscopie + biopsies
\item tumeur pancréatique : tête = ictère, prurite, corps = douleurs solaires, queu
= masse épigastrique/hypochondre gauche. AEG
\item pseudo-kyste du pancréas : contexte pancréatite
\end{itemize}
Hypochondre droit
\begin{itemize}
\item HMG
\item Grosse vésicule \footnote{Non palpable à l'état normal} : tumeur maligne pancréatique (ictère précédé d'un
prurit, hydrocholécyste (écho = diag), cholécystite aigüe (fébrile), tumeur
maligne d la vésicule (masse dure, fixée, irrégulière)
\item Autre : lésion angle colique D, rein D, surrénale D
\end{itemize}
Hypochondre gauche :
\begin{itemize}
\item \gls{SMG} : s'abaisse à l'inspiration, bord antérieur crénelé
\item queue du pancréas, angle colique gauche, grosse tubérosité gastrique, rein
gauche
\end{itemize}
Fosse iliaque droite :
\begin{itemize}
\item tumeur du c\ae{}cum : masse abdo, anémie ferriprive, méléna \thus diag =
coloscopie + biopsie
\item appendicite : si abcès, tuméfaction douloureuse fixée, fébrile. TDM si besoin
\item Crohn avec abcès
\item kyste de l'ovaire
\end{itemize}
Fosse iliaque gauche
\begin{itemize}
\item sigmoïdite avec abcès périsigmoïdien : douleur fosse iliaque gauche, troubles
transit, fièvre \thus diag = TDM
\item tumeur sigmoïdienne, kyste de l'ovaire
\end{itemize}
Région ombilicale
\begin{itemize}
\item anévrisme de l'aorte abdo : tuméfaction, battante, souffle systolique \thus
diag  angioscanner
\item K côlon transverse, tumeurs du grêle, tumeur mésentérique
\end{itemize}
Flancs : lésions rénale, psoas

Hypogastre : éliminer fécalome, globe vésical, grossesse
\begin{itemize}
\item fibromyome utérine : ménorragie, pesanteur pelvienne, pollakiurie. Masse
régulière, bien limitée, ferme indolore \thus diag = écho pelvienne
\item K endomètre : métrorragie post-ménopausique \thus diag = gynéco + biopsie
\item tumeur de l'ovaire : douleur pelvienne, pesanteur, ascite, palpation masse
pelvienne
\end{itemize}

Ubiquitaire : tuméfaction pariétale, nodules de carcinose péritonéale, ADP, corps étranger

\section{274 : Lithiases biliaires}
\label{sec:orgb6a93b9}
Fréquence : 20\% (Occident) et 60\% après 80 ans.

3 types :
\begin{itemize}
\item calculs cholestéroliques : 
\begin{itemize}
\item favorisés par \inc sécrétion biliaire de
cholestérol, défaut des facteurs le solubilisant, rétention vésiculaire
\item FR : âge, \female, surpoids, multiparité, jeûne prolongé, ethines, hyperTG,
certains médic
\end{itemize}
\item pigmentaire : déconjugaison bilirubine. FR = \inc production bilirubine,
infection ou obstacles biliaires
\item mixtes
\end{itemize}

\emph{Pas de dépistage}

\subsection{Lithiase vésiculaire symptomatique}
\label{sec:org6df10ef}
Typique = colique hépatique : douleur brutale, permanente, épigastre ou
hypochondre droit, irradiant vers l'épaule/fosse lombaire droite, qq min à qq
heures. Chercher un signe de Murphy

Bio = RAS. \emph{Échographie}

\subsection{Lithiase vésiculaire compliquée}
\label{sec:org0c4e857}

\subsubsection{Cholécystite aigüe\footnote{Infection aigüe de la vésicule (ici par obstruction prolongée !)}}
\label{sec:orga04f1b6}
Sd infectieux, douleur hypochondre droit \uline{> 6h} (> 24h), frissons \textpm{} défense,
contracture (= grave)

Bio : hyperleucocytose à PNN, hépatique normal

Imagerie : écho : paroi vésiculaire > 4mm

Complications : 
\begin{itemize}
\item gangrène paroi vésiculaire \thus perforation dans le foie ou
péritoine
\item ileus biliaire\footnote{Fistule biliodigestive \thus occlusion tube digestif par le calcul}, sd Mirizzi\footnote{Gros calcul dans le collet vésiculaire/canal cystique. Ictère,
dilatation voies biliaires, perturbation bilan hépatique}, K vésiculaire
\end{itemize}


\subsubsection{Migration lithiasique}
\label{sec:orgabde85b}
Douleur colique sans fièvre. \inc \uline{transitoire} transaminases

\subsubsection{Angiocholite aigüe}
\label{sec:org9f562eb}
Infection aigùe \gls{VBP} (généralement calcul, parfois parasite). Sous 48h :
\emph{douleur biliaire, fièvre, ictère}.
Sd infectieux parfois sévère

Bio : \uline{\inc bilirubine conjugée, \inc transaminases}, hyperleucocytose PNN

Écho (moyennement sensible) : cholangio-IRM, échoendoscopie (sensibles) 

Complications : choc septique, + IR

\subsubsection{Pancréatite aigüe (cf chap)}
\label{sec:org5a66e78}

\subsection{Traitement}
\label{sec:org074ed90}
\begin{itemize}
\item Asymptomatique : non
\item Colique hépatique : \danger urgence : antispasmodique, antalgiques,
anti-inflammatoire. Puis cholécystectomie < 1 mois
\item Cholécystite aigüe : remplissage vasc, ATB proba (germe digestif : (amoxicilline+
acide clavulanique) ou (CG3 + imidazolé)) puis adaptée. antalgique.
Cholécystectomie < 72h
\item Angiocholite : immédiatement ATB proba, décompression dans 24h (voire urgente si
ne répond pas) \thus \gls{CPRE} 1ere intention (extraction du calcul par
sphinctérotomie, 5-10\% de complications)
\item Calculs VBP hors angiocholite : calculs prédictifs, différentes approches
\item Pancrétite aigüe biliaire : si angiocholite aigüe en plus, ATB et extraction
calculs < 24h
\end{itemize}

\section{275 : Ictère}
\label{sec:orge11d20a}
Coloration jaune quand bilirubinémie > 40\$\(\mu\)\$mol/L

Physiopatho : 
\begin{itemize}
\item dégradation hémoglobine -> bilirubine dans le plasma (non conjugée surtout)
\item transportée par l'albumine dans les hépatocytes puis conjugée par \gls{BGT}
\item puis sécrétée dans la bile (mais une partie revient dans le plasma)
\end{itemize}

\begin{figure}[htpb]
  \centering
  \resizebox{0.5\linewidth}{!}{
     \tikz \graph [
    % Labels at the middle 
    edge quotes mid,
    % Needed for multi-lines
    nodes={align=center},
    edges={nodes={fill=white, align=center}}, 
    patho/.style={rectangle, draw=black},
    h path/.style = {to path={ - (\tikztotarget)}},
    layered layout]
    {
      "" ->  {
      	 "Urine claire\\b. non conjuguée" -> 
         "Ictère à bilirubine\\non conjugée" [patho] ->
         "\inc{} réticulocyte \dec{} haptoglobine ?"[level distance=50pt] -> {
             Hémolytique [>"oui", patho];
             "Non hémolytique" [>"non", patho];
           };
           "Urine brune\\b. conjuguée" -> "Ictère à bilirubine\\conjugée"[patho] -> {
           "Prurit\\ \inc{} phosphatases alcalines, $\gamma$ -GT" -> {
             "Cholestatique" [>"oui", patho] -> Imagerie -> {
	       "Avec obstacles\\sur gros canaux"[patho];
	       "Sans obstacles\\sur gros canaux"[patho];
	     };
             "Non cholestatique" [>"non", patho];
           };
	  };
        };
    };
  }
  \caption{Orientation devant un ictère}
\end{figure}

\subsection{Étiologies}
\label{sec:org10df51e}
\begin{itemize}
\item Ictère à bilirubinie non conjugée
\begin{itemize}
\item hémolyse, dysérythropoïèse\footnote{Destruction intra-médullaire des nouvelles hématies :}
\item \dec conjugaisons par la BGT
\begin{itemize}
\item sd de Gilbert : bénin, fréquent \thus diagnostic = ictère non persistant,
tests hépatiques \emph{normaux} et élimination autres causes
\item sd Crigler-Najjar : exceptionnel, très grave : ictère néonatal marqué permanent
\end{itemize}
\end{itemize}
\item Ictère à bilirubine conjugée
\begin{itemize}
\item cholestase++ (\dec sécrétion biliaire)
\begin{itemize}
\item obstruction canaux biliaires
\begin{itemize}
\item \gls{VBP} (freq) : \emph{K pancréas} (ictère, AEG), \emph{K primitif VBP} (ictère),
\emph{litihiase VBP} (précédée douleurs), sténose post-op voie biliaire,
pancréatite chronique calcifiante (par compression), ADP
\item atteinte des petits/moyens canaux : cirrhose biliaire primitive
(auto-immune, rare), cholangite immunoallergique (amox-acide
clavulanique, sulfamide, macrolide, allopurinol), cholangite scérosante
primite (rare, faire cholangio-IRM)
\end{itemize}
\item sans obstacle : génétique (très rare : infantile, "récurrente béningne" ou
gravidique") ou acquises (hépatite aigües, infections bactériennes, angiocholite)
\end{itemize}
\item transport canaliculaire de bilirubine conjugée (rarissime) : sd de Rotor,
maladie Dubin-Johnson
\item multiples mécanismes
\end{itemize}
\end{itemize}

Mnémotech: la bilirubine non conjuguée ne passe pas les urines \thus urine claire

\subsection{Urgences}
\label{sec:orgd6d11a6}
\begin{itemize}
\item Encéphalopathie bilirubinique du nouveau-né : séquelles cognitives/motrices
graves \thus ttt par UV/échanges pasma en urgence 
\item Angiocholite : cf table \ref{tab:org8a00f57}. Traitement :
\begin{enumerate}
\item ATB ASAP (bactéries intestinales), corrections désordres généraux
\item drainage endoscopique : sous 48h si aggravation ou en urgence si choc
septique 
\end{enumerate}
\item Insuffisance hépatique :
\begin{itemize}
\item cirrhose
\item K foie en phase terminale : confort du patient
\item insuf. hépatique aigüe : 
Quick et facteur V < 50\%
\begin{itemize}
\item transaminases > 20N \thus diagnostic = \dec taux
\item risque = insuf hépatique grave : 80\% mortalité 
\item chercher paracétamol systématiquement \thus N-acétylcystéine en urgence
\item si grave : transplantation
\end{itemize}
\end{itemize}
\end{itemize}

\begin{table}[htbp]
\caption{\label{tab:org8a00f57}
Diagnostic d'angiocholite}
\centering
\begin{tabular}{lll}
\toprule
Suspicion & Diagnostic & DD\\
\midrule
triade "douleur-fièvre-ictère" & cholestase & infections bactériennes sévères\\
 & sd inflammatoire systémique marqué & sd inflammatoires de lymphomes\\
 & obstruction voies biliaire : \emph{échographie} & hépatite herpétique/virale A\\
\bottomrule
\end{tabular}
\end{table}

\subsection{Imagerie}
\label{sec:org0697a26}
\begin{itemize}
\item Échographie : diagnostic pour lithiase biliaire, obstruction VBP (et siège de l'obstacle)
\item TDM : plus sensible que l'écho pour pancréas
\item cholangio-pancréatio-oRM
\item échoendoscopie \footnote{Anesthésie générale\label{org2249f26}}: litihase VBP++
\item \gls{CPRE} \textsuperscript{\ref{org2249f26}} (\danger pancréatite aigüe) : pas en diagnostic
\item cholangiographie percutanée transhépatique \textsuperscript{\ref{org2249f26}} : si échec CPRE mais risque
hémopéritoien, bilio-péritoine, angiocholite
\end{itemize}
\end{document}
